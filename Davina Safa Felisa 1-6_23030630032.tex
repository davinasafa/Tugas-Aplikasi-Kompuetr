\documentclass[a4paper,10pt]{article}
\usepackage{eumat}

\begin{document}
\begin{eulernotebook}
\eulerheading{Aplikasi Komputer}
\begin{eulercomment}
Nama: Davina Safa Felisa\\
NIM: 23030630032\\
Kelas: Matematika E

\begin{eulercomment}
\eulerheading{EMT untuk Perhitungan Aljabar}
\begin{eulercomment}
Pada notebook ini Anda belajar menggunakan EMT untuk melakukan
berbagai perhitungan terkait dengan materi atau topik dalam Aljabar.
Kegiatan yang harus Anda lakukan adalah sebagai berikut:

- Membaca secara cermat dan teliti notebook ini;\\
- Menerjemahkan teks bahasa Inggris ke bahasa Indonesia;\\
- Mencoba contoh-contoh perhitungan (perintah EMT) dengan cara
meng-ENTER setiap perintah EMT yang ada (pindahkan kursor ke baris
perintah)\\
- Jika perlu Anda dapat memodifikasi perintah yang ada dan memberikan
keterangan/penjelasan tambahan terkait hasilnya.\\
- Menyisipkan baris-baris perintah baru untuk mengerjakan soal-soal
Aljabar dari file PDF yang saya berikan;\\
- Memberi catatan hasilnya.\\
- Jika perlu tuliskan soalnya pada teks notebook (menggunakan format
LaTeX).\\
- Gunakan tampilan hasil semua perhitungan yang eksak atau simbolik
dengan format LaTeX. (Seperti contoh-contoh pada notebook ini.)

\end{eulercomment}
\eulersubheading{Contoh pertama}
\begin{eulercomment}
Menyederhanakan bentuk aljabar:

\end{eulercomment}
\begin{eulerformula}
\[
6x^{-3}y^5\times -7x^2y^{-9}
\]
\end{eulerformula}
\begin{eulercomment}
\end{eulercomment}
\begin{eulerprompt}
>$&6*x^(-3)*y^5*-7*x^2*y^(-9)
\end{eulerprompt}
\begin{eulerformula}
\[
-\frac{42}{x\,y^4}
\]
\end{eulerformula}
\begin{eulercomment}
Menjabarkan:

\end{eulercomment}
\begin{eulerformula}
\[
(6x^{-3}+y^5)(-7x^2-y^{-9})
\]
\end{eulerformula}
\begin{eulerprompt}
>$&showev('expand((6*x^(-3)+y^5)*(-7*x^2-y^(-9))))
\end{eulerprompt}
\begin{eulerformula}
\[
{\it expand}\left(\left(-\frac{1}{y^9}-7\,x^2\right)\,\left(y^5+  \frac{6}{x^3}\right)\right)=-7\,x^2\,y^5-\frac{1}{y^4}-\frac{6}{x^3  \,y^9}-\frac{42}{x}
\]
\end{eulerformula}
\eulersubheading{Baris Perintah}
\begin{eulercomment}
Baris perintah Euler terdiri dari satu atau beberapa perintah Euler
diikuti dengan titik koma ";" atau koma ",". Titik koma mencegah
pencetakan hasil. Koma setelah perintah terakhir dapat dihilangkan.

Baris perintah berikut hanya akan mencetak hasil ekspresi, bukan tugas
atau perintah format.
\end{eulercomment}
\begin{eulerprompt}
>r:=2; h:=4; pi*r^2*h/3
\end{eulerprompt}
\begin{euleroutput}
  16.7551608191
\end{euleroutput}
\begin{eulercomment}
Perintah harus dipisahkan dengan yang kosong. Baris perintah berikut
mencetak dua hasilnya.
\end{eulercomment}
\begin{eulerprompt}
>pi*2*r*h, %+2*pi*r*h // Ingat tanda % menyatakan hasil perhitungan terakhir sebelumnya
\end{eulerprompt}
\begin{euleroutput}
  50.2654824574
  100.530964915
\end{euleroutput}
\begin{eulercomment}
Baris perintah dieksekusi dalam urutan yang ditekan pengguna kembali.
Jadi Anda mendapatkan nilai baru setiap kali Anda menjalankan baris
kedua.
\end{eulercomment}
\begin{eulerprompt}
>x := 1;
>x := cos(x) // nilai cosinus (x dalam radian)
\end{eulerprompt}
\begin{euleroutput}
  0.540302305868
\end{euleroutput}
\begin{eulerprompt}
>x := cos(x)
\end{eulerprompt}
\begin{euleroutput}
  0.857553215846
\end{euleroutput}
\begin{eulercomment}
Jika dua garis terhubung dengan "..." kedua garis akan selalu
dieksekusi secara bersamaan.
\end{eulercomment}
\begin{eulerprompt}
>x := 1.5; ...
>x := (x+2/x)/2, x := (x+2/x)/2, x := (x+2/x)/2, 
\end{eulerprompt}
\begin{euleroutput}
  1.41666666667
  1.41421568627
  1.41421356237
\end{euleroutput}
\begin{eulercomment}
Ini juga merupakan cara yang baik untuk menyebarkan perintah panjang
pada dua atau lebih baris. Anda dapat menekan Ctrl+Return untuk
membagi garis menjadi dua pada posisi kursor saat ini, atau Ctrl+Back
untuk menggabungkan garis.

Untuk melipat semua multi-garis tekan Ctrl + L. Kemudian garis-garis
berikutnya hanya akan terlihat, jika salah satunya memiliki fokus.
Untuk melipat satu multi-baris, mulailah baris pertama dengan "\%+".
\end{eulercomment}
\begin{eulerprompt}
>%+ x=4+5; ...
\end{eulerprompt}
\begin{eulercomment}
Garis yang dimulai dengan \%\% tidak akan terlihat sama sekali.
\end{eulercomment}
\begin{euleroutput}
  81
\end{euleroutput}
\begin{eulercomment}
Euler mendukung loop di baris perintah, selama mereka masuk ke dalam
satu baris atau multi-baris. Dalam program, pembatasan ini tidak
berlaku, tentu saja. Untuk informasi lebih lanjut, lihat pengantar
berikut.
\end{eulercomment}
\begin{eulerprompt}
>x=1; for i=1 to 5; x := (x+2/x)/2, end; // menghitung akar 2
\end{eulerprompt}
\begin{euleroutput}
  1.5
  1.41666666667
  1.41421568627
  1.41421356237
  1.41421356237
\end{euleroutput}
\begin{eulercomment}
Tidak apa-apa untuk menggunakan multi-line. Pastikan baris diakhiri
dengan "...".
\end{eulercomment}
\begin{eulerprompt}
>x := 1.5; // comments go here before the ...
>repeat xnew:=(x+2/x)/2; until xnew~=x; ...
>   x := xnew; ...
>end; ...
>x,
\end{eulerprompt}
\begin{euleroutput}
  1.41421356237
\end{euleroutput}
\begin{eulercomment}
Struktur bersyarat juga berfungsi.
\end{eulercomment}
\begin{eulerprompt}
>if E^pi>pi^E; then "Thought so!", endif;
\end{eulerprompt}
\begin{euleroutput}
  Thought so!
\end{euleroutput}
\begin{eulercomment}
Saat Anda menjalankan perintah, kursor dapat berada di posisi mana pun
di baris perintah. Anda dapat kembali ke perintah sebelumnya atau
melompat ke perintah berikutnya dengan tombol panah. Atau Anda dapat
mengklik ke bagian komentar di atas perintah untuk menuju ke perintah.

Saat Anda menggerakkan kursor di sepanjang garis, pasangan tanda
kurung atau kurung buka dan tutup akan disorot. Juga, perhatikan baris
status. Setelah kurung buka fungsi sqrt(), baris status akan
menampilkan teks bantuan untuk fungsi tersebut. Jalankan perintah
dengan tombol kembali.
\end{eulercomment}
\begin{eulerprompt}
>sqrt(sin(10°)/cos(20°))
\end{eulerprompt}
\begin{euleroutput}
  0.429875017772
\end{euleroutput}
\begin{eulercomment}
Untuk melihat bantuan untuk perintah terbaru, buka jendela bantuan
dengan F1. Di sana, Anda dapat memasukkan teks untuk dicari. Pada
baris kosong, bantuan untuk jendela bantuan akan ditampilkan. Anda
dapat menekan escape untuk menghapus garis, atau untuk menutup jendela
bantuan.

Anda dapat mengklik dua kali pada perintah apa pun untuk membuka
bantuan untuk perintah ini. Coba klik dua kali perintah exp di bawah
ini di baris perintah.
\end{eulercomment}
\begin{eulerprompt}
>exp(log(2.5))
\end{eulerprompt}
\begin{euleroutput}
  2.5
\end{euleroutput}
\begin{eulercomment}
Anda dapat menyalin dan menempel di Euler juga. Gunakan Ctrl-C dan
Ctrl-V untuk ini. Untuk menandai teks, seret mouse atau gunakan shift
bersama dengan tombol kursor apa pun. Selain itu, Anda dapat menyalin
tanda kurung yang disorot.
\end{eulercomment}
\eulersubheading{Sintaks Dasar}
\begin{eulercomment}
Euler tahu fungsi matematika biasa. Seperti yang Anda lihat di atas,
fungsi trigonometri bekerja dalam radian atau derajat. Untuk
mengonversi ke derajat, tambahkan simbol derajat (dengan tombol F7) ke
nilainya, atau gunakan fungsi rad(x). Fungsi akar kuadrat disebut
kuadrat dalam Euler. Tentu saja, x\textasciicircum{}(1/2) juga dimungkinkan.

Untuk menyetel variabel, gunakan "=" atau ":=". Demi kejelasan,
pengantar ini menggunakan bentuk yang terakhir. Spasi tidak masalah.
Tapi ruang antara perintah diharapkan.

Beberapa perintah dalam satu baris dipisahkan dengan "," atau ";".
Titik koma menekan output dari perintah. Di akhir baris perintah ","
diasumsikan, jika ";" hilang.
\end{eulercomment}
\begin{eulerprompt}
>g:=9.81; t:=2.5; 1/2*g*t^2
\end{eulerprompt}
\begin{euleroutput}
  30.65625
\end{euleroutput}
\begin{eulercomment}
EMT menggunakan sintaks pemrograman untuk ekspresi. Memasuki

\end{eulercomment}
\begin{eulerformula}
\[
e^2 \cdot \left( \frac{1}{3+4 \log(0.6)}+\frac{1}{7} \right)
\]
\end{eulerformula}
\begin{eulercomment}
Anda harus mengatur tanda kurung yang benar dan menggunakan / untuk
pecahan. Perhatikan tanda kurung yang disorot untuk bantuan.
Perhatikan bahwa konstanta Euler e diberi nama E dalam EMT.
\end{eulercomment}
\begin{eulerprompt}
>E^2*(1/(3+4*log(0.6))+1/7)
\end{eulerprompt}
\begin{euleroutput}
  8.77908249441
\end{euleroutput}
\begin{eulercomment}
Untuk menghitung ekspresi rumit seperti

\end{eulercomment}
\begin{eulerformula}
\[
\left(\frac{\frac17 + \frac18 + 2}{\frac13 + \frac12}\right)^2 \pi
\]
\end{eulerformula}
\begin{eulercomment}
Anda harus memasukkannya dalam bentuk baris.
\end{eulercomment}
\begin{eulerprompt}
>((1/7 + 1/8 + 2) / (1/3 + 1/2))^2 * pi
\end{eulerprompt}
\begin{euleroutput}
  23.2671801626
\end{euleroutput}
\begin{eulercomment}
Letakkan tanda kurung dengan hati-hati di sekitar sub-ekspresi yang
perlu dihitung terlebih dahulu. EMT membantu Anda dengan menyorot
ekspresi bahwa braket penutup selesai. Anda juga harus memasukkan nama
"pi" untuk huruf Yunani pi.

Hasil dari perhitungan ini adalah bilangan floating point. Secara
default dicetak dengan akurasi sekitar 12 digit. Di baris perintah
berikut, kita juga belajar bagaimana kita bisa merujuk ke hasil
sebelumnya dalam baris yang sama.
\end{eulercomment}
\begin{eulerprompt}
>1/3+1/7, fraction %
\end{eulerprompt}
\begin{euleroutput}
  0.47619047619
  10/21
\end{euleroutput}
\begin{eulercomment}
Perintah Euler dapat berupa ekspresi atau perintah primitif. Ekspresi
dibuat dari operator dan fungsi. Jika perlu, itu harus mengandung
tanda kurung untuk memaksa urutan eksekusi yang benar. Jika ragu,
memasang braket adalah ide yang bagus. Perhatikan bahwa EMT
menunjukkan tanda kurung buka dan tutup saat mengedit baris perintah.
\end{eulercomment}
\begin{eulerprompt}
>(cos(pi/4)+1)^3*(sin(pi/4)+1)^2
\end{eulerprompt}
\begin{euleroutput}
  14.4978445072
\end{euleroutput}
\begin{eulercomment}
Operator numerik Euler meliputi

\end{eulercomment}
\begin{eulerttcomment}
 + unary atau operator plus
 - unary atau operator minus
 *, /
 . produk matriks
 a^b daya untuk positif a atau bilangan bulat b (a**b juga berfungsi)
 n! operator faktorial
\end{eulerttcomment}
\begin{eulercomment}

dan masih banyak lagi.

Berikut adalah beberapa fungsi yang mungkin Anda butuhkan. Ada banyak
lagi.

\end{eulercomment}
\begin{eulerttcomment}
 sin,cos,tan,atan,asin,acos,rad,deg
 log,exp,log10,sqrt,logbase
 bin,logbin,logfac,mod,lantai,ceil,bulat,abs,tanda
 conj,re,im,arg,conj,nyata,kompleks
 beta,betai,gamma,complexgamma,ellrf,ellf,ellrd,elle
 bitand, bitor, bitxor, bitnot
\end{eulerttcomment}
\begin{eulercomment}

Beberapa perintah memiliki alias, mis. Untuk log.
\end{eulercomment}
\begin{eulerprompt}
>ln(E^2), arctan(tan(0.5))
\end{eulerprompt}
\begin{euleroutput}
  2
  0.5
\end{euleroutput}
\begin{eulerprompt}
>sin(30°)
\end{eulerprompt}
\begin{euleroutput}
  0.5
\end{euleroutput}
\begin{eulercomment}
Pastikan untuk menggunakan tanda kurung (kurung bulat), setiap kali
ada keraguan tentang urutan eksekusi! Berikut ini tidak sama dengan
(2\textasciicircum{}3)\textasciicircum{}4, yang merupakan default untuk 2\textasciicircum{}3\textasciicircum{}4 di EMT (beberapa sistem
numerik melakukannya dengan cara lain).
\end{eulercomment}
\begin{eulerprompt}
>2^3^4, (2^3)^4, 2^(3^4)
\end{eulerprompt}
\begin{euleroutput}
  2.41785163923e+24
  4096
  2.41785163923e+24
\end{euleroutput}
\eulersubheading{Bilangan Asli}
\begin{eulercomment}
Tipe data utama dalam Euler adalah bilangan real. Real
direpresentasikan dalam format IEEE dengan akurasi sekitar 16 digit
desimal.
\end{eulercomment}
\begin{eulerprompt}
>longest 1/3
\end{eulerprompt}
\begin{euleroutput}
       0.3333333333333333 
\end{euleroutput}
\begin{eulercomment}
Representasi ganda internal membutuhkan 8 byte.
\end{eulercomment}
\begin{eulerprompt}
>printdual(1/3)
\end{eulerprompt}
\begin{euleroutput}
  1.0101010101010101010101010101010101010101010101010101*2^-2
\end{euleroutput}
\begin{eulerprompt}
>printhex(1/3)
\end{eulerprompt}
\begin{euleroutput}
  5.5555555555554*16^-1
\end{euleroutput}
\eulersubheading{String}
\begin{eulercomment}
Sebuah string dalam Euler didefinisikan dengan "...".
\end{eulercomment}
\begin{eulerprompt}
>"A string can contain anything."
\end{eulerprompt}
\begin{euleroutput}
  A string can contain anything.
\end{euleroutput}
\begin{eulercomment}
String dapat digabungkan dengan \textbar{} atau dengan +. Ini juga berfungsi
dengan angka, yang dikonversi menjadi string dalam kasus itu.
\end{eulercomment}
\begin{eulerprompt}
>"The area of the circle with radius " + 2 + " cm is " + pi*4 + " cm^2."
\end{eulerprompt}
\begin{euleroutput}
  The area of the circle with radius 2 cm is 12.5663706144 cm^2.
\end{euleroutput}
\begin{eulercomment}
Fungsi print juga mengonversi angka menjadi string. Ini dapat
mengambil sejumlah angka dan jumlah tempat (0 untuk keluaran padat),
dan secara optimal satu unit.
\end{eulercomment}
\begin{eulerprompt}
>"Golden Ratio : " + print((1+sqrt(5))/2,5,0)
\end{eulerprompt}
\begin{euleroutput}
  Golden Ratio : 1.61803
\end{euleroutput}
\begin{eulercomment}
Ada string khusus tidak ada, yang tidak dicetak. Itu dikembalikan oleh
beberapa fungsi, ketika hasilnya tidak masalah. (Ini dikembalikan
secara otomatis, jika fungsi tidak memiliki pernyataan pengembalian.)
\end{eulercomment}
\begin{eulerprompt}
>none
\end{eulerprompt}
\begin{eulercomment}
Untuk mengonversi string menjadi angka, cukup evaluasi saja. Ini juga
berfungsi untuk ekspresi (lihat di bawah).
\end{eulercomment}
\begin{eulerprompt}
>"1234.5"()
\end{eulerprompt}
\begin{euleroutput}
  1234.5
\end{euleroutput}
\begin{eulercomment}
Untuk mendefinisikan vektor string, gunakan notasi vektor [...].
\end{eulercomment}
\begin{eulerprompt}
>v:=["affe","charlie","bravo"]
\end{eulerprompt}
\begin{euleroutput}
  affe
  charlie
  bravo
\end{euleroutput}
\begin{eulercomment}
Vektor string kosong dilambangkan dengan [none]. Vektor string dapat
digabungkan.
\end{eulercomment}
\begin{eulerprompt}
>w:=[none]; w|v|v
\end{eulerprompt}
\begin{euleroutput}
  affe
  charlie
  bravo
  affe
  charlie
  bravo
\end{euleroutput}
\begin{eulercomment}
String dapat berisi karakter Unicode. Secara internal, string ini
berisi kode UTF-8. Untuk menghasilkan string seperti itu, gunakan
u"..." dan salah satu entitas HTML.

String Unicode dapat digabungkan seperti string lainnya.
\end{eulercomment}
\begin{eulerprompt}
>u"&alpha; = " + 45 + u"&deg;" // pdfLaTeX mungkin gagal menampilkan secara benar
\end{eulerprompt}
\begin{euleroutput}
  α = 45°
\end{euleroutput}
\begin{eulercomment}
I
\end{eulercomment}
\begin{eulercomment}
Dalam komentar, entitas yang sama seperti α, β dll dapat
digunakan. Ini mungkin alternatif cepat untuk Lateks. (Lebih detail di
komentar di bawah).
\end{eulercomment}
\begin{eulercomment}
Ada beberapa fungsi untuk membuat atau menganalisis string unicode.
Fungsi strtochar() akan mengenali string Unicode, dan menerjemahkannya
dengan benar.
\end{eulercomment}
\begin{eulerprompt}
>v=strtochar(u"&Auml; is a German letter")
\end{eulerprompt}
\begin{euleroutput}
  [196,  32,  105,  115,  32,  97,  32,  71,  101,  114,  109,  97,  110,
  32,  108,  101,  116,  116,  101,  114]
\end{euleroutput}
\begin{eulercomment}
Hasilnya adalah vektor angka Unicode. Fungsi kebalikannya adalah
chartoutf().
\end{eulercomment}
\begin{eulerprompt}
>v[1]=strtochar(u"&Uuml;")[1]; chartoutf(v)
\end{eulerprompt}
\begin{euleroutput}
  Ü is a German letter
\end{euleroutput}
\begin{eulercomment}
Fungsi utf() dapat menerjemahkan string dengan entitas dalam variabel
menjadi string Unicode.
\end{eulercomment}
\begin{eulerprompt}
>s="We have &alpha;=&beta;."; utf(s) // pdfLaTeX mungkin gagal menampilkan secara benar
\end{eulerprompt}
\begin{euleroutput}
  We have α=β.
\end{euleroutput}
\begin{eulercomment}
Dimungkinkan juga untuk menggunakan entitas numerik.
\end{eulercomment}
\begin{eulerprompt}
>u"&#196;hnliches"
\end{eulerprompt}
\begin{euleroutput}
  Ähnliches
\end{euleroutput}
\eulersubheading{Nilai Boolean}
\begin{eulercomment}
Nilai Boolean direpresentasikan dengan 1=true atau 0=false dalam
Euler. String dapat dibandingkan, seperti halnya angka.
\end{eulercomment}
\begin{eulerprompt}
>2<1, "apel"<"banana"
\end{eulerprompt}
\begin{euleroutput}
  0
  1
\end{euleroutput}
\begin{eulercomment}
"dan" adalah operator "\&\&" dan "atau" adalah operator "\textbar{}\textbar{}", seperti
dalam bahasa C. (Kata-kata "dan" dan "atau" hanya dapat digunakan
dalam kondisi untuk "jika".)
\end{eulercomment}
\begin{eulerprompt}
>2<E && E<3
\end{eulerprompt}
\begin{euleroutput}
  1
\end{euleroutput}
\begin{eulercomment}
Operator Boolean mematuhi aturan bahasa matriks.
\end{eulercomment}
\begin{eulerprompt}
>(1:10)>5, nonzeros(%)
\end{eulerprompt}
\begin{euleroutput}
  [0,  0,  0,  0,  0,  1,  1,  1,  1,  1]
  [6,  7,  8,  9,  10]
\end{euleroutput}
\begin{eulercomment}
Anda dapat menggunakan fungsi bukan nol() untuk mengekstrak elemen
tertentu dari vektor. Dalam contoh, kami menggunakan isprima
bersyarat(n).
\end{eulercomment}
\begin{eulerprompt}
>N=2|3:2:99 // N berisi elemen 2 dan bilangan2 ganjil dari 3 s.d. 99
\end{eulerprompt}
\begin{euleroutput}
  [2,  3,  5,  7,  9,  11,  13,  15,  17,  19,  21,  23,  25,  27,  29,
  31,  33,  35,  37,  39,  41,  43,  45,  47,  49,  51,  53,  55,  57,
  59,  61,  63,  65,  67,  69,  71,  73,  75,  77,  79,  81,  83,  85,
  87,  89,  91,  93,  95,  97,  99]
\end{euleroutput}
\begin{eulerprompt}
>N[nonzeros(isprime(N))] //pilih anggota2 N yang prima
\end{eulerprompt}
\begin{euleroutput}
  [2,  3,  5,  7,  11,  13,  17,  19,  23,  29,  31,  37,  41,  43,  47,
  53,  59,  61,  67,  71,  73,  79,  83,  89,  97]
\end{euleroutput}
\eulersubheading{Format Keluaran}
\begin{eulercomment}
Format output default EMT mencetak 12 digit. Untuk memastikan bahwa
kami melihat default, kami mengatur ulang format.
\end{eulercomment}
\begin{eulerprompt}
>defformat; pi
\end{eulerprompt}
\begin{euleroutput}
  3.14159265359
\end{euleroutput}
\begin{eulercomment}
Secara internal, EMT menggunakan standar IEEE untuk bilangan ganda
dengan sekitar 16 digit desimal. Untuk melihat jumlah digit penuh,
gunakan perintah "format terpanjang", atau kita gunakan operator
"terpanjang" untuk menampilkan hasil dalam format terpanjang.
\end{eulercomment}
\begin{eulerprompt}
>longest pi
\end{eulerprompt}
\begin{euleroutput}
        3.141592653589793 
\end{euleroutput}
\begin{eulercomment}
Berikut adalah representasi heksadesimal internal dari bilangan ganda.
\end{eulercomment}
\begin{eulerprompt}
>printhex(pi)
\end{eulerprompt}
\begin{euleroutput}
  3.243F6A8885A30*16^0
\end{euleroutput}
\begin{eulercomment}
Format output dapat diubah secara permanen dengan perintah format.
\end{eulercomment}
\begin{eulerprompt}
>format(12,5); 1/3, pi, sin(1)
\end{eulerprompt}
\begin{euleroutput}
      0.33333 
      3.14159 
      0.84147 
\end{euleroutput}
\begin{eulercomment}
Standarnya adalah format (12).
\end{eulercomment}
\begin{eulerprompt}
>format(12); 1/3
\end{eulerprompt}
\begin{euleroutput}
  0.333333333333
\end{euleroutput}
\begin{eulercomment}
Fungsi seperti "shortestformat", "shortformat", "longformat" bekerja
untuk vektor dengan cara berikut.
\end{eulercomment}
\begin{eulerprompt}
>shortestformat; random(3,8)
\end{eulerprompt}
\begin{euleroutput}
    0.66    0.2   0.89   0.28   0.53   0.31   0.44    0.3 
    0.28   0.88   0.27    0.7   0.22   0.45   0.31   0.91 
    0.19   0.46  0.095    0.6   0.43   0.73   0.47   0.32 
\end{euleroutput}
\begin{eulercomment}
Format default untuk skalar adalah format (12). Tapi ini bisa diubah.
\end{eulercomment}
\begin{eulerprompt}
>setscalarformat(5); pi
\end{eulerprompt}
\begin{euleroutput}
  3.1416
\end{euleroutput}
\begin{eulercomment}
Fungsi "format terpanjang" mengatur format skalar juga.
\end{eulercomment}
\begin{eulerprompt}
>longestformat; pi
\end{eulerprompt}
\begin{euleroutput}
  3.141592653589793
\end{euleroutput}
\begin{eulercomment}
Untuk referensi, berikut adalah daftar format output yang paling
penting.

\end{eulercomment}
\begin{eulerttcomment}
 format terpendek format pendek format panjang, format terpanjang
 format(panjang,digit) format baik(panjang)
 fracformat (panjang)
 mengubah bentuk
\end{eulerttcomment}
\begin{eulercomment}

Akurasi internal EMT adalah sekitar 16 tempat desimal, yang merupakan
standar IEEE. Angka disimpan dalam format internal ini.

Tetapi format output EMT dapat diatur dengan cara yang fleksibel.
\end{eulercomment}
\begin{eulerprompt}
>longestformat; pi,
\end{eulerprompt}
\begin{euleroutput}
  3.141592653589793
\end{euleroutput}
\begin{eulerprompt}
>format(10,5); pi
\end{eulerprompt}
\begin{euleroutput}
    3.14159 
\end{euleroutput}
\begin{eulercomment}
The default is defformat().
\end{eulercomment}
\begin{eulerprompt}
>defformat; // default
\end{eulerprompt}
\begin{eulercomment}
Ada operator pendek yang hanya mencetak satu nilai. Operator
"terpanjang" akan mencetak semua digit angka yang valid.
\end{eulercomment}
\begin{eulerprompt}
>longest pi^2/2
\end{eulerprompt}
\begin{euleroutput}
        4.934802200544679 
\end{euleroutput}
\begin{eulercomment}
Ada juga operator pendek untuk mencetak hasil dalam format pecahan.
Kami sudah menggunakannya di atas.
\end{eulercomment}
\begin{eulerprompt}
>fraction 1+1/2+1/3+1/4
\end{eulerprompt}
\begin{euleroutput}
  25/12
\end{euleroutput}
\begin{eulercomment}
Karena format internal menggunakan cara biner untuk menyimpan angka,
nilai 0,1 tidak akan direpresentasikan dengan tepat. Kesalahan
bertambah sedikit, seperti yang Anda lihat dalam perhitungan berikut.
\end{eulercomment}
\begin{eulerprompt}
>longest 0.1+0.1+0.1+0.1+0.1+0.1+0.1+0.1+0.1+0.1-1
\end{eulerprompt}
\begin{euleroutput}
   -1.110223024625157e-16 
\end{euleroutput}
\begin{eulercomment}
Tetapi dengan "format panjang" default Anda tidak akan melihat ini.
Untuk kenyamanan, output dari angka yang sangat kecil adalah 0.
\end{eulercomment}
\begin{eulerprompt}
>0.1+0.1+0.1+0.1+0.1+0.1+0.1+0.1+0.1+0.1-1
\end{eulerprompt}
\begin{euleroutput}
  0
\end{euleroutput}
\eulerheading{Ekspresi}
\begin{eulercomment}
String atau nama dapat digunakan untuk menyimpan ekspresi matematika,
yang dapat dievaluasi oleh EMT. Untuk ini, gunakan tanda kurung
setelah ekspresi. Jika Anda bermaksud menggunakan string sebagai
ekspresi, gunakan konvensi untuk menamainya "fx" atau "fxy" dll.
Ekspresi lebih diutamakan daripada fungsi.

Variabel global dapat digunakan dalam evaluasi.
\end{eulercomment}
\begin{eulerprompt}
>r:=2; fx:="pi*r^2"; longest fx()
\end{eulerprompt}
\begin{euleroutput}
        12.56637061435917 
\end{euleroutput}
\begin{eulercomment}
Parameter ditetapkan ke x, y, dan z dalam urutan itu. Parameter
tambahan dapat ditambahkan menggunakan parameter yang ditetapkan.
\end{eulercomment}
\begin{eulerprompt}
>fx:="a*sin(x)^2"; fx(5,a=-1)
\end{eulerprompt}
\begin{euleroutput}
  -0.919535764538
\end{euleroutput}
\begin{eulercomment}
Perhatikan bahwa ekspresi akan selalu menggunakan variabel global,
bahkan jika ada variabel dalam fungsi dengan nama yang sama. (Jika
tidak, evaluasi ekspresi dalam fungsi dapat memberikan hasil yang
sangat membingungkan bagi pengguna yang memanggil fungsi tersebut.)
\end{eulercomment}
\begin{eulerprompt}
>at:=4; function f(expr,x,at) := expr(x); ...
>f("at*x^2",3,5) // computes 4*3^2 not 5*3^2
\end{eulerprompt}
\begin{euleroutput}
  36
\end{euleroutput}
\begin{eulercomment}
Jika Anda ingin menggunakan nilai lain untuk "at" daripada nilai
global, Anda perlu menambahkan "at=value".
\end{eulercomment}
\begin{eulerprompt}
>at:=4; function f(expr,x,a) := expr(x,at=a); ...
>f("at*x^2",3,5)
\end{eulerprompt}
\begin{euleroutput}
  45
\end{euleroutput}
\begin{eulercomment}
Untuk referensi, kami berkomentar bahwa koleksi panggilan (dibahas di
tempat lain) dapat berisi ekspresi. Jadi kita bisa membuat contoh di
atas sebagai berikut.
\end{eulercomment}
\begin{eulerprompt}
>at:=4; function f(expr,x) := expr(x); ...
>f(\{\{"at*x^2",at=5\}\},3)
\end{eulerprompt}
\begin{euleroutput}
  45
\end{euleroutput}
\begin{eulercomment}
Ekspresi dalam x sering digunakan seperti fungsi.\\
Perhatikan bahwa mendefinisikan fungsi dengan nama yang sama seperti
ekspresi simbolik global menghapus variabel ini untuk menghindari
kebingungan antara ekspresi simbolik dan fungsi.
\end{eulercomment}
\begin{eulerprompt}
>f &= 5*x;
>function f(x) := 6*x;
>f(2)
\end{eulerprompt}
\begin{euleroutput}
  12
\end{euleroutput}
\begin{eulercomment}
Dengan cara konvensi, ekspresi simbolik atau numerik harus diberi nama
fx, fxy dll. Skema penamaan ini tidak boleh digunakan untuk fungsi.
\end{eulercomment}
\begin{eulerprompt}
>fx &= diff(x^x,x); $&fx
\end{eulerprompt}
\begin{eulerformula}
\[
x^{x}\,\left(\log x+1\right)
\]
\end{eulerformula}
\begin{eulercomment}
Bentuk khusus dari ekspresi memungkinkan variabel apa pun sebagai
parameter tanpa nama untuk evaluasi ekspresi, bukan hanya "x", "y"
dll. Untuk ini, mulai ekspresi dengan "@(variabel) ...".
\end{eulercomment}
\begin{eulerprompt}
>"@(a,b) a^2+b^2", %(4,5)
\end{eulerprompt}
\begin{euleroutput}
  @(a,b) a^2+b^2
  41
\end{euleroutput}
\begin{eulercomment}
Ini memungkinkan untuk memanipulasi ekspresi dalam variabel lain untuk
fungsi EMT yang membutuhkan ekspresi dalam "x".

Cara paling dasar untuk mendefinisikan fungsi sederhana adalah dengan
menyimpan rumusnya dalam ekspresi simbolis atau numerik. Jika variabel
utama adalah x, ekspresi dapat dievaluasi seperti fungsi.

Seperti yang Anda lihat dalam contoh berikut, variabel global terlihat
selama evaluasi.
\end{eulercomment}
\begin{eulerprompt}
>fx &= x^3-a*x;  ...
>a=1.2; fx(0.5)
\end{eulerprompt}
\begin{euleroutput}
  -0.475
\end{euleroutput}
\begin{eulercomment}
Semua variabel lain dalam ekspresi dapat ditentukan dalam evaluasi
menggunakan parameter yang ditetapkan.
\end{eulercomment}
\begin{eulerprompt}
>fx(0.5,a=1.1)
\end{eulerprompt}
\begin{euleroutput}
  -0.425
\end{euleroutput}
\begin{eulercomment}
Sebuah ekspresi tidak perlu simbolis. Ini diperlukan, jika ekspresi
berisi fungsi, yang hanya diketahui di kernel numerik, bukan di
Maxima.

\begin{eulercomment}
\eulerheading{Matematika Simbolik}
\begin{eulercomment}
EMT melakukan matematika simbolis dengan bantuan Maxima. Untuk
detailnya, mulailah dengan tutorial berikut, atau telusuri referensi
untuk Maxima. Para ahli di Maxima harus mencatat bahwa ada perbedaan
sintaks antara sintaks asli Maxima dan sintaks default ekspresi
simbolik di EMT.

Matematika simbolik terintegrasi dengan mulus ke dalam Euler dengan \&.
Ekspresi apa pun yang dimulai dengan \& adalah ekspresi simbolis. Itu
dievaluasi dan dicetak oleh Maxima.

Pertama-tama, Maxima memiliki aritmatika "tak terbatas" yang dapat
menangani angka yang sangat besar.
\end{eulercomment}
\begin{eulerprompt}
>$&44!
\end{eulerprompt}
\begin{eulerformula}
\[
2658271574788448768043625811014615890319638528000000000
\]
\end{eulerformula}
\begin{eulercomment}
Dengan cara ini, Anda dapat menghitung hasil yang besar dengan tepat.
Mari kita hitung

\end{eulercomment}
\begin{eulerformula}
\[
C(44,10) = \frac{44!}{34! \cdot 10!}
\]
\end{eulerformula}
\begin{eulerprompt}
>$& 44!/(34!*10!) // nilai C(44,10)
\end{eulerprompt}
\begin{eulerformula}
\[
2481256778
\]
\end{eulerformula}
\begin{eulercomment}
Tentu saja, Maxima memiliki fungsi yang lebih efisien untuk ini
(seperti halnya bagian numerik dari EMT).
\end{eulercomment}
\begin{eulerprompt}
>$binomial(44,10) //menghitung C(44,10) menggunakan fungsi binomial()
\end{eulerprompt}
\begin{eulerformula}
\[
2481256778
\]
\end{eulerformula}
\begin{eulercomment}
Untuk mempelajari lebih lanjut tentang fungsi tertentu klik dua kali
di atasnya. Misalnya, coba klik dua kali pada "\&binomial" di baris
perintah sebelumnya. Ini membuka dokumentasi Maxima seperti yang
disediakan oleh penulis program itu.

Anda akan belajar bahwa yang berikut ini juga berfungsi.

\end{eulercomment}
\begin{eulerformula}
\[
C(x,3)=\frac{x!}{(x-3)!3!}=\frac{(x-2)(x-1)x}{6}
\]
\end{eulerformula}
\begin{eulerprompt}
>$binomial(x,3) // C(x,3)
\end{eulerprompt}
\begin{eulerformula}
\[
\frac{\left(x-2\right)\,\left(x-1\right)\,x}{6}
\]
\end{eulerformula}
\begin{eulercomment}
Jika Anda ingin mengganti x dengan nilai tertentu, gunakan "dengan".
\end{eulercomment}
\begin{eulerprompt}
>$&binomial(x,3) with x=10 // substitusi x=10 ke C(x,3)
\end{eulerprompt}
\begin{eulerformula}
\[
120
\]
\end{eulerformula}
\begin{eulercomment}
Dengan begitu Anda dapat menggunakan solusi persamaan dalam persamaan
lain.

Ekspresi simbolik dicetak oleh Maxima dalam bentuk 2D. Alasan untuk
ini adalah bendera simbolis khusus dalam string.

Seperti yang akan Anda lihat pada contoh sebelumnya dan berikut, jika
Anda telah menginstal LaTeX, Anda dapat mencetak ekspresi simbolis
dengan Lateks. Jika tidak, perintah berikut akan mengeluarkan pesan
kesalahan.

Untuk mencetak ekspresi simbolis dengan LaTeX, gunakan \textdollar{} di depan \&
(atau Anda dapat menghilangkan \&) sebelum perintah. Jangan menjalankan
perintah Maxima dengan \textdollar{}, jika Anda tidak menginstal LaTeX.
\end{eulercomment}
\begin{eulerprompt}
>$(3+x)/(x^2+1)
\end{eulerprompt}
\begin{eulerformula}
\[
\frac{x+3}{x^2+1}
\]
\end{eulerformula}
\begin{eulercomment}
Ekspresi simbolik diuraikan oleh Euler. Jika Anda membutuhkan sintaks
yang kompleks dalam satu ekspresi, Anda dapat menyertakan ekspresi
dalam "...". Untuk menggunakan lebih dari ekspresi sederhana adalah
mungkin, tetapi sangat tidak disarankan.
\end{eulercomment}
\begin{eulerprompt}
>&"v := 5; v^2"
\end{eulerprompt}
\begin{euleroutput}
  
                                    25
  
\end{euleroutput}
\begin{eulercomment}
Untuk kelengkapan, kami menyatakan bahwa ekspresi simbolik dapat
digunakan dalam program, tetapi perlu diapit dalam tanda kutip. Selain
itu, jauh lebih efektif untuk memanggil Maxima pada waktu kompilasi
jika memungkinkan.
\end{eulercomment}
\begin{eulerprompt}
>$&expand((1+x)^4), $&factor(diff(%,x)) // diff: turunan, factor: faktor
\end{eulerprompt}
\begin{eulerformula}
\[
4\,\left(x+1\right)^3
\]
\end{eulerformula}
\eulerimg{0}{images/Davina Safa Felisa 1-6-017-large.png}
\begin{eulercomment}
Sekali lagi, \% mengacu pada hasil sebelumnya.

Untuk mempermudah, kami menyimpan solusi ke variabel simbolik.
Variabel simbolik didefinisikan dengan "\&=".
\end{eulercomment}
\begin{eulerprompt}
>fx &= (x+1)/(x^4+1); $&fx
\end{eulerprompt}
\begin{eulerformula}
\[
\frac{x+1}{x^4+1}
\]
\end{eulerformula}
\begin{eulercomment}
Ekspresi simbolik dapat digunakan dalam ekspresi simbolik lainnya.
\end{eulercomment}
\begin{eulerprompt}
>$&factor(diff(fx,x))
\end{eulerprompt}
\begin{eulerformula}
\[
\frac{-3\,x^4-4\,x^3+1}{\left(x^4+1\right)^2}
\]
\end{eulerformula}
\begin{eulercomment}
Masukan langsung dari perintah Maxima juga tersedia. Mulai baris
perintah dengan "::". Sintaks Maxima disesuaikan dengan sintaks EMT
(disebut "mode kompatibilitas").
\end{eulercomment}
\begin{eulerprompt}
>&factor(20!)
\end{eulerprompt}
\begin{euleroutput}
  
                           2432902008176640000
  
\end{euleroutput}
\begin{eulerprompt}
>::: factor(10!)
\end{eulerprompt}
\begin{euleroutput}
  
                                 8  4  2
                                2  3  5  7
  
\end{euleroutput}
\begin{eulerprompt}
>:: factor(20!)
\end{eulerprompt}
\begin{euleroutput}
  
                          18  8  4  2
                         2   3  5  7  11 13 17 19
  
\end{euleroutput}
\begin{eulercomment}
Jika Anda ahli dalam Maxima, Anda mungkin ingin menggunakan sintaks
asli Maxima. Anda dapat melakukannya dengan ":::".
\end{eulercomment}
\begin{eulerprompt}
>::: av:g$ av^2;
\end{eulerprompt}
\begin{euleroutput}
  
                                     2
                                    g
  
\end{euleroutput}
\begin{eulerprompt}
>fx &= x^3*exp(x), $fx
\end{eulerprompt}
\begin{euleroutput}
  
                                   3  x
                                  x  E
  
\end{euleroutput}
\begin{eulerformula}
\[
x^3\,e^{x}
\]
\end{eulerformula}
\begin{eulercomment}
Jika Anda ahli dalam Maxima, Anda mungkin ingin menggunakan sintaks
asli Maxima. Anda dapat melakukannya dengan ":::".
\end{eulercomment}
\begin{eulerprompt}
>& (fx with x=5), $%, &float(%)
\end{eulerprompt}
\begin{euleroutput}
  
                                       5
                                  125 E
  
\end{euleroutput}
\begin{eulerformula}
\[
125\,e^5
\]
\end{eulerformula}
\begin{euleroutput}
  
                            18551.64488782208
  
\end{euleroutput}
\begin{eulerprompt}
>fx(5)
\end{eulerprompt}
\begin{euleroutput}
  18551.6448878
\end{euleroutput}
\begin{eulerprompt}
>&(fx with x=10)-(fx with x=5), &float(%)
\end{eulerprompt}
\begin{euleroutput}
  
                                  10        5
                            1000 E   - 125 E
  
  
                           2.20079141499189e+7
  
\end{euleroutput}
\begin{eulercomment}
Untuk evaluasi ekspresi dengan nilai variabel tertentu, Anda dapat
menggunakan operator "with".

Baris perintah berikut juga menunjukkan bahwa Maxima dapat
mengevaluasi ekspresi secara numerik dengan float().
\end{eulercomment}
\begin{eulerprompt}
>&(fx with x=10)-(fx with x=5), &float(%)
\end{eulerprompt}
\begin{euleroutput}
  
                                  10        5
                            1000 E   - 125 E
  
  
                           2.20079141499189e+7
  
\end{euleroutput}
\begin{eulerprompt}
>$factor(diff(fx,x,2))
\end{eulerprompt}
\begin{eulerformula}
\[
x\,\left(x^2+6\,x+6\right)\,e^{x}
\]
\end{eulerformula}
\begin{eulercomment}
Untuk mendapatkan kode Lateks untuk ekspresi, Anda dapat menggunakan
perintah tex.
\end{eulercomment}
\begin{eulerprompt}
>tex(fx)
\end{eulerprompt}
\begin{euleroutput}
  x^3\(\backslash\),e^\{x\}
\end{euleroutput}
\begin{eulercomment}
Ekspresi simbolik dapat dievaluasi seperti ekspresi numerik.
\end{eulercomment}
\begin{eulerprompt}
>fx(0.5)
\end{eulerprompt}
\begin{euleroutput}
  0.206090158838
\end{euleroutput}
\begin{eulercomment}
Dalam ekspresi simbolis, ini tidak berfungsi, karena Maxima tidak
mendukungnya. Sebagai gantinya, gunakan sintaks "with" (bentuk yang
lebih bagus dari perintah at(...) dari Maxima).
\end{eulercomment}
\begin{eulerprompt}
>$&fx with x=1/2
\end{eulerprompt}
\begin{eulerformula}
\[
\frac{\sqrt{e}}{8}
\]
\end{eulerformula}
\begin{eulercomment}
Penugasan juga bisa bersifat simbolis.
\end{eulercomment}
\begin{eulerprompt}
>$&fx with x=1+t
\end{eulerprompt}
\begin{eulerformula}
\[
\left(t+1\right)^3\,e^{t+1}
\]
\end{eulerformula}
\begin{eulercomment}
Perintah solve memecahkan ekspresi simbolik untuk variabel di Maxima.
Hasilnya adalah vektor solusi.
\end{eulercomment}
\begin{eulerprompt}
>$&solve(x^2+x=4,x)
\end{eulerprompt}
\begin{eulerformula}
\[
\left[ x=\frac{-\sqrt{17}-1}{2} , x=\frac{\sqrt{17}-1}{2} \right] 
\]
\end{eulerformula}
\begin{eulercomment}
Bandingkan dengan perintah numerik "selesaikan" di Euler, yang
membutuhkan nilai awal, dan secara opsional nilai target.
\end{eulercomment}
\begin{eulerprompt}
>solve("x^2+x",1,y=4)
\end{eulerprompt}
\begin{euleroutput}
  1.56155281281
\end{euleroutput}
\begin{eulercomment}
Nilai numerik dari solusi simbolik dapat dihitung dengan evaluasi
hasil simbolis. Euler akan membaca tugas x= dll. Jika Anda tidak
memerlukan hasil numerik untuk perhitungan lebih lanjut, Anda juga
dapat membiarkan Maxima menemukan nilai numerik.
\end{eulercomment}
\begin{eulerprompt}
>sol &= solve(x^2+2*x=4,x); $&sol, sol(), $&float(sol)
\end{eulerprompt}
\begin{eulerformula}
\[
\left[ x=-\sqrt{5}-1 , x=\sqrt{5}-1 \right] 
\]
\end{eulerformula}
\begin{euleroutput}
  [-3.23607,  1.23607]
\end{euleroutput}
\begin{eulerformula}
\[
\left[ x=-3.23606797749979 , x=1.23606797749979 \right] 
\]
\end{eulerformula}
\begin{eulercomment}
Untuk mendapatkan solusi simbolis tertentu, seseorang dapat
menggunakan "dengan" dan indeks.
\end{eulercomment}
\begin{eulerprompt}
>$&solve(x^2+x=1,x), x2 &= x with %[2]; $&x2
\end{eulerprompt}
\begin{eulerformula}
\[
\frac{\sqrt{5}-1}{2}
\]
\end{eulerformula}
\eulerimg{1}{images/Davina Safa Felisa 1-6-029-large.png}
\begin{eulercomment}
Untuk menyelesaikan sistem persamaan, gunakan vektor persamaan.
Hasilnya adalah vektor solusi.
\end{eulercomment}
\begin{eulerprompt}
>sol &= solve([x+y=3,x^2+y^2=5],[x,y]); $&sol, $&x*y with sol[1]
\end{eulerprompt}
\begin{eulerformula}
\[
2
\]
\end{eulerformula}
\eulerimg{0}{images/Davina Safa Felisa 1-6-031-large.png}
\begin{eulercomment}
Ekspresi simbolis dapat memiliki bendera, yang menunjukkan perlakuan
khusus di Maxima. Beberapa flag dapat digunakan sebagai perintah juga,
yang lain tidak. Bendera ditambahkan dengan "\textbar{}" (bentuk yang lebih
bagus dari "ev(...,flags)")
\end{eulercomment}
\begin{eulerprompt}
>$& diff((x^3-1)/(x+1),x) //turunan bentuk pecahan
\end{eulerprompt}
\begin{eulerformula}
\[
\frac{3\,x^2}{x+1}-\frac{x^3-1}{\left(x+1\right)^2}
\]
\end{eulerformula}
\begin{eulerprompt}
>$& diff((x^3-1)/(x+1),x) | ratsimp //menyederhanakan pecahan
\end{eulerprompt}
\begin{eulerformula}
\[
\frac{2\,x^3+3\,x^2+1}{x^2+2\,x+1}
\]
\end{eulerformula}
\begin{eulerprompt}
>$&factor(%)
\end{eulerprompt}
\begin{eulerformula}
\[
\frac{2\,x^3+3\,x^2+1}{\left(x+1\right)^2}
\]
\end{eulerformula}
\eulerheading{Fungsi}
\begin{eulercomment}
Dalam EMT, fungsi adalah program yang didefinisikan dengan perintah
"fungsi". Ini bisa berupa fungsi satu baris atau fungsi multibaris.\\
Fungsi satu baris dapat berupa numerik atau simbolis. Fungsi satu
baris numerik didefinisikan oleh ":=".
\end{eulercomment}
\begin{eulerprompt}
>function f(x) := x*sqrt(x^2+1)
\end{eulerprompt}
\begin{eulercomment}
Untuk gambaran umum, kami menunjukkan semua kemungkinan definisi untuk
fungsi satu baris. Suatu fungsi dapat dievaluasi sama seperti fungsi
Euler bawaan lainnya.
\end{eulercomment}
\begin{eulerprompt}
>f(2)
\end{eulerprompt}
\begin{euleroutput}
  4.472135955
\end{euleroutput}
\begin{eulercomment}
Fungsi ini akan bekerja untuk vektor juga, dengan mematuhi bahasa
matriks Euler, karena ekspresi yang digunakan dalam fungsi
divektorkan.
\end{eulercomment}
\begin{eulerprompt}
>f(0:0.1:1)
\end{eulerprompt}
\begin{euleroutput}
  [0,  0.100499,  0.203961,  0.313209,  0.430813,  0.559017,  0.699714,
  0.854459,  1.0245,  1.21083,  1.41421]
\end{euleroutput}
\begin{eulercomment}
Fungsi dapat diplot. Alih-alih ekspresi, kita hanya perlu memberikan
nama fungsi.

Berbeda dengan ekspresi simbolik atau numerik, nama fungsi harus
diberikan dalam string.
\end{eulercomment}
\begin{eulerprompt}
>solve("f",1,y=1)
\end{eulerprompt}
\begin{euleroutput}
  0.786151377757
\end{euleroutput}
\begin{eulercomment}
Secara default, jika Anda perlu menimpa fungsi bawaan, Anda harus
menambahkan kata kunci "menimpa". Menimpa fungsi bawaan berbahaya dan
dapat menyebabkan masalah untuk fungsi lain tergantung pada fungsi
tersebut.

Anda masih dapat memanggil fungsi bawaan sebagai "\_...", jika itu
adalah fungsi di inti Euler.
\end{eulercomment}
\begin{eulerprompt}
>function overwrite sin (x) := _sin(x°) // redine sine in degrees
>sin(45)
\end{eulerprompt}
\begin{euleroutput}
  0.707106781187
\end{euleroutput}
\begin{eulercomment}
Lebih baik kita menghapus redefinisi dosa ini.
\end{eulercomment}
\begin{eulerprompt}
>forget sin; sin(pi/4)
\end{eulerprompt}
\begin{euleroutput}
  0.707106781187
\end{euleroutput}
\eulersubheading{Parameter Default}
\begin{eulercomment}
Fungsi numerik dapat memiliki parameter default.
\end{eulercomment}
\begin{eulerprompt}
>function f(x,a=1) := a*x^2
\end{eulerprompt}
\begin{eulercomment}
Menghilangkan parameter ini menggunakan nilai default.
\end{eulercomment}
\begin{eulerprompt}
>f(4)
\end{eulerprompt}
\begin{euleroutput}
  16
\end{euleroutput}
\begin{eulercomment}
Menyetelnya akan menimpa nilai default.
\end{eulercomment}
\begin{eulerprompt}
>f(4,5) ...
\end{eulerprompt}
\begin{euleroutput}
  80
\end{euleroutput}
\begin{eulercomment}
Parameter yang ditetapkan menimpanya juga. Ini digunakan oleh banyak
fungsi Euler seperti plot2d, plot3d.
\end{eulercomment}
\begin{eulerprompt}
>f(4,a=1)
\end{eulerprompt}
\begin{euleroutput}
  16
\end{euleroutput}
\begin{eulercomment}
Jika suatu variabel bukan parameter, itu harus global. Fungsi satu
baris dapat melihat variabel global.
\end{eulercomment}
\begin{eulerprompt}
>function f(x) := a*x^2
>a=6; f(2)
\end{eulerprompt}
\begin{euleroutput}
  24
\end{euleroutput}
\begin{eulercomment}
Tetapi parameter yang ditetapkan menimpa nilai global.

Jika argumen tidak ada dalam daftar parameter yang telah ditentukan
sebelumnya, argumen tersebut harus dideklarasikan dengan ":="!
\end{eulercomment}
\begin{eulerprompt}
>f(2,a:=5)
\end{eulerprompt}
\begin{euleroutput}
  20
\end{euleroutput}
\begin{eulercomment}
Fungsi simbolik didefinisikan dengan "\&=". Mereka didefinisikan dalam
Euler dan Maxima, dan bekerja di kedua dunia. Ekspresi yang
mendefinisikan dijalankan melalui Maxima sebelum definisi.
\end{eulercomment}
\begin{eulerprompt}
>function g(x) &= x^3-x*exp(-x); $&g(x)
\end{eulerprompt}
\begin{eulerformula}
\[
x^3-x\,e^ {- x }
\]
\end{eulerformula}
\begin{eulercomment}
Fungsi simbolik dapat digunakan dalam ekspresi simbolik.
\end{eulercomment}
\begin{eulerprompt}
>$&diff(g(x),x), $&% with x=4/3
\end{eulerprompt}
\begin{eulerformula}
\[
\frac{e^ {- \frac{4}{3} }}{3}+\frac{16}{3}
\]
\end{eulerformula}
\eulerimg{1}{images/Davina Safa Felisa 1-6-037-large.png}
\begin{eulercomment}
Mereka juga dapat digunakan dalam ekspresi numerik. Tentu saja, ini
hanya akan berfungsi jika EMT dapat menginterpretasikan semua yang ada
di dalam fungsi tersebut.
\end{eulercomment}
\begin{eulerprompt}
>g(5+g(1))
\end{eulerprompt}
\begin{euleroutput}
  178.635099908
\end{euleroutput}
\begin{eulercomment}
Mereka dapat digunakan untuk mendefinisikan fungsi atau ekspresi
simbolis lainnya.
\end{eulercomment}
\begin{eulerprompt}
>function G(x) &= factor(integrate(g(x),x)); $&G(c) // integrate: mengintegralkan
\end{eulerprompt}
\begin{eulerformula}
\[
\frac{e^ {- c }\,\left(c^4\,e^{c}+4\,c+4\right)}{4}
\]
\end{eulerformula}
\begin{eulerprompt}
>solve(&g(x),0.5)
\end{eulerprompt}
\begin{euleroutput}
  0.703467422498
\end{euleroutput}
\begin{eulercomment}
Berikut ini juga berfungsi, karena Euler menggunakan ekspresi simbolis
dalam fungsi g, jika tidak menemukan variabel simbolik g, dan jika ada
fungsi simbolis g.
\end{eulercomment}
\begin{eulerprompt}
>solve(&g,0.5)
\end{eulerprompt}
\begin{euleroutput}
  0.703467422498
\end{euleroutput}
\begin{eulerprompt}
>function P(x,n) &= (2*x-1)^n; $&P(x,n)
\end{eulerprompt}
\begin{eulerformula}
\[
\left(2\,x-1\right)^{n}
\]
\end{eulerformula}
\begin{eulerprompt}
>function Q(x,n) &= (x+2)^n; $&Q(x,n)
\end{eulerprompt}
\begin{eulerformula}
\[
\left(x+2\right)^{n}
\]
\end{eulerformula}
\begin{eulerprompt}
>$&P(x,4), $&expand(%)
\end{eulerprompt}
\begin{eulerformula}
\[
16\,x^4-32\,x^3+24\,x^2-8\,x+1
\]
\end{eulerformula}
\eulerimg{0}{images/Davina Safa Felisa 1-6-042-large.png}
\begin{eulerprompt}
>P(3,4)
\end{eulerprompt}
\begin{euleroutput}
  625
\end{euleroutput}
\begin{eulerprompt}
>$&P(x,4)+ Q(x,3), $&expand(%)
\end{eulerprompt}
\begin{eulerformula}
\[
16\,x^4-31\,x^3+30\,x^2+4\,x+9
\]
\end{eulerformula}
\eulerimg{0}{images/Davina Safa Felisa 1-6-044-large.png}
\begin{eulerprompt}
>$&P(x,4)-Q(x,3), $&expand(%), $&factor(%)
\end{eulerprompt}
\begin{eulerformula}
\[
16\,x^4-33\,x^3+18\,x^2-20\,x-7
\]
\end{eulerformula}
\eulerimg{0}{images/Davina Safa Felisa 1-6-046-large.png}
\eulerimg{0}{images/Davina Safa Felisa 1-6-047-large.png}
\begin{eulerprompt}
>$&P(x,4)*Q(x,3), $&expand(%), $&factor(%)
\end{eulerprompt}
\begin{eulerformula}
\[
\left(x+2\right)^3\,\left(2\,x-1\right)^4
\]
\end{eulerformula}
\eulerimg{0}{images/Davina Safa Felisa 1-6-049-large.png}
\eulerimg{0}{images/Davina Safa Felisa 1-6-050-large.png}
\begin{eulerprompt}
>$&P(x,4)/Q(x,1), $&expand(%), $&factor(%)
\end{eulerprompt}
\begin{eulerformula}
\[
\frac{\left(2\,x-1\right)^4}{x+2}
\]
\end{eulerformula}
\eulerimg{1}{images/Davina Safa Felisa 1-6-052-large.png}
\eulerimg{1}{images/Davina Safa Felisa 1-6-053-large.png}
\begin{eulerprompt}
>function f(x) &= x^3-x; $&f(x)
\end{eulerprompt}
\begin{eulerformula}
\[
x^3-x
\]
\end{eulerformula}
\begin{eulercomment}
Dengan \&= fungsinya simbolis, dan dapat digunakan dalam ekspresi
simbolik lainnya.
\end{eulercomment}
\begin{eulerprompt}
>$&integrate(f(x),x)
\end{eulerprompt}
\begin{eulerformula}
\[
\frac{x^4}{4}-\frac{x^2}{2}
\]
\end{eulerformula}
\begin{eulercomment}
Dengan := fungsinya numerik. Contoh yang baik adalah integral tak
tentu seperti

\end{eulercomment}
\begin{eulerformula}
\[
f(x) = \int_1^x t^t \, dt,
\]
\end{eulerformula}
\begin{eulercomment}
yang tidak dapat dinilai secara simbolis.

Jika kita mendefinisikan kembali fungsi dengan kata kunci "peta" dapat
digunakan untuk vektor x. Secara internal, fungsi dipanggil untuk
semua nilai x satu kali, dan hasilnya disimpan dalam vektor.
\end{eulercomment}
\begin{eulerprompt}
>function map f(x) := integrate("x^x",1,x)
>f(0:0.5:2)
\end{eulerprompt}
\begin{euleroutput}
  [-0.783431,  -0.410816,  0,  0.676863,  2.05045]
\end{euleroutput}
\begin{eulercomment}
Fungsi dapat memiliki nilai default untuk parameter.
\end{eulercomment}
\begin{eulerprompt}
>function mylog (x,base=10) := ln(x)/ln(base);
\end{eulerprompt}
\begin{eulercomment}
Sekarang fungsi dapat dipanggil dengan atau tanpa parameter "basis".
\end{eulercomment}
\begin{eulerprompt}
>mylog(100), mylog(2^6.7,2)
\end{eulerprompt}
\begin{euleroutput}
  2
  6.7
\end{euleroutput}
\begin{eulercomment}
Selain itu, dimungkinkan untuk menggunakan parameter yang ditetapkan.
\end{eulercomment}
\begin{eulerprompt}
>mylog(E^2,base=E)
\end{eulerprompt}
\begin{euleroutput}
  2
\end{euleroutput}
\begin{eulercomment}
Seringkali, kita ingin menggunakan fungsi untuk vektor di satu tempat,
dan untuk elemen individual di tempat lain. Ini dimungkinkan dengan
parameter vektor.
\end{eulercomment}
\begin{eulerprompt}
>function f([a,b]) &= a^2+b^2-a*b+b; $&f(a,b), $&f(x,y)
\end{eulerprompt}
\begin{eulerformula}
\[
y^2-x\,y+y+x^2
\]
\end{eulerformula}
\eulerimg{0}{images/Davina Safa Felisa 1-6-058-large.png}
\begin{eulercomment}
Fungsi simbolik seperti itu dapat digunakan untuk variabel simbolik.

Tetapi fungsi tersebut juga dapat digunakan untuk vektor numerik.
\end{eulercomment}
\begin{eulerprompt}
>v=[3,4]; f(v)
\end{eulerprompt}
\begin{euleroutput}
  17
\end{euleroutput}
\begin{eulercomment}
Ada juga fungsi simbolis murni, yang tidak dapat digunakan secara
numerik.
\end{eulercomment}
\begin{eulerprompt}
>function lapl(expr,x,y) &&= diff(expr,x,2)+diff(expr,y,2)//turunan parsial kedua
\end{eulerprompt}
\begin{euleroutput}
  
                   diff(expr, y, 2) + diff(expr, x, 2)
  
\end{euleroutput}
\begin{eulerprompt}
>$&realpart((x+I*y)^4), $&lapl(%,x,y)
\end{eulerprompt}
\begin{eulerformula}
\[
0
\]
\end{eulerformula}
\eulerimg{0}{images/Davina Safa Felisa 1-6-060-large.png}
\begin{eulercomment}
Tetapi tentu saja, mereka dapat digunakan dalam ekspresi simbolik atau
dalam definisi fungsi simbolik.
\end{eulercomment}
\begin{eulerprompt}
>function f(x,y) &= factor(lapl((x+y^2)^5,x,y)); $&f(x,y)
\end{eulerprompt}
\begin{eulerformula}
\[
10\,\left(y^2+x\right)^3\,\left(9\,y^2+x+2\right)
\]
\end{eulerformula}
\begin{eulercomment}
Untuk meringkas

- \&= mendefinisikan fungsi simbolis,\\
- := mendefinisikan fungsi numerik,\\
- \&\&= mendefinisikan fungsi simbolis murni.

\begin{eulercomment}
\eulerheading{Memecahkan Ekspresi}
\begin{eulercomment}
Ekspresi dapat diselesaikan secara numerik dan simbolis.

Untuk menyelesaikan ekspresi sederhana dari satu variabel, kita dapat
menggunakan fungsi solve(). Perlu nilai awal untuk memulai pencarian.
Secara internal, solve() menggunakan metode secant.
\end{eulercomment}
\begin{eulerprompt}
>solve("x^2-2",1)
\end{eulerprompt}
\begin{euleroutput}
  1.41421356237
\end{euleroutput}
\begin{eulercomment}
Ini juga berfungsi untuk ekspresi simbolis. Ambil fungsi berikut.
\end{eulercomment}
\begin{eulerprompt}
>$&solve(x^2=2,x)
\end{eulerprompt}
\begin{eulerformula}
\[
\left[ x=-\sqrt{2} , x=\sqrt{2} \right] 
\]
\end{eulerformula}
\begin{eulerprompt}
>$&solve(x^2-2,x)
\end{eulerprompt}
\begin{eulerformula}
\[
\left[ x=-\sqrt{2} , x=\sqrt{2} \right] 
\]
\end{eulerformula}
\begin{eulerprompt}
>$&solve(a*x^2+b*x+c=0,x)
\end{eulerprompt}
\begin{eulerformula}
\[
\left[ x=\frac{-\sqrt{b^2-4\,a\,c}-b}{2\,a} , x=\frac{\sqrt{b^2-4\,  a\,c}-b}{2\,a} \right] 
\]
\end{eulerformula}
\begin{eulerprompt}
>$&solve([a*x+b*y=c,d*x+e*y=f],[x,y])
\end{eulerprompt}
\begin{eulerformula}
\[
\left[ \left[ x=-\frac{c\,e}{b\,\left(d-5\right)-a\,e} , y=\frac{c  \,\left(d-5\right)}{b\,\left(d-5\right)-a\,e} \right]  \right] 
\]
\end{eulerformula}
\begin{eulerprompt}
>px &= 4*x^8+x^7-x^4-x; $&px
\end{eulerprompt}
\begin{eulerformula}
\[
4\,x^8+x^7-x^4-x
\]
\end{eulerformula}
\begin{eulercomment}
Sekarang kita mencari titik, di mana polinomialnya adalah 2. Dalam
solve(), nilai target default y=0 dapat diubah dengan variabel yang
ditetapkan.\\
Kami menggunakan y=2 dan memeriksa dengan mengevaluasi polinomial pada
hasil sebelumnya.
\end{eulercomment}
\begin{eulerprompt}
>solve(px,1,y=2), px(%)
\end{eulerprompt}
\begin{euleroutput}
  0.966715594851
  2
\end{euleroutput}
\begin{eulercomment}
Memecahkan ekspresi simbolis dalam bentuk simbolis mengembalikan
daftar solusi. Kami menggunakan pemecah simbolik solve() yang
disediakan oleh Maxima.
\end{eulercomment}
\begin{eulerprompt}
>sol &= solve(x^2-x-1,x); $&sol
\end{eulerprompt}
\begin{eulerformula}
\[
\left[ x=\frac{1-\sqrt{5}}{2} , x=\frac{\sqrt{5}+1}{2} \right] 
\]
\end{eulerformula}
\begin{eulercomment}
Cara termudah untuk mendapatkan nilai numerik adalah dengan
mengevaluasi solusi secara numerik seperti ekspresi.
\end{eulercomment}
\begin{eulerprompt}
>longest sol()
\end{eulerprompt}
\begin{euleroutput}
      -0.6180339887498949       1.618033988749895 
\end{euleroutput}
\begin{eulercomment}
Untuk menggunakan solusi secara simbolis dalam ekspresi lain, cara
termudah adalah "dengan".
\end{eulercomment}
\begin{eulerprompt}
>$&x^2 with sol[1], $&expand(x^2-x-1 with sol[2])
\end{eulerprompt}
\begin{eulerformula}
\[
0
\]
\end{eulerformula}
\eulerimg{0}{images/Davina Safa Felisa 1-6-069-large.png}
\begin{eulercomment}
Memecahkan sistem persamaan secara simbolis dapat dilakukan dengan
vektor persamaan dan solver simbolis solve(). Jawabannya adalah daftar
daftar persamaan.
\end{eulercomment}
\begin{eulerprompt}
>$&solve([x+y=2,x^3+2*y+x=4],[x,y])
\end{eulerprompt}
\begin{eulerformula}
\[
\left[ \left[ x=-1 , y=3 \right]  , \left[ x=1 , y=1 \right]  ,   \left[ x=0 , y=2 \right]  \right] 
\]
\end{eulerformula}
\begin{eulercomment}
Fungsi f() dapat melihat variabel global. Namun seringkali kita ingin
menggunakan parameter lokal.

\end{eulercomment}
\begin{eulerformula}
\[
a^x-x^a = 0.1
\]
\end{eulerformula}
\begin{eulercomment}
dengan a=3.
\end{eulercomment}
\begin{eulerprompt}
>function f(x,a) := x^a-a^x;
\end{eulerprompt}
\begin{eulercomment}
Salah satu cara untuk meneruskan parameter tambahan ke f() adalah
dengan menggunakan daftar dengan nama fungsi dan parameter (sebaliknya
adalah parameter titik koma).
\end{eulercomment}
\begin{eulerprompt}
>solve(\{\{"f",3\}\},2,y=0.1)
\end{eulerprompt}
\begin{euleroutput}
  2.54116291558
\end{euleroutput}
\begin{eulercomment}
Ini juga bekerja dengan ekspresi. Tapi kemudian, elemen daftar bernama
harus digunakan. (Lebih lanjut tentang daftar di tutorial tentang
sintaks EMT).
\end{eulercomment}
\begin{eulerprompt}
>solve(\{\{"x^a-a^x",a=3\}\},2,y=0.1)
\end{eulerprompt}
\begin{euleroutput}
  2.54116291558
\end{euleroutput}
\eulerheading{Menyelesaikan Pertidaksamaan}
\begin{eulercomment}
Untuk menyelesaikan pertidaksamaan, EMT tidak akan dapat melakukannya,
melainkan dengan bantuan Maxima, artinya secara eksak (simbolik).
Perintah Maxima yang digunakan adalah fourier\_elim(), yang harus
dipanggil dengan perintah "load(fourier\_elim)" terlebih dahulu.
\end{eulercomment}
\begin{eulerprompt}
>&load(fourier_elim)
\end{eulerprompt}
\begin{euleroutput}
  
          C:/Program Files/Euler x64/maxima/share/maxima/5.35.1/share/f\(\backslash\)
  ourier_elim/fourier_elim.lisp
  
\end{euleroutput}
\begin{eulerprompt}
>$&fourier_elim([x^2 - 1>0],[x]) // x^2-1 > 0
\end{eulerprompt}
\begin{eulerformula}
\[
\left[ 1<x \right] \lor \left[ x<-1 \right] 
\]
\end{eulerformula}
\begin{eulerprompt}
>$&fourier_elim([x^2 - 1<0],[x]) // x^2-1 < 0
\end{eulerprompt}
\begin{eulerformula}
\[
\left[ -1<x , x<1 \right] 
\]
\end{eulerformula}
\begin{eulerprompt}
>$&fourier_elim([x^2 - 1 # 0],[x]) // x^-1 <> 0
\end{eulerprompt}
\begin{eulerformula}
\[
\left[ -1<x , x<1 \right] \lor \left[ 1<x \right] \lor \left[ x<-1   \right] 
\]
\end{eulerformula}
\begin{eulerprompt}
>$&fourier_elim([x # 6],[x])
\end{eulerprompt}
\begin{eulerformula}
\[
\left[ x<6 \right] \lor \left[ 6<x \right] 
\]
\end{eulerformula}
\begin{eulerprompt}
>$&fourier_elim([x < 1, x > 1],[x]) // tidak memiliki penyelesaian
\end{eulerprompt}
\begin{eulerformula}
\[
{\it emptyset}
\]
\end{eulerformula}
\begin{eulerprompt}
>$&fourier_elim([minf < x, x < inf],[x]) // solusinya R
\end{eulerprompt}
\begin{eulerformula}
\[
{\it universalset}
\]
\end{eulerformula}
\begin{eulerprompt}
>$&fourier_elim([x^3 - 1 > 0],[x])
\end{eulerprompt}
\begin{eulerformula}
\[
\left[ 1<x , x^2+x+1>0 \right] \lor \left[ x<1 , -x^2-x-1>0   \right] 
\]
\end{eulerformula}
\begin{eulerprompt}
>$&fourier_elim([cos(x) < 1/2],[x]) // ??? gagal
\end{eulerprompt}
\begin{eulerformula}
\[
\left[ 1-2\,\cos x>0 \right] 
\]
\end{eulerformula}
\begin{eulerprompt}
>$&fourier_elim([y-x < 5, x - y < 7, 10 < y],[x,y]) // sistem pertidaksamaan
\end{eulerprompt}
\begin{eulerformula}
\[
\left[ y-5<x , x<y+7 , 10<y \right] 
\]
\end{eulerformula}
\begin{eulerprompt}
>$&fourier_elim([y-x < 5, x - y < 7, 10 < y],[y,x])
\end{eulerprompt}
\begin{eulerformula}
\[
\left[ {\it max}\left(10 , x-7\right)<y , y<x+5 , 5<x \right] 
\]
\end{eulerformula}
\begin{eulerprompt}
>$&fourier_elim((x + y < 5) and (x - y >8),[x,y])
\end{eulerprompt}
\begin{eulerformula}
\[
\left[ y+8<x , x<5-y , y<-\frac{3}{2} \right] 
\]
\end{eulerformula}
\begin{eulerprompt}
>$&fourier_elim(((x + y < 5) and x < 1) or  (x - y >8),[x,y])
\end{eulerprompt}
\begin{eulerformula}
\[
\left[ y+8<x \right] \lor \left[ x<{\it min}\left(1 , 5-y\right)   \right] 
\]
\end{eulerformula}
\begin{eulerprompt}
>&fourier_elim([max(x,y) > 6, x # 8, abs(y-1) > 12],[x,y])
\end{eulerprompt}
\begin{euleroutput}
  
          [6 < x, x < 8, y < - 11] or [8 < x, y < - 11]
   or [x < 8, 13 < y] or [x = y, 13 < y] or [8 < x, x < y, 13 < y]
   or [y < x, 13 < y]
  
\end{euleroutput}
\begin{eulerprompt}
>$&fourier_elim([(x+6)/(x-9) <= 6],[x])
\end{eulerprompt}
\begin{eulerformula}
\[
\left[ x=12 \right] \lor \left[ 12<x \right] \lor \left[ x<9   \right] 
\]
\end{eulerformula}
\eulerheading{Bahasa Matriks}
\begin{eulercomment}
Dokumentasi inti EMT berisi diskusi terperinci tentang bahasa matriks
Euler.

Vektor dan matriks dimasukkan dengan tanda kurung siku, elemen
dipisahkan dengan koma, baris dipisahkan dengan titik koma.
\end{eulercomment}
\begin{eulerprompt}
>A=[1,2;3,4]
\end{eulerprompt}
\begin{euleroutput}
              1             2 
              3             4 
\end{euleroutput}
\begin{eulercomment}
Produk matriks dilambangkan dengan titik.
\end{eulercomment}
\begin{eulerprompt}
>b=[3;4]
\end{eulerprompt}
\begin{euleroutput}
              3 
              4 
\end{euleroutput}
\begin{eulerprompt}
>b' // transpose b
\end{eulerprompt}
\begin{euleroutput}
  [3,  4]
\end{euleroutput}
\begin{eulerprompt}
>inv(A) //inverse A
\end{eulerprompt}
\begin{euleroutput}
             -2             1 
            1.5          -0.5 
\end{euleroutput}
\begin{eulerprompt}
>A.b //perkalian matriks
\end{eulerprompt}
\begin{euleroutput}
             11 
             25 
\end{euleroutput}
\begin{eulerprompt}
>A.inv(A)
\end{eulerprompt}
\begin{euleroutput}
              1             0 
              0             1 
\end{euleroutput}
\begin{eulercomment}
Poin utama dari bahasa matriks adalah bahwa semua fungsi dan operator
bekerja elemen untuk elemen.
\end{eulercomment}
\begin{eulerprompt}
>A.A
\end{eulerprompt}
\begin{euleroutput}
              7            10 
             15            22 
\end{euleroutput}
\begin{eulerprompt}
>A^2 //perpangkatan elemen2 A
\end{eulerprompt}
\begin{euleroutput}
              1             4 
              9            16 
\end{euleroutput}
\begin{eulerprompt}
>A.A.A
\end{eulerprompt}
\begin{euleroutput}
             37            54 
             81           118 
\end{euleroutput}
\begin{eulerprompt}
>power(A,3) //perpangkatan matriks
\end{eulerprompt}
\begin{euleroutput}
             37            54 
             81           118 
\end{euleroutput}
\begin{eulerprompt}
>A/A //pembagian elemen-elemen matriks yang seletak
\end{eulerprompt}
\begin{euleroutput}
              1             1 
              1             1 
\end{euleroutput}
\begin{eulerprompt}
>A/b //pembagian elemen2 A oleh elemen2 b kolom demi kolom (karena b vektor kolom)
\end{eulerprompt}
\begin{euleroutput}
       0.333333      0.666667 
           0.75             1 
\end{euleroutput}
\begin{eulerprompt}
>A\(\backslash\)b // hasilkali invers A dan b, A^(-1)b 
\end{eulerprompt}
\begin{euleroutput}
             -2 
            2.5 
\end{euleroutput}
\begin{eulerprompt}
>inv(A).b
\end{eulerprompt}
\begin{euleroutput}
             -2 
            2.5 
\end{euleroutput}
\begin{eulerprompt}
>A\(\backslash\)A   //A^(-1)A
\end{eulerprompt}
\begin{euleroutput}
              1             0 
              0             1 
\end{euleroutput}
\begin{eulerprompt}
>inv(A).A
\end{eulerprompt}
\begin{euleroutput}
              1             0 
              0             1 
\end{euleroutput}
\begin{eulerprompt}
>A*A //perkalin elemen-elemen matriks seletak
\end{eulerprompt}
\begin{euleroutput}
              1             4 
              9            16 
\end{euleroutput}
\begin{eulercomment}
Ini bukan produk matriks, tetapi perkalian elemen demi elemen. Hal
yang sama berlaku untuk vektor.
\end{eulercomment}
\begin{eulerprompt}
>b^2 // perpangkatan elemen-elemen matriks/vektor
\end{eulerprompt}
\begin{euleroutput}
              9 
             16 
\end{euleroutput}
\begin{eulercomment}
Jika salah satu operan adalah vektor atau skalar, itu diperluas secara
alami.
\end{eulercomment}
\begin{eulerprompt}
>2*A
\end{eulerprompt}
\begin{euleroutput}
              2             4 
              6             8 
\end{euleroutput}
\begin{eulercomment}
Misalnya, jika operan adalah vektor kolom, elemennya diterapkan ke
semua baris A.
\end{eulercomment}
\begin{eulerprompt}
>[1,2]*A
\end{eulerprompt}
\begin{euleroutput}
              1             4 
              3             8 
\end{euleroutput}
\begin{eulercomment}
Jika itu adalah vektor baris, itu diterapkan ke semua kolom A.
\end{eulercomment}
\begin{eulerprompt}
>A*[2,3]
\end{eulerprompt}
\begin{euleroutput}
              2             6 
              6            12 
\end{euleroutput}
\begin{eulercomment}
Seseorang dapat membayangkan perkalian ini seolah-olah vektor baris v
telah digandakan untuk membentuk matriks dengan ukuran yang sama
dengan A.
\end{eulercomment}
\begin{eulerprompt}
>dup([1,2],2) // dup: menduplikasi/menggandakan vektor [1,2] sebanyak 2 kali (baris)
\end{eulerprompt}
\begin{euleroutput}
              1             2 
              1             2 
\end{euleroutput}
\begin{eulerprompt}
>A*dup([1,2],2) 
\end{eulerprompt}
\begin{euleroutput}
              1             4 
              3             8 
\end{euleroutput}
\begin{eulercomment}
Ini juga berlaku untuk dua vektor di mana satu adalah vektor baris dan
yang lainnya adalah vektor kolom. Kami menghitung i*j untuk i,j dari 1
hingga 5. Caranya adalah dengan mengalikan 1:5 dengan transposnya.
Bahasa matriks Euler secara otomatis menghasilkan tabel nilai.
\end{eulercomment}
\begin{eulerprompt}
>(1:5)*(1:5)' // hasilkali elemen-elemen vektor baris dan vektor kolom
\end{eulerprompt}
\begin{euleroutput}
              1             2             3             4             5 
              2             4             6             8            10 
              3             6             9            12            15 
              4             8            12            16            20 
              5            10            15            20            25 
\end{euleroutput}
\begin{eulercomment}
Sekali lagi, ingat bahwa ini bukan produk matriks!
\end{eulercomment}
\begin{eulerprompt}
>(1:5).(1:5)' // hasilkali vektor baris dan vektor kolom
\end{eulerprompt}
\begin{euleroutput}
  55
\end{euleroutput}
\begin{eulerprompt}
>sum((1:5)*(1:5)) // sama hasilnya
\end{eulerprompt}
\begin{euleroutput}
  55
\end{euleroutput}
\begin{eulercomment}
Bahkan operator seperti \textless{} atau == bekerja dengan cara yang sama.
\end{eulercomment}
\begin{eulerprompt}
>(1:10)<6 // menguji elemen-elemen yang kurang dari 6
\end{eulerprompt}
\begin{euleroutput}
  [1,  1,  1,  1,  1,  0,  0,  0,  0,  0]
\end{euleroutput}
\begin{eulercomment}
Misalnya, kita dapat menghitung jumlah elemen yang memenuhi kondisi
tertentu dengan fungsi sum().
\end{eulercomment}
\begin{eulerprompt}
>sum((1:10)<6) // banyak elemen yang kurang dari 6
\end{eulerprompt}
\begin{euleroutput}
  5
\end{euleroutput}
\begin{eulercomment}
Euler memiliki operator perbandingan, seperti "==", yang memeriksa
kesetaraan.

Kami mendapatkan vektor 0 dan 1, di mana 1 berarti benar.
\end{eulercomment}
\begin{eulerprompt}
>t=(1:10)^2; t==25 //menguji elemen2 t yang sama dengan 25 (hanya ada 1)
\end{eulerprompt}
\begin{euleroutput}
  [0,  0,  0,  0,  1,  0,  0,  0,  0,  0]
\end{euleroutput}
\begin{eulercomment}
Dari vektor seperti itu, "bukan nol" memilih elemen bukan nol.

Dalam hal ini, kami mendapatkan indeks semua elemen lebih besar dari
50.
\end{eulercomment}
\begin{eulerprompt}
>nonzeros(t>50) //indeks elemen2 t yang lebih besar daripada 50
\end{eulerprompt}
\begin{euleroutput}
  [8,  9,  10]
\end{euleroutput}
\begin{eulercomment}
Tentu saja, kita dapat menggunakan vektor indeks ini untuk mendapatkan
nilai yang sesuai dalam t.
\end{eulercomment}
\begin{eulerprompt}
>t[nonzeros(t>50)] //elemen2 t yang lebih besar daripada 50
\end{eulerprompt}
\begin{euleroutput}
  [64,  81,  100]
\end{euleroutput}
\begin{eulercomment}
Sebagai contoh, mari kita cari semua kuadrat dari angka 1 hingga 1000,
yaitu 5 modulo 11 dan 3 modulo 13.
\end{eulercomment}
\begin{eulerprompt}
>t=1:1000; nonzeros(mod(t^2,11)==5 && mod(t^2,13)==3)
\end{eulerprompt}
\begin{euleroutput}
  [4,  48,  95,  139,  147,  191,  238,  282,  290,  334,  381,  425,
  433,  477,  524,  568,  576,  620,  667,  711,  719,  763,  810,  854,
  862,  906,  953,  997]
\end{euleroutput}
\begin{eulercomment}
EMT tidak sepenuhnya efektif untuk perhitungan bilangan bulat. Ini
menggunakan titik mengambang presisi ganda secara internal. Namun,
seringkali sangat berguna.

Kita dapat memeriksa keutamaan. Mari kita cari tahu, berapa banyak
kuadrat ditambah 1 adalah bilangan prima.
\end{eulercomment}
\begin{eulerprompt}
>t=1:1000; length(nonzeros(isprime(t^2+1)))
\end{eulerprompt}
\begin{euleroutput}
  112
\end{euleroutput}
\begin{eulercomment}
Fungsi bukan nol() hanya berfungsi untuk vektor. Untuk matriks, ada
mnonzeros().
\end{eulercomment}
\begin{eulerprompt}
>seed(2); A=random(3,4)
\end{eulerprompt}
\begin{euleroutput}
       0.765761      0.401188      0.406347      0.267829 
        0.13673      0.390567      0.495975      0.952814 
       0.548138      0.006085      0.444255      0.539246 
\end{euleroutput}
\begin{eulercomment}
Ini mengembalikan indeks elemen, yang bukan nol.
\end{eulercomment}
\begin{eulerprompt}
>k=mnonzeros(A<0.4) //indeks elemen2 A yang kurang dari 0,4
\end{eulerprompt}
\begin{euleroutput}
              1             4 
              2             1 
              2             2 
              3             2 
\end{euleroutput}
\begin{eulercomment}
Indeks ini dapat digunakan untuk mengatur elemen ke beberapa nilai.
\end{eulercomment}
\begin{eulerprompt}
>mset(A,k,0) //mengganti elemen2 suatu matriks pada indeks tertentu
\end{eulerprompt}
\begin{euleroutput}
       0.765761      0.401188      0.406347             0 
              0             0      0.495975      0.952814 
       0.548138             0      0.444255      0.539246 
\end{euleroutput}
\begin{eulercomment}
Fungsi mset() juga dapat mengatur elemen pada indeks ke entri dari
beberapa matriks lainnya.
\end{eulercomment}
\begin{eulerprompt}
>mset(A,k,-random(size(A)))
\end{eulerprompt}
\begin{euleroutput}
       0.765761      0.401188      0.406347     -0.126917 
      -0.122404     -0.691673      0.495975      0.952814 
       0.548138     -0.483902      0.444255      0.539246 
\end{euleroutput}
\begin{eulercomment}
Dan dimungkinkan untuk mendapatkan elemen dalam vektor.
\end{eulercomment}
\begin{eulerprompt}
>mget(A,k)
\end{eulerprompt}
\begin{euleroutput}
  [0.267829,  0.13673,  0.390567,  0.006085]
\end{euleroutput}
\begin{eulercomment}
Fungsi lain yang berguna adalah ekstrem, yang mengembalikan nilai
minimal dan maksimal di setiap baris matriks dan posisinya.
\end{eulercomment}
\begin{eulerprompt}
>ex=extrema(A)
\end{eulerprompt}
\begin{euleroutput}
       0.267829             4      0.765761             1 
        0.13673             1      0.952814             4 
       0.006085             2      0.548138             1 
\end{euleroutput}
\begin{eulercomment}
Kita dapat menggunakan ini untuk mengekstrak nilai maksimal di setiap
baris.
\end{eulercomment}
\begin{eulerprompt}
>ex[,3]'
\end{eulerprompt}
\begin{euleroutput}
  [0.765761,  0.952814,  0.548138]
\end{euleroutput}
\begin{eulercomment}
Ini, tentu saja, sama dengan fungsi max().
\end{eulercomment}
\begin{eulerprompt}
>max(A)'
\end{eulerprompt}
\begin{euleroutput}
  [0.765761,  0.952814,  0.548138]
\end{euleroutput}
\begin{eulercomment}
Tetapi dengan mget(), kita dapat mengekstrak indeks dan menggunakan
informasi ini untuk mengekstrak elemen pada posisi yang sama dari
matriks lain.
\end{eulercomment}
\begin{eulerprompt}
>j=(1:rows(A))'|ex[,4], mget(-A,j)
\end{eulerprompt}
\begin{euleroutput}
              1             1 
              2             4 
              3             1 
  [-0.765761,  -0.952814,  -0.548138]
\end{euleroutput}
\eulerheading{Fungsi Matriks Lainnya (Membangun Matriks)}
\begin{eulercomment}
Untuk membangun matriks, kita dapat menumpuk satu matriks di atas yang
lain. Jika keduanya tidak memiliki jumlah kolom yang sama, kolom yang
lebih pendek akan diisi dengan 0.
\end{eulercomment}
\begin{eulerprompt}
>v=1:3; v_v
\end{eulerprompt}
\begin{euleroutput}
              1             2             3 
              1             2             3 
\end{euleroutput}
\begin{eulercomment}
Demikian juga, kita dapat melampirkan matriks ke yang lain secara
berdampingan, jika keduanya memiliki jumlah baris yang sama.
\end{eulercomment}
\begin{eulerprompt}
>A=random(3,4); A|v'
\end{eulerprompt}
\begin{euleroutput}
       0.032444     0.0534171      0.595713      0.564454             1 
        0.83916      0.175552      0.396988       0.83514             2 
      0.0257573      0.658585      0.629832      0.770895             3 
\end{euleroutput}
\begin{eulercomment}
Jika mereka tidak memiliki jumlah baris yang sama, matriks yang lebih
pendek diisi dengan 0.

Ada pengecualian untuk aturan ini. Bilangan real yang dilampirkan pada
matriks akan digunakan sebagai kolom yang diisi dengan bilangan real
tersebut.
\end{eulercomment}
\begin{eulerprompt}
>A|1
\end{eulerprompt}
\begin{euleroutput}
       0.032444     0.0534171      0.595713      0.564454             1 
        0.83916      0.175552      0.396988       0.83514             1 
      0.0257573      0.658585      0.629832      0.770895             1 
\end{euleroutput}
\begin{eulercomment}
Dimungkinkan untuk membuat matriks vektor baris dan kolom.
\end{eulercomment}
\begin{eulerprompt}
>[v;v]
\end{eulerprompt}
\begin{euleroutput}
              1             2             3 
              1             2             3 
\end{euleroutput}
\begin{eulerprompt}
>[v',v']
\end{eulerprompt}
\begin{euleroutput}
              1             1 
              2             2 
              3             3 
\end{euleroutput}
\begin{eulercomment}
Tujuan utama dari ini adalah untuk menafsirkan vektor ekspresi untuk
vektor kolom.
\end{eulercomment}
\begin{eulerprompt}
>"[x,x^2]"(v')
\end{eulerprompt}
\begin{euleroutput}
              1             1 
              2             4 
              3             9 
\end{euleroutput}
\begin{eulercomment}
Untuk mendapatkan ukuran A, kita dapat menggunakan fungsi berikut.
\end{eulercomment}
\begin{eulerprompt}
>C=zeros(2,4); rows(C), cols(C), size(C), length(C)
\end{eulerprompt}
\begin{euleroutput}
  2
  4
  [2,  4]
  4
\end{euleroutput}
\begin{eulercomment}
Untuk vektor, ada panjang().
\end{eulercomment}
\begin{eulerprompt}
>length(2:10)
\end{eulerprompt}
\begin{euleroutput}
  9
\end{euleroutput}
\begin{eulercomment}
Ada banyak fungsi lain, yang menghasilkan matriks.
\end{eulercomment}
\begin{eulerprompt}
>ones(2,2)
\end{eulerprompt}
\begin{euleroutput}
              1             1 
              1             1 
\end{euleroutput}
\begin{eulercomment}
Ini juga dapat digunakan dengan satu parameter. Untuk mendapatkan
vektor dengan angka selain 1, gunakan yang berikut ini.
\end{eulercomment}
\begin{eulerprompt}
>ones(5)*6
\end{eulerprompt}
\begin{euleroutput}
  [6,  6,  6,  6,  6]
\end{euleroutput}
\begin{eulercomment}
Juga matriks bilangan acak dapat dihasilkan dengan acak (distribusi
seragam) atau normal (distribusi Gau).
\end{eulercomment}
\begin{eulerprompt}
>random(2,2)
\end{eulerprompt}
\begin{euleroutput}
        0.66566      0.831835 
          0.977      0.544258 
\end{euleroutput}
\begin{eulercomment}
Berikut adalah fungsi lain yang berguna, yang merestrukturisasi elemen
matriks menjadi matriks lain.
\end{eulercomment}
\begin{eulerprompt}
>redim(1:9,3,3) // menyusun elemen2 1, 2, 3, ..., 9 ke bentuk matriks 3x3
\end{eulerprompt}
\begin{euleroutput}
              1             2             3 
              4             5             6 
              7             8             9 
\end{euleroutput}
\begin{eulercomment}
Dengan fungsi berikut, kita dapat menggunakan ini dan fungsi dup untuk
menulis fungsi rep(), yang mengulang vektor n kali.
\end{eulercomment}
\begin{eulerprompt}
>function rep(v,n) := redim(dup(v,n),1,n*cols(v))
\end{eulerprompt}
\begin{eulercomment}
Let us test.
\end{eulercomment}
\begin{eulerprompt}
>rep(1:3,5)
\end{eulerprompt}
\begin{euleroutput}
  [1,  2,  3,  1,  2,  3,  1,  2,  3,  1,  2,  3,  1,  2,  3]
\end{euleroutput}
\begin{eulercomment}
Fungsi multdup() menduplikasi elemen vektor.
\end{eulercomment}
\begin{eulerprompt}
>multdup(1:3,5), multdup(1:3,[2,3,2])
\end{eulerprompt}
\begin{euleroutput}
  [1,  1,  1,  1,  1,  2,  2,  2,  2,  2,  3,  3,  3,  3,  3]
  [1,  1,  2,  2,  2,  3,  3]
\end{euleroutput}
\begin{eulercomment}
Fungsi flipx() dan flipy() mengembalikan urutan baris atau kolom
matriks. Yaitu, fungsi flipx() membalik secara horizontal.
\end{eulercomment}
\begin{eulerprompt}
>flipx(1:5) //membalik elemen2 vektor baris
\end{eulerprompt}
\begin{euleroutput}
  [5,  4,  3,  2,  1]
\end{euleroutput}
\begin{eulercomment}
Untuk rotasi, Euler memiliki rotleft() dan rotright().
\end{eulercomment}
\begin{eulerprompt}
>rotleft(1:5) // memutar elemen2 vektor baris
\end{eulerprompt}
\begin{euleroutput}
  [2,  3,  4,  5,  1]
\end{euleroutput}
\begin{eulercomment}
Sebuah fungsi khusus adalah drop(v,i), yang menghilangkan elemen
dengan indeks di i dari vektor v.
\end{eulercomment}
\begin{eulerprompt}
>drop(10:20,3)
\end{eulerprompt}
\begin{euleroutput}
  [10,  11,  13,  14,  15,  16,  17,  18,  19,  20]
\end{euleroutput}
\begin{eulercomment}
Perhatikan bahwa vektor i di drop(v,i) mengacu pada indeks elemen di
v, bukan nilai elemen. Jika Anda ingin menghapus elemen, Anda harus
menemukan elemennya terlebih dahulu. Fungsi indexof(v,x) dapat
digunakan untuk mencari elemen x dalam vektor terurut v.
\end{eulercomment}
\begin{eulerprompt}
>v=primes(50), i=indexof(v,10:20), drop(v,i)
\end{eulerprompt}
\begin{euleroutput}
  [2,  3,  5,  7,  11,  13,  17,  19,  23,  29,  31,  37,  41,  43,  47]
  [0,  5,  0,  6,  0,  0,  0,  7,  0,  8,  0]
  [2,  3,  5,  7,  23,  29,  31,  37,  41,  43,  47]
\end{euleroutput}
\begin{eulercomment}
Seperti yang Anda lihat, tidak ada salahnya untuk memasukkan indeks di
luar rentang (seperti 0), indeks ganda, atau indeks yang tidak
diurutkan.
\end{eulercomment}
\begin{eulerprompt}
>drop(1:10,shuffle([0,0,5,5,7,12,12]))
\end{eulerprompt}
\begin{euleroutput}
  [1,  2,  3,  4,  6,  8,  9,  10]
\end{euleroutput}
\begin{eulercomment}
Ada beberapa fungsi khusus untuk mengatur diagonal atau untuk
menghasilkan matriks diagonal.

Kita mulai dengan matriks identitas.
\end{eulercomment}
\begin{eulerprompt}
>A=id(5) // matriks identitas 5x5
\end{eulerprompt}
\begin{euleroutput}
              1             0             0             0             0 
              0             1             0             0             0 
              0             0             1             0             0 
              0             0             0             1             0 
              0             0             0             0             1 
\end{euleroutput}
\begin{eulercomment}
Kemudian kita atur diagonal bawah (-1) menjadi 1:4.
\end{eulercomment}
\begin{eulerprompt}
>setdiag(A,-1,1:4) //mengganti diagonal di bawah diagonal utama
\end{eulerprompt}
\begin{euleroutput}
              1             0             0             0             0 
              1             1             0             0             0 
              0             2             1             0             0 
              0             0             3             1             0 
              0             0             0             4             1 
\end{euleroutput}
\begin{eulercomment}
Perhatikan bahwa kami tidak mengubah matriks A. Kami mendapatkan
matriks baru sebagai hasil dari setdiag().

Berikut adalah fungsi, yang mengembalikan matriks tri-diagonal.
\end{eulercomment}
\begin{eulerprompt}
>function tridiag (n,a,b,c) := setdiag(setdiag(b*id(n),1,c),-1,a); ...
>tridiag(5,1,2,3)
\end{eulerprompt}
\begin{euleroutput}
              2             3             0             0             0 
              1             2             3             0             0 
              0             1             2             3             0 
              0             0             1             2             3 
              0             0             0             1             2 
\end{euleroutput}
\begin{eulercomment}
Diagonal suatu matriks juga dapat diekstraksi dari matriks tersebut.
Untuk mendemonstrasikan ini, kami merestrukturisasi vektor 1:9 menjadi
matriks 3x3.
\end{eulercomment}
\begin{eulerprompt}
>A=redim(1:9,3,3)
\end{eulerprompt}
\begin{euleroutput}
              1             2             3 
              4             5             6 
              7             8             9 
\end{euleroutput}
\begin{eulercomment}
Sekarang kita dapat mengekstrak diagonal.
\end{eulercomment}
\begin{eulerprompt}
>d=getdiag(A,0)
\end{eulerprompt}
\begin{euleroutput}
  [1,  5,  9]
\end{euleroutput}
\begin{eulercomment}
Misalnya. Kita dapat membagi matriks dengan diagonalnya. Bahasa
matriks memperhatikan bahwa vektor kolom d diterapkan ke matriks baris
demi baris.
\end{eulercomment}
\begin{eulerprompt}
>fraction A/d'
\end{eulerprompt}
\begin{euleroutput}
          1         2         3 
        4/5         1       6/5 
        7/9       8/9         1 
\end{euleroutput}
\eulerheading{Vektorisasi}
\begin{eulercomment}
Hampir semua fungsi di Euler juga berfungsi untuk input matriks dan
vektor, kapan pun ini masuk akal.

Misalnya, fungsi sqrt() menghitung akar kuadrat dari semua elemen
vektor atau matriks.
\end{eulercomment}
\begin{eulerprompt}
>sqrt(1:3)
\end{eulerprompt}
\begin{euleroutput}
  [1,  1.41421,  1.73205]
\end{euleroutput}
\begin{eulercomment}
Jadi Anda dapat dengan mudah membuat tabel nilai. Ini adalah salah
satu cara untuk memplot suatu fungsi (alternatifnya menggunakan
ekspresi).
\end{eulercomment}
\begin{eulerprompt}
>x=1:0.01:5; y=log(x)/x^2; // terlalu panjang untuk ditampikan
\end{eulerprompt}
\begin{eulercomment}
Dengan ini dan operator titik dua a:delta:b, vektor nilai fungsi dapat
dihasilkan dengan mudah.

Pada contoh berikut, kita membangkitkan vektor nilai t[i] dengan spasi
0,1 dari -1 hingga 1. Kemudian kita membangkitkan vektor nilai fungsi

\end{eulercomment}
\begin{eulerformula}
\[
s = t^3-t
\]
\end{eulerformula}
\begin{eulerprompt}
>t=-1:0.1:1; s=t^3-t
\end{eulerprompt}
\begin{euleroutput}
  [0,  0.171,  0.288,  0.357,  0.384,  0.375,  0.336,  0.273,  0.192,
  0.099,  0,  -0.099,  -0.192,  -0.273,  -0.336,  -0.375,  -0.384,
  -0.357,  -0.288,  -0.171,  0]
\end{euleroutput}
\begin{eulercomment}
EMT memperluas operator untuk skalar, vektor, dan matriks dengan cara
yang jelas.

Misalnya, vektor kolom dikalikan vektor baris menjadi matriks, jika
operator diterapkan. Berikut ini, v' adalah vektor yang
ditransposisikan (vektor kolom).
\end{eulercomment}
\begin{eulerprompt}
>shortest (1:5)*(1:5)'
\end{eulerprompt}
\begin{euleroutput}
       1      2      3      4      5 
       2      4      6      8     10 
       3      6      9     12     15 
       4      8     12     16     20 
       5     10     15     20     25 
\end{euleroutput}
\begin{eulercomment}
Perhatikan, bahwa ini sangat berbeda dari produk matriks. Produk
matriks dilambangkan dengan titik "." di EMT.
\end{eulercomment}
\begin{eulerprompt}
>(1:5).(1:5)'
\end{eulerprompt}
\begin{euleroutput}
  55
\end{euleroutput}
\begin{eulercomment}
Secara default, vektor baris dicetak dalam format yang ringkas.
\end{eulercomment}
\begin{eulerprompt}
>[1,2,3,4]
\end{eulerprompt}
\begin{euleroutput}
  [1,  2,  3,  4]
\end{euleroutput}
\begin{eulercomment}
Untuk matriks operator khusus . menunjukkan perkalian matriks, dan A'
menunjukkan transpos. Matriks 1x1 dapat digunakan seperti bilangan
real.
\end{eulercomment}
\begin{eulerprompt}
>v:=[1,2]; v.v', %^2
\end{eulerprompt}
\begin{euleroutput}
  5
  25
\end{euleroutput}
\begin{eulercomment}
Untuk mentranspos matriks kita menggunakan apostrof.
\end{eulercomment}
\begin{eulerprompt}
>v=1:4; v'
\end{eulerprompt}
\begin{euleroutput}
              1 
              2 
              3 
              4 
\end{euleroutput}
\begin{eulercomment}
Jadi kita dapat menghitung matriks A kali vektor b.
\end{eulercomment}
\begin{eulerprompt}
>A=[1,2,3,4;5,6,7,8]; A.v'
\end{eulerprompt}
\begin{euleroutput}
             30 
             70 
\end{euleroutput}
\begin{eulercomment}
Perhatikan bahwa v masih merupakan vektor baris. Jadi v'.v berbeda
dari v.v'.
\end{eulercomment}
\begin{eulerprompt}
>v'.v
\end{eulerprompt}
\begin{euleroutput}
              1             2             3             4 
              2             4             6             8 
              3             6             9            12 
              4             8            12            16 
\end{euleroutput}
\begin{eulercomment}
v.v' menghitung norma v kuadrat untuk vektor baris v. Hasilnya adalah
vektor 1x1, yang bekerja seperti bilangan real.
\end{eulercomment}
\begin{eulerprompt}
>v.v'
\end{eulerprompt}
\begin{euleroutput}
  30
\end{euleroutput}
\begin{eulercomment}
Ada juga fungsi norma (bersama dengan banyak fungsi lain dari Aljabar
Linier).
\end{eulercomment}
\begin{eulerprompt}
>norm(v)^2
\end{eulerprompt}
\begin{euleroutput}
  30
\end{euleroutput}
\begin{eulercomment}
Operator dan fungsi mematuhi bahasa matriks Euler.

Berikut ringkasan aturannya.

- Fungsi yang diterapkan ke vektor atau matriks diterapkan ke setiap
elemen.

- Operator yang beroperasi pada dua matriks dengan ukuran yang sama
diterapkan berpasangan ke elemen matriks.

- Jika kedua matriks memiliki dimensi yang berbeda, keduanya diperluas
dengan cara yang masuk akal, sehingga memiliki ukuran yang sama.

Misalnya, nilai skalar kali vektor mengalikan nilai dengan setiap
elemen vektor. Atau matriks kali vektor (dengan *, bukan .) memperluas
vektor ke ukuran matriks dengan menduplikasinya.

Berikut ini adalah kasus sederhana dengan operator \textasciicircum{}.
\end{eulercomment}
\begin{eulerprompt}
>[1,2,3]^2
\end{eulerprompt}
\begin{euleroutput}
  [1,  4,  9]
\end{euleroutput}
\begin{eulercomment}
Berikut adalah kasus yang lebih rumit. Vektor baris dikalikan dengan
vektor kolom mengembang keduanya dengan menduplikasi.
\end{eulercomment}
\begin{eulerprompt}
>v:=[1,2,3]; v*v'
\end{eulerprompt}
\begin{euleroutput}
              1             2             3 
              2             4             6 
              3             6             9 
\end{euleroutput}
\begin{eulercomment}
Perhatikan bahwa produk skalar menggunakan produk matriks, bukan *!
\end{eulercomment}
\begin{eulerprompt}
>v.v'
\end{eulerprompt}
\begin{euleroutput}
  14
\end{euleroutput}
\begin{eulercomment}
Ada banyak fungsi matriks. Kami memberikan daftar singkat. Anda harus
berkonsultasi dengan dokumentasi untuk informasi lebih lanjut tentang
perintah ini.

\end{eulercomment}
\begin{eulerttcomment}
  sum,prod menghitung jumlah dan produk dari baris
  cumsum,cumprod melakukan hal yang sama secara kumulatif
  menghitung nilai ekstrem dari setiap baris
  extrema mengembalikan vektor dengan informasi ekstrim
  diag(A,i) mengembalikan diagonal ke-i
  setdiag(A,i,v) mengatur diagonal ke-i
  id(n) matriks identitas
  det(A) penentu
  charpoly(A) polinomial karakteristik
  nilai eigen(A) nilai eigen
\end{eulerttcomment}
\begin{eulerprompt}
>v*v, sum(v*v), cumsum(v*v)
\end{eulerprompt}
\begin{euleroutput}
  [1,  4,  9]
  14
  [1,  5,  14]
\end{euleroutput}
\begin{eulercomment}
Operator : menghasilkan vektor baris spasi yang sama, opsional dengan
ukuran langkah.
\end{eulercomment}
\begin{eulerprompt}
>1:4, 1:2:10
\end{eulerprompt}
\begin{euleroutput}
  [1,  2,  3,  4]
  [1,  3,  5,  7,  9]
\end{euleroutput}
\begin{eulercomment}
Untuk menggabungkan matriks dan vektor ada operator "\textbar{}" dan "\_".
\end{eulercomment}
\begin{eulerprompt}
>[1,2,3]|[4,5], [1,2,3]_1
\end{eulerprompt}
\begin{euleroutput}
  [1,  2,  3,  4,  5]
              1             2             3 
              1             1             1 
\end{euleroutput}
\begin{eulercomment}
Unsur-unsur matriks disebut dengan "A[i,j]".
\end{eulercomment}
\begin{eulerprompt}
>A:=[1,2,3;4,5,6;7,8,9]; A[2,3]
\end{eulerprompt}
\begin{euleroutput}
  6
\end{euleroutput}
\begin{eulercomment}
Untuk vektor baris atau kolom, v[i] adalah elemen ke-i dari vektor.
Untuk matriks, ini mengembalikan baris ke-i lengkap dari matriks.
\end{eulercomment}
\begin{eulerprompt}
>v:=[2,4,6,8]; v[3], A[3]
\end{eulerprompt}
\begin{euleroutput}
  6
  [7,  8,  9]
\end{euleroutput}
\begin{eulercomment}
Indeks juga bisa menjadi vektor baris dari indeks. : menunjukkan semua
indeks.
\end{eulercomment}
\begin{eulerprompt}
>v[1:2], A[:,2]
\end{eulerprompt}
\begin{euleroutput}
  [2,  4]
              2 
              5 
              8 
\end{euleroutput}
\begin{eulercomment}
Bentuk singkat untuk : adalah menghilangkan indeks sepenuhnya.
\end{eulercomment}
\begin{eulerprompt}
>A[,2:3]
\end{eulerprompt}
\begin{euleroutput}
              2             3 
              5             6 
              8             9 
\end{euleroutput}
\begin{eulercomment}
Untuk tujuan vektorisasi, elemen matriks dapat diakses seolah-olah
mereka adalah vektor.
\end{eulercomment}
\begin{eulerprompt}
>A\{4\}
\end{eulerprompt}
\begin{euleroutput}
  4
\end{euleroutput}
\begin{eulercomment}
Matriks juga dapat diratakan, menggunakan fungsi redim(). Ini
diimplementasikan dalam fungsi flatten().
\end{eulercomment}
\begin{eulerprompt}
>redim(A,1,prod(size(A))), flatten(A)
\end{eulerprompt}
\begin{euleroutput}
  [1,  2,  3,  4,  5,  6,  7,  8,  9]
  [1,  2,  3,  4,  5,  6,  7,  8,  9]
\end{euleroutput}
\begin{eulercomment}
Untuk menggunakan matriks untuk tabel, mari kita reset ke format
default, dan menghitung tabel nilai sinus dan kosinus. Perhatikan
bahwa sudut dalam radian secara default.
\end{eulercomment}
\begin{eulerprompt}
>defformat; w=0°:45°:360°; w=w'; deg(w)
\end{eulerprompt}
\begin{euleroutput}
              0 
             45 
             90 
            135 
            180 
            225 
            270 
            315 
            360 
\end{euleroutput}
\begin{eulercomment}
Sekarang kita menambahkan kolom ke matriks.
\end{eulercomment}
\begin{eulerprompt}
>M = deg(w)|w|cos(w)|sin(w)
\end{eulerprompt}
\begin{euleroutput}
              0             0             1             0 
             45      0.785398      0.707107      0.707107 
             90        1.5708             0             1 
            135       2.35619     -0.707107      0.707107 
            180       3.14159            -1             0 
            225       3.92699     -0.707107     -0.707107 
            270       4.71239             0            -1 
            315       5.49779      0.707107     -0.707107 
            360       6.28319             1             0 
\end{euleroutput}
\begin{eulercomment}
Dengan menggunakan bahasa matriks, kita dapat menghasilkan beberapa
tabel dari beberapa fungsi sekaligus.

Dalam contoh berikut, kita menghitung t[j]\textasciicircum{}i untuk i dari 1 hingga n.
Kami mendapatkan matriks, di mana setiap baris adalah tabel t\textasciicircum{}i untuk
satu i. Yaitu, matriks memiliki elemen\\
\end{eulercomment}
\begin{eulerformula}
\[
a_{i,j} = t_j^i, \quad 1 \le j \le 101, \quad 1 \le i \le n
\]
\end{eulerformula}
\begin{eulercomment}
Fungsi yang tidak berfungsi untuk input vektor harus "divektorkan".
Ini dapat dicapai dengan kata kunci "peta" dalam definisi fungsi.
Kemudian fungsi tersebut akan dievaluasi untuk setiap elemen dari
parameter vektor.

Integrasi numerik terintegrasi() hanya berfungsi untuk batas interval
skalar. Jadi kita perlu membuat vektor.
\end{eulercomment}
\begin{eulerprompt}
>function map f(x) := integrate("x^x",1,x)
\end{eulerprompt}
\begin{eulercomment}
Kata kunci "peta" membuat vektor fungsi. Fungsinya sekarang akan
bekerja\\
untuk vektor bilangan.
\end{eulercomment}
\begin{eulerprompt}
>f([1:5])
\end{eulerprompt}
\begin{euleroutput}
  [0,  2.05045,  13.7251,  113.336,  1241.03]
\end{euleroutput}
\eulerheading{Sub-Matriks dan Matriks-Elemen}
\begin{eulercomment}
Untuk mengakses elemen matriks, gunakan notasi braket.
\end{eulercomment}
\begin{eulerprompt}
>A=[1,2,3;4,5,6;7,8,9], A[2,2]
\end{eulerprompt}
\begin{euleroutput}
              1             2             3 
              4             5             6 
              7             8             9 
  5
\end{euleroutput}
\begin{eulercomment}
Kita dapat mengakses satu baris matriks yang lengkap.
\end{eulercomment}
\begin{eulerprompt}
>A[2]
\end{eulerprompt}
\begin{euleroutput}
  [4,  5,  6]
\end{euleroutput}
\begin{eulercomment}
Dalam kasus vektor baris atau kolom, ini mengembalikan elemen vektor.
\end{eulercomment}
\begin{eulerprompt}
>v=1:3; v[2]
\end{eulerprompt}
\begin{euleroutput}
  2
\end{euleroutput}
\begin{eulercomment}
Untuk memastikan, Anda mendapatkan baris pertama untuk matriks 1xn dan
mxn, tentukan semua kolom menggunakan indeks kedua kosong.
\end{eulercomment}
\begin{eulerprompt}
>A[2,]
\end{eulerprompt}
\begin{euleroutput}
  [4,  5,  6]
\end{euleroutput}
\begin{eulercomment}
Jika indeks adalah vektor indeks, Euler akan mengembalikan baris
matriks yang sesuai.

Di sini kita menginginkan baris pertama dan kedua dari A.
\end{eulercomment}
\begin{eulerprompt}
>A[[1,2]]
\end{eulerprompt}
\begin{euleroutput}
              1             2             3 
              4             5             6 
\end{euleroutput}
\begin{eulercomment}
Kita bahkan dapat menyusun ulang A menggunakan vektor indeks.
Tepatnya, kami tidak mengubah A di sini, tetapi menghitung versi A
yang disusun ulang.
\end{eulercomment}
\begin{eulerprompt}
>A[[3,2,1]]
\end{eulerprompt}
\begin{euleroutput}
              7             8             9 
              4             5             6 
              1             2             3 
\end{euleroutput}
\begin{eulercomment}
Trik indeks bekerja dengan kolom juga.

Contoh ini memilih semua baris A dan kolom kedua dan ketiga.
\end{eulercomment}
\begin{eulerprompt}
>A[1:3,2:3]
\end{eulerprompt}
\begin{euleroutput}
              2             3 
              5             6 
              8             9 
\end{euleroutput}
\begin{eulercomment}
Untuk singkatan ":" menunjukkan semua indeks baris atau kolom.
\end{eulercomment}
\begin{eulerprompt}
>A[:,3]
\end{eulerprompt}
\begin{euleroutput}
              3 
              6 
              9 
\end{euleroutput}
\begin{eulercomment}
Atau, biarkan indeks pertama kosong.
\end{eulercomment}
\begin{eulerprompt}
>A[,2:3]
\end{eulerprompt}
\begin{euleroutput}
              2             3 
              5             6 
              8             9 
\end{euleroutput}
\begin{eulercomment}
Kita juga bisa mendapatkan baris terakhir dari A.
\end{eulercomment}
\begin{eulerprompt}
>A[-1]
\end{eulerprompt}
\begin{euleroutput}
  [7,  8,  9]
\end{euleroutput}
\begin{eulercomment}
Sekarang mari kita ubah elemen A dengan menetapkan submatriks A ke
beberapa nilai. Ini sebenarnya mengubah matriks A yang tersimpan.
\end{eulercomment}
\begin{eulerprompt}
>A[1,1]=4
\end{eulerprompt}
\begin{euleroutput}
              4             2             3 
              4             5             6 
              7             8             9 
\end{euleroutput}
\begin{eulercomment}
Kami juga dapat menetapkan nilai ke baris A.
\end{eulercomment}
\begin{eulerprompt}
>A[1]=[-1,-1,-1]
\end{eulerprompt}
\begin{euleroutput}
             -1            -1            -1 
              4             5             6 
              7             8             9 
\end{euleroutput}
\begin{eulercomment}
Kami bahkan dapat menetapkan sub-matriks jika memiliki ukuran yang
tepat.
\end{eulercomment}
\begin{eulerprompt}
>A[1:2,1:2]=[5,6;7,8]
\end{eulerprompt}
\begin{euleroutput}
              5             6            -1 
              7             8             6 
              7             8             9 
\end{euleroutput}
\begin{eulercomment}
Selain itu, beberapa jalan pintas diperbolehkan.
\end{eulercomment}
\begin{eulerprompt}
>A[1:2,1:2]=0
\end{eulerprompt}
\begin{euleroutput}
              0             0            -1 
              0             0             6 
              7             8             9 
\end{euleroutput}
\begin{eulercomment}
Peringatan: Indeks di luar batas mengembalikan matriks kosong, atau
pesan kesalahan, tergantung pada pengaturan sistem. Standarnya adalah
pesan kesalahan. Ingat, bagaimanapun, bahwa indeks negatif dapat
digunakan untuk mengakses elemen matriks yang dihitung dari akhir.
\end{eulercomment}
\begin{eulerprompt}
>A[4]
\end{eulerprompt}
\begin{euleroutput}
  Row index 4 out of bounds!
  Error in:
  A[4] ...
      ^
\end{euleroutput}
\eulerheading{Menyortir dan Mengacak}
\begin{eulercomment}
Fungsi sort() mengurutkan vektor baris.
\end{eulercomment}
\begin{eulerprompt}
>sort([5,6,4,8,1,9])
\end{eulerprompt}
\begin{euleroutput}
  [1,  4,  5,  6,  8,  9]
\end{euleroutput}
\begin{eulercomment}
Seringkali perlu untuk mengetahui indeks dari vektor yang diurutkan
dalam vektor aslinya. Ini dapat digunakan untuk menyusun ulang vektor
lain dengan cara yang sama.

Mari kita mengocok vektor.
\end{eulercomment}
\begin{eulerprompt}
>v=shuffle(1:10)
\end{eulerprompt}
\begin{euleroutput}
  [4,  5,  10,  6,  8,  9,  1,  7,  2,  3]
\end{euleroutput}
\begin{eulercomment}
Indeks berisi urutan yang tepat dari v.
\end{eulercomment}
\begin{eulerprompt}
>\{vs,ind\}=sort(v); v[ind]
\end{eulerprompt}
\begin{euleroutput}
  [1,  2,  3,  4,  5,  6,  7,  8,  9,  10]
\end{euleroutput}
\begin{eulercomment}
Ini bekerja untuk vektor string juga.
\end{eulercomment}
\begin{eulerprompt}
>s=["a","d","e","a","aa","e"]
\end{eulerprompt}
\begin{euleroutput}
  a
  d
  e
  a
  aa
  e
\end{euleroutput}
\begin{eulerprompt}
>\{ss,ind\}=sort(s); ss
\end{eulerprompt}
\begin{euleroutput}
  a
  a
  aa
  d
  e
  e
\end{euleroutput}
\begin{eulercomment}
Seperti yang Anda lihat, posisi entri ganda agak acak.
\end{eulercomment}
\begin{eulerprompt}
>ind
\end{eulerprompt}
\begin{euleroutput}
  [4,  1,  5,  2,  6,  3]
\end{euleroutput}
\begin{eulercomment}
Fungsi unik mengembalikan daftar elemen unik vektor yang diurutkan.
\end{eulercomment}
\begin{eulerprompt}
>intrandom(1,10,10), unique(%)
\end{eulerprompt}
\begin{euleroutput}
  [4,  4,  9,  2,  6,  5,  10,  6,  5,  1]
  [1,  2,  4,  5,  6,  9,  10]
\end{euleroutput}
\begin{eulercomment}
Ini bekerja untuk vektor string juga.
\end{eulercomment}
\begin{eulerprompt}
>unique(s)
\end{eulerprompt}
\begin{euleroutput}
  a
  aa
  d
  e
\end{euleroutput}
\eulerheading{Aljabar linier}
\begin{eulercomment}
EMT memiliki banyak fungsi untuk menyelesaikan sistem linier, sistem
sparse, atau masalah regresi.

Untuk sistem linier Ax=b, Anda dapat menggunakan algoritma Gauss,
matriks invers atau kecocokan linier. Operator A\textbackslash{}b menggunakan versi
algoritma Gauss.
\end{eulercomment}
\begin{eulerprompt}
>A=[1,2;3,4]; b=[5;6]; A\(\backslash\)b
\end{eulerprompt}
\begin{euleroutput}
             -4 
            4.5 
\end{euleroutput}
\begin{eulercomment}
Untuk contoh lain, kami membuat matriks 200x200 dan jumlah barisnya.
Kemudian kita selesaikan Ax=b menggunakan matriks invers. Kami
mengukur kesalahan sebagai deviasi maksimal semua elemen dari 1, yang
tentu saja merupakan solusi yang benar.
\end{eulercomment}
\begin{eulerprompt}
>A=normal(200,200); b=sum(A); longest totalmax(abs(inv(A).b-1))
\end{eulerprompt}
\begin{euleroutput}
    8.790745908981989e-13 
\end{euleroutput}
\begin{eulercomment}
Jika sistem tidak memiliki solusi, kecocokan linier meminimalkan norma
kesalahan Ax-b.
\end{eulercomment}
\begin{eulerprompt}
>A=[1,2,3;4,5,6;7,8,9]
\end{eulerprompt}
\begin{euleroutput}
              1             2             3 
              4             5             6 
              7             8             9 
\end{euleroutput}
\begin{eulercomment}
Determinan matriks ini adalah 0.
\end{eulercomment}
\begin{eulerprompt}
>det(A)
\end{eulerprompt}
\begin{euleroutput}
  0
\end{euleroutput}
\eulerheading{Matriks Simbolik}
\begin{eulercomment}
Maxima memiliki matriks simbolis. Tentu saja, Maxima dapat digunakan
untuk masalah aljabar linier sederhana seperti itu. Kita dapat
mendefinisikan matriks untuk Euler dan Maxima dengan \&:=, dan kemudian
menggunakannya dalam ekspresi simbolis. Bentuk [...] biasa untuk
mendefinisikan matriks dapat digunakan di Euler untuk mendefinisikan
matriks simbolik.
\end{eulercomment}
\begin{eulerprompt}
>A &= [a,1,1;1,a,1;1,1,a]; $A
\end{eulerprompt}
\begin{eulerformula}
\[
\begin{pmatrix}a & 1 & 1 \\ 1 & a & 1 \\ 1 & 1 & a \\ \end{pmatrix}
\]
\end{eulerformula}
\begin{eulerprompt}
>$&det(A), $&factor(%)
\end{eulerprompt}
\begin{eulerformula}
\[
\left(a-1\right)^2\,\left(a+2\right)
\]
\end{eulerformula}
\eulerimg{0}{images/Davina Safa Felisa 1-6-089-large.png}
\begin{eulerprompt}
>$&invert(A) with a=0
\end{eulerprompt}
\begin{eulerformula}
\[
\begin{pmatrix}-\frac{1}{2} & \frac{1}{2} & \frac{1}{2} \\ \frac{1  }{2} & -\frac{1}{2} & \frac{1}{2} \\ \frac{1}{2} & \frac{1}{2} & -  \frac{1}{2} \\ \end{pmatrix}
\]
\end{eulerformula}
\begin{eulerprompt}
>A &= [1,a;b,2]; $A
\end{eulerprompt}
\begin{eulerformula}
\[
\begin{pmatrix}1 & a \\ b & 2 \\ \end{pmatrix}
\]
\end{eulerformula}
\begin{eulercomment}
Seperti semua variabel simbolik, matriks ini dapat digunakan dalam
ekspresi simbolik lainnya.
\end{eulercomment}
\begin{eulerprompt}
>$&det(A-x*ident(2)), $&solve(%,x)
\end{eulerprompt}
\begin{eulerformula}
\[
\left[ x=\frac{3-\sqrt{4\,a\,b+1}}{2} , x=\frac{\sqrt{4\,a\,b+1}+3  }{2} \right] 
\]
\end{eulerformula}
\eulerimg{1}{images/Davina Safa Felisa 1-6-093-large.png}
\begin{eulercomment}
Nilai eigen juga dapat dihitung secara otomatis. Hasilnya adalah
vektor dengan dua vektor nilai eigen dan multiplisitas.
\end{eulercomment}
\begin{eulerprompt}
>$&eigenvalues([a,1;1,a])
\end{eulerprompt}
\begin{eulerformula}
\[
\left[ \left[ a-1 , a+1 \right]  , \left[ 1 , 1 \right]  \right] 
\]
\end{eulerformula}
\begin{eulercomment}
Untuk mengekstrak vektor eigen tertentu perlu pengindeksan yang
cermat.
\end{eulercomment}
\begin{eulerprompt}
>$&eigenvectors([a,1;1,a]), &%[2][1][1]
\end{eulerprompt}
\begin{eulerformula}
\[
\left[ \left[ \left[ a-1 , a+1 \right]  , \left[ 1 , 1 \right]    \right]  , \left[ \left[ \left[ 1 , -1 \right]  \right]  , \left[   \left[ 1 , 1 \right]  \right]  \right]  \right] 
\]
\end{eulerformula}
\begin{euleroutput}
  
                                 [1, - 1]
  
\end{euleroutput}
\begin{eulercomment}
Matriks simbolik dapat dievaluasi dalam Euler secara numerik seperti
ekspresi simbolik lainnya.
\end{eulercomment}
\begin{eulerprompt}
>A(a=4,b=5)
\end{eulerprompt}
\begin{euleroutput}
              1             4 
              5             2 
\end{euleroutput}
\begin{eulercomment}
Dalam ekspresi simbolik, gunakan dengan.
\end{eulercomment}
\begin{eulerprompt}
>$&A with [a=4,b=5]
\end{eulerprompt}
\begin{eulerformula}
\[
\begin{pmatrix}1 & 4 \\ 5 & 2 \\ \end{pmatrix}
\]
\end{eulerformula}
\begin{eulercomment}
Akses ke baris matriks simbolik bekerja seperti halnya dengan matriks
numerik.
\end{eulercomment}
\begin{eulerprompt}
>$&A[1]
\end{eulerprompt}
\begin{eulerformula}
\[
\left[ 1 , a \right] 
\]
\end{eulerformula}
\begin{eulercomment}
Ekspresi simbolis dapat berisi tugas. Dan itu mengubah matriks A.
\end{eulercomment}
\begin{eulerprompt}
>&A[1,1]:=t+1; $&A
\end{eulerprompt}
\begin{eulerformula}
\[
\begin{pmatrix}t+1 & a \\ b & 2 \\ \end{pmatrix}
\]
\end{eulerformula}
\begin{eulercomment}
Ada fungsi simbolik di Maxima untuk membuat vektor dan matriks. Untuk
ini, lihat dokumentasi Maxima atau tutorial tentang Maxima di EMT.
\end{eulercomment}
\begin{eulerprompt}
>v &= makelist(1/(i+j),i,1,3); $v
\end{eulerprompt}
\begin{eulerformula}
\[
\left[ \frac{1}{j+1} , \frac{1}{j+2} , \frac{1}{j+3} \right] 
\]
\end{eulerformula}
\begin{eulerttcomment}
 
\end{eulerttcomment}
\begin{eulerprompt}
>B &:= [1,2;3,4]; $B, $&invert(B)
\end{eulerprompt}
\begin{eulerformula}
\[
\begin{pmatrix}-2 & 1 \\ \frac{3}{2} & -\frac{1}{2} \\   \end{pmatrix}
\]
\end{eulerformula}
\eulerimg{1}{images/Davina Safa Felisa 1-6-101-large.png}
\begin{eulercomment}
Hasilnya dapat dievaluasi secara numerik dalam Euler. Untuk informasi
lebih lanjut tentang Maxima, lihat pengantar Maxima.
\end{eulercomment}
\begin{eulerprompt}
>$&invert(B)()
\end{eulerprompt}
\begin{euleroutput}
             -2             1 
            1.5          -0.5 
\end{euleroutput}
\begin{eulercomment}
Euler juga memiliki fungsi xinv() yang kuat, yang membuat upaya lebih
besar dan mendapatkan hasil yang lebih tepat.

Perhatikan, bahwa dengan \&:= matriks B telah didefinisikan sebagai
simbolik dalam ekspresi simbolik dan sebagai numerik dalam ekspresi
numerik. Jadi kita bisa menggunakannya di sini.
\end{eulercomment}
\begin{eulerprompt}
>longest B.xinv(B)
\end{eulerprompt}
\begin{euleroutput}
                        1                       0 
                        0                       1 
\end{euleroutput}
\begin{eulercomment}
Misalnya. nilai eigen dari A dapat dihitung secara numerik.
\end{eulercomment}
\begin{eulerprompt}
>A=[1,2,3;4,5,6;7,8,9]; real(eigenvalues(A))
\end{eulerprompt}
\begin{euleroutput}
  [16.1168,  -1.11684,  0]
\end{euleroutput}
\begin{eulercomment}
Atau secara simbolis. Lihat tutorial tentang Maxima untuk detailnya.
\end{eulercomment}
\begin{eulerprompt}
>$&eigenvalues(@A)
\end{eulerprompt}
\begin{eulerformula}
\[
\left[ \left[ \frac{15-3\,\sqrt{33}}{2} , \frac{3\,\sqrt{33}+15}{2}   , 0 \right]  , \left[ 1 , 1 , 1 \right]  \right] 
\]
\end{eulerformula}
\eulerheading{Nilai Numerik dalam Ekspresi simbolis}
\begin{eulercomment}
Ekspresi simbolis hanyalah string yang berisi ekspresi. Jika kita
ingin mendefinisikan nilai baik untuk ekspresi simbolik maupun
ekspresi numerik, kita harus menggunakan "\&:=".
\end{eulercomment}
\begin{eulerprompt}
>A &:= [1,pi;4,5]
\end{eulerprompt}
\begin{euleroutput}
              1       3.14159 
              4             5 
\end{euleroutput}
\begin{eulercomment}
Masih ada perbedaan antara bentuk numerik dan simbolik. Saat
mentransfer matriks ke bentuk simbolis, pendekatan fraksional untuk
real akan digunakan.
\end{eulercomment}
\begin{eulerprompt}
>$&A
\end{eulerprompt}
\begin{eulerformula}
\[
\begin{pmatrix}1 & \frac{1146408}{364913} \\ 4 & 5 \\ \end{pmatrix}
\]
\end{eulerformula}
\begin{eulercomment}
Untuk menghindarinya, ada fungsi "mxmset(variable)".
\end{eulercomment}
\begin{eulerprompt}
>mxmset(A); $&A
\end{eulerprompt}
\begin{eulerformula}
\[
\begin{pmatrix}1 & 3.141592653589793 \\ 4 & 5 \\ \end{pmatrix}
\]
\end{eulerformula}
\begin{eulercomment}
Maxima juga dapat menghitung dengan angka floating point, dan bahkan
dengan angka floating besar dengan 32 digit. Namun, evaluasinya jauh
lebih lambat.
\end{eulercomment}
\begin{eulerprompt}
>$&bfloat(sqrt(2)), $&float(sqrt(2))
\end{eulerprompt}
\begin{eulerformula}
\[
1.414213562373095
\]
\end{eulerformula}
\eulerimg{0}{images/Davina Safa Felisa 1-6-106-large.png}
\begin{eulercomment}
Ketepatan angka floating point besar dapat diubah.
\end{eulercomment}
\begin{eulerprompt}
>&fpprec:=100; &bfloat(pi)
\end{eulerprompt}
\begin{euleroutput}
  
          3.14159265358979323846264338327950288419716939937510582097494\(\backslash\)
  4592307816406286208998628034825342117068b0
  
\end{euleroutput}
\begin{eulercomment}
Variabel numerik dapat digunakan dalam ekspresi simbolis apa pun
menggunakan "@var".

Perhatikan bahwa ini hanya diperlukan, jika variabel telah
didefinisikan dengan ":=" atau "=" sebagai variabel numerik.
\end{eulercomment}
\begin{eulerprompt}
>B:=[1,pi;3,4]; $&det(@B)
\end{eulerprompt}
\begin{eulerformula}
\[
-5.424777960769379
\]
\end{eulerformula}
\eulerheading{Demo - Suku Bunga}
\begin{eulercomment}
Di bawah ini, kami menggunakan Euler Math Toolbox (EMT) untuk
perhitungan suku bunga. Kami melakukannya secara numerik dan simbolis
untuk menunjukkan kepada Anda bagaimana Euler dapat digunakan untuk
memecahkan masalah kehidupan nyata.

Asumsikan Anda memiliki modal awal 5000 (katakanlah dalam dolar).
\end{eulercomment}
\begin{eulerprompt}
>K=5000
\end{eulerprompt}
\begin{euleroutput}
  5000
\end{euleroutput}
\begin{eulercomment}
Sekarang kita asumsikan tingkat bunga 3\% per tahun. Mari kita
tambahkan satu tarif sederhana dan hitung hasilnya.
\end{eulercomment}
\begin{eulerprompt}
>K*1.03
\end{eulerprompt}
\begin{euleroutput}
  5150
\end{euleroutput}
\begin{eulercomment}
Euler akan memahami sintaks berikut juga.
\end{eulercomment}
\begin{eulerprompt}
>K+K*3%
\end{eulerprompt}
\begin{euleroutput}
  5150
\end{euleroutput}
\begin{eulercomment}
Tetapi lebih mudah menggunakan faktornya
\end{eulercomment}
\begin{eulerprompt}
>q=1+3%, K*q
\end{eulerprompt}
\begin{euleroutput}
  1.03
  5150
\end{euleroutput}
\begin{eulercomment}
Selama 10 tahun, kita cukup mengalikan faktornya dan mendapatkan nilai
akhir dengan suku bunga majemuk.
\end{eulercomment}
\begin{eulerprompt}
>K*q^10
\end{eulerprompt}
\begin{euleroutput}
  6719.58189672
\end{euleroutput}
\begin{eulercomment}
Untuk tujuan kita, kita dapat mengatur format menjadi 2 digit setelah
titik desimal.
\end{eulercomment}
\begin{eulerprompt}
>format(12,2); K*q^10
\end{eulerprompt}
\begin{euleroutput}
      6719.58 
\end{euleroutput}
\begin{eulercomment}
Mari kita cetak yang dibulatkan menjadi 2 digit dalam kalimat lengkap.
\end{eulercomment}
\begin{eulerprompt}
>"Starting from " + K + "$ you get " + round(K*q^10,2) + "$."
\end{eulerprompt}
\begin{euleroutput}
  Starting from 5000$ you get 6719.58$.
\end{euleroutput}
\begin{eulercomment}
Bagaimana jika kita ingin mengetahui hasil antara dari tahun 1 sampai
tahun 9? Untuk ini, bahasa matriks Euler sangat membantu. Anda tidak
harus menulis loop, tetapi cukup masukkan
\end{eulercomment}
\begin{eulerprompt}
>K*q^(0:10)
\end{eulerprompt}
\begin{euleroutput}
  Real 1 x 11 matrix
  
      5000.00     5150.00     5304.50     5463.64     ...
\end{euleroutput}
\begin{eulercomment}
Bagaimana keajaiban ini bekerja? Pertama ekspresi 0:10 mengembalikan
vektor bilangan bulat.
\end{eulercomment}
\begin{eulerprompt}
>short 0:10
\end{eulerprompt}
\begin{euleroutput}
  [0,  1,  2,  3,  4,  5,  6,  7,  8,  9,  10]
\end{euleroutput}
\begin{eulercomment}
Kemudian semua operator dan fungsi dalam Euler dapat diterapkan pada
elemen vektor untuk elemen. Jadi
\end{eulercomment}
\begin{eulerprompt}
>short q^(0:10)
\end{eulerprompt}
\begin{euleroutput}
  [1,  1.03,  1.0609,  1.0927,  1.1255,  1.1593,  1.1941,  1.2299,
  1.2668,  1.3048,  1.3439]
\end{euleroutput}
\begin{eulercomment}
adalah vektor faktor q\textasciicircum{}0 sampai q\textasciicircum{}10. Ini dikalikan dengan K, dan kami
mendapatkan vektor nilai.
\end{eulercomment}
\begin{eulerprompt}
>VK=K*q^(0:10);
\end{eulerprompt}
\begin{eulercomment}
Tentu saja, cara realistis untuk menghitung suku bunga ini adalah
dengan membulatkan ke sen terdekat setelah setiap tahun. Mari kita
tambahkan fungsi untuk ini.
\end{eulercomment}
\begin{eulerprompt}
>function oneyear (K) := round(K*q,2)
\end{eulerprompt}
\begin{eulercomment}
Mari kita bandingkan dua hasil, dengan dan tanpa pembulatan.
\end{eulercomment}
\begin{eulerprompt}
>longest oneyear(1234.57), longest 1234.57*q
\end{eulerprompt}
\begin{euleroutput}
                  1271.61 
                1271.6071 
\end{euleroutput}
\begin{eulercomment}
Sekarang tidak ada rumus sederhana untuk tahun ke-n, dan kita harus
mengulang selama bertahun-tahun. Euler memberikan banyak solusi untuk
ini.

Cara termudah adalah iterasi fungsi, yang mengulangi fungsi tertentu
beberapa kali.
\end{eulercomment}
\begin{eulerprompt}
>VKr=iterate("oneyear",5000,10)
\end{eulerprompt}
\begin{euleroutput}
  Real 1 x 11 matrix
  
      5000.00     5150.00     5304.50     5463.64     ...
\end{euleroutput}
\begin{eulercomment}
Kami dapat mencetaknya dengan cara yang ramah, menggunakan format kami
dengan tempat desimal tetap.
\end{eulercomment}
\begin{eulerprompt}
>VKr'
\end{eulerprompt}
\begin{euleroutput}
      5000.00 
      5150.00 
      5304.50 
      5463.64 
      5627.55 
      5796.38 
      5970.27 
      6149.38 
      6333.86 
      6523.88 
      6719.60 
\end{euleroutput}
\begin{eulercomment}
Untuk mendapatkan elemen tertentu dari vektor, kami menggunakan indeks
dalam tanda kurung siku.
\end{eulercomment}
\begin{eulerprompt}
>VKr[2], VKr[1:3]
\end{eulerprompt}
\begin{euleroutput}
      5150.00 
      5000.00     5150.00     5304.50 
\end{euleroutput}
\begin{eulercomment}
Anehnya, kita juga bisa menggunakan vektor indeks. Ingat bahwa 1:3
menghasilkan vektor [1,2,3].

Mari kita bandingkan elemen terakhir dari nilai yang dibulatkan dengan
nilai penuh.
\end{eulercomment}
\begin{eulerprompt}
>VKr[-1], VK[-1]
\end{eulerprompt}
\begin{euleroutput}
      6719.60 
      6719.58 
\end{euleroutput}
\begin{eulercomment}
Perbedaannya sangat kecil.

\begin{eulercomment}
\eulerheading{Memecahkan Persamaan}
\begin{eulercomment}
Sekarang kita mengambil fungsi yang lebih maju, yang menambahkan
tingkat uang tertentu setiap tahun.
\end{eulercomment}
\begin{eulerprompt}
>function onepay (K) := K*q+R
\end{eulerprompt}
\begin{eulercomment}
Kita tidak perlu menentukan q atau R untuk definisi fungsi. Hanya jika
kita menjalankan perintah, kita harus mendefinisikan nilai-nilai ini.
Kami memilih R=200.
\end{eulercomment}
\begin{eulerprompt}
>R=200; iterate("onepay",5000,10)
\end{eulerprompt}
\begin{euleroutput}
  Real 1 x 11 matrix
  
      5000.00     5350.00     5710.50     6081.82     ...
\end{euleroutput}
\begin{eulercomment}
Bagaimana jika kita menghapus jumlah yang sama setiap tahun?
\end{eulercomment}
\begin{eulerprompt}
>R=-200; iterate("onepay",5000,10)
\end{eulerprompt}
\begin{euleroutput}
  Real 1 x 11 matrix
  
      5000.00     4950.00     4898.50     4845.45     ...
\end{euleroutput}
\begin{eulercomment}
Kami melihat bahwa uang berkurang. Jelas, jika kita hanya mendapatkan
150 bunga di tahun pertama, tetapi menghapus 200, kita kehilangan uang
setiap tahun.

Bagaimana kita bisa menentukan berapa tahun uang itu akan bertahan?
Kita harus menulis loop untuk ini. Cara termudah adalah dengan iterasi
cukup lama.
\end{eulercomment}
\begin{eulerprompt}
>VKR=iterate("onepay",5000,50)
\end{eulerprompt}
\begin{euleroutput}
  Real 1 x 51 matrix
  
      5000.00     4950.00     4898.50     4845.45     ...
\end{euleroutput}
\begin{eulercomment}
Dengan menggunakan bahasa matriks, kita dapat menentukan nilai negatif
pertama dengan cara berikut.
\end{eulercomment}
\begin{eulerprompt}
>min(nonzeros(VKR<0))
\end{eulerprompt}
\begin{euleroutput}
        48.00 
\end{euleroutput}
\begin{eulercomment}
Alasan untuk ini adalah bahwa bukan nol(VKR\textless{}0) mengembalikan vektor
indeks i, di mana VKR[i]\textless{}0, dan min menghitung indeks minimal.

Karena vektor selalu dimulai dengan indeks 1, jawabannya adalah 47
tahun.

Fungsi iterate() memiliki satu trik lagi. Itu bisa mengambil kondisi
akhir sebagai argumen. Kemudian akan mengembalikan nilai dan jumlah
iterasi.
\end{eulercomment}
\begin{eulerprompt}
>\{x,n\}=iterate("onepay",5000,till="x<0"); x, n,
\end{eulerprompt}
\begin{euleroutput}
       -19.83 
        47.00 
\end{euleroutput}
\begin{eulercomment}
Mari kita coba menjawab pertanyaan yang lebih ambigu. Asumsikan kita
tahu bahwa nilainya adalah 0 setelah 50 tahun. Apa yang akan menjadi
tingkat bunga?

Ini adalah pertanyaan yang hanya bisa dijawab dengan angka. Di bawah
ini, kita akan mendapatkan formula yang diperlukan. Kemudian Anda akan
melihat bahwa tidak ada formula yang mudah untuk tingkat bunga. Tapi
untuk saat ini, kami bertujuan untuk solusi numerik.

Langkah pertama adalah mendefinisikan fungsi yang melakukan iterasi
sebanyak n kali. Kami menambahkan semua parameter ke fungsi ini.
\end{eulercomment}
\begin{eulerprompt}
>function f(K,R,P,n) := iterate("x*(1+P/100)+R",K,n;P,R)[-1]
\end{eulerprompt}
\begin{eulercomment}
Iterasinya sama seperti di atas

\end{eulercomment}
\begin{eulerformula}
\[
x_{n+1} = x_n \cdot \left(1+ \frac{P}{100}\right) + R
\]
\end{eulerformula}
\begin{eulercomment}
Tapi kami tidak lagi menggunakan nilai global R dalam ekspresi kami.
Fungsi seperti iterate() memiliki trik khusus di Euler. Anda dapat
meneruskan nilai variabel dalam ekspresi sebagai parameter titik koma.
Dalam hal ini P dan R.

Selain itu, kami hanya tertarik pada nilai terakhir. Jadi kita ambil
indeks [-1].

Mari kita coba tes.
\end{eulercomment}
\begin{eulerprompt}
>f(5000,-200,3,47)
\end{eulerprompt}
\begin{euleroutput}
       -19.83 
\end{euleroutput}
\begin{eulercomment}
Sekarang kita bisa menyelesaikan masalah kita.
\end{eulercomment}
\begin{eulerprompt}
>solve("f(5000,-200,x,50)",3)
\end{eulerprompt}
\begin{euleroutput}
         3.15 
\end{euleroutput}
\begin{eulercomment}
Rutin memecahkan memecahkan ekspresi=0 untuk variabel x. Jawabannya
adalah 3,15\% per tahun. Kami mengambil nilai awal 3\% untuk algoritma.
Fungsi solve() selalu membutuhkan nilai awal.

Kita dapat menggunakan fungsi yang sama untuk menyelesaikan pertanyaan
berikut: Berapa banyak yang dapat kita keluarkan per tahun sehingga
modal awal habis setelah 20 tahun dengan asumsi tingkat bunga 3\% per
tahun.
\end{eulercomment}
\begin{eulerprompt}
>solve("f(5000,x,3,20)",-200)
\end{eulerprompt}
\begin{euleroutput}
      -336.08 
\end{euleroutput}
\begin{eulercomment}
Perhatikan bahwa Anda tidak dapat menyelesaikan jumlah tahun, karena
fungsi kami mengasumsikan n sebagai nilai integer.

\end{eulercomment}
\eulersubheading{Solusi Simbolik untuk Masalah Suku Bunga}
\begin{eulercomment}
Kita dapat menggunakan bagian simbolik dari Euler untuk mempelajari
masalah tersebut. Pertama kita mendefinisikan fungsi onepay() kita
secara simbolis.
\end{eulercomment}
\begin{eulerprompt}
>function op(K) &= K*q+R; $&op(K)
\end{eulerprompt}
\begin{eulerformula}
\[
R+q\,K
\]
\end{eulerformula}
\begin{eulercomment}
Kita sekarang dapat mengulangi ini.
\end{eulercomment}
\begin{eulerprompt}
>$&op(op(op(op(K)))), $&expand(%)
\end{eulerprompt}
\begin{eulerformula}
\[
q^3\,R+q^2\,R+q\,R+R+q^4\,K
\]
\end{eulerformula}
\eulerimg{0}{images/Davina Safa Felisa 1-6-111-large.png}
\begin{eulercomment}
Kami melihat sebuah pola. Setelah n periode yang kita miliki

\end{eulercomment}
\begin{eulerformula}
\[
K_n = q^n K + R (1+q+\ldots+q^{n-1}) = q^n K + \frac{q^n-1}{q-1} R
\]
\end{eulerformula}
\begin{eulercomment}
Rumusnya adalah rumus untuk jumlah geometri, yang diketahui Maxima.
\end{eulercomment}
\begin{eulerprompt}
>&sum(q^k,k,0,n-1); $& % = ev(%,simpsum)
\end{eulerprompt}
\begin{eulerformula}
\[
\sum_{k=0}^{n-1}{q^{k}}=\frac{q^{n}-1}{q-1}
\]
\end{eulerformula}
\begin{eulercomment}
Ini agak rumit. Jumlahnya dievaluasi dengan bendera "simpsum" untuk
menguranginya menjadi hasil bagi.

Mari kita membuat fungsi untuk ini.
\end{eulercomment}
\begin{eulerprompt}
>function fs(K,R,P,n) &= (1+P/100)^n*K + ((1+P/100)^n-1)/(P/100)*R; $&fs(K,R,P,n)
\end{eulerprompt}
\begin{eulerformula}
\[
\frac{100\,\left(\left(\frac{P}{100}+1\right)^{n}-1\right)\,R}{P}+K  \,\left(\frac{P}{100}+1\right)^{n}
\]
\end{eulerformula}
\begin{eulercomment}
Fungsi tersebut melakukan hal yang sama seperti fungsi f kita
sebelumnya. Tapi itu lebih efektif.
\end{eulercomment}
\begin{eulerprompt}
>longest f(5000,-200,3,47), longest fs(5000,-200,3,47)
\end{eulerprompt}
\begin{euleroutput}
       -19.82504734650985 
       -19.82504734652684 
\end{euleroutput}
\begin{eulercomment}
Kita sekarang dapat menggunakannya untuk menanyakan waktu n. Kapan
modal kita habis? Dugaan awal kami adalah 30 tahun.
\end{eulercomment}
\begin{eulerprompt}
>solve("fs(5000,-330,3,x)",30)
\end{eulerprompt}
\begin{euleroutput}
        20.51 
\end{euleroutput}
\begin{eulercomment}
Jawaban ini mengatakan bahwa itu akan menjadi negatif setelah 21
tahun.

Kita juga dapat menggunakan sisi simbolis Euler untuk menghitung
formula pembayaran.

Asumsikan kita mendapatkan pinjaman sebesar K, dan membayar n
pembayaran sebesar R (dimulai setelah tahun pertama) meninggalkan sisa
hutang sebesar Kn (pada saat pembayaran terakhir). Rumus untuk ini
jelas
\end{eulercomment}
\begin{eulerprompt}
>equ &= fs(K,R,P,n)=Kn; $&equ
\end{eulerprompt}
\begin{eulerformula}
\[
\frac{100\,\left(\left(\frac{P}{100}+1\right)^{n}-1\right)\,R}{P}+K  \,\left(\frac{P}{100}+1\right)^{n}={\it Kn}
\]
\end{eulerformula}
\begin{eulercomment}
Biasanya rumus ini diberikan dalam bentuk

\end{eulercomment}
\begin{eulerformula}
\[
i = \frac{P}{100}
\]
\end{eulerformula}
\begin{eulerprompt}
>equ &= (equ with P=100*i); $&equ
\end{eulerprompt}
\begin{eulerformula}
\[
\frac{\left(\left(i+1\right)^{n}-1\right)\,R}{i}+\left(i+1\right)^{  n}\,K={\it Kn}
\]
\end{eulerformula}
\begin{eulercomment}
Kita dapat memecahkan tingkat R secara simbolis.
\end{eulercomment}
\begin{eulerprompt}
>$&solve(equ,R)
\end{eulerprompt}
\begin{eulerformula}
\[
\left[ R=\frac{i\,{\it Kn}-i\,\left(i+1\right)^{n}\,K}{\left(i+1  \right)^{n}-1} \right] 
\]
\end{eulerformula}
\begin{eulercomment}
Seperti yang Anda lihat dari rumus, fungsi ini mengembalikan kesalahan
titik mengambang untuk i=0. Euler tetap merencanakannya.

Tentu saja, kami memiliki batasan berikut.
\end{eulercomment}
\begin{eulerprompt}
>$&limit(R(5000,0,x,10),x,0)
\end{eulerprompt}
\begin{eulerformula}
\[
\lim_{x\rightarrow 0}{R\left(5000 , 0 , x , 10\right)}
\]
\end{eulerformula}
\begin{eulercomment}
Jelas, tanpa bunga kita harus membayar kembali 10 tarif 500.

Persamaan juga dapat diselesaikan untuk n. Kelihatannya lebih bagus,
jika kita menerapkan beberapa penyederhanaan untuk itu.
\end{eulercomment}
\begin{eulerprompt}
>fn &= solve(equ,n) | ratsimp; $&fn
\end{eulerprompt}
\begin{eulerformula}
\[
\left[ n=\frac{\log \left(\frac{R+i\,{\it Kn}}{R+i\,K}\right)}{  \log \left(i+1\right)} \right] 
\]
\end{eulerformula}
\eulerheading{Menggambar Grafik 2D dengan EMT}
\begin{eulercomment}
Notebook ini menjelaskan tentang cara menggambar berbagaikurva dan
grafik 2D dengan software EMT. EMT menyediakan fungsi plot2d() untuk
menggambar berbagai kurva dan grafik dua dimensi (2D).\\
\end{eulercomment}
\eulersubheading{Plot Dasar}
\begin{eulercomment}
Ada fungsi yang sangat mendasar dari plot. Ada koordinat layar, yang
selalu berkisar dari 0 hingga 1024 di setiap sumbu, tidak peduli
apakah layarnya persegi atau tidak. Semut ada koordinat plot, yang
dapat diatur dengan setplot(). Pemetaan antara koordinat tergantung
pada jendela plot saat ini. Misalnya, shrinkwindow() default
menyisakan ruang untuk label sumbu dan judul plot.

Dalam contoh, kita hanya menggambar beberapa garis acak dalam berbagai
warna. Untuk detail tentang fungsi ini, pelajari fungsi inti EMT.
\end{eulercomment}
\begin{eulerprompt}
>clg; // clear screen
>window(0,0,1024,1024); // use all of the window
>setplot(0,1,0,1); // set plot coordinates
>hold on; // start overwrite mode
>n=100; X=random(n,2); Y=random(n,2);  // get random points
>colors=rgb(random(n),random(n),random(n)); // get random colors
>loop 1 to n; color(colors[#]); plot(X[#],Y[#]); end; // plot
>hold off; // end overwrite mode
>insimg; // insert to notebook
\end{eulerprompt}
\eulerimg{27}{images/Davina Safa Felisa 1-6-121.png}
\begin{eulerprompt}
>reset;
\end{eulerprompt}
\begin{eulercomment}
Grafik perlu ditahan, karena perintah plot() akan menghapus jendela
plot.

Untuk menghapus semua yang kami lakukan, kami menggunakan reset().

Untuk menampilkan gambar hasil plot di layar notebook, perintah
plot2d() dapat diakhiri dengan titik dua (:). Cara lain adalah
perintah plot2d() diakhiri dengan titik koma (;), kemudian menggunakan
perintah insimg() untuk menampilkan gambar hasil plot.

Untuk contoh lain, kami menggambar plot sebagai sisipan di plot lain.
Ini dilakukan dengan mendefinisikan jendela plot yang lebih kecil.
Perhatikan bahwa jendela ini tidak menyediakan ruang untuk label sumbu
di luar jendela plot. Kita harus menambahkan beberapa margin untuk ini
sesuai kebutuhan. Perhatikan bahwa kami menyimpan dan memulihkan
jendela penuh, dan menahan plot saat ini saat kami memplot inset.
\end{eulercomment}
\begin{eulerprompt}
>plot2d("x^3-x");
>xw=200; yw=100; ww=300; hw=300;
>ow=window();
>window(xw,yw,xw+ww,yw+hw);
>hold on;
>barclear(xw-50,yw-10,ww+60,ww+60);
>plot2d("x^4-x",grid=6):
\end{eulerprompt}
\eulerimg{27}{images/Davina Safa Felisa 1-6-122.png}
\begin{eulerprompt}
>hold off;
>window(ow);
\end{eulerprompt}
\begin{eulercomment}
Plot dengan banyak angka dicapai dengan cara yang sama. Ada fungsi
figure() utilitas untuk ini.

\end{eulercomment}
\eulersubheading{Aspek Plot}
\begin{eulercomment}
Plot default menggunakan jendela plot persegi. Anda dapat mengubah ini
dengan fungsi aspek(). Jangan lupa untuk mengatur ulang aspek nanti.
Anda juga dapat mengubah default ini di menu dengan "Set Aspect" ke
rasio aspek tertentu atau ke ukuran jendela grafis saat ini.

Tetapi Anda juga dapat mengubahnya untuk satu plot. Untuk ini, ukuran
area plot saat ini diubah, dan jendela diatur sehingga label memiliki
cukup ruang.
\end{eulercomment}
\begin{eulerprompt}
>aspect(2); // rasio panjang dan lebar 2:1
>plot2d(["sin(x)","cos(x)"],0,2pi):
\end{eulerprompt}
\eulerimg{13}{images/Davina Safa Felisa 1-6-123.png}
\begin{eulerprompt}
>aspect();
>reset;
\end{eulerprompt}
\begin{eulercomment}
Fungsi reset() mengembalikan default plot termasuk rasio aspek.\\
Plot 2D di Euler

EMT Math Toolbox memiliki plot dalam 2D, baik untuk data maupun
fungsi. EMT menggunakan fungsi plot2d. Fungsi ini dapat memplot fungsi
dan data.

Dimungkinkan untuk membuat plot di Maxima menggunakan Gnuplot atau
dengan Python menggunakan Math Plot Lib.

Euler dapat memplot plot 2D dari

- ekspresi\\
- fungsi, variabel, atau kurva parameter,\\
- vektor nilai x-y,\\
- awan titik di pesawat,\\
- kurva implisit dengan level atau wilayah level.\\
- Fungsi kompleks

Gaya plot mencakup berbagai gaya untuk garis dan titik, plot batang
dan plot berbayang.\\
Plot Ekspresi atau Variabel

Ekspresi tunggal dalam "x" (mis. "4*x\textasciicircum{}2") atau nama fungsi (mis. "f")
menghasilkan grafik fungsi.

Berikut adalah contoh paling dasar, yang menggunakan rentang default
dan menetapkan rentang y yang tepat agar sesuai dengan plot fungsi.

Catatan: Jika Anda mengakhiri baris perintah dengan titik dua ":",
plot akan dimasukkan ke dalam jendela teks. Jika tidak, tekan TAB
untuk melihat plot jika jendela plot tertutup.
\end{eulercomment}
\begin{eulerprompt}
>plot2d("x^2"):
\end{eulerprompt}
\eulerimg{27}{images/Davina Safa Felisa 1-6-124.png}
\begin{eulerprompt}
>aspect(1.5); plot2d("x^3-x"):
\end{eulerprompt}
\eulerimg{17}{images/Davina Safa Felisa 1-6-125.png}
\begin{eulerprompt}
>a:=5.6; plot2d("exp(-a*x^2)/a"); insimg(30); // menampilkan gambar hasil plot setinggi 25 baris
\end{eulerprompt}
\eulerimg{17}{images/Davina Safa Felisa 1-6-126.png}
\begin{eulercomment}
Dari beberapa contoh sebelumnya Anda dapat melihat bahwa Gambaran
gambar plot menggunakan sumbu X dengan rentang nilai dari -2 sampai
dengan 2. Untuk mengubah rentang nilai X dan Y, Anda dapat menambahkan
nilai batas X (dan Y) di belakang ekspresi yang digambar.

Rentang plot diatur dengan parameter yang ditetapkan berikut:

- a,b: rentang-x (default -2,2)\\
- c,d: y-range (default: skala dengan nilai)\\
- r: sebagai alternatif radius di sekitar pusat plot\\
- cx,cy: koordinat pusat plot (default 0,0)
\end{eulercomment}
\begin{eulerprompt}
>plot2d("x^3-x",-1,2):
\end{eulerprompt}
\eulerimg{17}{images/Davina Safa Felisa 1-6-127.png}
\begin{eulerprompt}
>plot2d("sin(x)",-2*pi,2*pi): // plot sin(x) pada interval [-2pi, 2pi]
\end{eulerprompt}
\eulerimg{17}{images/Davina Safa Felisa 1-6-128.png}
\begin{eulerprompt}
>plot2d("cos(x)","sin(3*x)",xmin=0,xmax=2pi):
\end{eulerprompt}
\eulerimg{17}{images/Davina Safa Felisa 1-6-129.png}
\begin{eulercomment}
Alternatif untuk titik dua adalah perintah insimg(baris), yang
menyisipkan plot yang menempati sejumlah baris teks tertentu.

Dalam opsi, plot dapat diatur untuk muncul

- di jendela terpisah yang dapat diubah ukurannya,\\
- di jendela buku catatan.

Lebih banyak gaya dapat dicapai dengan perintah plot tertentu.

Bagaimanapun, tekan tombol tabulator untuk melihat plot, jika
disembunyikan.

Untuk membagi jendela menjadi beberapa plot, gunakan perintah
figure(). Dalam contoh, kami memplot x\textasciicircum{}1 hingga x\textasciicircum{}4 menjadi 4 bagian
jendela. figure(0) mengatur ulang jendela default.
\end{eulercomment}
\begin{eulerprompt}
>reset;
>figure(2,2); ...
>for n=1 to 4; figure(n); plot2d("x^"+n); end; ...
>figure(0):
\end{eulerprompt}
\eulerimg{27}{images/Davina Safa Felisa 1-6-130.png}
\begin{eulercomment}
Di plot2d(), ada gaya alternatif yang tersedia dengan grid=x. Untuk
gambaran umum, kami menunjukkan berbagai gaya kisi dalam satu gambar
(lihat di bawah untuk perintah figure()). Gaya kisi=0 tidak
disertakan. Ini menunjukkan tidak ada grid dan tidak ada bingkai.
\end{eulercomment}
\begin{eulerprompt}
>figure(3,3); ...
>for k=1:9; figure(k); plot2d("x^3-x",-2,1,grid=k); end; ...
>figure(0):
\end{eulerprompt}
\eulerimg{27}{images/Davina Safa Felisa 1-6-131.png}
\begin{eulercomment}
Jika argumen ke plot2d() adalah ekspresi yang diikuti oleh empat
angka, angka-angka ini adalah rentang x dan y untuk plot.

Atau, a, b, c, d dapat ditentukan sebagai parameter yang ditetapkan
sebagai a=... dll.

Dalam contoh berikut, kita mengubah gaya kisi, menambahkan label, dan
menggunakan label vertikal untuk sumbu y.
\end{eulercomment}
\begin{eulerprompt}
>aspect(1.5); plot2d("sin(x)",0,2pi,-1.2,1.2,grid=3,xl="x",yl="sin(x)"):
\end{eulerprompt}
\eulerimg{17}{images/Davina Safa Felisa 1-6-132.png}
\begin{eulerprompt}
>plot2d("sin(x)+cos(2*x)",0,4pi):
\end{eulerprompt}
\eulerimg{17}{images/Davina Safa Felisa 1-6-133.png}
\begin{eulercomment}
Gambar yang dihasilkan dengan memasukkan plot ke dalam jendela teks
disimpan di direktori yang sama dengan buku catatan, secara default di
subdirektori bernama "gambar". Mereka juga digunakan oleh ekspor HTML.

Anda cukup menandai gambar apa saja dan menyalinnya ke clipboard
dengan Ctrl-C. Tentu saja, Anda juga dapat mengekspor grafik saat ini
dengan fungsi di menu File.

Fungsi atau ekspresi dalam plot2d dievaluasi secara adaptif. Untuk
kecepatan lebih, matikan plot adaptif dengan \textless{}adaptive dan tentukan
jumlah subinterval dengan n=... Ini hanya diperlukan dalam kasus yang
jarang terjadi.
\end{eulercomment}
\begin{eulerprompt}
>plot2d("sign(x)*exp(-x^2)",-1,1,<adaptive,n=10000):
\end{eulerprompt}
\eulerimg{17}{images/Davina Safa Felisa 1-6-134.png}
\begin{eulerprompt}
>plot2d("x^x",r=1.2,cx=1,cy=1):
\end{eulerprompt}
\eulerimg{17}{images/Davina Safa Felisa 1-6-135.png}
\begin{eulercomment}
Perhatikan bahwa x\textasciicircum{}x tidak didefinisikan untuk x\textless{}=0. Fungsi plot2d
menangkap kesalahan ini, dan mulai merencanakan segera setelah fungsi
didefinisikan. Ini berfungsi untuk semua fungsi yang mengembalikan NAN
keluar dari jangkauan definisinya.
\end{eulercomment}
\begin{eulerprompt}
>plot2d("log(x)",-0.1,2):
\end{eulerprompt}
\eulerimg{17}{images/Davina Safa Felisa 1-6-136.png}
\begin{eulercomment}
Parameter square=true (atau \textgreater{}square) memilih y-range secara otomatis
sehingga hasilnya adalah jendela plot persegi. Perhatikan bahwa secara
default, Euler menggunakan ruang persegi di dalam jendela plot.
\end{eulercomment}
\begin{eulerprompt}
>plot2d("x^3-x",>square):
\end{eulerprompt}
\eulerimg{17}{images/Davina Safa Felisa 1-6-137.png}
\begin{eulerprompt}
>plot2d(''integrate("sin(x)*exp(-x^2)",0,x)'',0,2): // plot integral
\end{eulerprompt}
\eulerimg{17}{images/Davina Safa Felisa 1-6-138.png}
\begin{eulercomment}
Jika Anda membutuhkan lebih banyak ruang untuk label-y, panggil
shrinkwindow() dengan parameter yang lebih kecil, atau tetapkan nilai
positif untuk "lebih kecil" di plot2d().
\end{eulercomment}
\begin{eulerprompt}
>plot2d("gamma(x)",1,10,yl="y-values",smaller=6,<vertical):
\end{eulerprompt}
\eulerimg{17}{images/Davina Safa Felisa 1-6-139.png}
\begin{eulercomment}
Ekspresi simbolik juga dapat digunakan, karena disimpan sebagai
ekspresi string sederhana.
\end{eulercomment}
\begin{eulerprompt}
>x=linspace(0,2pi,1000); plot2d(sin(5x),cos(7x)):
\end{eulerprompt}
\eulerimg{17}{images/Davina Safa Felisa 1-6-140.png}
\begin{eulerprompt}
>a:=5.6; expr &= exp(-a*x^2)/a; // define expression
>plot2d(expr,-2,2): // plot from -2 to 2
\end{eulerprompt}
\eulerimg{17}{images/Davina Safa Felisa 1-6-141.png}
\begin{eulerprompt}
>plot2d(expr,r=1,thickness=2): // plot in a square around (0,0)
\end{eulerprompt}
\eulerimg{17}{images/Davina Safa Felisa 1-6-142.png}
\begin{eulerprompt}
>plot2d(&diff(expr,x),>add,style="--",color=red): // add another plot
\end{eulerprompt}
\eulerimg{17}{images/Davina Safa Felisa 1-6-143.png}
\begin{eulerprompt}
>plot2d(&diff(expr,x,2),a=-2,b=2,c=-2,d=1): // plot in rectangle
\end{eulerprompt}
\eulerimg{17}{images/Davina Safa Felisa 1-6-144.png}
\begin{eulerprompt}
>plot2d(&diff(expr,x),a=-2,b=2,>square): // keep plot square
\end{eulerprompt}
\eulerimg{17}{images/Davina Safa Felisa 1-6-145.png}
\begin{eulerprompt}
>plot2d("x^2",0,1,steps=1,color=red,n=10):
\end{eulerprompt}
\eulerimg{17}{images/Davina Safa Felisa 1-6-146.png}
\begin{eulerprompt}
>plot2d("x^2",>add,steps=2,color=blue,n=10):
\end{eulerprompt}
\eulerimg{17}{images/Davina Safa Felisa 1-6-147.png}
\eulerheading{Fungsi dalam satu Parameter}
\begin{eulercomment}
Fungsi plot yang paling penting untuk plot planar adalah plot2d().
Fungsi ini diimplementasikan dalam bahasa Euler dalam file "plot.e",
yang dimuat di awal program.

Berikut adalah beberapa contoh menggunakan fungsi. Seperti biasa di
EMT, fungsi yang berfungsi untuk fungsi atau ekspresi lain, Anda dapat
meneruskan parameter tambahan (selain x) yang bukan variabel global ke
fungsi dengan parameter titik koma atau dengan koleksi panggilan.
\end{eulercomment}
\begin{eulerprompt}
>function f(x,a) := x^2/a+a*x^2-x; // define a function
>a=0.3; plot2d("f",0,1;a): // plot with a=0.3
\end{eulerprompt}
\eulerimg{17}{images/Davina Safa Felisa 1-6-148.png}
\begin{eulerprompt}
>plot2d("f",0,1;0.4): // plot with a=0.4
\end{eulerprompt}
\eulerimg{17}{images/Davina Safa Felisa 1-6-149.png}
\begin{eulerprompt}
>plot2d(\{\{"f",0.2\}\},0,1): // plot with a=0.2
\end{eulerprompt}
\eulerimg{17}{images/Davina Safa Felisa 1-6-150.png}
\begin{eulerprompt}
>plot2d(\{\{"f(x,b)",b=0.1\}\},0,1): // plot with 0.1
\end{eulerprompt}
\eulerimg{17}{images/Davina Safa Felisa 1-6-151.png}
\begin{eulerprompt}
>function f(x) := x^3-x; ...
>plot2d("f",r=1):
\end{eulerprompt}
\eulerimg{17}{images/Davina Safa Felisa 1-6-152.png}
\begin{eulercomment}
Berikut adalah ringkasan dari fungsi yang diterima

- ekspresi atau ekspresi simbolik dalam x\\
- fungsi atau fungsi simbolis dengan nama sebagai "f"\\
- fungsi simbolis hanya dengan nama f

Fungsi plot2d() juga menerima fungsi simbolis. Untuk fungsi simbolis,
nama saja yang berfungsi.
\end{eulercomment}
\begin{eulerprompt}
>function f(x) &= diff(x^x,x)
\end{eulerprompt}
\begin{euleroutput}
  
                              x
                             x  (log(x) + 1)
  
\end{euleroutput}
\begin{eulerprompt}
>plot2d(f,0,2):
\end{eulerprompt}
\eulerimg{17}{images/Davina Safa Felisa 1-6-153.png}
\begin{eulercomment}
Tentu saja, untuk ekspresi atau ekspresi simbolik, nama variabel sudah
cukup untuk memplotnya.
\end{eulercomment}
\begin{eulerprompt}
>expr &= sin(x)*exp(-x)
\end{eulerprompt}
\begin{euleroutput}
  
                                - x
                               E    sin(x)
  
\end{euleroutput}
\begin{eulerprompt}
>plot2d(expr,0,3pi):
\end{eulerprompt}
\eulerimg{17}{images/Davina Safa Felisa 1-6-154.png}
\begin{eulerprompt}
>function f(x) &= x^x;
>plot2d(f,r=1,cx=1,cy=1,color=blue,thickness=2);
>plot2d(&diff(f(x),x),>add,color=red,style="-.-"):
\end{eulerprompt}
\eulerimg{17}{images/Davina Safa Felisa 1-6-155.png}
\begin{eulercomment}
Untuk gaya garis ada berbagai pilihan.

- gaya="...". Pilih dari "-", "--", "-.", ".", ".-.", "-.-".\\
- warna: Lihat di bawah untuk warna.\\
- ketebalan: Default adalah 1.

Warna dapat dipilih sebagai salah satu warna default, atau sebagai
warna RGB.

- 0.15: indeks warna default.\\
- konstanta warna: putih, hitam, merah, hijau, biru, cyan, zaitun,
abu-abu muda, abu-abu, abu-abu tua, oranye, hijau muda, pirus, biru
muda, oranye terang, kuning\\
- rgb(merah, hijau, biru): parameter adalah real dalam [0,1].
\end{eulercomment}
\begin{eulerprompt}
>plot2d("exp(-x^2)",r=2,color=red,thickness=3,style="--"):
\end{eulerprompt}
\eulerimg{17}{images/Davina Safa Felisa 1-6-156.png}
\begin{eulercomment}
Berikut adalah tampilan warna EMT yang telah ditentukan sebelumnya.
\end{eulercomment}
\begin{eulerprompt}
>aspect(2); columnsplot(ones(1,16),lab=0:15,grid=0,color=0:15):
\end{eulerprompt}
\eulerimg{13}{images/Davina Safa Felisa 1-6-157.png}
\begin{eulercomment}
But you can use any color.
\end{eulercomment}
\begin{eulerprompt}
>columnsplot(ones(1,16),grid=0,color=rgb(0,0,linspace(0,1,15))):
\end{eulerprompt}
\eulerimg{13}{images/Davina Safa Felisa 1-6-158.png}
\eulersubheading{Menggambar Beberapa Kurva pada bidang koordinat yang sama}
\begin{eulercomment}
Plot lebih dari satu fungsi (multiple function) ke dalam satu jendela
dapat dilakukan dengan berbagai cara. Salah satu metode menggunakan
\textgreater{}add untuk beberapa panggilan ke plot2d secara keseluruhan, tetapi
panggilan pertama. Kami telah menggunakan fitur ini dalam contoh di
atas.
\end{eulercomment}
\begin{eulerprompt}
>aspect(); plot2d("cos(x)",r=2,grid=6); plot2d("x",style=".",>add):
\end{eulerprompt}
\eulerimg{27}{images/Davina Safa Felisa 1-6-159.png}
\begin{eulerprompt}
>aspect(1.5); plot2d("sin(x)",0,2pi); plot2d("cos(x)",color=blue,style="--",>add):
\end{eulerprompt}
\eulerimg{17}{images/Davina Safa Felisa 1-6-160.png}
\begin{eulercomment}
Salah satu kegunaan \textgreater{}add adalah untuk menambahkan titik pada kurva.
\end{eulercomment}
\begin{eulerprompt}
>plot2d("sin(x)",0,pi); plot2d(2,sin(2),>points,>add):
\end{eulerprompt}
\eulerimg{17}{images/Davina Safa Felisa 1-6-161.png}
\begin{eulercomment}
Kami menambahkan titik persimpangan dengan label (pada posisi "cl"
untuk kiri tengah), dan memasukkan hasilnya ke dalam notebook. Kami
juga menambahkan judul ke plot.
\end{eulercomment}
\begin{eulerprompt}
>plot2d(["cos(x)","x"],r=1.1,cx=0.5,cy=0.5, ...
>  color=[black,blue],style=["-","."], ...
>  grid=1);
>x0=solve("cos(x)-x",1);  ...
>  plot2d(x0,x0,>points,>add,title="Intersection Demo");  ...
>  label("cos(x) = x",x0,x0,pos="cl",offset=20):
\end{eulerprompt}
\eulerimg{17}{images/Davina Safa Felisa 1-6-162.png}
\begin{eulercomment}
Dalam demo berikut, kami memplot fungsi sinc(x)=sin(x)/x dan ekspansi
Taylor ke-8 dan ke-16. Kami menghitung ekspansi ini menggunakan Maxima
melalui ekspresi simbolis.\\
Plot ini dilakukan dalam perintah multi-baris berikut dengan tiga
panggilan ke plot2d(). Yang kedua dan yang ketiga memiliki set flag
\textgreater{}add, yang membuat plot menggunakan rentang sebelumnya.

Kami menambahkan kotak label yang menjelaskan fungsi.
\end{eulercomment}
\begin{eulerprompt}
>$taylor(sin(x)/x,x,0,4)
\end{eulerprompt}
\begin{eulerformula}
\[
\frac{x^4}{120}-\frac{x^2}{6}+1
\]
\end{eulerformula}
\begin{eulerprompt}
>plot2d("sinc(x)",0,4pi,color=green,thickness=2); ...
>  plot2d(&taylor(sin(x)/x,x,0,8),>add,color=blue,style="--"); ...
>  plot2d(&taylor(sin(x)/x,x,0,16),>add,color=red,style="-.-"); ...
>  labelbox(["sinc","T8","T16"],styles=["-","--","-.-"], ...
>    colors=[black,blue,red]):
\end{eulerprompt}
\eulerimg{17}{images/Davina Safa Felisa 1-6-164.png}
\begin{eulercomment}
Dalam contoh berikut, kami menghasilkan Bernstein-Polinomial.

\end{eulercomment}
\begin{eulerformula}
\[
B_i(x) = \binom{n}{i} x^i (1-x)^{n-i}
\]
\end{eulerformula}
\begin{eulerprompt}
>plot2d("(1-x)^10",0,1); // plot first function
>for i=1 to 10; plot2d("bin(10,i)*x^i*(1-x)^(10-i)",>add); end;
>insimg;
\end{eulerprompt}
\eulerimg{17}{images/Davina Safa Felisa 1-6-166.png}
\begin{eulercomment}
Metode kedua menggunakan pasangan matriks nilai-x dan matriks nilai-y
yang berukuran sama.

Kami menghasilkan matriks nilai dengan satu Polinomial Bernstein di
setiap baris. Untuk ini, kita cukup menggunakan vektor kolom i. Lihat
pengantar tentang bahasa matriks untuk mempelajari lebih detail.
\end{eulercomment}
\begin{eulerprompt}
>x=linspace(0,1,500);
>n=10; k=(0:n)'; // n is row vector, k is column vector
>y=bin(n,k)*x^k*(1-x)^(n-k); // y is a matrix then
>plot2d(x,y):
\end{eulerprompt}
\eulerimg{17}{images/Davina Safa Felisa 1-6-167.png}
\begin{eulercomment}
Perhatikan bahwa parameter warna dapat berupa vektor. Kemudian setiap
warna digunakan untuk setiap baris matriks.
\end{eulercomment}
\begin{eulerprompt}
>x=linspace(0,1,200); y=x^(1:10)'; plot2d(x,y,color=1:10):
\end{eulerprompt}
\eulerimg{17}{images/Davina Safa Felisa 1-6-168.png}
\begin{eulercomment}
Metode lain adalah menggunakan vektor ekspresi (string). Anda kemudian
dapat menggunakan larik warna, larik gaya, dan larik ketebalan dengan
panjang yang sama.
\end{eulercomment}
\begin{eulerprompt}
>plot2d(["sin(x)","cos(x)"],0,2pi,color=4:5): 
\end{eulerprompt}
\eulerimg{17}{images/Davina Safa Felisa 1-6-169.png}
\begin{eulerprompt}
>plot2d(["sin(x)","cos(x)"],0,2pi): // plot vector of expressions
\end{eulerprompt}
\eulerimg{17}{images/Davina Safa Felisa 1-6-170.png}
\begin{eulercomment}
Kita bisa mendapatkan vektor seperti itu dari Maxima menggunakan
makelist() dan mxm2str().
\end{eulercomment}
\begin{eulerprompt}
>v &= makelist(binomial(10,i)*x^i*(1-x)^(10-i),i,0,10) // make list
\end{eulerprompt}
\begin{euleroutput}
  
                  10            9              8  2             7  3
          [(1 - x)  , 10 (1 - x)  x, 45 (1 - x)  x , 120 (1 - x)  x , 
             6  4             5  5             4  6             3  7
  210 (1 - x)  x , 252 (1 - x)  x , 210 (1 - x)  x , 120 (1 - x)  x , 
            2  8              9   10
  45 (1 - x)  x , 10 (1 - x) x , x  ]
  
\end{euleroutput}
\begin{eulerprompt}
>mxm2str(v) // get a vector of strings from the symbolic vector
\end{eulerprompt}
\begin{euleroutput}
  (1-x)^10
  10*(1-x)^9*x
  45*(1-x)^8*x^2
  120*(1-x)^7*x^3
  210*(1-x)^6*x^4
  252*(1-x)^5*x^5
  210*(1-x)^4*x^6
  120*(1-x)^3*x^7
  45*(1-x)^2*x^8
  10*(1-x)*x^9
  x^10
\end{euleroutput}
\begin{eulerprompt}
>plot2d(mxm2str(v),0,1): // plot functions
\end{eulerprompt}
\eulerimg{17}{images/Davina Safa Felisa 1-6-171.png}
\begin{eulercomment}
Alternatif lain adalah dengan menggunakan bahasa matriks Euler.

Jika ekspresi menghasilkan matriks fungsi, dengan satu fungsi di
setiap baris, semua fungsi ini akan diplot ke dalam satu plot.

Untuk ini, gunakan vektor parameter dalam bentuk vektor kolom. Jika
array warna ditambahkan, itu akan digunakan untuk setiap baris plot.
\end{eulercomment}
\begin{eulerprompt}
>n=(1:10)'; plot2d("x^n",0,1,color=1:10):
\end{eulerprompt}
\eulerimg{17}{images/Davina Safa Felisa 1-6-172.png}
\begin{eulercomment}
Ekspresi dan fungsi satu baris dapat melihat variabel global.

Jika Anda tidak dapat menggunakan variabel global, Anda perlu
menggunakan fungsi dengan parameter tambahan, dan meneruskan parameter
ini sebagai parameter titik koma.

Berhati-hatilah, untuk meletakkan semua parameter yang ditetapkan di
akhir perintah plot2d. Dalam contoh kita meneruskan a=5 ke fungsi f,
yang kita plot dari -10 hingga 10.
\end{eulercomment}
\begin{eulerprompt}
>function f(x,a) := 1/a*exp(-x^2/a); ...
>plot2d("f",-10,10;5,thickness=2,title="a=5"):
\end{eulerprompt}
\eulerimg{17}{images/Davina Safa Felisa 1-6-173.png}
\begin{eulercomment}
Atau, gunakan koleksi dengan nama fungsi dan semua parameter tambahan.
Daftar khusus ini disebut koleksi panggilan, dan itu adalah cara yang
lebih disukai untuk meneruskan argumen ke fungsi yang dengan
sendirinya diteruskan sebagai argumen ke fungsi lain.

Dalam contoh berikut, kami menggunakan loop untuk memplot beberapa
fungsi (lihat tutorial tentang pemrograman untuk loop).
\end{eulercomment}
\begin{eulerprompt}
>plot2d(\{\{"f",1\}\},-10,10); ...
>for a=2:10; plot2d(\{\{"f",a\}\},>add); end:
\end{eulerprompt}
\eulerimg{17}{images/Davina Safa Felisa 1-6-174.png}
\begin{eulercomment}
Kami dapat mencapai hasil yang sama dengan cara berikut menggunakan
bahasa matriks EMT. Setiap baris matriks f(x,a) adalah satu fungsi.
Selain itu, kita dapat mengatur warna untuk setiap baris matriks. Klik
dua kali pada fungsi getspectral() untuk penjelasannya.
\end{eulercomment}
\begin{eulerprompt}
>x=-10:0.01:10; a=(1:10)'; plot2d(x,f(x,a),color=getspectral(a/10)):
\end{eulerprompt}
\eulerimg{17}{images/Davina Safa Felisa 1-6-175.png}
\eulersubheading{Label Teks}
\begin{eulercomment}
Dekorasi sederhana bisa

- judul dengan judul="..."\\
- x- dan y-label dengan xl="...", yl="..."\\
- label teks lain dengan label("...",x,y)

Perintah label akan memplot ke dalam plot saat ini pada koordinat plot
(x,y). Itu bisa mengambil argumen posisi.
\end{eulercomment}
\begin{eulerprompt}
>plot2d("x^3-x",-1,2,title="y=x^3-x",yl="y",xl="x"):
\end{eulerprompt}
\eulerimg{17}{images/Davina Safa Felisa 1-6-176.png}
\begin{eulerprompt}
>expr := "log(x)/x"; ...
>  plot2d(expr,0.5,5,title="y="+expr,xl="x",yl="y"); ...
>  label("(1,0)",1,0); label("Max",E,expr(E),pos="lc"):
\end{eulerprompt}
\eulerimg{17}{images/Davina Safa Felisa 1-6-177.png}
\begin{eulercomment}
Ada juga fungsi labelbox(), yang dapat menampilkan fungsi dan teks.
Dibutuhkan vektor string dan warna, satu item untuk setiap fungsi.
\end{eulercomment}
\begin{eulerprompt}
>function f(x) &= x^2*exp(-x^2);  ...
>plot2d(&f(x),a=-3,b=3,c=-1,d=1);  ...
>plot2d(&diff(f(x),x),>add,color=blue,style="--"); ...
>labelbox(["function","derivative"],styles=["-","--"], ...
>   colors=[black,blue],w=0.4):
\end{eulerprompt}
\eulerimg{17}{images/Davina Safa Felisa 1-6-178.png}
\begin{eulercomment}
Kotak ditambatkan di kanan atas secara default, tetapi \textgreater{} kiri
menambatkannya di kiri atas. Anda dapat memindahkannya ke tempat yang
Anda suka. Posisi jangkar adalah sudut kanan atas kotak, dan angkanya
adalah pecahan dari ukuran jendela grafik. Lebarnya otomatis.

Untuk plot titik, kotak label juga berfungsi. Tambahkan parameter
\textgreater{}points, atau vektor flag, satu untuk setiap label.

Dalam contoh berikut, hanya ada satu fungsi. Jadi kita bisa
menggunakan string sebagai pengganti vektor string. Kami mengatur
warna teks menjadi hitam untuk contoh ini.
\end{eulercomment}
\begin{eulerprompt}
>n=10; plot2d(0:n,bin(n,0:n),>addpoints); ...
>labelbox("Binomials",styles="[]",>points,x=0.1,y=0.1, ...
>tcolor=black,>left):
\end{eulerprompt}
\eulerimg{17}{images/Davina Safa Felisa 1-6-179.png}
\begin{eulercomment}
Gaya plot ini juga tersedia di statplot(). Seperti di plot2d() warna
dapat diatur untuk setiap baris plot. Ada lebih banyak plot khusus
untuk keperluan statistik (lihat tutorial tentang statistik).
\end{eulercomment}
\begin{eulerprompt}
>statplot(1:10,random(2,10),color=[red,blue]):
\end{eulerprompt}
\eulerimg{17}{images/Davina Safa Felisa 1-6-180.png}
\begin{eulercomment}
Fitur serupa adalah fungsi textbox().

Lebar secara default adalah lebar maksimal dari baris teks. Tapi itu
bisa diatur oleh pengguna juga.
\end{eulercomment}
\begin{eulerprompt}
>function f(x) &= exp(-x)*sin(2*pi*x); ...
>plot2d("f(x)",0,2pi); ...
>textbox(latex("\(\backslash\)text\{Example of a damped oscillation\}\(\backslash\) f(x)=e^\{-x\}sin(2\(\backslash\)pi x)"),w=0.85):
\end{eulerprompt}
\eulerimg{17}{images/Davina Safa Felisa 1-6-181.png}
\begin{eulercomment}
Label teks, judul, kotak label, dan teks lainnya dapat berisi string
Unicode (lihat sintaks EMT untuk mengetahui lebih lanjut tentang
string Unicode).
\end{eulercomment}
\begin{eulerprompt}
>plot2d("x^3-x",title=u"x &rarr; x&sup3; - x"):
\end{eulerprompt}
\eulerimg{17}{images/Davina Safa Felisa 1-6-182.png}
\begin{eulercomment}
Label pada sumbu x dan y bisa vertikal, begitu juga sumbunya.
\end{eulercomment}
\begin{eulerprompt}
>plot2d("sinc(x)",0,2pi,xl="x",yl=u"x &rarr; sinc(x)",>vertical):
\end{eulerprompt}
\eulerimg{17}{images/Davina Safa Felisa 1-6-183.png}
\eulersubheading{LaTeX}
\begin{eulercomment}
Anda juga dapat memplot rumus LaTeX jika Anda telah menginstal sistem
LaTeX. Saya merekomendasikan MiKTeX. Jalur ke biner "lateks" dan
"dvipng" harus berada di jalur sistem, atau Anda harus mengatur LaTeX
di menu opsi.

Perhatikan, bahwa penguraian LaTeX lambat. Jika Anda ingin menggunakan
LaTeX dalam plot animasi, Anda harus memanggil latex() sebelum loop
sekali dan menggunakan hasilnya (gambar dalam matriks RGB).

Dalam plot berikut, kami menggunakan LaTeX untuk label x dan y, label,
kotak label, dan judul plot.
\end{eulercomment}
\begin{eulerprompt}
>plot2d("exp(-x)*sin(x)/x",a=0,b=2pi,c=0,d=1,grid=6,color=blue, ...
>  title=latex("\(\backslash\)text\{Function $\(\backslash\)Phi$\}"), ...
>  xl=latex("\(\backslash\)phi"),yl=latex("\(\backslash\)Phi(\(\backslash\)phi)")); ...
>textbox( ...
>  latex("\(\backslash\)Phi(\(\backslash\)phi) = e^\{-\(\backslash\)phi\} \(\backslash\)frac\{\(\backslash\)sin(\(\backslash\)phi)\}\{\(\backslash\)phi\}"),x=0.8,y=0.5); ...
>label(latex("\(\backslash\)Phi",color=blue),1,0.4):
\end{eulerprompt}
\eulerimg{17}{images/Davina Safa Felisa 1-6-184.png}
\begin{eulercomment}
Seringkali, kami menginginkan spasi dan label teks non-konformal pada
sumbu x. Kita dapat menggunakan xaxis() dan yaxis() seperti yang akan
kita tunjukkan nanti.

Cara termudah adalah dengan membuat plot kosong dengan bingkai
menggunakan grid=4, lalu menambahkan grid dengan ygrid() dan xgrid().
Dalam contoh berikut, kami menggunakan tiga string LaTeX untuk label
pada sumbu x dengan xtick().
\end{eulercomment}
\begin{eulerprompt}
>plot2d("sinc(x)",0,2pi,grid=4,<ticks); ...
>ygrid(-2:0.5:2,grid=6); ...
>xgrid([0:2]*pi,<ticks,grid=6);  ...
>xtick([0,pi,2pi],["0","\(\backslash\)pi","2\(\backslash\)pi"],>latex):
\end{eulerprompt}
\eulerimg{17}{images/Davina Safa Felisa 1-6-185.png}
\begin{eulercomment}
Tentu saja, fungsi juga dapat digunakan.
\end{eulercomment}
\begin{eulerprompt}
>function map f(x) ...
\end{eulerprompt}
\begin{eulerudf}
  if x>0 then return x^4
  else return x^2
  endif
  endfunction
\end{eulerudf}
\begin{eulercomment}
Parameter "peta" membantu menggunakan fungsi untuk vektor. Untuk\\
plot, itu tidak perlu. Tetapi untuk mendemonstrasikan vektorisasi itu\\
berguna, kami menambahkan beberapa poin kunci ke plot di x=-1, x=0 dan
x=1.

Pada plot berikut, kami juga memasukkan beberapa kode LaTeX. Kami
menggunakannya untuk\\
dua label dan kotak teks. Tentu saja, Anda hanya akan dapat
menggunakan\\
LaTeX jika Anda telah menginstal LaTeX dengan benar.
\end{eulercomment}
\begin{eulerprompt}
>plot2d("f",-1,1,xl="x",yl="f(x)",grid=6);  ...
>plot2d([-1,0,1],f([-1,0,1]),>points,>add); ...
>label(latex("x^3"),0.72,f(0.72)); ...
>label(latex("x^2"),-0.52,f(-0.52),pos="ll"); ...
>textbox( ...
>  latex("f(x)=\(\backslash\)begin\{cases\} x^3 & x>0 \(\backslash\)\(\backslash\) x^2 & x \(\backslash\)le 0\(\backslash\)end\{cases\}"), ...
>  x=0.7,y=0.2):
\end{eulerprompt}
\eulerimg{17}{images/Davina Safa Felisa 1-6-186.png}
\eulersubheading{Interaksi pengguna}
\begin{eulercomment}
Saat memplot fungsi atau ekspresi, parameter \textgreater{}user memungkinkan
pengguna untuk memperbesar dan menggeser plot dengan tombol kursor
atau mouse. Pengguna dapat

- perbesar dengan + atau -\\
- pindahkan plot dengan tombol kursor\\
- pilih jendela plot dengan mouse\\
- atur ulang tampilan dengan spasi\\
- keluar dengan kembali

Tombol spasi akan mengatur ulang plot ke jendela plot asli.

Saat memplot data, flag \textgreater{}user hanya akan menunggu penekanan tombol.
\end{eulercomment}
\begin{eulerprompt}
>plot2d(\{\{"x^3-a*x",a=1\}\},>user,title="Press any key!"):
\end{eulerprompt}
\eulerimg{17}{images/Davina Safa Felisa 1-6-187.png}
\begin{eulerprompt}
>plot2d("exp(x)*sin(x)",user=true, ...
>  title="+/- or cursor keys (return to exit)"):
\end{eulerprompt}
\eulerimg{17}{images/Davina Safa Felisa 1-6-188.png}
\begin{eulercomment}
Berikut ini menunjukkan cara interaksi pengguna tingkat lanjut (lihat
tutorial tentang pemrograman untuk detailnya).

Fungsi bawaan mousedrag() menunggu event mouse atau keyboard. Ini
melaporkan mouse ke bawah, mouse dipindahkan atau mouse ke atas, dan
penekanan tombol. Fungsi dragpoints() memanfaatkan ini, dan
memungkinkan pengguna menyeret titik mana pun dalam plot.

Kita membutuhkan fungsi plot terlebih dahulu. Sebagai contoh, kita
interpolasi dalam 5 titik dengan polinomial. Fungsi harus diplot ke
area plot tetap.
\end{eulercomment}
\begin{eulerprompt}
>function plotf(xp,yp,select) ...
\end{eulerprompt}
\begin{eulerudf}
    d=interp(xp,yp);
    plot2d("interpval(xp,d,x)";d,xp,r=2);
    plot2d(xp,yp,>points,>add);
    if select>0 then
      plot2d(xp[select],yp[select],color=red,>points,>add);
    endif;
    title("Drag one point, or press space or return!");
  endfunction
\end{eulerudf}
\begin{eulercomment}
Perhatikan parameter titik koma di plot2d (d dan xp), yang diteruskan
ke evaluasi fungsi interp(). Tanpa ini, kita harus menulis fungsi
plotinterp() terlebih dahulu, mengakses nilai secara global.

Sekarang kita menghasilkan beberapa nilai acak, dan membiarkan
pengguna menyeret poin.
\end{eulercomment}
\begin{eulerprompt}
>t=-1:0.5:1; dragpoints("plotf",t,random(size(t))-0.5):
\end{eulerprompt}
\eulerimg{17}{images/Davina Safa Felisa 1-6-189.png}
\begin{eulercomment}
Ada juga fungsi, yang memplot fungsi lain tergantung pada vektor
parameter, dan memungkinkan pengguna menyesuaikan parameter ini.

Pertama kita membutuhkan fungsi plot.
\end{eulercomment}
\begin{eulerprompt}
>function plotf([a,b]) := plot2d("exp(a*x)*cos(2pi*b*x)",0,2pi;a,b);
\end{eulerprompt}
\begin{eulercomment}
Kemudian kita membutuhkan nama untuk parameter, nilai awal dan matriks
rentang nx2, opsional baris judul.\\
Ada slider interaktif, yang dapat mengatur nilai oleh pengguna. Fungsi
dragvalues() menyediakan ini.
\end{eulercomment}
\begin{eulerprompt}
>dragvalues("plotf",["a","b"],[-1,2],[[-2,2];[1,10]], ...
>  heading="Drag these values:",hcolor=black):
\end{eulerprompt}
\eulerimg{17}{images/Davina Safa Felisa 1-6-190.png}
\begin{eulercomment}
Dimungkinkan untuk membatasi nilai yang diseret ke bilangan bulat.
Sebagai contoh, kita menulis fungsi plot, yang memplot polinomial
Taylor derajat n ke fungsi kosinus.
\end{eulercomment}
\begin{eulerprompt}
>function plotf(n) ...
\end{eulerprompt}
\begin{eulerudf}
  plot2d("cos(x)",0,2pi,>square,grid=6);
  plot2d(&"taylor(cos(x),x,0,@n)",color=blue,>add);
  textbox("Taylor polynomial of degree "+n,0.1,0.02,style="t",>left);
  endfunction
\end{eulerudf}
\begin{eulercomment}
Sekarang kami mengizinkan derajat n bervariasi dari 0 hingga 20 dalam
20 pemberhentian. Hasil dragvalues() digunakan untuk memplot sketsa
dengan n ini, dan untuk memasukkan plot ke dalam buku catatan.
\end{eulercomment}
\begin{eulerprompt}
>nd=dragvalues("plotf","degree",2,[0,20],20,y=0.8, ...
>   heading="Drag the value:"); ...
>plotf(nd):
\end{eulerprompt}
\eulerimg{17}{images/Davina Safa Felisa 1-6-191.png}
\begin{eulercomment}
Berikut ini adalah demonstrasi sederhana dari fungsi tersebut.
Pengguna dapat menggambar di atas jendela plot, meninggalkan jejak
poin.
\end{eulercomment}
\begin{eulerprompt}
>function dragtest ...
\end{eulerprompt}
\begin{eulerudf}
    plot2d(none,r=1,title="Drag with the mouse, or press any key!");
    start=0;
    repeat
      \{flag,m,time\}=mousedrag();
      if flag==0 then return; endif;
      if flag==2 then
        hold on; mark(m[1],m[2]); hold off;
      endif;
    end
  endfunction
\end{eulerudf}
\begin{eulerprompt}
>dragtest // lihat hasilnya dan cobalah lakukan!
\end{eulerprompt}
\eulersubheading{Gaya Plot 2D}
\begin{eulercomment}
Secara default, EMT menghitung tick sumbu otomatis dan menambahkan
label ke setiap tick. Ini dapat diubah dengan parameter grid. Gaya
default sumbu dan label dapat dimodifikasi. Selain itu, label dan
judul dapat ditambahkan secara manual. Untuk mengatur ulang ke gaya
default, gunakan reset().
\end{eulercomment}
\begin{eulerprompt}
>aspect();
>figure(3,4); ...
> figure(1); plot2d("x^3-x",grid=0); ... // no grid, frame or axis
> figure(2); plot2d("x^3-x",grid=1); ... // x-y-axis
> figure(3); plot2d("x^3-x",grid=2); ... // default ticks
> figure(4); plot2d("x^3-x",grid=3); ... // x-y- axis with labels inside
> figure(5); plot2d("x^3-x",grid=4); ... // no ticks, only labels
> figure(6); plot2d("x^3-x",grid=5); ... // default, but no margin
> figure(7); plot2d("x^3-x",grid=6); ... // axes only
> figure(8); plot2d("x^3-x",grid=7); ... // axes only, ticks at axis
> figure(9); plot2d("x^3-x",grid=8); ... // axes only, finer ticks at axis
> figure(10); plot2d("x^3-x",grid=9); ... // default, small ticks inside
> figure(11); plot2d("x^3-x",grid=10); ...// no ticks, axes only
> figure(0):
\end{eulerprompt}
\eulerimg{27}{images/Davina Safa Felisa 1-6-192.png}
\begin{eulercomment}
Parameter \textless{}frame mematikan frame, dan framecolor=blue mengatur frame
ke warna biru.

Jika Anda ingin centang sendiri, Anda dapat menggunakan style=0, dan
menambahkan semuanya nanti.
\end{eulercomment}
\begin{eulerprompt}
>aspect(1.5); 
>plot2d("x^3-x",grid=0); // plot
>frame; xgrid([-1,0,1]); ygrid(0): // add frame and grid
\end{eulerprompt}
\eulerimg{17}{images/Davina Safa Felisa 1-6-193.png}
\begin{eulercomment}
Untuk judul plot dan label sumbu, lihat contoh berikut.
\end{eulercomment}
\begin{eulerprompt}
>plot2d("exp(x)",-1,1);
>textcolor(black); // set the text color to black
>title(latex("y=e^x")); // title above the plot
>xlabel(latex("x")); // "x" for x-axis
>ylabel(latex("y"),>vertical); // vertical "y" for y-axis
>label(latex("(0,1)"),0,1,color=blue): // label a point
\end{eulerprompt}
\eulerimg{17}{images/Davina Safa Felisa 1-6-194.png}
\begin{eulercomment}
Sumbu dapat digambar secara terpisah dengan xaxis() dan yaxis().
\end{eulercomment}
\begin{eulerprompt}
>plot2d("x^3-x",<grid,<frame);
>xaxis(0,xx=-2:1,style="->"); yaxis(0,yy=-5:5,style="->"):
\end{eulerprompt}
\eulerimg{17}{images/Davina Safa Felisa 1-6-195.png}
\begin{eulercomment}
Teks pada plot dapat diatur dengan label(). Dalam contoh berikut, "lc"
berarti tengah bawah. Ini mengatur posisi label relatif terhadap
koordinat plot.
\end{eulercomment}
\begin{eulerprompt}
>function f(x) &= x^3-x
\end{eulerprompt}
\begin{euleroutput}
  
                                   3
                                  x  - x
  
\end{euleroutput}
\begin{eulerprompt}
>plot2d(f,-1,1,>square);
>x0=fmin(f,0,1); // compute point of minimum
>label("Rel. Min.",x0,f(x0),pos="lc"): // add a label there
\end{eulerprompt}
\eulerimg{17}{images/Davina Safa Felisa 1-6-196.png}
\begin{eulercomment}
Ada juga kotak teks.
\end{eulercomment}
\begin{eulerprompt}
>plot2d(&f(x),-1,1,-2,2); // function
>plot2d(&diff(f(x),x),>add,style="--",color=red); // derivative
>labelbox(["f","f'"],["-","--"],[black,red]): // label box
\end{eulerprompt}
\eulerimg{17}{images/Davina Safa Felisa 1-6-197.png}
\begin{eulerprompt}
>plot2d(["exp(x)","1+x"],color=[black,blue],style=["-","-.-"]):
\end{eulerprompt}
\eulerimg{17}{images/Davina Safa Felisa 1-6-198.png}
\begin{eulerprompt}
>gridstyle("->",color=gray,textcolor=gray,framecolor=gray);  ...
> plot2d("x^3-x",grid=1);   ...
> settitle("y=x^3-x",color=black); ...
> label("x",2,0,pos="bc",color=gray);  ...
> label("y",0,6,pos="cl",color=gray); ...
> reset():
\end{eulerprompt}
\eulerimg{27}{images/Davina Safa Felisa 1-6-199.png}
\begin{eulercomment}
Untuk kontrol lebih, sumbu x dan sumbu y dapat dilakukan secara
manual.

Perintah fullwindow() memperluas jendela plot karena kita tidak lagi
membutuhkan tempat untuk label di luar jendela plot. Gunakan
shrinkwindow() atau reset() untuk mengatur ulang ke default.
\end{eulercomment}
\begin{eulerprompt}
>fullwindow; ...
> gridstyle(color=darkgray,textcolor=darkgray); ...
> plot2d(["2^x","1","2^(-x)"],a=-2,b=2,c=0,d=4,<grid,color=4:6,<frame); ...
> xaxis(0,-2:1,style="->"); xaxis(0,2,"x",<axis); ...
> yaxis(0,4,"y",style="->"); ...
> yaxis(-2,1:4,>left); ...
> yaxis(2,2^(-2:2),style=".",<left); ...
> labelbox(["2^x","1","2^-x"],colors=4:6,x=0.8,y=0.2); ...
> reset:
\end{eulerprompt}
\eulerimg{27}{images/Davina Safa Felisa 1-6-200.png}
\begin{eulercomment}
Berikut adalah contoh lain, di mana string Unicode digunakan dan sumbu
di luar area plot.
\end{eulercomment}
\begin{eulerprompt}
>aspect(1.5); 
>plot2d(["sin(x)","cos(x)"],0,2pi,color=[red,green],<grid,<frame); ...
> xaxis(-1.1,(0:2)*pi,xt=["0",u"&pi;",u"2&pi;"],style="-",>ticks,>zero);  ...
> xgrid((0:0.5:2)*pi,<ticks); ...
> yaxis(-0.1*pi,-1:0.2:1,style="-",>zero,>grid); ...
> labelbox(["sin","cos"],colors=[red,green],x=0.5,y=0.2,>left); ...
> xlabel(u"&phi;"); ylabel(u"f(&phi;)"):
\end{eulerprompt}
\eulerimg{17}{images/Davina Safa Felisa 1-6-201.png}
\eulerheading{Merencanakan Data 2D}
\begin{eulercomment}
Jika x dan y adalah vektor data, data ini akan digunakan sebagai
koordinat x dan y dari suatu kurva. Dalam hal ini, a, b, c, dan d,
atau radius r dapat ditentukan, atau jendela plot akan menyesuaikan
secara otomatis dengan data. Atau, \textgreater{}persegi dapat diatur untuk menjaga
rasio aspek persegi.

Memplot ekspresi hanyalah singkatan untuk plot data. Untuk plot data,
Anda memerlukan satu atau beberapa baris nilai x, dan satu atau
beberapa baris nilai y. Dari rentang dan nilai-x, fungsi plot2d akan
menghitung data yang akan diplot, secara default dengan evaluasi
fungsi yang adaptif. Untuk plot titik gunakan "\textgreater{}titik", untuk garis
campuran dan titik gunakan "\textgreater{}tambahan".

Tapi Anda bisa memasukkan data secara langsung.

- Gunakan vektor baris untuk x dan y untuk satu fungsi.\\
- Matriks untuk x dan y diplot baris demi baris.

Berikut adalah contoh dengan satu baris untuk x dan y.
\end{eulercomment}
\begin{eulerprompt}
>x=-10:0.1:10; y=exp(-x^2)*x; plot2d(x,y):
\end{eulerprompt}
\eulerimg{17}{images/Davina Safa Felisa 1-6-202.png}
\begin{eulercomment}
Data juga dapat diplot sebagai titik. Gunakan poin=true untuk ini.
Plotnya bekerja seperti poligon, tetapi hanya menggambar
sudut-sudutnya.

- style="...": Pilih dari "[]", "\textless{}\textgreater{}", "o", ".", "..", "+", "*", "[]#",
"\textless{} \textgreater{}#", "o#", "..#", "#", "\textbar{}".

Untuk memplot set poin gunakan \textgreater{}points. Jika warna adalah vektor
warna, setiap titik\\
mendapat warna yang berbeda. Untuk matriks koordinat dan vektor kolom,
warna berlaku untuk baris matriks.\\
Parameter \textgreater{}addpoints menambahkan titik ke segmen garis untuk plot
data.
\end{eulercomment}
\begin{eulerprompt}
>xdata=[1,1.5,2.5,3,4]; ydata=[3,3.1,2.8,2.9,2.7]; // data
>plot2d(xdata,ydata,a=0.5,b=4.5,c=2.5,d=3.5,style="."); // lines
>plot2d(xdata,ydata,>points,>add,style="o"): // add points
\end{eulerprompt}
\eulerimg{17}{images/Davina Safa Felisa 1-6-203.png}
\begin{eulerprompt}
>p=polyfit(xdata,ydata,1); // get regression line
>plot2d("polyval(p,x)",>add,color=red): // add plot of line
\end{eulerprompt}
\eulerimg{17}{images/Davina Safa Felisa 1-6-204.png}
\eulerheading{Menggambar Daerah Yang Dibatasi Kurva}
\begin{eulercomment}
Plot data benar-benar poligon. Kita juga dapat memplot kurva atau
kurva terisi.

- terisi=benar mengisi plot.\\
- style="...": Pilih dari "#", "/", "\textbackslash{}", "\textbackslash{}/".\\
- fillcolor: Lihat di atas untuk warna yang tersedia.

Warna isian ditentukan oleh argumen "fillcolor", dan pada \textless{}outline
opsional mencegah menggambar batas untuk semua gaya kecuali yang
default.
\end{eulercomment}
\begin{eulerprompt}
>t=linspace(0,2pi,1000); // parameter for curve
>x=sin(t)*exp(t/pi); y=cos(t)*exp(t/pi); // x(t) and y(t)
>figure(1,2); aspect(16/9)
>figure(1); plot2d(x,y,r=10); // plot curve
>figure(2); plot2d(x,y,r=10,>filled,style="/",fillcolor=red); // fill curve
>figure(0):
\end{eulerprompt}
\eulerimg{14}{images/Davina Safa Felisa 1-6-205.png}
\begin{eulercomment}
Dalam contoh berikut kami memplot elips terisi dan dua segi enam
terisi menggunakan kurva tertutup dengan 6 titik dengan gaya isian
berbeda.
\end{eulercomment}
\begin{eulerprompt}
>x=linspace(0,2pi,1000); plot2d(sin(x),cos(x)*0.5,r=1,>filled,style="/"):
\end{eulerprompt}
\eulerimg{14}{images/Davina Safa Felisa 1-6-206.png}
\begin{eulerprompt}
>t=linspace(0,2pi,6); ...
>plot2d(cos(t),sin(t),>filled,style="/",fillcolor=red,r=1.2):
\end{eulerprompt}
\eulerimg{14}{images/Davina Safa Felisa 1-6-207.png}
\begin{eulerprompt}
>t=linspace(0,2pi,6); plot2d(cos(t),sin(t),>filled,style="#"):
\end{eulerprompt}
\eulerimg{14}{images/Davina Safa Felisa 1-6-208.png}
\begin{eulercomment}
Contoh lainnya adalah segi empat, yang kita buat dengan 7 titik pada
lingkaran satuan.
\end{eulercomment}
\begin{eulerprompt}
>t=linspace(0,2pi,7);  ...
> plot2d(cos(t),sin(t),r=1,>filled,style="/",fillcolor=red):
\end{eulerprompt}
\eulerimg{14}{images/Davina Safa Felisa 1-6-209.png}
\begin{eulercomment}
Berikut ini adalah himpunan nilai maksimal dari empat kondisi linier
yang kurang dari atau sama dengan 3. Ini adalah A[k].v\textless{}=3 untuk semua
baris A. Untuk mendapatkan sudut yang bagus, kita menggunakan n yang
relatif besar.
\end{eulercomment}
\begin{eulerprompt}
>A=[2,1;1,2;-1,0;0,-1];
>function f(x,y) := max([x,y].A');
>plot2d("f",r=4,level=[0;3],color=green,n=111):
\end{eulerprompt}
\eulerimg{14}{images/Davina Safa Felisa 1-6-210.png}
\begin{eulercomment}
Poin utama dari bahasa matriks adalah memungkinkan untuk menghasilkan
tabel fungsi dengan mudah.
\end{eulercomment}
\begin{eulerprompt}
>t=linspace(0,2pi,1000); x=cos(3*t); y=sin(4*t);
\end{eulerprompt}
\begin{eulercomment}
Kami sekarang memiliki vektor x dan y nilai. plot2d() dapat memplot
nilai-nilai ini\\
sebagai kurva yang menghubungkan titik-titik. Plotnya bisa diisi. Pada
kasus ini\\
ini menghasilkan hasil yang bagus karena aturan lilitan, yang
digunakan untuk\\
isi.
\end{eulercomment}
\begin{eulerprompt}
>plot2d(x,y,<grid,<frame,>filled):
\end{eulerprompt}
\eulerimg{14}{images/Davina Safa Felisa 1-6-211.png}
\begin{eulercomment}
Sebuah vektor interval diplot terhadap nilai x sebagai daerah terisi\\
antara nilai interval bawah dan atas.

Hal ini dapat berguna untuk memplot kesalahan perhitungan. Tapi itu
bisa\\
juga digunakan untuk memplot kesalahan statistik.
\end{eulercomment}
\begin{eulerprompt}
>t=0:0.1:1; ...
> plot2d(t,interval(t-random(size(t)),t+random(size(t))),style="|");  ...
> plot2d(t,t,add=true):
\end{eulerprompt}
\eulerimg{14}{images/Davina Safa Felisa 1-6-212.png}
\begin{eulercomment}
Jika x adalah vektor yang diurutkan, dan y adalah vektor interval,
maka plot2d akan memplot rentang interval yang terisi dalam bidang.
Gaya isian sama dengan gaya poligon.
\end{eulercomment}
\begin{eulerprompt}
>t=-1:0.01:1; x=~t-0.01,t+0.01~; y=x^3-x;
>plot2d(t,y):
\end{eulerprompt}
\eulerimg{14}{images/Davina Safa Felisa 1-6-213.png}
\begin{eulercomment}
Jika x adalah vektor yang diurutkan, dan y adalah vektor interval,
maka plot2d akan memplot rentang interval yang terisi dalam bidang.
Gaya isian sama dengan gaya poligon.
\end{eulercomment}
\begin{eulerprompt}
>expr := "2*x^2+x*y+3*y^4+y"; // define an expression f(x,y)
>plot2d(expr,level=[0;1],style="-",color=blue): // 0 <= f(x,y) <= 1
\end{eulerprompt}
\eulerimg{14}{images/Davina Safa Felisa 1-6-214.png}
\begin{eulercomment}
Kami juga dapat mengisi rentang nilai seperti

\end{eulercomment}
\begin{eulerformula}
\[
-1 \le (x^2+y^2)^2-x^2+y^2 \le 0.
\]
\end{eulerformula}
\begin{eulerprompt}
>plot2d("(x^2+y^2)^2-x^2+y^2",r=1.2,level=[-1;0],style="/"):
\end{eulerprompt}
\eulerimg{14}{images/Davina Safa Felisa 1-6-216.png}
\begin{eulerprompt}
>plot2d("cos(x)","sin(x)^3",xmin=0,xmax=2pi,>filled,style="/"):
\end{eulerprompt}
\eulerimg{14}{images/Davina Safa Felisa 1-6-217.png}
\eulerheading{Grafik Fungsi Parametrik}
\begin{eulercomment}
Nilai-x tidak perlu diurutkan. (x,y) hanya menggambarkan kurva. Jika x
diurutkan, kurva tersebut merupakan grafik fungsi.

Dalam contoh berikut, kami memplot spiral

\end{eulercomment}
\begin{eulerformula}
\[
\gamma(t) = t \cdot (\cos(2\pi t),\sin(2\pi t))
\]
\end{eulerformula}
\begin{eulercomment}
Kita perlu menggunakan banyak titik untuk tampilan yang halus atau
fungsi adaptif() untuk mengevaluasi ekspresi (lihat fungsi adaptif()
untuk lebih jelasnya).
\end{eulercomment}
\begin{eulerprompt}
>t=linspace(0,1,1000); ...
>plot2d(t*cos(2*pi*t),t*sin(2*pi*t),r=1):
\end{eulerprompt}
\eulerimg{14}{images/Davina Safa Felisa 1-6-219.png}
\begin{eulercomment}
Atau, dimungkinkan untuk menggunakan dua ekspresi untuk kurva. Berikut
ini plot kurva yang sama seperti di atas.
\end{eulercomment}
\begin{eulerprompt}
>plot2d("x*cos(2*pi*x)","x*sin(2*pi*x)",xmin=0,xmax=1,r=1):
\end{eulerprompt}
\eulerimg{14}{images/Davina Safa Felisa 1-6-220.png}
\begin{eulerprompt}
>t=linspace(0,1,1000); r=exp(-t); x=r*cos(2pi*t); y=r*sin(2pi*t);
>plot2d(x,y,r=1):
\end{eulerprompt}
\eulerimg{14}{images/Davina Safa Felisa 1-6-221.png}
\begin{eulercomment}
Dalam contoh berikutnya, kami memplot kurva

\end{eulercomment}
\begin{eulerformula}
\[
\gamma(t) = (r(t) \cos(t), r(t) \sin(t))
\]
\end{eulerformula}
\begin{eulercomment}
dengan

\end{eulercomment}
\begin{eulerformula}
\[
r(t) = 1 + \dfrac{\sin(3t)}{2}.
\]
\end{eulerformula}
\begin{eulerprompt}
>t=linspace(0,2pi,1000); r=1+sin(3*t)/2; x=r*cos(t); y=r*sin(t); ...
>plot2d(x,y,>filled,fillcolor=red,style="/",r=1.5):
\end{eulerprompt}
\eulerimg{14}{images/Davina Safa Felisa 1-6-224.png}
\eulerheading{Menggambar Grafik Bilangan Kompleks}
\begin{eulercomment}
Array bilangan kompleks juga dapat diplot. Kemudian titik-titik grid
akan terhubung. Jika sejumlah garis kisi ditentukan (atau vektor garis
kisi 1x2) dalam argumen cgrid, hanya garis kisi tersebut yang
terlihat.

Matriks bilangan kompleks akan secara otomatis diplot sebagai kisi di
bidang kompleks.

Dalam contoh berikut, kami memplot gambar lingkaran satuan di bawah
fungsi eksponensial. Parameter cgrid menyembunyikan beberapa kurva
grid.
\end{eulercomment}
\begin{eulerprompt}
>aspect(); r=linspace(0,1,50); a=linspace(0,2pi,80)'; z=r*exp(I*a);...
>plot2d(z,a=-1.25,b=1.25,c=-1.25,d=1.25,cgrid=10):
\end{eulerprompt}
\eulerimg{27}{images/Davina Safa Felisa 1-6-225.png}
\begin{eulerprompt}
>aspect(1.25); r=linspace(0,1,50); a=linspace(0,2pi,200)'; z=r*exp(I*a);
>plot2d(exp(z),cgrid=[40,10]):
\end{eulerprompt}
\eulerimg{21}{images/Davina Safa Felisa 1-6-226.png}
\begin{eulerprompt}
>r=linspace(0,1,10); a=linspace(0,2pi,40)'; z=r*exp(I*a);
>plot2d(exp(z),>points,>add):
\end{eulerprompt}
\eulerimg{21}{images/Davina Safa Felisa 1-6-227.png}
\begin{eulercomment}
Sebuah vektor bilangan kompleks secara otomatis diplot sebagai kurva
pada bidang kompleks dengan bagian real dan bagian imajiner.

Dalam contoh, kami memplot lingkaran satuan dengan

\end{eulercomment}
\begin{eulerformula}
\[
\gamma(t) = e^{it}
\]
\end{eulerformula}
\begin{eulerprompt}
>t=linspace(0,2pi,1000); ...
>plot2d(exp(I*t)+exp(4*I*t),r=2):
\end{eulerprompt}
\eulerimg{21}{images/Davina Safa Felisa 1-6-229.png}
\eulerheading{Plot Statistik}
\begin{eulercomment}
Ada banyak fungsi yang dikhususkan pada plot statistik. Salah satu
plot yang sering digunakan adalah plot kolom.

Jumlah kumulatif dari nilai terdistribusi 0-1-normal menghasilkan
jalan acak.
\end{eulercomment}
\begin{eulerprompt}
>plot2d(cumsum(randnormal(1,1000))):
\end{eulerprompt}
\eulerimg{21}{images/Davina Safa Felisa 1-6-230.png}
\begin{eulercomment}
Menggunakan dua baris menunjukkan jalan dalam dua dimensi.
\end{eulercomment}
\begin{eulerprompt}
>X=cumsum(randnormal(2,1000)); plot2d(X[1],X[2]):
\end{eulerprompt}
\eulerimg{21}{images/Davina Safa Felisa 1-6-231.png}
\begin{eulerprompt}
>columnsplot(cumsum(random(10)),style="/",color=blue):
\end{eulerprompt}
\eulerimg{21}{images/Davina Safa Felisa 1-6-232.png}
\begin{eulercomment}
Itu juga dapat menampilkan string sebagai label.
\end{eulercomment}
\begin{eulerprompt}
>months=["Jan","Feb","Mar","Apr","May","Jun", ...
>  "Jul","Aug","Sep","Oct","Nov","Dec"];
>values=[10,12,12,18,22,28,30,26,22,18,12,8];
>columnsplot(values,lab=months,color=red,style="-");
>title("Temperature"):
\end{eulerprompt}
\eulerimg{21}{images/Davina Safa Felisa 1-6-233.png}
\begin{eulerprompt}
>k=0:10;
>plot2d(k,bin(10,k),>bar):
\end{eulerprompt}
\eulerimg{21}{images/Davina Safa Felisa 1-6-234.png}
\begin{eulerprompt}
>plot2d(k,bin(10,k)); plot2d(k,bin(10,k),>points,>add):
\end{eulerprompt}
\eulerimg{21}{images/Davina Safa Felisa 1-6-235.png}
\begin{eulerprompt}
>plot2d(normal(1000),normal(1000),>points,grid=6,style=".."):
\end{eulerprompt}
\eulerimg{21}{images/Davina Safa Felisa 1-6-236.png}
\begin{eulerprompt}
>plot2d(normal(1,1000),>distribution,style="O"):
\end{eulerprompt}
\eulerimg{21}{images/Davina Safa Felisa 1-6-237.png}
\begin{eulerprompt}
>plot2d("qnormal",0,5;2.5,0.5,>filled):
\end{eulerprompt}
\eulerimg{21}{images/Davina Safa Felisa 1-6-238.png}
\begin{eulercomment}
Untuk memplot distribusi statistik eksperimental, Anda dapat
menggunakan distribution=n dengan plot2d.
\end{eulercomment}
\begin{eulerprompt}
>w=randexponential(1,1000); // exponential distribution
>plot2d(w,>distribution): // or distribution=n with n intervals
\end{eulerprompt}
\eulerimg{21}{images/Davina Safa Felisa 1-6-239.png}
\begin{eulercomment}
Atau Anda dapat menghitung distribusi dari data dan memplot hasilnya
dengan \textgreater{}bar di plot3d, atau dengan plot kolom.
\end{eulercomment}
\begin{eulerprompt}
>w=normal(1000); // 0-1-normal distribution
>\{x,y\}=histo(w,10,v=[-6,-4,-2,-1,0,1,2,4,6]); // interval bounds v
>plot2d(x,y,>bar):
\end{eulerprompt}
\eulerimg{21}{images/Davina Safa Felisa 1-6-240.png}
\begin{eulercomment}
Fungsi statplot() menyetel gaya dengan string sederhana.
\end{eulercomment}
\begin{eulerprompt}
>statplot(1:10,cumsum(random(10)),"b"):
\end{eulerprompt}
\eulerimg{21}{images/Davina Safa Felisa 1-6-241.png}
\begin{eulerprompt}
>n=10; i=0:n; ...
>plot2d(i,bin(n,i)/2^n,a=0,b=10,c=0,d=0.3); ...
>plot2d(i,bin(n,i)/2^n,points=true,style="ow",add=true,color=blue):
\end{eulerprompt}
\eulerimg{21}{images/Davina Safa Felisa 1-6-242.png}
\begin{eulercomment}
Selain itu, data dapat diplot sebagai batang. Dalam hal ini, x harus
diurutkan dan satu elemen lebih panjang dari y. Bilah akan memanjang
dari x[i] ke x[i+1] dengan nilai y[i]. Jika x memiliki ukuran yang
sama dengan y, maka akan diperpanjang satu elemen dengan spasi
terakhir.

Gaya isian dapat digunakan seperti di atas.
\end{eulercomment}
\begin{eulerprompt}
>n=10; k=bin(n,0:n); ...
>plot2d(-0.5:n+0.5,k,bar=true,fillcolor=lightgray):
\end{eulerprompt}
\eulerimg{21}{images/Davina Safa Felisa 1-6-243.png}
\begin{eulercomment}
Data untuk plot batang (bar=1) dan histogram (histogram=1) dapat
dinyatakan secara eksplisit dalam xv dan yv, atau dapat dihitung dari
distribusi empiris dalam xv dengan \textgreater{}distribusi (atau distribusi=n).
Histogram nilai xv akan dihitung secara otomatis dengan \textgreater{}histogram.
Jika \textgreater{}genap ditentukan, nilai xv akan dihitung dalam interval bilangan
bulat.
\end{eulercomment}
\begin{eulerprompt}
>plot2d(normal(10000),distribution=50):
\end{eulerprompt}
\eulerimg{21}{images/Davina Safa Felisa 1-6-244.png}
\begin{eulerprompt}
>k=0:10; m=bin(10,k); x=(0:11)-0.5; plot2d(x,m,>bar):
\end{eulerprompt}
\eulerimg{21}{images/Davina Safa Felisa 1-6-245.png}
\begin{eulerprompt}
>columnsplot(m,k):
\end{eulerprompt}
\eulerimg{21}{images/Davina Safa Felisa 1-6-246.png}
\begin{eulerprompt}
>plot2d(random(600)*6,histogram=6):
\end{eulerprompt}
\eulerimg{21}{images/Davina Safa Felisa 1-6-247.png}
\begin{eulercomment}
Untuk distribusi, ada parameter distribution=n, yang menghitung nilai
secara otomatis dan mencetak distribusi relatif dengan n sub-interval.
\end{eulercomment}
\begin{eulerprompt}
>plot2d(normal(1,1000),distribution=10,style="\(\backslash\)/"):
\end{eulerprompt}
\eulerimg{21}{images/Davina Safa Felisa 1-6-248.png}
\begin{eulercomment}
Dengan parameter even=true, ini akan menggunakan interval integer.
\end{eulercomment}
\begin{eulerprompt}
>plot2d(intrandom(1,1000,10),distribution=10,even=true):
\end{eulerprompt}
\eulerimg{21}{images/Davina Safa Felisa 1-6-249.png}
\begin{eulercomment}
Perhatikan bahwa ada banyak plot statistik, yang mungkin berguna.
Silahkan lihat tutorial tentang statistik.
\end{eulercomment}
\begin{eulerprompt}
>columnsplot(getmultiplicities(1:6,intrandom(1,6000,6))):
\end{eulerprompt}
\eulerimg{21}{images/Davina Safa Felisa 1-6-250.png}
\begin{eulerprompt}
>plot2d(normal(1,1000),>distribution); ...
>  plot2d("qnormal(x)",color=red,thickness=2,>add):
\end{eulerprompt}
\eulerimg{21}{images/Davina Safa Felisa 1-6-251.png}
\begin{eulercomment}
Ada juga banyak plot khusus untuk statistik. Boxplot menunjukkan
kuartil dari distribusi ini dan banyak outlier. Menurut definisi,
outlier dalam boxplot adalah data yang melebihi 1,5 kali kisaran 50\%
tengah plot.
\end{eulercomment}
\begin{eulerprompt}
>M=normal(5,1000); boxplot(quartiles(M)):
\end{eulerprompt}
\eulerimg{21}{images/Davina Safa Felisa 1-6-252.png}
\eulerheading{Fungsi Implisit}
\begin{eulercomment}
Plot implisit menunjukkan garis level yang menyelesaikan f(x,y)=level,
di mana "level" dapat berupa nilai tunggal atau vektor nilai. Jika
level="auto", akan ada garis level nc, yang akan menyebar antara
fungsi minimum dan maksimum secara merata. Warna yang lebih gelap atau
lebih terang dapat ditambahkan dengan \textgreater{}hue untuk menunjukkan nilai
fungsi. Untuk fungsi implisit, xv harus berupa fungsi atau ekspresi
dari parameter x dan y, atau, sebagai alternatif, xv dapat berupa
matriks nilai.

Euler dapat menandai garis level

\end{eulercomment}
\begin{eulerformula}
\[
f(x,y) = c
\]
\end{eulerformula}
\begin{eulercomment}
dari fungsi apapun.

Untuk menggambar himpunan f(x,y)=c untuk satu atau lebih konstanta c,
Anda dapat menggunakan plot2d() dengan plot implisitnya di dalam
bidang. Parameter untuk c adalah level=c, di mana c dapat berupa
vektor garis level. Selain itu, skema warna dapat digambar di latar
belakang untuk menunjukkan nilai fungsi untuk setiap titik dalam plot.
Parameter "n" menentukan kehalusan plot.
\end{eulercomment}
\begin{eulerprompt}
>aspect(1.5); 
>plot2d("x^2+y^2-x*y-x",r=1.5,level=0,contourcolor=red):
\end{eulerprompt}
\eulerimg{17}{images/Davina Safa Felisa 1-6-254.png}
\begin{eulerprompt}
>expr := "2*x^2+x*y+3*y^4+y"; // define an expression f(x,y)
>plot2d(expr,level=0): // Solutions of f(x,y)=0
\end{eulerprompt}
\eulerimg{17}{images/Davina Safa Felisa 1-6-255.png}
\begin{eulerprompt}
>plot2d(expr,level=0:0.5:20,>hue,contourcolor=white,n=200): // nice
\end{eulerprompt}
\eulerimg{17}{images/Davina Safa Felisa 1-6-256.png}
\begin{eulerprompt}
>plot2d(expr,level=0:0.5:20,>hue,>spectral,n=200,grid=4): // nicer
\end{eulerprompt}
\eulerimg{17}{images/Davina Safa Felisa 1-6-257.png}
\begin{eulercomment}
Ini berfungsi untuk plot data juga. Tetapi Anda harus menentukan
rentangnya\\
untuk label sumbu.
\end{eulercomment}
\begin{eulerprompt}
>x=-2:0.05:1; y=x'; z=expr(x,y);
>plot2d(z,level=0,a=-1,b=2,c=-2,d=1,>hue):
\end{eulerprompt}
\eulerimg{17}{images/Davina Safa Felisa 1-6-258.png}
\begin{eulerprompt}
>plot2d("x^3-y^2",>contour,>hue,>spectral):
\end{eulerprompt}
\eulerimg{17}{images/Davina Safa Felisa 1-6-259.png}
\begin{eulerprompt}
>plot2d("x^3-y^2",level=0,contourwidth=3,>add,contourcolor=red):
\end{eulerprompt}
\eulerimg{17}{images/Davina Safa Felisa 1-6-260.png}
\begin{eulerprompt}
>z=z+normal(size(z))*0.2;
>plot2d(z,level=0.5,a=-1,b=2,c=-2,d=1):
\end{eulerprompt}
\eulerimg{17}{images/Davina Safa Felisa 1-6-261.png}
\begin{eulerprompt}
>plot2d(expr,level=[0:0.2:5;0.05:0.2:5.05],color=lightgray):
\end{eulerprompt}
\eulerimg{17}{images/Davina Safa Felisa 1-6-262.png}
\begin{eulerprompt}
>plot2d("x^2+y^3+x*y",level=1,r=4,n=100):
\end{eulerprompt}
\eulerimg{17}{images/Davina Safa Felisa 1-6-263.png}
\begin{eulerprompt}
>plot2d("x^2+2*y^2-x*y",level=0:0.1:10,n=100,contourcolor=white,>hue):
\end{eulerprompt}
\eulerimg{17}{images/Davina Safa Felisa 1-6-264.png}
\begin{eulercomment}
Juga dimungkinkan untuk mengisi set

\end{eulercomment}
\begin{eulerformula}
\[
a \le f(x,y) \le b
\]
\end{eulerformula}
\begin{eulercomment}
dengan rentang tingkat.

Dimungkinkan untuk mengisi wilayah nilai untuk fungsi tertentu. Untuk
ini, level harus berupa matriks 2xn. Baris pertama adalah batas bawah
dan baris kedua berisi batas atas.
\end{eulercomment}
\begin{eulerprompt}
>plot2d(expr,level=[0;1],style="-",color=blue): // 0 <= f(x,y) <= 1
\end{eulerprompt}
\eulerimg{17}{images/Davina Safa Felisa 1-6-266.png}
\begin{eulercomment}
Plot implisit juga dapat menunjukkan rentang level. Kemudian level
harus berupa matriks 2xn dari interval level, di mana baris pertama
berisi awal dan baris kedua adalah akhir dari setiap interval. Atau,
vektor baris sederhana dapat digunakan untuk level, dan parameter dl
memperluas nilai level ke interval.
\end{eulercomment}
\begin{eulerprompt}
>plot2d("x^4+y^4",r=1.5,level=[0;1],color=blue,style="/"):
\end{eulerprompt}
\eulerimg{17}{images/Davina Safa Felisa 1-6-267.png}
\begin{eulerprompt}
>plot2d("x^2+y^3+x*y",level=[0,2,4;1,3,5],style="/",r=2,n=100):
\end{eulerprompt}
\eulerimg{17}{images/Davina Safa Felisa 1-6-268.png}
\begin{eulerprompt}
>plot2d("x^2+y^3+x*y",level=-10:20,r=2,style="-",dl=0.1,n=100):
\end{eulerprompt}
\eulerimg{17}{images/Davina Safa Felisa 1-6-269.png}
\begin{eulerprompt}
>plot2d("sin(x)*cos(y)",r=pi,>hue,>levels,n=100):
\end{eulerprompt}
\eulerimg{17}{images/Davina Safa Felisa 1-6-270.png}
\begin{eulercomment}
Dimungkinkan juga untuk menandai suatu wilayah

\end{eulercomment}
\begin{eulerformula}
\[
a \le f(x,y) \le b.
\]
\end{eulerformula}
\begin{eulercomment}
Ini dilakukan dengan menambahkan level dengan dua baris.
\end{eulercomment}
\begin{eulerprompt}
>plot2d("(x^2+y^2-1)^3-x^2*y^3",r=1.3, ...
>  style="#",color=red,<outline, ...
>  level=[-2;0],n=100):
\end{eulerprompt}
\eulerimg{17}{images/Davina Safa Felisa 1-6-272.png}
\begin{eulercomment}
Dimungkinkan untuk menentukan level tertentu. Misalnya, kita dapat
memplot solusi persamaan seperti

\end{eulercomment}
\begin{eulerformula}
\[
x^3-xy+x^2y^2=6
\]
\end{eulerformula}
\begin{eulerprompt}
>plot2d("x^3-x*y+x^2*y^2",r=6,level=1,n=100):
\end{eulerprompt}
\eulerimg{17}{images/Davina Safa Felisa 1-6-274.png}
\begin{eulerprompt}
>function starplot1 (v, style="/", color=green, lab=none) ...
\end{eulerprompt}
\begin{eulerudf}
    if !holding() then clg; endif;
    w=window(); window(0,0,1024,1024);
    h=holding(1);
    r=max(abs(v))*1.2;
    setplot(-r,r,-r,r);
    n=cols(v); t=linspace(0,2pi,n);
    v=v|v[1]; c=v*cos(t); s=v*sin(t);
    cl=barcolor(color); st=barstyle(style);
    loop 1 to n
      polygon([0,c[#],c[#+1]],[0,s[#],s[#+1]],1);
      if lab!=none then
        rlab=v[#]+r*0.1;
        \{col,row\}=toscreen(cos(t[#])*rlab,sin(t[#])*rlab);
        ctext(""+lab[#],col,row-textheight()/2);
      endif;
    end;
    barcolor(cl); barstyle(st);
    holding(h);
    window(w);
  endfunction
\end{eulerudf}
\begin{eulercomment}
Tidak ada kotak atau sumbu kutu di sini. Selain itu, kami menggunakan
jendela penuh untuk plot.

Kami memanggil reset sebelum kami menguji plot ini untuk mengembalikan
default grafis. Ini tidak perlu, jika Anda yakin plot Anda berhasil.
\end{eulercomment}
\begin{eulerprompt}
>reset; starplot1(normal(1,10)+5,color=red,lab=1:10):
\end{eulerprompt}
\eulerimg{27}{images/Davina Safa Felisa 1-6-275.png}
\begin{eulercomment}
Terkadang, Anda mungkin ingin merencanakan sesuatu yang tidak dapat
dilakukan plot2d, tetapi hampir.

Dalam fungsi berikut, kami melakukan plot impuls logaritmik. plot2d
dapat melakukan plot logaritmik, tetapi tidak untuk batang impuls.
\end{eulercomment}
\begin{eulerprompt}
>function logimpulseplot1 (x,y) ...
\end{eulerprompt}
\begin{eulerudf}
    \{x0,y0\}=makeimpulse(x,log(y)/log(10));
    plot2d(x0,y0,>bar,grid=0);
    h=holding(1);
    frame();
    xgrid(ticks(x));
    p=plot();
    for i=-10 to 10;
      if i<=p[4] and i>=p[3] then
         ygrid(i,yt="10^"+i);
      endif;
    end;
    holding(h);
  endfunction
\end{eulerudf}
\begin{eulercomment}
Mari kita uji dengan nilai yang terdistribusi secara eksponensial.
\end{eulercomment}
\begin{eulerprompt}
>aspect(1.5); x=1:10; y=-log(random(size(x)))*200; ...
>logimpulseplot1(x,y):
\end{eulerprompt}
\eulerimg{17}{images/Davina Safa Felisa 1-6-276.png}
\begin{eulercomment}
Mari kita menganimasikan kurva 2D menggunakan plot langsung. Perintah
plot(x,y) hanya memplot kurva ke jendela plot. setplot(a,b,c,d)
mengatur jendela ini.

Fungsi wait(0) memaksa plot untuk muncul di jendela grafik. Jika
tidak, menggambar ulang terjadi dalam interval waktu yang jarang.
\end{eulercomment}
\begin{eulerprompt}
>function animliss (n,m) ...
\end{eulerprompt}
\begin{eulerudf}
  t=linspace(0,2pi,500);
  f=0;
  c=framecolor(0);
  l=linewidth(2);
  setplot(-1,1,-1,1);
  repeat
    clg;
    plot(sin(n*t),cos(m*t+f));
    wait(0);
    if testkey() then break; endif;
    f=f+0.02;
  end;
  framecolor(c);
  linewidth(l);
  endfunction
\end{eulerudf}
\begin{eulercomment}
Tekan sembarang tombol untuk menghentikan animasi ini.
\end{eulercomment}
\begin{eulerprompt}
>animliss(2,3); // lihat hasilnya, jika sudah puas, tekan ENTER
\end{eulerprompt}
\eulerheading{Plot Logaritmik}
\begin{eulercomment}
EMT menggunakan parameter "logplot" untuk skala logaritmik.\\
Plot logaritma dapat diplot baik menggunakan skala logaritma dalam y
dengan logplot=1, atau menggunakan skala logaritma dalam x dan y
dengan logplot=2, atau dalam x dengan logplot=3.

\end{eulercomment}
\begin{eulerttcomment}
 - logplot=1: y-logaritma
 - logplot=2: x-y-logaritma
 - logplot=3: x-logaritma
\end{eulerttcomment}
\begin{eulerprompt}
>plot2d("exp(x^3-x)*x^2",1,5,logplot=1):
\end{eulerprompt}
\eulerimg{17}{images/Davina Safa Felisa 1-6-277.png}
\begin{eulerprompt}
>plot2d("exp(x+sin(x))",0,100,logplot=1):
\end{eulerprompt}
\eulerimg{17}{images/Davina Safa Felisa 1-6-278.png}
\begin{eulerprompt}
>plot2d("exp(x+sin(x))",10,100,logplot=2):
\end{eulerprompt}
\eulerimg{17}{images/Davina Safa Felisa 1-6-279.png}
\begin{eulerprompt}
>plot2d("gamma(x)",1,10,logplot=1):
\end{eulerprompt}
\eulerimg{17}{images/Davina Safa Felisa 1-6-280.png}
\begin{eulerprompt}
>plot2d("log(x*(2+sin(x/100)))",10,1000,logplot=3):
\end{eulerprompt}
\eulerimg{17}{images/Davina Safa Felisa 1-6-281.png}
\begin{eulercomment}
Ini juga berfungsi dengan plot data.
\end{eulercomment}
\begin{eulerprompt}
>x=10^(1:20); y=x^2-x;
>plot2d(x,y,logplot=2):
\end{eulerprompt}
\eulerimg{17}{images/Davina Safa Felisa 1-6-282.png}
\eulersubheading{Contoh Soal}
\begin{eulercomment}
Plot grafik fungsi kuadrat berikut:\\
\end{eulercomment}
\begin{eulerformula}
\[
f(x) = x^2-4x+3
\]
\end{eulerformula}
\begin{eulerprompt}
>plot2d("x^2-4x+3"):
\end{eulerprompt}
\eulerimg{17}{images/Davina Safa Felisa 1-6-284.png}
\begin{eulerprompt}
>function f(x) := 3x^2-5x;
>plot2d("f",r=5):
\end{eulerprompt}
\eulerimg{17}{images/Davina Safa Felisa 1-6-285.png}
\eulerheading{Rujukan Lengkap Fungsi plot2d()}
\begin{eulercomment}
\end{eulercomment}
\begin{eulerttcomment}
  function plot2d (xv, yv, btest, a, b, c, d, xmin, xmax, r, n,  ..
  logplot, grid, frame, framecolor, square, color, thickness, style,
\end{eulerttcomment}
\begin{eulercomment}
..\\
\end{eulercomment}
\begin{eulerttcomment}
  auto, add, user, delta, points, addpoints, pointstyle, bar,
\end{eulerttcomment}
\begin{eulercomment}
histogram,  ..\\
\end{eulercomment}
\begin{eulerttcomment}
  distribution, even, steps, own, adaptive, hue, level, contour,  ..
  nc, filled, fillcolor, outline, title, xl, yl, maps, contourcolor,
\end{eulerttcomment}
\begin{eulercomment}
..\\
\end{eulercomment}
\begin{eulerttcomment}
  contourwidth, ticks, margin, clipping, cx, cy, insimg, spectral,  ..
  cgrid, vertical, smaller, dl, niveau, levels)
\end{eulerttcomment}
\begin{eulercomment}

Multipurpose plot function for plots in the plane (2D plots). This
function can do plots of functions of one variables, data plots,
curves in the plane, bar plots, grids of complex numbers, and implicit
plots of functions of two variables.

Parameters

x,y       : equations, functions or data vectors\\
a,b,c,d   : Plot area (default a=-2,b=2)\\
r         : if r is set, then a=cx-r, b=cx+r, c=cy-r, d=cy+r\\
\end{eulercomment}
\begin{eulerttcomment}
            r can be a vector [rx,ry] or a vector [rx1,rx2,ry1,ry2].
\end{eulerttcomment}
\begin{eulercomment}
xmin,xmax : range of the parameter for curves\\
auto      : Determine y-range automatically (default)\\
square    : if true, try to keep square x-y-ranges\\
n         : number of intervals (default is adaptive)\\
grid      : 0 = no grid and labels,\\
\end{eulercomment}
\begin{eulerttcomment}
            1 = axis only,
            2 = normal grid (see below for the number of grid lines)
            3 = inside axis
            4 = no grid
            5 = full grid including margin
            6 = ticks at the frame
            7 = axis only
            8 = axis only, sub-ticks
\end{eulerttcomment}
\begin{eulercomment}
frame     : 0 = no frame\\
framecolor: color of the frame and the grid\\
margin    : number between 0 and 0.4 for the margin around the plot\\
color     : Color of curves. If this is a vector of colors,\\
\end{eulercomment}
\begin{eulerttcomment}
            it will be used for each row of a matrix of plots. In the
\end{eulerttcomment}
\begin{eulercomment}
case of\\
\end{eulercomment}
\begin{eulerttcomment}
            point plots, it should be a column vector. If a row vector
\end{eulerttcomment}
\begin{eulercomment}
or a\\
\end{eulercomment}
\begin{eulerttcomment}
            full matrix of colors is used for point plots, it will be
\end{eulerttcomment}
\begin{eulercomment}
used for\\
\end{eulercomment}
\begin{eulerttcomment}
            each data point.
\end{eulerttcomment}
\begin{eulercomment}
thickness : line thickness for curves\\
\end{eulercomment}
\begin{eulerttcomment}
            This value can be smaller than 1 for very thin lines.
\end{eulerttcomment}
\begin{eulercomment}
style     : Plot style for lines, markers, and fills.\\
\end{eulercomment}
\begin{eulerttcomment}
            For points use
            "[]", "<>", ".", "..", "...",
            "*", "+", "|", "-", "o"
            "[]#", "<>#", "o#" (filled shapes)
            "[]w", "<>w", "ow" (non-transparent)
            For lines use
            "-", "--", "-.", ".", ".-.", "-.-", "->"
            For filled polygons or bar plots use
            "#", "#O", "O", "/", "\(\backslash\)", "\(\backslash\)/",
            "+", "|", "-", "t"
\end{eulerttcomment}
\begin{eulercomment}
points    : plot single points instead of line segments\\
addpoints : if true, plots line segments and points\\
add       : add the plot to the existing plot\\
user      : enable user interaction for functions\\
delta     : step size for user interaction\\
bar       : bar plot (x are the interval bounds, y the interval
values)\\
histogram : plots the frequencies of x in n subintervals\\
distribution=n : plots the distribution of x with n subintervals\\
even      : use inter values for automatic histograms.\\
steps     : plots the function as a step function (steps=1,2)\\
adaptive  : use adaptive plots (n is the minimal number of steps)\\
level     : plot level lines of an implicit function of two variables\\
outline   : draws boundary of level ranges.

If the level value is a 2xn matrix, ranges of levels will be drawn\\
in the color using the given fill style. If outline is true, it\\
will be drawn in the contour color. Using this feature, regions of\\
f(x,y) between limits can be marked.

hue       : add hue color to the level plot to indicate the function\\
\end{eulercomment}
\begin{eulerttcomment}
            value
\end{eulerttcomment}
\begin{eulercomment}
contour   : Use level plot with automatic levels\\
nc        : number of automatic level lines\\
title     : plot title (default "")\\
xl, yl    : labels for the x- and y-axis\\
smaller   : if \textgreater{}0, there will be more space to the left for labels.\\
vertical  :\\
\end{eulercomment}
\begin{eulerttcomment}
  Turns vertical labels on or off. This changes the global variable
  verticallabels locally for one plot. The value 1 sets only vertical
  text, the value 2 uses vertical numerical labels on the y axis.
\end{eulerttcomment}
\begin{eulercomment}
filled    : fill the plot of a curve\\
fillcolor : fill color for bar and filled curves\\
outline   : boundary for filled polygons\\
logplot   : set logarithmic plots\\
\end{eulercomment}
\begin{eulerttcomment}
            1 = logplot in y,
            2 = logplot in xy,
            3 = logplot in x
\end{eulerttcomment}
\begin{eulercomment}
own       :\\
\end{eulercomment}
\begin{eulerttcomment}
  A string, which points to an own plot routine. With >user, you get
  the same user interaction as in plot2d. The range will be set
  before each call to your function.
\end{eulerttcomment}
\begin{eulercomment}
maps      : map expressions (0 is faster), functions are always
mapped.\\
contourcolor : color of contour lines\\
contourwidth : width of contour lines\\
clipping  : toggles the clipping (default is true)\\
title     :\\
\end{eulercomment}
\begin{eulerttcomment}
  This can be used to describe the plot. The title will appear above
  the plot. Moreover, a label for the x and y axis can be added with
  xl="string" or yl="string". Other labels can be added with the
  functions label() or labelbox(). The title can be a unicode
  string or an image of a Latex formula.
\end{eulerttcomment}
\begin{eulercomment}
cgrid     :\\
\end{eulercomment}
\begin{eulerttcomment}
  Determines the number of grid lines for plots of complex grids.
  Should be a divisor of the the matrix size minus 1 (number of
  subintervals). cgrid can be a vector [cx,cy].
\end{eulerttcomment}
\begin{eulercomment}

Overview

The function can plot

- expressions, call collections or functions of one variable,\\
- parametric curves,\\
- x data against y data,\\
- implicit functions,\\
- bar plots,\\
- complex grids,\\
- polygons.

If a function or expression for xv is given, plot2d() will compute\\
values in the given range using the function or expression. The\\
expression must be an expression in the variable x. The range must\\
be defined in the parameters a and b unless the default range\\
[-2,2] should be used. The y-range will be computed automatically,\\
unless c and d are specified, or a radius r, which yields the range\\
[-r,r] for x and y. For plots of functions, plot2d will use an\\
adaptive evaluation of the function by default. To speed up the\\
plot for complicated functions, switch this off with \textless{}adaptive, and\\
optionally decrease the number of intervals n. Moreover, plot2d()\\
will by default use mapping. I.e., it will compute the plot element\\
for element. If your expression or your functions can handle a\\
vector x, you can switch that off with \textless{}maps for faster evaluation.

Note that adaptive plots are always computed element for element.\\
If functions or expressions for both xv and for yv are specified,\\
plot2d() will compute a curve with the xv values as x-coordinates\\
and the yv values as y-coordinates. In this case, a range should be\\
defined for the parameter using xmin, xmax. Expressions contained in
strings must always be expressions in the parameter variable x.
\end{eulercomment}
\begin{eulercomment}

\begin{eulercomment}
\eulerheading{Menggambar Plot 3D dengan EMT}
\begin{eulercomment}
Ini adalah pengenalan plot 3D di Euler. Kita membutuhkan plot 3D untuk
memvisualisasikan fungsi dari dua variabel.

Euler menggambar fungsi tersebut menggunakan algoritma pengurutan
untuk menyembunyikan bagian di latar belakang. Secara umum, Euler
menggunakan proyeksi pusat. Standarnya adalah dari kuadran x-y positif
menuju titik asal x=y=z=0, tetapi sudut=0° terlihat dari arah sumbu y.
Sudut pandang dan tinggi dapat diubah.

Euler dapat merencanakan

- permukaan dengan bayangan dan garis level atau rentang level,\\
- awan poin,\\
- kurva parametrik,\\
- permukaan implisit.

Plot 3D dari suatu fungsi menggunakan plot3d. Cara termudah adalah
dengan memplot ekspresi dalam x dan y. Parameter r mengatur kisaran
plot di sekitar (0,0).
\end{eulercomment}
\begin{eulerprompt}
>aspect(1.5); plot3d("x^2+sin(y)",-5,5,0,6*pi):
\end{eulerprompt}
\eulerimg{17}{images/Davina Safa Felisa 1-6-286.png}
\begin{eulerprompt}
>plot3d("x^2+x*sin(y)",-5,5,0,6*pi):
\end{eulerprompt}
\eulerimg{17}{images/Davina Safa Felisa 1-6-287.png}
\eulerheading{Fungsi dua Variabel}
\begin{eulercomment}
Untuk grafik fungsi, gunakan

- ekspresi sederhana dalam x dan y,\\
- nama fungsi dari dua variabell\\
- atau matriks data.

Standarnya adalah kotak kawat yang diisi dengan warna berbeda di kedua
sisi. Perhatikan bahwa jumlah default interval grid adalah 10, tetapi
plot menggunakan jumlah default 40x40 persegi panjang untuk membangun
permukaan. Ini bisa diubah.

- n=40, n=[40,40]: jumlah garis grid di setiap arah\\
- grid=10, grid=[10,10]: jumlah garis grid di setiap arah.

Kami menggunakan default n=40 dan grid=10.
\end{eulercomment}
\begin{eulerprompt}
>plot3d("x^2+y^2"):
\end{eulerprompt}
\eulerimg{17}{images/Davina Safa Felisa 1-6-288.png}
\begin{eulercomment}
Interaksi pengguna dimungkinkan dengan \textgreater{}parameter pengguna. Pengguna
dapat menekan tombol berikut.

- kiri, kanan, atas, bawah: putar sudut pandang\\
- +,-: memperbesar atau memperkecil\\
- a: menghasilkan anaglyph (lihat di bawah)\\
- l: beralih memutar sumber cahaya (lihat di bawah)\\
- spasi: reset ke default\\
- kembali: akhiri interaksi
\end{eulercomment}
\begin{eulerprompt}
>plot3d("exp(-x^2+y^2)",>user, ...
>  title="Turn with the vector keys (press return to finish)"):
\end{eulerprompt}
\eulerimg{17}{images/Davina Safa Felisa 1-6-289.png}
\begin{eulercomment}
Rentang plot untuk fungsi dapat ditentukan dengan

- a,b: rentang-x\\
- c,d: rentang-y\\
- r: persegi simetris di sekitar (0,0).\\
- n: jumlah subinterval untuk plot.

Ada beberapa parameter untuk menskalakan fungsi atau mengubah tampilan
grafik.

fscale: skala ke nilai fungsi (defaultnya adalah \textless{}fscale).\\
skala: angka atau vektor 1x2 untuk skala ke arah x dan y.\\
bingkai: jenis bingkai (default 1).
\end{eulercomment}
\begin{eulerprompt}
>plot3d("exp(-(x^2+y^2)/5)",r=10,n=80,fscale=4,scale=1.2,frame=3):
\end{eulerprompt}
\eulerimg{17}{images/Davina Safa Felisa 1-6-290.png}
\begin{eulercomment}
Tampilan dapat diubah dengan berbagai cara.

- jarak: jarak pandang ke plot.\\
- zoom: nilai zoom.\\
- sudut: sudut terhadap sumbu y negatif dalam radian.\\
- tinggi: ketinggian tampilan dalam radian.

Nilai default dapat diperiksa atau diubah dengan fungsi view(). Ini
mengembalikan parameter dalam urutan di atas.
\end{eulercomment}
\begin{eulerprompt}
>view
\end{eulerprompt}
\begin{euleroutput}
  [5,  2.6,  2,  0.4]
\end{euleroutput}
\begin{eulercomment}
Jarak yang lebih dekat membutuhkan lebih sedikit zoom. Efeknya lebih
seperti lensa sudut lebar.

Dalam contoh berikut, sudut=0 dan tinggi=0 terlihat dari sumbu y
negatif. Label sumbu untuk y disembunyikan dalam kasus ini.
\end{eulercomment}
\begin{eulerprompt}
>plot3d("x^2+y",distance=3,zoom=2,angle=pi/2,height=0):
\end{eulerprompt}
\eulerimg{17}{images/Davina Safa Felisa 1-6-291.png}
\begin{eulercomment}
Plot terlihat selalu ke pusat kubus plot. Anda dapat memindahkan pusat
dengan parameter tengah.
\end{eulercomment}
\begin{eulerprompt}
>plot3d("x^4+y^2",a=0,b=1,c=-1,d=1,angle=-20°,height=20°, ...
>  center=[0.4,0,0],zoom=5):
\end{eulerprompt}
\eulerimg{17}{images/Davina Safa Felisa 1-6-292.png}
\begin{eulercomment}
Plot diskalakan agar sesuai dengan kubus satuan untuk dilihat. Jadi
tidak perlu mengubah jarak atau zoom tergantung pada ukuran plot.
Namun, label mengacu pada ukuran sebenarnya.

Jika Anda mematikannya dengan scale=false, Anda perlu berhati-hati,
bahwa plot masih cocok dengan jendela plot, dengan mengubah jarak
pandang atau zoom, dan memindahkan pusat.
\end{eulercomment}
\begin{eulerprompt}
>plot3d("5*exp(-x^2-y^2)",r=2,<fscale,<scale,distance=13,height=50°, ...
>  center=[0,0,-2],frame=3):
\end{eulerprompt}
\eulerimg{17}{images/Davina Safa Felisa 1-6-293.png}
\begin{eulercomment}
Sebuah plot kutub juga tersedia. Parameter polar=true menggambar plot
polar. Fungsi tersebut harus tetap merupakan fungsi dari x dan y.
Parameter "fscale" menskalakan fungsi dengan skala sendiri. Jika
tidak, fungsi diskalakan agar sesuai dengan kubus.
\end{eulercomment}
\begin{eulerprompt}
>plot3d("1/(x^2+y^2+1)",r=5,>polar, ...
>fscale=2,>hue,n=100,zoom=4,>contour,color=gray):
\end{eulerprompt}
\eulerimg{17}{images/Davina Safa Felisa 1-6-294.png}
\begin{eulerprompt}
>function f(r) := exp(-r/2)*cos(r); ...
>plot3d("f(x^2+y^2)",>polar,scale=[1,1,0.4],r=pi,frame=3,zoom=4):
\end{eulerprompt}
\eulerimg{17}{images/Davina Safa Felisa 1-6-295.png}
\begin{eulercomment}
Rotasi parameter memutar fungsi dalam x di sekitar sumbu x.

- rotate=1: Menggunakan sumbu x\\
- rotate=2: Menggunakan sumbu z
\end{eulercomment}
\begin{eulerprompt}
>plot3d("x^2+1",a=-1,b=1,rotate=true,grid=5):
\end{eulerprompt}
\eulerimg{17}{images/Davina Safa Felisa 1-6-296.png}
\begin{eulerprompt}
>plot3d("x^2+1",a=-1,b=1,rotate=2,grid=5):
\end{eulerprompt}
\eulerimg{17}{images/Davina Safa Felisa 1-6-297.png}
\begin{eulerprompt}
>plot3d("sqrt(25-x^2)",a=0,b=5,rotate=1):
\end{eulerprompt}
\eulerimg{17}{images/Davina Safa Felisa 1-6-298.png}
\begin{eulerprompt}
>plot3d("x","x^2+y^2","y",r=2,zoom=3.5,frame=3):
\end{eulerprompt}
\eulerimg{17}{images/Davina Safa Felisa 1-6-299.png}
\eulerheading{Plot Kontur}
\begin{eulercomment}
Untuk plot, Euler menambahkan garis grid. Sebagai gantinya
dimungkinkan untuk menggunakan garis level dan rona satu warna atau
rona berwarna spektral. Euler dapat menggambar tinggi fungsi pada plot
dengan bayangan. Di semua plot 3D, Euler dapat menghasilkan anaglyph
merah/sian.

-\textgreater{} hue: Menyalakan bayangan cahaya alih-alih kabel.\\
-\textgreater{} kontur: Memplot garis kontur otomatis pada plot.\\
- level=... (atau level): Sebuah vektor nilai untuk garis kontur.

Standarnya adalah level="auto", yang menghitung beberapa garis level
secara otomatis. Seperti yang Anda lihat di plot, level sebenarnya
adalah rentang level.

Gaya default dapat diubah. Untuk plot kontur berikut, kami menggunakan
grid yang lebih halus untuk 100x100 poin, skala fungsi dan plot, dan
menggunakan sudut pandang yang berbeda.
\end{eulercomment}
\begin{eulerprompt}
>plot3d("exp(-x^2-y^2)",r=2,n=100,level="thin", ...
> >contour,>spectral,fscale=1,scale=1.1,angle=45°,height=20°):
\end{eulerprompt}
\eulerimg{17}{images/Davina Safa Felisa 1-6-300.png}
\begin{eulerprompt}
>plot3d("exp(x*y)",angle=100°,>contour,color=green):
\end{eulerprompt}
\eulerimg{17}{images/Davina Safa Felisa 1-6-301.png}
\begin{eulercomment}
Bayangan default menggunakan warna abu-abu. Tetapi rentang warna
spektral juga tersedia.

-\textgreater{} spektral: Menggunakan skema spektral default\\
- color=...: Menggunakan warna khusus atau skema spektral

Untuk plot berikut, kami menggunakan skema spektral default dan
menambah jumlah titik untuk mendapatkan tampilan yang sangat halus.
\end{eulercomment}
\begin{eulerprompt}
>plot3d("x^2+y^2",>spectral,>contour,n=100):
\end{eulerprompt}
\eulerimg{17}{images/Davina Safa Felisa 1-6-302.png}
\begin{eulercomment}
Alih-alih garis level otomatis, kita juga dapat mengatur nilai garis
level. Ini akan menghasilkan garis level tipis alih-alih rentang
level.
\end{eulercomment}
\begin{eulerprompt}
>plot3d("x^2-y^2",0,5,0,5,level=-1:0.1:1,color=redgreen):
\end{eulerprompt}
\eulerimg{17}{images/Davina Safa Felisa 1-6-303.png}
\begin{eulercomment}
Dalam plot berikut, kami menggunakan dua pita level yang sangat luas
dari -0,1 hingga 1, dan dari 0,9 hingga 1. Ini dimasukkan sebagai
matriks dengan batas level sebagai kolom.

Selain itu, kami melapisi kisi dengan 10 interval di setiap arah.
\end{eulercomment}
\begin{eulerprompt}
>plot3d("x^2+y^3",level=[-0.1,0.9;0,1], ...
>  >spectral,angle=30°,grid=10,contourcolor=gray):
\end{eulerprompt}
\eulerimg{17}{images/Davina Safa Felisa 1-6-304.png}
\begin{eulercomment}
Dalam contoh berikut, kami memplot himpunan, di mana

\end{eulercomment}
\begin{eulerformula}
\[
f(x,y) = x^y-y^x = 0
\]
\end{eulerformula}
\begin{eulercomment}
Kami menggunakan satu garis tipis untuk garis level.
\end{eulercomment}
\begin{eulerprompt}
>plot3d("x^y-y^x",level=0,a=0,b=6,c=0,d=6,contourcolor=red,n=100):
\end{eulerprompt}
\eulerimg{17}{images/Davina Safa Felisa 1-6-306.png}
\begin{eulercomment}
Dimungkinkan untuk menunjukkan bidang kontur di bawah plot. Warna dan
jarak ke plot dapat ditentukan.
\end{eulercomment}
\begin{eulerprompt}
>plot3d("x^2+y^4",>cp,cpcolor=green,cpdelta=0.2):
\end{eulerprompt}
\eulerimg{17}{images/Davina Safa Felisa 1-6-307.png}
\begin{eulercomment}
Here are a few more styles. We always turn off the frame, and use
various color schemes for the plot and the grid.
\end{eulercomment}
\begin{eulerprompt}
>figure(2,2); ...
\end{eulerprompt}
\begin{eulercomment}
Dimungkinkan untuk menunjukkan bidang kontur di bawah plot. Warna dan
jarak ke plot dapat ditentukan.
\end{eulercomment}
\begin{eulerprompt}
>expr="y^3-x^2"; ...
>figure(1);  ...
>  plot3d(expr,<frame,>cp,cpcolor=spectral); ...
>figure(2);  ...
>  plot3d(expr,<frame,>spectral,grid=10,cp=2); ...
>figure(3);  ...
>  plot3d(expr,<frame,>contour,color=gray,nc=5,cp=3,cpcolor=greenred); ...
>figure(4);  ...
>  plot3d(expr,<frame,>hue,grid=10,>transparent,>cp,cpcolor=gray); ...
>figure(0):
\end{eulerprompt}
\eulerimg{17}{images/Davina Safa Felisa 1-6-308.png}
\begin{eulercomment}
Ada beberapa skema spektral lainnya, bernomor dari 1 hingga 9. Tetapi
Anda juga dapat menggunakan warna=nilai, di mana nilai

- spektral: untuk rentang dari biru ke merah\\
- putih: untuk rentang yang lebih redup\\
-kuningbiru,ungu hijau,birukuning,hijaumerah\\
- birukuning, hijau ungu, kuning biru, merah hijau
\end{eulercomment}
\begin{eulerprompt}
>figure(3,3); ...
>for i=1:9;  ...
>  figure(i); plot3d("x^2+y^2",spectral=i,>contour,>cp,<frame,zoom=4);  ...
>end; ...
>figure(0):
\end{eulerprompt}
\eulerimg{17}{images/Davina Safa Felisa 1-6-309.png}
\begin{eulercomment}
Sumber cahaya dapat diubah dengan l dan tombol kursor selama interaksi
pengguna. Itu juga dapat diatur dengan parameter.

- cahaya: arah untuk cahaya\\
- amb: cahaya sekitar antara 0 dan 1

Perhatikan bahwa program tidak membuat perbedaan antara sisi plot.
Tidak ada bayangan. Untuk ini, Anda perlu Povray.
\end{eulercomment}
\begin{eulerprompt}
>plot3d("-x^2-y^2", ...
>  hue=true,light=[0,1,1],amb=0,user=true, ...
>  title="Press l and cursor keys (return to exit)"):
\end{eulerprompt}
\eulerimg{17}{images/Davina Safa Felisa 1-6-310.png}
\begin{eulercomment}
Parameter warna mengubah warna permukaan. Warna garis level juga dapat
diubah.
\end{eulercomment}
\begin{eulerprompt}
>plot3d("-x^2-y^2",color=rgb(0.2,0.2,0),hue=true,frame=false, ...
>  zoom=3,contourcolor=red,level=-2:0.1:1,dl=0.01):
\end{eulerprompt}
\eulerimg{17}{images/Davina Safa Felisa 1-6-311.png}
\begin{eulercomment}
Warna 0 memberikan efek pelangi khusus.
\end{eulercomment}
\begin{eulerprompt}
>plot3d("x^2/(x^2+y^2+1)",color=0,hue=true,grid=10):
\end{eulerprompt}
\eulerimg{17}{images/Davina Safa Felisa 1-6-312.png}
\begin{eulercomment}
Permukaannya juga bisa transparan.
\end{eulercomment}
\begin{eulerprompt}
>plot3d("x^2+y^2",>transparent,grid=10,wirecolor=red):
\end{eulerprompt}
\eulerimg{17}{images/Davina Safa Felisa 1-6-313.png}
\eulerheading{Plot Implisit}
\begin{eulercomment}
Ada juga plot implisit dalam tiga dimensi. Euler menghasilkan
pemotongan melalui objek. Fitur plot3d termasuk plot implisit.
Plot-plot ini menunjukkan himpunan nol dari suatu fungsi dalam tiga
variabel.\\
Solusi dari

\end{eulercomment}
\begin{eulerformula}
\[
f(x,y,z) = 0
\]
\end{eulerformula}
\begin{eulercomment}
dapat divisualisasikan dalam potongan sejajar dengan bidang x-y-, x-z-
dan y-z.

- implisit=1: potong sejajar dengan bidang y-z\\
- implisit=2: potong sejajar dengan bidang x-z\\
- implisit=4: potong sejajar dengan bidang x-y

Tambahkan nilai-nilai ini, jika Anda suka. Dalam contoh kita plot

\end{eulercomment}
\begin{eulerformula}
\[
M = \{ (x,y,z) : x^2+y^3+zy=1 \}
\]
\end{eulerformula}
\begin{eulerprompt}
>plot3d("x^2+y^3+z*y-1",r=5,implicit=3):
\end{eulerprompt}
\eulerimg{17}{images/Davina Safa Felisa 1-6-316.png}
\begin{eulerprompt}
>plot3d("x^2+y^2+4*x*z+z^3",>implicit,r=2,zoom=2.5):
\end{eulerprompt}
\eulerimg{17}{images/Davina Safa Felisa 1-6-317.png}
\eulerheading{Merencanakan Data 3D}
\begin{eulercomment}
Sama seperti plot2d, plot3d menerima data. Untuk objek 3D, Anda perlu
menyediakan matriks nilai x-, y- dan z, atau tiga fungsi atau ekspresi
fx(x,y), fy(x,y), fz(x,y).

\end{eulercomment}
\begin{eulerformula}
\[
\gamma(t,s) = (x(t,s),y(t,s),z(t,s))
\]
\end{eulerformula}
\begin{eulercomment}
Karena x,y,z adalah matriks, kita asumsikan bahwa (t,s) melalui sebuah
kotak persegi. Hasilnya, Anda dapat memplot gambar persegi panjang di
ruang angkasa.

Anda dapat menggunakan bahasa matriks Euler untuk menghasilkan
koordinat secara efektif.

Dalam contoh berikut, kami menggunakan vektor nilai t dan vektor kolom
nilai s untuk membuat parameter permukaan bola. Dalam gambar kita
dapat menandai daerah, dalam kasus kita daerah kutub.
\end{eulercomment}
\begin{eulerprompt}
>t=linspace(0,2pi,180); s=linspace(-pi/2,pi/2,90)'; ...
>x=cos(s)*cos(t); y=cos(s)*sin(t); z=sin(s); ...
>plot3d(x,y,z,>hue, ...
>color=blue,<frame,grid=[10,20], ...
>values=s,contourcolor=red,level=[90°-24°;90°-22°], ...
>scale=1.4,height=50°):
\end{eulerprompt}
\eulerimg{17}{images/Davina Safa Felisa 1-6-319.png}
\begin{eulercomment}
Berikut adalah contoh, yang merupakan grafik fungsi.
\end{eulercomment}
\begin{eulerprompt}
>t=-1:0.1:1; s=(-1:0.1:1)'; plot3d(t,s,t*s,grid=10):
\end{eulerprompt}
\eulerimg{17}{images/Davina Safa Felisa 1-6-320.png}
\begin{eulercomment}
Namun, kita bisa membuat segala macam permukaan. Berikut adalah
permukaan yang sama dengan fungsi

\end{eulercomment}
\begin{eulerformula}
\[
x = y \, z
\]
\end{eulerformula}
\begin{eulerprompt}
>plot3d(t*s,t,s,angle=180°,grid=10):
\end{eulerprompt}
\eulerimg{17}{images/Davina Safa Felisa 1-6-322.png}
\begin{eulercomment}
Dengan lebih banyak usaha, kami dapat menghasilkan banyak permukaan.

Dalam contoh berikut, kita membuat tampilan bayangan dari bola yang
terdistorsi. Koordinat biasa untuk bola adalah

\end{eulercomment}
\begin{eulerformula}
\[
\gamma(t,s) = (\cos(t)\cos(s),\sin(t)\sin(s),\cos(s))
\]
\end{eulerformula}
\begin{eulercomment}
dengan

\end{eulercomment}
\begin{eulerformula}
\[
0 \le t \le 2\pi, \quad \frac{-\pi}{2} \le s \le \frac{\pi}{2}.
\]
\end{eulerformula}
\begin{eulercomment}
Kami mendistorsi ini dengan sebuah faktor

\end{eulercomment}
\begin{eulerformula}
\[
d(t,s) = \frac{\cos(4t)+\cos(8s)}{4}.
\]
\end{eulerformula}
\begin{eulerprompt}
>t=linspace(0,2pi,320); s=linspace(-pi/2,pi/2,160)'; ...
>d=1+0.2*(cos(4*t)+cos(8*s)); ...
>plot3d(cos(t)*cos(s)*d,sin(t)*cos(s)*d,sin(s)*d,hue=1, ...
>  light=[1,0,1],frame=0,zoom=5):
\end{eulerprompt}
\eulerimg{17}{images/Davina Safa Felisa 1-6-326.png}
\begin{eulercomment}
Tentu saja, titik cloud juga dimungkinkan. Untuk memplot data titik
dalam ruang, kita membutuhkan tiga vektor untuk koordinat titik-titik
tersebut.

Gayanya sama seperti di plot2d dengan points=true;
\end{eulercomment}
\begin{eulerprompt}
>n=500;  ...
>  plot3d(normal(1,n),normal(1,n),normal(1,n),points=true,style="."):
\end{eulerprompt}
\eulerimg{17}{images/Davina Safa Felisa 1-6-327.png}
\begin{eulercomment}
Dimungkinkan juga untuk memplot kurva dalam 3D. Dalam hal ini, lebih
mudah untuk menghitung titik-titik kurva. Untuk kurva di pesawat kami
menggunakan urutan koordinat dan parameter wire=true.
\end{eulercomment}
\begin{eulerprompt}
>t=linspace(0,8pi,500); ...
>plot3d(sin(t),cos(t),t/10,>wire,zoom=3):
\end{eulerprompt}
\eulerimg{17}{images/Davina Safa Felisa 1-6-328.png}
\begin{eulerprompt}
>t=linspace(0,4pi,1000); plot3d(cos(t),sin(t),t/2pi,>wire, ...
>linewidth=3,wirecolor=blue):
\end{eulerprompt}
\eulerimg{17}{images/Davina Safa Felisa 1-6-329.png}
\begin{eulerprompt}
>X=cumsum(normal(3,100)); ...
> plot3d(X[1],X[2],X[3],>anaglyph,>wire):
\end{eulerprompt}
\eulerimg{17}{images/Davina Safa Felisa 1-6-330.png}
\begin{eulercomment}
EMT juga dapat memplot dalam mode anaglyph. Untuk melihat plot seperti
itu, Anda memerlukan kacamata merah/sian.
\end{eulercomment}
\begin{eulerprompt}
> plot3d("x^2+y^3",>anaglyph,>contour,angle=30°):
\end{eulerprompt}
\eulerimg{17}{images/Davina Safa Felisa 1-6-331.png}
\begin{eulercomment}
Seringkali, skema warna spektral digunakan untuk plot. Ini menekankan
ketinggian fungsi.
\end{eulercomment}
\begin{eulerprompt}
>plot3d("x^2*y^3-y",>spectral,>contour,zoom=3.2):
\end{eulerprompt}
\eulerimg{17}{images/Davina Safa Felisa 1-6-332.png}
\begin{eulercomment}
Euler juga dapat memplot permukaan berparameter, ketika parameternya
adalah nilai x-, y-, dan z dari gambar kotak persegi panjang dalam
ruang.

Untuk demo berikut, kami mengatur parameter u- dan v-, dan
menghasilkan koordinat ruang dari ini.
\end{eulercomment}
\begin{eulerprompt}
>u=linspace(-1,1,10); v=linspace(0,2*pi,50)'; ...
>X=(3+u*cos(v/2))*cos(v); Y=(3+u*cos(v/2))*sin(v); Z=u*sin(v/2); ...
>plot3d(X,Y,Z,>anaglyph,<frame,>wire,scale=2.3):
\end{eulerprompt}
\eulerimg{17}{images/Davina Safa Felisa 1-6-333.png}
\begin{eulercomment}
Berikut adalah contoh yang lebih rumit, yang megah dengan kacamata
merah/sian.
\end{eulercomment}
\begin{eulerprompt}
>u:=linspace(-pi,pi,160); v:=linspace(-pi,pi,400)';  ...
>x:=(4*(1+.25*sin(3*v))+cos(u))*cos(2*v); ...
>y:=(4*(1+.25*sin(3*v))+cos(u))*sin(2*v); ...
> z=sin(u)+2*cos(3*v); ...
>plot3d(x,y,z,frame=0,scale=1.5,hue=1,light=[1,0,-1],zoom=2.8,>anaglyph):
\end{eulerprompt}
\eulerimg{17}{images/Davina Safa Felisa 1-6-334.png}
\eulerheading{Plot Statistik}
\begin{eulercomment}
Plot bar juga dimungkinkan. Untuk ini, kita harus menyediakan

- x: vektor baris dengan n+1 elemen\\
- y: vektor kolom dengan n+1 elemen\\
- z: matriks nilai nxn.

z bisa lebih besar, tetapi hanya nilai nxn yang akan digunakan.

Dalam contoh, pertama-tama kita menghitung nilainya. Kemudian kita
sesuaikan x dan y, sehingga vektor berpusat pada nilai yang digunakan.
\end{eulercomment}
\begin{eulerprompt}
>x=-1:0.1:1; y=x'; z=x^2+y^2; ...
>xa=(x|1.1)-0.05; ya=(y_1.1)-0.05; ...
>plot3d(xa,ya,z,bar=true):
\end{eulerprompt}
\eulerimg{17}{images/Davina Safa Felisa 1-6-335.png}
\begin{eulercomment}
Dimungkinkan untuk membagi plot permukaan menjadi dua atau lebih
bagian.
\end{eulercomment}
\begin{eulerprompt}
>x=-1:0.1:1; y=x'; z=x+y; d=zeros(size(x)); ...
>plot3d(x,y,z,disconnect=2:2:20):
\end{eulerprompt}
\eulerimg{17}{images/Davina Safa Felisa 1-6-336.png}
\begin{eulercomment}
Jika memuat atau menghasilkan matriks data M dari file dan perlu
memplotnya dalam 3D, Anda dapat menskalakan matriks ke [-1,1] dengan
scale(M), atau menskalakan matriks dengan \textgreater{}zscale. Ini dapat
dikombinasikan dengan faktor penskalaan individu yang diterapkan
sebagai tambahan.
\end{eulercomment}
\begin{eulerprompt}
>i=1:20; j=i'; ...
>plot3d(i*j^2+100*normal(20,20),>zscale,scale=[1,1,1.5],angle=-40°,zoom=1.8):
\end{eulerprompt}
\eulerimg{17}{images/Davina Safa Felisa 1-6-337.png}
\begin{eulerprompt}
>Z=intrandom(5,100,6); v=zeros(5,6); ...
>loop 1 to 5; v[#]=getmultiplicities(1:6,Z[#]); end; ...
>columnsplot3d(v',scols=1:5,ccols=[1:5]):
\end{eulerprompt}
\eulerimg{17}{images/Davina Safa Felisa 1-6-338.png}
\eulerheading{Permukaan Benda Putar}
\begin{eulerprompt}
>plot2d("(x^2+y^2-1)^3-x^2*y^3",r=1.3, ...
>style="#",color=red,<outline, ...
>level=[-2;0],n=100):
\end{eulerprompt}
\eulerimg{17}{images/Davina Safa Felisa 1-6-339.png}
\begin{eulerprompt}
>ekspresi &= (x^2+y^2-1)^3-x^2*y^3; $ekspresi
\end{eulerprompt}
\begin{eulerformula}
\[
\left(y^2+x^2-1\right)^3-x^2\,y^3
\]
\end{eulerformula}
\begin{eulercomment}
Kami ingin memutar kurva jantung di sekitar sumbu y. Berikut adalah
ungkapan, yang mendefinisikan hati:

\end{eulercomment}
\begin{eulerformula}
\[
f(x,y)=(x^2+y^2-1)^3-x^2.y^3.
\]
\end{eulerformula}
\begin{eulercomment}
Selanjutnya kita atur

\end{eulercomment}
\begin{eulerformula}
\[
x=r.cos(a),\quad y=r.sin(a).
\]
\end{eulerformula}
\begin{eulerprompt}
>function fr(r,a) &= ekspresi with [x=r*cos(a),y=r*sin(a)] | trigreduce; $fr(r,a)
\end{eulerprompt}
\begin{eulerformula}
\[
\left(r^2-1\right)^3+\frac{\left(\sin \left(5\,a\right)-\sin \left(  3\,a\right)-2\,\sin a\right)\,r^5}{16}
\]
\end{eulerformula}
\begin{eulercomment}
Hal ini memungkinkan untuk mendefinisikan fungsi numerik, yang
memecahkan r, jika a diberikan. Dengan fungsi itu kita dapat memplot
jantung yang diputar sebagai permukaan parametrik.
\end{eulercomment}
\begin{eulerprompt}
>function map f(a) := bisect("fr",0,2;a); ...
>t=linspace(-pi/2,pi/2,100); r=f(t);  ...
>s=linspace(pi,2pi,100)'; ...
>plot3d(r*cos(t)*sin(s),r*cos(t)*cos(s),r*sin(t), ...
>>hue,<frame,color=red,zoom=4,amb=0,max=0.7,grid=12,height=50°):
\end{eulerprompt}
\eulerimg{17}{images/Davina Safa Felisa 1-6-344.png}
\begin{eulercomment}
Berikut ini adalah plot 3D dari gambar di atas yang diputar di sekitar
sumbu z. Kami mendefinisikan fungsi, yang menggambarkan objek.
\end{eulercomment}
\begin{eulerprompt}
>function f(x,y,z) ...
\end{eulerprompt}
\begin{eulerudf}
  r=x^2+y^2;
  return (r+z^2-1)^3-r*z^3;
   endfunction
\end{eulerudf}
\begin{eulerprompt}
>plot3d("f(x,y,z)", ...
>xmin=0,xmax=1.2,ymin=-1.2,ymax=1.2,zmin=-1.2,zmax=1.4, ...
>implicit=1,angle=-30°,zoom=2.5,n=[10,60,60],>anaglyph):
\end{eulerprompt}
\eulerimg{17}{images/Davina Safa Felisa 1-6-345.png}
\eulerheading{Plot 3D Khusus}
\begin{eulercomment}
Fungsi plot3d bagus untuk dimiliki, tetapi tidak memenuhi semua
kebutuhan. Selain rutinitas yang lebih mendasar, dimungkinkan untuk
mendapatkan plot berbingkai dari objek apa pun yang Anda suka.

Meskipun Euler bukan program 3D, ia dapat menggabungkan beberapa objek
dasar. Kami mencoba memvisualisasikan paraboloid dan garis
singgungnya.
\end{eulercomment}
\begin{eulerprompt}
>function myplot ...
\end{eulerprompt}
\begin{eulerudf}
    y=0:0.01:1; x=(0.1:0.01:1)';
    plot3d(x,y,0.2*(x-0.1)/2,<scale,<frame,>hue, ..
      hues=0.5,>contour,color=orange);
    h=holding(1);
    plot3d(x,y,(x^2+y^2)/2,<scale,<frame,>contour,>hue);
    holding(h);
  endfunction
\end{eulerudf}
\begin{eulercomment}
Sekarang framedplot() menyediakan frame, dan mengatur tampilan.
\end{eulercomment}
\begin{eulerprompt}
>framedplot("myplot",[0.1,1,0,1,0,1],angle=-45°, ...
>  center=[0,0,-0.7],zoom=6):
\end{eulerprompt}
\eulerimg{17}{images/Davina Safa Felisa 1-6-346.png}
\begin{eulercomment}
Dengan cara yang sama, Anda dapat memplot bidang kontur secara manual.
Perhatikan bahwa plot3d() menyetel jendela ke fullwindow() secara
default, tetapi plotcontourplane() mengasumsikan itu.
\end{eulercomment}
\begin{eulerprompt}
>x=-1:0.02:1.1; y=x'; z=x^2-y^4;
>function myplot (x,y,z) ...
\end{eulerprompt}
\begin{eulerudf}
    zoom(2);
    wi=fullwindow();
    plotcontourplane(x,y,z,level="auto",<scale);
    plot3d(x,y,z,>hue,<scale,>add,color=white,level="thin");
    window(wi);
    reset();
  endfunction
\end{eulerudf}
\begin{eulerprompt}
>myplot(x,y,z):
\end{eulerprompt}
\eulerimg{27}{images/Davina Safa Felisa 1-6-347.png}
\eulerheading{Animasi}
\begin{eulercomment}
Euler dapat menggunakan frame untuk menghitung animasi terlebih
dahulu.

Salah satu fungsi yang memanfaatkan teknik ini adalah rotate. Itu
dapat mengubah sudut pandang dan menggambar ulang plot 3D. Fungsi
memanggil addpage() untuk setiap plot baru. Akhirnya itu menjiwai
plot.

Silakan pelajari sumber rotasi untuk melihat lebih detail.
\end{eulercomment}
\begin{eulerprompt}
>function testplot () := plot3d("x^2+y^3"); ...
>rotate("testplot"); testplot():
\end{eulerprompt}
\eulerimg{27}{images/Davina Safa Felisa 1-6-348.png}
\eulersubheading{Menggambar Povray}
\begin{eulercomment}
Dengan bantuan file Euler povray.e, Euler dapat menghasilkan file
Povray. Hasilnya sangat bagus untuk dilihat.

Anda perlu menginstal Povray (32bit atau 64bit) dari
http://www.povray.org/, dan meletakkan sub-direktori "bin" dari Povray ke jalur lingkungan, atau mengatur variabel "defaultpovray" dengan path lengkap yang menunjuk ke "pvengine.exe".

Antarmuka Povray dari Euler menghasilkan file Povray di direktori home
pengguna, dan memanggil Povray untuk mengurai file-file ini. Nama file
default adalah current.pov, dan direktori default adalah eulerhome(),
biasanya c:\textbackslash{}Users\textbackslash{}Username\textbackslash{}Euler. Povray menghasilkan file PNG, yang
dapat dimuat oleh Euler ke dalam buku catatan. Untuk membersihkan
file-file ini, gunakan povclear().

Fungsi pov3d memiliki semangat yang sama dengan plot3d. Ini dapat
menghasilkan grafik fungsi f(x,y), atau permukaan dengan koordinat
X,Y,Z dalam matriks, termasuk garis level opsional. Fungsi ini memulai
raytracer secara otomatis, dan memuat adegan ke dalam notebook Euler.

Selain pov3d(), ada banyak fungsi yang menghasilkan objek Povray.
Fungsi-fungsi ini mengembalikan string, yang berisi kode Povray untuk
objek. Untuk menggunakan fungsi ini, mulai file Povray dengan
povstart(). Kemudian gunakan writeln(...) untuk menulis objek ke file
adegan. Terakhir, akhiri file dengan povend(). Secara default,
raytracer akan dimulai, dan PNG akan dimasukkan ke dalam notebook
Euler.

Fungsi objek memiliki parameter yang disebut "look", yang membutuhkan
string dengan kode Povray untuk tekstur dan hasil akhir objek. Fungsi
povlook() dapat digunakan untuk menghasilkan string ini. Ini memiliki
parameter untuk warna, transparansi, Phong Shading dll.

Perhatikan bahwa alam semesta Povray memiliki sistem koordinat lain.
Antarmuka ini menerjemahkan semua koordinat ke sistem Povray. Jadi
Anda dapat terus berpikir dalam sistem koordinat Euler dengan z
menunjuk vertikal ke atas, a nd x,y,z sumbu dalam arti tangan kanan.\\
Anda perlu memuat file povray.
\end{eulercomment}
\begin{eulerprompt}
>load povray;
\end{eulerprompt}
\begin{eulercomment}
Pastikan, direktori bin Povray ada di jalur. Jika tidak, edit variabel
berikut sehingga berisi path ke povray yang dapat dieksekusi.
\end{eulercomment}
\begin{eulerprompt}
>defaultpovray="C:\(\backslash\)Program Files\(\backslash\)POV-Ray\(\backslash\)v3.7\(\backslash\)bin\(\backslash\)pvengine.exe"
\end{eulerprompt}
\begin{euleroutput}
  C:\(\backslash\)Program Files\(\backslash\)POV-Ray\(\backslash\)v3.7\(\backslash\)bin\(\backslash\)pvengine.exe
\end{euleroutput}
\begin{eulercomment}
Untuk kesan pertama, kami memplot fungsi sederhana. Perintah berikut
menghasilkan file povray di direktori pengguna Anda, dan menjalankan
Povray untuk ray tracing file ini.

Jika Anda memulai perintah berikut, GUI Povray akan terbuka,
menjalankan file, dan menutup secara otomatis. Karena alasan keamanan,
Anda akan ditanya, apakah Anda ingin mengizinkan file exe untuk
dijalankan. Anda dapat menekan batal untuk menghentikan pertanyaan
lebih lanjut. Anda mungkin harus menekan OK di jendela Povray untuk
mengakui dialog awal Povray.
\end{eulercomment}
\begin{eulerprompt}
>pov3d("x^2+y^2",zoom=3);
\end{eulerprompt}
\eulerimg{27}{images/Davina Safa Felisa 1-6-349.png}
\begin{eulercomment}
Kita dapat membuat fungsi menjadi transparan dan menambahkan hasil
akhir lainnya. Kami juga dapat menambahkan garis level ke plot fungsi.
\end{eulercomment}
\begin{eulerprompt}
>pov3d("x^2+y^3",axiscolor=red,angle=20°, ...
>  look=povlook(blue,0.2),level=-1:0.5:1,zoom=3.8);
\end{eulerprompt}
\eulerimg{27}{images/Davina Safa Felisa 1-6-350.png}
\begin{eulercomment}
Terkadang perlu untuk mencegah penskalaan fungsi, dan menskalakan
fungsi dengan tangan.

Kami memplot himpunan titik di bidang kompleks, di mana produk dari
jarak ke 1 dan -1 sama dengan 1.
\end{eulercomment}
\begin{eulerprompt}
>pov3d("((x-1)^2+y^2)*((x+1)^2+y^2)/40",r=1.5, ...
>  angle=-120°,level=1/40,dlevel=0.005,light=[-1,1,1],height=45°,n=50, ...
>  <fscale,zoom=3.8);
\end{eulerprompt}
\eulerimg{27}{images/Davina Safa Felisa 1-6-351.png}
\eulerheading{Merencanakan dengan Koordinat}
\begin{eulercomment}
Alih-alih fungsi, kita dapat memplot dengan koordinat. Seperti pada
plot3d, kita membutuhkan tiga matriks untuk mendefinisikan objek.

Dalam contoh kita memutar fungsi di sekitar sumbu z.
\end{eulercomment}
\begin{eulerprompt}
>function f(x) := x^3-x+1; ...
>x=-1:0.01:1; t=linspace(0,2pi,8)'; ...
>Z=x; X=cos(t)*f(x); Y=sin(t)*f(x); ...
>pov3d(X,Y,Z,angle=40°,height=20°,axis=0,zoom=4,light=[10,-5,5]);
\end{eulerprompt}
\eulerimg{27}{images/Davina Safa Felisa 1-6-352.png}
\begin{eulercomment}
Dalam contoh berikut, kami memplot gelombang teredam. Kami
menghasilkan gelombang dengan bahasa matriks Euler.

Kami juga menunjukkan, bagaimana objek tambahan dapat ditambahkan ke
adegan pov3d. Untuk pembuatan objek, lihat contoh berikut. Perhatikan
bahwa plot3d menskalakan plot, sehingga cocok dengan kubus satuan.
\end{eulercomment}
\begin{eulerprompt}
>r=linspace(0,1,80); phi=linspace(0,2pi,80)'; ...
>x=r*cos(phi); y=r*sin(phi); z=exp(-5*r)*cos(8*pi*r)/3;  ...
>pov3d(x,y,z,zoom=5,axis=0,add=povsphere([0,0,0.5],0.1,povlook(green)), ...
>  w=500,h=300);
\end{eulerprompt}
\eulerimg{16}{images/Davina Safa Felisa 1-6-353.png}
\begin{eulercomment}
Dengan metode bayangan canggih dari Povray, sangat sedikit titik yang
dapat menghasilkan permukaan yang sangat halus. Hanya di perbatasan
dan dalam bayang-bayang triknya mungkin menjadi jelas.

Untuk ini, kita perlu menambahkan vektor normal di setiap titik
matriks.
\end{eulercomment}
\begin{eulerprompt}
>Z &= x^2*y^3
\end{eulerprompt}
\begin{euleroutput}
  
                                   2  3
                                  x  y
  
\end{euleroutput}
\begin{eulercomment}
Persamaan permukaannya adalah [x,y,Z]. Kami menghitung dua turunan ke
x dan y ini dan mengambil produk silang sebagai normal.
\end{eulercomment}
\begin{eulerprompt}
>dx &= diff([x,y,Z],x); dy &= diff([x,y,Z],y);
\end{eulerprompt}
\begin{eulercomment}
Kami mendefinisikan normal sebagai produk silang dari turunan ini, dan
mendefinisikan fungsi koordinat.
\end{eulercomment}
\begin{eulerprompt}
>N &= crossproduct(dx,dy); NX &= N[1]; NY &= N[2]; NZ &= N[3]; N,
\end{eulerprompt}
\begin{euleroutput}
  
                                 3       2  2
                         [- 2 x y , - 3 x  y , 1]
  
\end{euleroutput}
\begin{eulercomment}
Kami hanya menggunakan 25 poin.
\end{eulercomment}
\begin{eulerprompt}
>x=-1:0.5:1; y=x';
>pov3d(x,y,Z(x,y),angle=10°, ...
>  xv=NX(x,y),yv=NY(x,y),zv=NZ(x,y),<shadow);
\end{eulerprompt}
\eulerimg{27}{images/Davina Safa Felisa 1-6-354.png}
\begin{eulercomment}
Berikut ini adalah simpul Trefoil yang dilakukan oleh A. Busser di
Povray. Ada versi yang ditingkatkan dari ini dalam contoh.

Lihat: Contoh\textbackslash{}Trefoil Simpul \textbar{} Simpul trefoil

Untuk tampilan yang bagus dengan tidak terlalu banyak titik, kami
menambahkan vektor normal di sini. Kami menggunakan Maxima untuk
menghitung normal bagi kami. Pertama, ketiga fungsi koordinat sebagai
ekspresi simbolik.
\end{eulercomment}
\begin{eulerprompt}
>X &= ((4+sin(3*y))+cos(x))*cos(2*y); ...
>Y &= ((4+sin(3*y))+cos(x))*sin(2*y); ...
>Z &= sin(x)+2*cos(3*y);
\end{eulerprompt}
\begin{eulercomment}
Kemudian kedua vektor turunan ke x dan y.
\end{eulercomment}
\begin{eulerprompt}
>dx &= diff([X,Y,Z],x); dy &= diff([X,Y,Z],y);
\end{eulerprompt}
\begin{eulercomment}
Sekarang normal, yang merupakan produk silang dari dua turunan.
\end{eulercomment}
\begin{eulerprompt}
>dn &= crossproduct(dx,dy);
\end{eulerprompt}
\begin{eulercomment}
Kami sekarang mengevaluasi semua ini secara numerik.
\end{eulercomment}
\begin{eulerprompt}
>x:=linspace(-%pi,%pi,40); y:=linspace(-%pi,%pi,100)';
\end{eulerprompt}
\begin{eulercomment}
Vektor normal adalah evaluasi dari ekspresi simbolik dn[i] untuk
i=1,2,3. Sintaks untuk ini adalah \&"expression"(parameters). Ini
adalah alternatif dari metode pada contoh sebelumnya, di mana kita
mendefinisikan ekspresi simbolik NX, NY, NZ terlebih dahulu.
\end{eulercomment}
\begin{eulerprompt}
>pov3d(X(x,y),Y(x,y),Z(x,y),axis=0,zoom=5,w=450,h=350, ...
>  <shadow,look=povlook(gray), ...
>  xv=&"dn[1]"(x,y), yv=&"dn[2]"(x,y), zv=&"dn[3]"(x,y));
\end{eulerprompt}
\eulerimg{21}{images/Davina Safa Felisa 1-6-355.png}
\begin{eulercomment}
Kami juga dapat menghasilkan grid dalam 3D.
\end{eulercomment}
\begin{eulerprompt}
>povstart(zoom=4); ...
>x=-1:0.5:1; r=1-(x+1)^2/6; ...
>t=(0°:30°:360°)'; y=r*cos(t); z=r*sin(t); ...
>writeln(povgrid(x,y,z,d=0.02,dballs=0.05)); ...
>povend();
\end{eulerprompt}
\eulerimg{27}{images/Davina Safa Felisa 1-6-356.png}
\begin{eulercomment}
Dengan povgrid(), kurva dimungkinkan.
\end{eulercomment}
\begin{eulerprompt}
>povstart(center=[0,0,1],zoom=3.6); ...
>t=linspace(0,2,1000); r=exp(-t); ...
>x=cos(2*pi*10*t)*r; y=sin(2*pi*10*t)*r; z=t; ...
>writeln(povgrid(x,y,z,povlook(red))); ...
>writeAxis(0,2,axis=3); ...
>povend();
\end{eulerprompt}
\eulerimg{27}{images/Davina Safa Felisa 1-6-357.png}
\eulerheading{Objek Povray}
\begin{eulercomment}
Di atas, kami menggunakan pov3d untuk memplot permukaan. Antarmuka
povray di Euler juga dapat menghasilkan objek Povray. Objek-objek ini
disimpan sebagai string di Euler, dan perlu ditulis ke file Povray.

Kami memulai output dengan povstart().
\end{eulercomment}
\begin{eulerprompt}
>povstart(zoom=4);
\end{eulerprompt}
\begin{eulercomment}
Pertama kita mendefinisikan tiga silinder, dan menyimpannya dalam
string di Euler.

Fungsi povx() dll. hanya mengembalikan vektor [1,0,0], yang dapat
digunakan sebagai gantinya.
\end{eulercomment}
\begin{eulerprompt}
>c1=povcylinder(-povx,povx,1,povlook(red)); ...
>c2=povcylinder(-povy,povy,1,povlook(green)); ...
>c3=povcylinder(-povz,povz,1,povlook(blue)); ...
\end{eulerprompt}
\begin{eulercomment}
Pertama kita mendefinisikan tiga silinder, dan menyimpannya dalam
string di Euler.

Fungsi povx() dll. hanya mengembalikan vektor [1,0,0], yang dapat
digunakan sebagai pengingat.
\end{eulercomment}
\begin{eulerprompt}
>c1
\end{eulerprompt}
\begin{euleroutput}
  cylinder \{ <-1,0,0>, <1,0,0>, 1
   texture \{ pigment \{ color rgb <0.564706,0.0627451,0.0627451> \}  \} 
   finish \{ ambient 0.2 \} 
   \}
\end{euleroutput}
\begin{eulercomment}
Seperti yang Anda lihat, kami menambahkan tekstur ke objek dalam tiga
warna berbeda.

Itu dilakukan oleh povlook(), yang mengembalikan string dengan kode
Povray yang relevan. Kita dapat menggunakan warna Euler default, atau
menentukan warna kita sendiri. Kami juga dapat menambahkan
transparansi, atau mengubah cahaya sekitar.
\end{eulercomment}
\begin{eulerprompt}
>povlook(rgb(0.1,0.2,0.3),0.1,0.5)
\end{eulerprompt}
\begin{euleroutput}
   texture \{ pigment \{ color rgbf <0.101961,0.2,0.301961,0.1> \}  \} 
   finish \{ ambient 0.5 \} 
  
\end{euleroutput}
\begin{eulercomment}
Sekarang kita mendefinisikan objek persimpangan, dan menulis hasilnya
ke file.
\end{eulercomment}
\begin{eulerprompt}
>writeln(povintersection([c1,c2,c3]));
\end{eulerprompt}
\begin{eulercomment}
Persimpangan tiga silinder sulit untuk divisualisasikan, jika Anda
belum pernah melihatnya sebelumnya.
\end{eulercomment}
\begin{eulerprompt}
>povend;
\end{eulerprompt}
\eulerimg{27}{images/Davina Safa Felisa 1-6-358.png}
\begin{eulercomment}
Fungsi berikut menghasilkan fraktal secara rekursif.

Fungsi pertama menunjukkan, bagaimana Euler menangani objek Povray
sederhana. Fungsi povbox() mengembalikan string, yang berisi koordinat
kotak, tekstur, dan hasil akhir.
\end{eulercomment}
\begin{eulerprompt}
>function onebox(x,y,z,d) := povbox([x,y,z],[x+d,y+d,z+d],povlook());
>function fractal (x,y,z,h,n) ...
\end{eulerprompt}
\begin{eulerudf}
   if n==1 then writeln(onebox(x,y,z,h));
   else
     h=h/3;
     fractal(x,y,z,h,n-1);
     fractal(x+2*h,y,z,h,n-1);
     fractal(x,y+2*h,z,h,n-1);
     fractal(x,y,z+2*h,h,n-1);
     fractal(x+2*h,y+2*h,z,h,n-1);
     fractal(x+2*h,y,z+2*h,h,n-1);
     fractal(x,y+2*h,z+2*h,h,n-1);
     fractal(x+2*h,y+2*h,z+2*h,h,n-1);
     fractal(x+h,y+h,z+h,h,n-1);
   endif;
  endfunction
\end{eulerudf}
\begin{eulerprompt}
>povstart(fade=10,<shadow);
>fractal(-1,-1,-1,2,4);
>povend();
\end{eulerprompt}
\eulerimg{27}{images/Davina Safa Felisa 1-6-359.png}
\begin{eulercomment}
Perbedaan memungkinkan memotong satu objek dari yang lain. Seperti
persimpangan, ada bagian dari objek CSG Povray.
\end{eulercomment}
\begin{eulerprompt}
>povstart(light=[5,-5,5],fade=10);
\end{eulerprompt}
\begin{eulercomment}
Untuk demonstrasi ini, kami mendefinisikan objek di Povray, alih-alih
menggunakan string di Euler. Definisi ditulis ke file segera.

Koordinat kotak -1 berarti [-1,-1,-1].
\end{eulercomment}
\begin{eulerprompt}
>povdefine("mycube",povbox(-1,1));
\end{eulerprompt}
\begin{eulercomment}
Kita dapat menggunakan objek ini di povobject(), yang mengembalikan
string seperti biasa.
\end{eulercomment}
\begin{eulerprompt}
>c1=povobject("mycube",povlook(red));
\end{eulerprompt}
\begin{eulercomment}
Kami menghasilkan kubus kedua, dan memutar dan menskalakannya sedikit.
\end{eulercomment}
\begin{eulerprompt}
>c2=povobject("mycube",povlook(yellow),translate=[1,1,1], ...
>  rotate=xrotate(10°)+yrotate(10°), scale=1.2);
\end{eulerprompt}
\begin{eulercomment}
Kemudian kita ambil selisih kedua benda tersebut.
\end{eulercomment}
\begin{eulerprompt}
>writeln(povdifference(c1,c2));
\end{eulerprompt}
\begin{eulercomment}
Sekarang tambahkan tiga sumbu.
\end{eulercomment}
\begin{eulerprompt}
>writeAxis(-1.2,1.2,axis=1); ...
>writeAxis(-1.2,1.2,axis=2); ...
>writeAxis(-1.2,1.2,axis=4); ...
>povend();
\end{eulerprompt}
\eulerimg{27}{images/Davina Safa Felisa 1-6-360.png}
\eulerheading{Fungsi Implisit}
\begin{eulercomment}
Povray dapat memplot himpunan di mana f(x,y,z)=0, seperti parameter
implisit di plot3d. Namun, hasilnya terlihat jauh lebih baik.

Sintaks untuk fungsinya sedikit berbeda. Anda tidak dapat menggunakan
output dari ekspresi Maxima atau Euler.
\end{eulercomment}
\begin{eulerprompt}
>povstart(angle=70°,height=50°,zoom=4);
\end{eulerprompt}
\begin{eulercomment}
Buat permukaan implisit. Perhatikan sintaks yang berbeda dalam
ekspresi.
\end{eulercomment}
\begin{eulerprompt}
>writeln(povsurface("pow(x,2)*y-pow(y,3)-pow(z,2)",povlook(green))); ...
>writeAxes(); ...
>povend();
\end{eulerprompt}
\eulerimg{27}{images/Davina Safa Felisa 1-6-361.png}
\eulerheading{Objek Jala}
\begin{eulercomment}
Dalam contoh ini, kami menunjukkan cara membuat objek mesh, dan
menggambarnya dengan informasi tambahan.

Kami ingin memaksimalkan xy di bawah kondisi x+y=1 dan menunjukkan
sentuhan tangensial dari garis level.
\end{eulercomment}
\begin{eulerprompt}
>povstart(angle=-10°,center=[0.5,0.5,0.5],zoom=7);
\end{eulerprompt}
\begin{eulercomment}
Kami tidak dapat menyimpan objek dalam string seperti sebelumnya,
karena terlalu besar. Jadi kita mendefinisikan objek dalam file Povray
menggunakan #declare. Fungsi povtriangle() melakukan ini secara
otomatis. Itu dapat menerima vektor normal seperti pov3d().

Berikut ini mendefinisikan objek mesh, dan langsung menulisnya ke
dalam file.
\end{eulercomment}
\begin{eulerprompt}
>x=0:0.02:1; y=x'; z=x*y; vx=-y; vy=-x; vz=1;
>mesh=povtriangles(x,y,z,"",vx,vy,vz);
\end{eulerprompt}
\begin{eulercomment}
Sekarang kita mendefinisikan dua buah cakram, yang akan berpotongan
dengan permukaan.
\end{eulercomment}
\begin{eulerprompt}
>cl=povdisc([0.5,0.5,0],[1,1,0],2); ...
>ll=povdisc([0,0,1/4],[0,0,1],2);
\end{eulerprompt}
\begin{eulercomment}
Tuliskan permukaan dikurangi dengan dua cakram.
\end{eulercomment}
\begin{eulerprompt}
>writeln(povdifference(mesh,povunion([cl,ll]),povlook(green)));
\end{eulerprompt}
\begin{eulercomment}
Tuliskan dua persimpangan
\end{eulercomment}
\begin{eulerprompt}
>writeln(povintersection([mesh,cl],povlook(red))); ...
>writeln(povintersection([mesh,ll],povlook(gray)));
\end{eulerprompt}
\begin{eulercomment}
Menuliskan sebuah titik secara maksimal.
\end{eulercomment}
\begin{eulerprompt}
>writeln(povpoint([1/2,1/2,1/4],povlook(gray),size=2*defaultpointsize));
\end{eulerprompt}
\begin{eulercomment}
Tambahkan sumbu dan selesaikan.
\end{eulercomment}
\begin{eulerprompt}
>writeAxes(0,1,0,1,0,1,d=0.015); ...
>povend();
\end{eulerprompt}
\eulerimg{27}{images/Davina Safa Felisa 1-6-362.png}
\eulerheading{Anaglyphs di Povray}
\begin{eulercomment}
Untuk menghasilkan anaglyph untuk kacamata merah/cyan, Povray harus
dijalankan dua kali\\
dari posisi kamera yang berbeda. Ini menghasilkan dua file Povray dan
dua file PNG, yang dimuat dengan fungsi loadanaglyph().

Tentu saja, Anda membutuhkan kacamata merah/cyan untuk melihat contoh
berikut\\
dengan benar.

Fungsi pov3d() memiliki sebuah saklar sederhana untuk menghasilkan
anaglyph.
\end{eulercomment}
\begin{eulerprompt}
>pov3d("-exp(-x^2-y^2)/2",r=2,height=45°,>anaglyph, ...
>  center=[0,0,0.5],zoom=3.5);
\end{eulerprompt}
\eulerimg{27}{images/Davina Safa Felisa 1-6-363.png}
\begin{eulercomment}
Jika Anda membuat adegan dengan objek, Anda perlu memasukkan pembuatan
adegan ke dalam sebuah fungsi, dan jalankan dua kali dengan nilai yang
berbeda untuk parameter anaglyph.
\end{eulercomment}
\begin{eulerprompt}
>function myscene ...
\end{eulerprompt}
\begin{eulerudf}
    s=povsphere(povc,1);
    cl=povcylinder(-povz,povz,0.5);
    clx=povobject(cl,rotate=xrotate(90°));
    cly=povobject(cl,rotate=yrotate(90°));
    c=povbox([-1,-1,0],1);
    un=povunion([cl,clx,cly,c]);
    obj=povdifference(s,un,povlook(red));
    writeln(obj);
    writeAxes();
  endfunction
\end{eulerudf}
\begin{eulercomment}
Fungsi povanaglyph() melakukan semua ini. Parameter-parameternya
adalah seperti di dalam\\
povstart() dan povend() digabungkan.
\end{eulercomment}
\begin{eulerprompt}
>povanaglyph("myscene",zoom=4.5);
\end{eulerprompt}
\eulerimg{27}{images/Davina Safa Felisa 1-6-364.png}
\eulerheading{Mendefinisikan Objek Sendiri}
\begin{eulercomment}
Antarmuka povray Euler berisi banyak objek. Tapi Anda\\
idak terbatas pada objek-objek tersebut. Anda dapat membuat objek
sendiri, yang menggabungkan objek lain, atau objek yang benar-benar
baru.

Kami mendemonstrasikan sebuah torus. Perintah Povray untuk ini adalah
“torus”. Jadi kami mengembalikan sebuah string dengan perintah ini dan
parameternya. Perhatikan bahwa torus selalu berpusat pada titik asal.
\end{eulercomment}
\begin{eulerprompt}
>function povdonat (r1,r2,look="") ...
\end{eulerprompt}
\begin{eulerudf}
    return "torus \{"+r1+","+r2+look+"\}";
  endfunction
\end{eulerudf}
\begin{eulercomment}
Ini adalah torus pertama kita.
\end{eulercomment}
\begin{eulerprompt}
>t1=povdonat(0.8,0.2)
\end{eulerprompt}
\begin{euleroutput}
  torus \{0.8,0.2\}
\end{euleroutput}
\begin{eulercomment}
Mari kita gunakan objek ini untuk membuat torus kedua, ditranslasikan
dan diputar.
\end{eulercomment}
\begin{eulerprompt}
>t2=povobject(t1,rotate=xrotate(90°),translate=[0.8,0,0])
\end{eulerprompt}
\begin{euleroutput}
  object \{ torus \{0.8,0.2\}
   rotate 90 *x 
   translate <0.8,0,0>
   \}
\end{euleroutput}
\begin{eulercomment}
Sekarang kita tempatkan objek-objek ini ke dalam sebuah scene. Untuk
tampilan, kita menggunakan Phong
\end{eulercomment}
\begin{eulerprompt}
>povstart(center=[0.4,0,0],angle=0°,zoom=3.8,aspect=1.5); ...
>writeln(povobject(t1,povlook(green,phong=1))); ...
>writeln(povobject(t2,povlook(green,phong=1))); ...
\end{eulerprompt}
\begin{eulerttcomment}
 >povend();
\end{eulerttcomment}
\begin{eulercomment}
memanggil program Povray. Namun, jika terjadi kesalahan program ini
tidak\\
menampilkan kesalahan. Oleh karena itu, Anda harus menggunakan

\end{eulercomment}
\begin{eulerttcomment}
 >povend(<exit);
\end{eulerttcomment}
\begin{eulercomment}

jika ada yang tidak berhasil. Ini akan membiarkan jendela Povray
terbuka.
\end{eulercomment}
\begin{eulerprompt}
>povend(h=320,w=480);
\end{eulerprompt}
\eulerimg{18}{images/Davina Safa Felisa 1-6-365.png}
\begin{eulercomment}
Berikut adalah contoh yang lebih rumit. Kami menyelesaikan

\end{eulercomment}
\begin{eulerformula}
\[
Ax \le b, \quad x \ge 0, \quad c.x \to \text{Max.}
\]
\end{eulerformula}
\begin{eulercomment}
dan menunjukkan titik-titik yang layak dan optimal dalam plot 3D.
\end{eulercomment}
\begin{eulerprompt}
>A=[10,8,4;5,6,8;6,3,2;9,5,6];
>b=[10,10,10,10]';
>c=[1,1,1];
\end{eulerprompt}
\begin{eulercomment}
Pertama, mari kita periksa, apakah contoh ini memiliki solusi atau
tidak.
\end{eulercomment}
\begin{eulerprompt}
>x=simplex(A,b,c,>max,>check)'
\end{eulerprompt}
\begin{euleroutput}
  [0,  1,  0.5]
\end{euleroutput}
\begin{eulercomment}
Ya, ada.


Selanjutnya kita mendefinisikan dua objek. Yang pertama adalah pesawat

\end{eulercomment}
\begin{eulerformula}
\[
a \cdot x \le b
\]
\end{eulerformula}
\begin{eulerprompt}
>function oneplane (a,b,look="") ...
\end{eulerprompt}
\begin{eulerudf}
    return povplane(a,b,look)
  endfunction
\end{eulerudf}
\begin{eulercomment}
Kemudian kita mendefinisikan perpotongan dari semua setengah ruang dan
sebuah kubus.
\end{eulercomment}
\begin{eulerprompt}
>function adm (A, b, r, look="") ...
\end{eulerprompt}
\begin{eulerudf}
    ol=[];
    loop 1 to rows(A); ol=ol|oneplane(A[#],b[#]); end;
    ol=ol|povbox([0,0,0],[r,r,r]);
    return povintersection(ol,look);
  endfunction
\end{eulerudf}
\begin{eulercomment}
Kita sekarang dapat memplot adegan.
\end{eulercomment}
\begin{eulerprompt}
>povstart(angle=120°,center=[0.5,0.5,0.5],zoom=3.5); ...
>writeln(adm(A,b,2,povlook(green,0.4))); ...
>writeAxes(0,1.3,0,1.6,0,1.5); ...
\end{eulerprompt}
\begin{eulercomment}
Berikut ini adalah sebuah lingkaran di sekitar titik optimal.
\end{eulercomment}
\begin{eulerprompt}
>writeln(povintersection([povsphere(x,0.5),povplane(c,c.x')], ...
>  povlook(red,0.9)));
\end{eulerprompt}
\begin{eulercomment}
Dan kesalahan ke arah optimum
\end{eulercomment}
\begin{eulerprompt}
>writeln(povarrow(x,c*0.5,povlook(red)));
\end{eulerprompt}
\begin{eulercomment}
Kami menambahkan teks ke layar. Teks hanyalah sebuah objek 3D. Kita
perlu menempatkan dan mengubahnya sesuai dengan pandangan kita.
\end{eulercomment}
\begin{eulerprompt}
>writeln(povtext("Linear Problem",[0,0.2,1.3],size=0.05,rotate=125°)); ...
>povend();
\end{eulerprompt}
\eulerimg{27}{images/Davina Safa Felisa 1-6-368.png}
\eulersubheading{Contoh Lainnya}
\begin{eulercomment}
Anda dapat menemukan beberapa contoh lain untuk Povray di Euler dalam
file-file berikut.

Lihat: Examples/Dandelin Spheres\\
Lihat: Contoh/Contoh/Donat Matematika\\
Lihat: Contoh/Simpul Trefoil\\
Lihat: Contoh/Optimalisasi dengan Penskalaan Affine

\end{eulercomment}
\eulersubheading{Contoh Soal}
\begin{eulercomment}
Grafik dari fungsi f dengan dua variabel yang dimaksud adalah grafik
dari persamaan z = f(x,y). Biasanya grafik ini berupa permukaan dan
karena setiap (x,y) di daerah asal hanya berpadanan dengan satu nilai
z, maka setiap garis tegaklurus bidang-xy memotong permukaan pada
paling banyak satu titik.\\
Soal :\\
1. Grafik merupakan sebuah paraboloida\\
\end{eulercomment}
\begin{eulerformula}
\[
f(x,y) = y^2-x^2
\]
\end{eulerformula}
\begin{eulercomment}
2.\\
\end{eulercomment}
\begin{eulerformula}
\[
z = -4x^3y^2
\]
\end{eulerformula}
\begin{eulercomment}
3.\\
\end{eulercomment}
\begin{eulerformula}
\[
z = xy exp(-x^2-y^2)
\]
\end{eulerformula}
\begin{eulercomment}
4.\\
\end{eulercomment}
\begin{eulerformula}
\[
z = x - 1/8x^3 - 1/3y^2
\]
\end{eulerformula}
\begin{eulerprompt}
> aspect(1.5); plot3d("y^2-x^2"):
\end{eulerprompt}
\eulerimg{17}{images/Davina Safa Felisa 1-6-373.png}
\begin{eulerprompt}
> aspect(1.5); plot3d("-4x^3*y^2"):
\end{eulerprompt}
\eulerimg{17}{images/Davina Safa Felisa 1-6-374.png}
\begin{eulerprompt}
> plot3d("x*y*exp(-x^2-y^2)",r=2,<fscale,<scale,distance=13,height=20°, ...
>  center=[0,0,-0.2],frame=3):
\end{eulerprompt}
\eulerimg{17}{images/Davina Safa Felisa 1-6-375.png}
\begin{eulerprompt}
> aspect(1.5); plot3d("x-1/8*x^3-1/3*y^2"): 
\end{eulerprompt}
\eulerimg{17}{images/Davina Safa Felisa 1-6-376.png}
\eulerheading{Kalkulus dengan EMT}
\begin{eulercomment}
Materi Kalkulus mencakup di antaranya:

- Fungsi (fungsi aljabar, trigonometri, eksponensial, logaritma,
komposisi fungsi)\\
- Limit Fungsi,\\
- Turunan Fungsi,\\
- Integral Tak Tentu,\\
- Integral Tentu dan Aplikasinya,\\
- Barisan dan Deret (kekonvergenan barisan dan deret).

EMT (bersama Maxima) dapat digunakan untuk melakukan semua perhitungan
di dalam kalkulus, baik secara numerik maupun analitik (eksak).

\end{eulercomment}
\eulersubheading{Mendefinisikan Fungsi}
\begin{eulercomment}
Terdapat beberapa cara mendefinisikan fungsi pada EMT, yakni:

- Menggunakan format nama\_fungsi := rumus fungsi (untuk fungsi
numerik),\\
- Menggunakan format nama\_fungsi \&= rumus fungsi (untuk fungsi
simbolik, namun dapat dihitung secara numerik),\\
- Menggunakan format nama\_fungsi \&\&= rumus fungsi (untuk fungsi
simbolik murni, tidak dapat dihitung langsung),\\
- Fungsi sebagai program EMT.

Setiap format harus diawali dengan perintah function (bukan sebagai
ekspresi).

Berikut adalah adalah beberapa contoh cara mendefinisikan fungsi.
\end{eulercomment}
\begin{eulerprompt}
>function f(x) := 2*x^2+exp(sin(x)) // fungsi numerik
>f(0), f(1), f(pi)
\end{eulerprompt}
\begin{euleroutput}
  1
  4.31977682472
  20.7392088022
\end{euleroutput}
\begin{eulerprompt}
>function g(x) := sqrt(x^2-3*x)/(x+1)
>g(3)
\end{eulerprompt}
\begin{euleroutput}
  0
\end{euleroutput}
\begin{eulerprompt}
>g(0)
\end{eulerprompt}
\begin{euleroutput}
  0
\end{euleroutput}
\begin{eulerprompt}
>g(f(1))
\end{eulerprompt}
\begin{euleroutput}
  0.448835801122
\end{euleroutput}
\begin{eulerprompt}
>f(g(5)) // komposisi fungsi
\end{eulerprompt}
\begin{euleroutput}
  2.20920171961
\end{euleroutput}
\begin{eulerprompt}
>g(f(5))
\end{eulerprompt}
\begin{euleroutput}
  0.950898070639
\end{euleroutput}
\begin{eulerprompt}
>function h(x) := f(g(x)) // definisi komposisi fungsi
>h(5) // sama dengan f(g(5))
\end{eulerprompt}
\begin{euleroutput}
  2.20920171961
\end{euleroutput}
\begin{eulerprompt}
>f(0:10) // nilai-nilai f(1), f(2), ..., f(10)
\end{eulerprompt}
\begin{euleroutput}
  [1,  4.31978,  10.4826,  19.1516,  32.4692,  50.3833,  72.7562,
  99.929,  130.69,  163.51,  200.58]
\end{euleroutput}
\begin{eulerprompt}
>fmap(0:10) // sama dengan f(0:10), berlaku untuk semua fungsi
\end{eulerprompt}
\begin{euleroutput}
  [1,  4.31978,  10.4826,  19.1516,  32.4692,  50.3833,  72.7562,
  99.929,  130.69,  163.51,  200.58]
\end{euleroutput}
\begin{eulercomment}
Misalkan kita akan mendefinisikan fungsi

\end{eulercomment}
\begin{eulerformula}
\[
f(x) = \begin{cases} x^3 & x>0 \\ x^2 & x\le 0. \end{cases}
\]
\end{eulerformula}
\begin{eulercomment}
Fungsi tersebut tidak dapat didefinisikan sebagai fungsi numerik
secara "inline" menggunakan format :=, melainkan didefinisikan sebagai
program. Perhatikan, kata "map" digunakan agar fungsi dapat menerima
vektor sebagai input, dan hasilnya berupa vektor. Jika tanpa kata
"map" fungsinya hanya dapat menerima input satu nilai.
\end{eulercomment}
\begin{eulerprompt}
>function map f(x) ...
\end{eulerprompt}
\begin{eulerudf}
    if x>0 then return x^3
    else return x^2
    endif;
  endfunction
\end{eulerudf}
\begin{eulerprompt}
>f(1)
\end{eulerprompt}
\begin{euleroutput}
  1
\end{euleroutput}
\begin{eulerprompt}
>f(-2)
\end{eulerprompt}
\begin{euleroutput}
  4
\end{euleroutput}
\begin{eulerprompt}
>f(-5:5)
\end{eulerprompt}
\begin{euleroutput}
  [25,  16,  9,  4,  1,  0,  1,  8,  27,  64,  125]
\end{euleroutput}
\begin{eulerprompt}
>aspect(1.5); plot2d("f(x)",-5,5):
\end{eulerprompt}
\eulerimg{17}{images/Davina Safa Felisa 1-6-378.png}
\begin{eulerprompt}
>function f(x) &= 2*E^x // fungsi simbolik
\end{eulerprompt}
\begin{euleroutput}
  
                                      x
                                   2 E
  
\end{euleroutput}
\begin{eulerprompt}
>function g(x) &= 3*x+1
\end{eulerprompt}
\begin{euleroutput}
  
                                 3 x + 1
  
\end{euleroutput}
\begin{eulerprompt}
>function h(x) &= f(g(x)) // komposisi fungsi
\end{eulerprompt}
\begin{euleroutput}
  
                                   3 x + 1
                                2 E
  
\end{euleroutput}
\eulerheading{Latihan}
\begin{eulercomment}
Bukalah buku Kalkulus. Cari dan pilih beberapa (paling sedikit 5
fungsi berbeda tipe/bentuk/jenis) fungsi dari buku tersebut, kemudian
definisikan di EMT pada baris-baris perintah berikut (jika perlu
tambahkan lagi). Untuk setiap fungsi, hitung beberapa nilainya, baik
untuk satu nilai maupun vektor. Gambar grafik tersebut.

Juga, carilah fungsi beberapa (dua) variabel. Lakukan hal sama seperti
di atas.

Jawab:\\
\end{eulercomment}
\begin{eulerformula}
\[
\text{A). FUNGSI 1 VARIABEL}
\]
\end{eulerformula}
\begin{eulercomment}
1. Fungsi 1
\end{eulercomment}
\begin{eulerprompt}
>function k(x) := x*(x^5+3)^3
>k(3), k(5), k(7)
\end{eulerprompt}
\begin{euleroutput}
  44660808
  153027765760
  3.3250729687e+13
\end{euleroutput}
\begin{eulerprompt}
>kmap(-3:3)
\end{eulerprompt}
\begin{euleroutput}
  [4.1472e+07,  48778,  -8,  0,  64,  85750,  4.46608e+07]
\end{euleroutput}
\begin{eulerprompt}
>plot2d("k(x)"):
\end{eulerprompt}
\eulerimg{17}{images/Davina Safa Felisa 1-6-380.png}
\begin{eulercomment}
2. Fungsi 2
\end{eulercomment}
\begin{eulerprompt}
>function m(x) := (x)^4/(3-x^2) 
>m(2), m(-2), m(1)
\end{eulerprompt}
\begin{euleroutput}
  -16
  -16
  0.5
\end{euleroutput}
\begin{eulerprompt}
>mmap(-5:-5)
\end{eulerprompt}
\begin{euleroutput}
  -28.4090909091
\end{euleroutput}
\begin{eulerprompt}
>plot2d("m(x)"):
\end{eulerprompt}
\eulerimg{17}{images/Davina Safa Felisa 1-6-381.png}
\begin{eulercomment}
3. Fungsi 3
\end{eulercomment}
\begin{eulerprompt}
>function n(x) := 3*x/(x+5)+2
>n(2), n(-1), n(-3), n(4)
\end{eulerprompt}
\begin{euleroutput}
  2.85714285714
  1.25
  -2.5
  3.33333333333
\end{euleroutput}
\begin{eulerprompt}
>nmap(2:5)
\end{eulerprompt}
\begin{euleroutput}
  [2.85714,  3.125,  3.33333,  3.5]
\end{euleroutput}
\begin{eulerprompt}
>plot2d("n(x)"):
\end{eulerprompt}
\eulerimg{17}{images/Davina Safa Felisa 1-6-382.png}
\begin{eulercomment}
4. Fungsi 4
\end{eulercomment}
\begin{eulerprompt}
>function l(x) := 3*x^3/(x^4-3)
>l(5), l(4), l(3)
\end{eulerprompt}
\begin{euleroutput}
  0.602893890675
  0.758893280632
  1.03846153846
\end{euleroutput}
\begin{eulerprompt}
>lmap(5:8)
\end{eulerprompt}
\begin{euleroutput}
  [0.602894,  0.50116,  0.429108,  0.375275]
\end{euleroutput}
\begin{eulerprompt}
>plot2d("l(x)",-3,3,-600,600):
\end{eulerprompt}
\eulerimg{17}{images/Davina Safa Felisa 1-6-383.png}
\begin{eulercomment}
5. Fungsi 5
\end{eulercomment}
\begin{eulerprompt}
>function j(x) := (cos(x))*sin(2*x)
>j(pi), j(0), j(pi/3)
\end{eulerprompt}
\begin{euleroutput}
  0
  0
  0.433012701892
\end{euleroutput}
\begin{eulerprompt}
>jmap(0:3pi)
\end{eulerprompt}
\begin{euleroutput}
  [0,  0.491295,  0.314941,  0.276619,  -0.646688,  -0.154318,
  -0.515201,  0.746821,  0.0418899,  0.684247]
\end{euleroutput}
\begin{eulerprompt}
>plot2d("j(x)"):
\end{eulerprompt}
\eulerimg{17}{images/Davina Safa Felisa 1-6-384.png}
\begin{eulercomment}
6. Fungsi 6
\end{eulercomment}
\begin{eulerprompt}
>function o(x) := x*sqrt(x+2)
>o(3), o(5), o(7)
\end{eulerprompt}
\begin{euleroutput}
  6.7082039325
  13.2287565553
  21
\end{euleroutput}
\begin{eulerprompt}
>omap(3:12)
\end{eulerprompt}
\begin{euleroutput}
  [6.7082,  9.79796,  13.2288,  16.9706,  21,  25.2982,  29.8496,
  34.641,  39.6611,  44.8999]
\end{euleroutput}
\begin{eulerprompt}
>plot2d("o(x)"):
\end{eulerprompt}
\eulerimg{17}{images/Davina Safa Felisa 1-6-385.png}
\begin{eulercomment}
\end{eulercomment}
\begin{eulerformula}
\[
\text{B). FUNGSI 2 VARIABEL}
\]
\end{eulerformula}
\begin{eulercomment}
1. Fungsi 1
\end{eulercomment}
\begin{eulerprompt}
>function a(x,y) ...
\end{eulerprompt}
\begin{eulerudf}
  return x^2+y^2-24
  endfunction
\end{eulerudf}
\begin{eulerprompt}
>a(2,1), a(5,4), a(2,4)
\end{eulerprompt}
\begin{euleroutput}
  -19
  17
  -4
\end{euleroutput}
\begin{eulerprompt}
>amap(-2:2,3:3)
\end{eulerprompt}
\begin{euleroutput}
  [-11,  -14,  -15,  -14,  -11]
\end{euleroutput}
\begin{eulerprompt}
>aspect=1.5; plot3d("a(x,y)",a=-100,b=100,c=-80,d=80,angle=35°,height=30°,r=pi,n=100):
\end{eulerprompt}
\eulerimg{17}{images/Davina Safa Felisa 1-6-387.png}
\begin{eulercomment}
2. Fungsi 2
\end{eulercomment}
\begin{eulerprompt}
>function q(x,y) ...
\end{eulerprompt}
\begin{eulerudf}
  return y^2/(x^2/3)
  endfunction
\end{eulerudf}
\begin{eulerprompt}
>q(4,2), q(2,3), q(4,3)
\end{eulerprompt}
\begin{euleroutput}
  0.75
  6.75
  1.6875
\end{euleroutput}
\begin{eulerprompt}
>qmap(2:2,-2:2)
\end{eulerprompt}
\begin{euleroutput}
  [3,  0.75,  0,  0.75,  3]
\end{euleroutput}
\begin{eulerprompt}
>aspect=1.5; plot3d("q(x,y)",a=-100,b=100,c=-80,d=80,angle=35°,height=30°,r=pi,n=100):
\end{eulerprompt}
\eulerimg{17}{images/Davina Safa Felisa 1-6-388.png}
\eulerheading{Menghitung Limit}
\begin{eulercomment}
Perhitungan limit pada EMT dapat dilakukan dengan menggunakan fungsi
Maxima, yakni "limit". Fungsi "limit" dapat digunakan untuk menghitung
limit fungsi dalam bentuk ekspresi maupun fungsi yang sudah
didefinisikan sebelumnya. Nilai limit dapat dihitung pada sebarang
nilai atau pada tak hingga (-inf, minf, dan inf). Limit kiri dan limit
kanan juga dapat dihitung, dengan cara memberi opsi "plus" atau
"minus". Hasil limit dapat berupa nilai, "und' (tak definisi), "ind"
(tak tentu namun terbatas), "infinity" (kompleks tak hingga).

Perhatikan beberapa contoh berikut. Perhatikan cara menampilkan
perhitungan secara lengkap, tidak hanya menampilkan hasilnya saja.
\end{eulercomment}
\begin{eulerprompt}
>$limit((x^3-13*x^2+51*x-63)/(x^3-4*x^2-3*x+18),x,3)
\end{eulerprompt}
\begin{eulerformula}
\[
-\frac{4}{5}
\]
\end{eulerformula}
\begin{eulerprompt}
>aspect(1.5); plot2d("(x^3-13*x^2+51*x-63)/(x^3-4*x^2-3*x+18)",0,4); plot2d(3,-4/5,>points,style="ow",>add):
\end{eulerprompt}
\eulerimg{17}{images/Davina Safa Felisa 1-6-390.png}
\begin{eulerprompt}
>$limit(2*x*sin(x)/(1-cos(x)),x,0)
\end{eulerprompt}
\begin{eulerformula}
\[
4
\]
\end{eulerformula}
\begin{eulerprompt}
>plot2d("2*x*sin(x)/(1-cos(x))",-pi,pi); plot2d(0,4,>points,style="ow",>add):
\end{eulerprompt}
\eulerimg{17}{images/Davina Safa Felisa 1-6-392.png}
\begin{eulerprompt}
>$limit(cot(7*h)/cot(5*h),h,0)
\end{eulerprompt}
\begin{eulerformula}
\[
\frac{5}{7}
\]
\end{eulerformula}
\begin{eulerprompt}
>plot2d("cot(7*x)/cot(5*x)",-0.001,0.001); plot2d(0,5/7,>points,style="ow",>add):
\end{eulerprompt}
\eulerimg{17}{images/Davina Safa Felisa 1-6-394.png}
\begin{eulerprompt}
>$showev('limit(1/(2*x-1),x,0))
\end{eulerprompt}
\begin{eulerformula}
\[
\lim_{x\rightarrow 0}{\frac{1}{2\,x-1}}=-1
\]
\end{eulerformula}
\begin{eulerprompt}
>$showev('limit((x^2-3*x-10)/(x-5),x,5))
\end{eulerprompt}
\begin{eulerformula}
\[
\lim_{x\rightarrow 5}{\frac{x^2-3\,x-10}{x-5}}=7
\]
\end{eulerformula}
\begin{eulerprompt}
>$showev('limit(sin(x)/x,x,0))
\end{eulerprompt}
\begin{eulerformula}
\[
\lim_{x\rightarrow 0}{\frac{\sin x}{x}}=1
\]
\end{eulerformula}
\begin{eulerprompt}
>plot2d("sin(x)/x",-pi,pi):
\end{eulerprompt}
\eulerimg{17}{images/Davina Safa Felisa 1-6-398.png}
\begin{eulerprompt}
>$showev('limit(sin(x^3)/x,x,0))
\end{eulerprompt}
\begin{eulerformula}
\[
\lim_{x\rightarrow 0}{\frac{\sin x^3}{x}}=0
\]
\end{eulerformula}
\begin{eulerprompt}
>$showev('limit(log(x), x, minf))
\end{eulerprompt}
\begin{eulerformula}
\[
\lim_{x\rightarrow  -\infty }{\log x}={\it infinity}
\]
\end{eulerformula}
\begin{eulerprompt}
>$showev('limit((-2)^x,x, inf))
\end{eulerprompt}
\begin{eulerformula}
\[
\lim_{x\rightarrow \infty }{\left(-2\right)^{x}}={\it infinity}
\]
\end{eulerformula}
\begin{eulerprompt}
>$showev('limit(t-sqrt(2-t),t,2,minus))
\end{eulerprompt}
\begin{eulerformula}
\[
\lim_{t\uparrow 2}{t-\sqrt{2-t}}=2
\]
\end{eulerformula}
\begin{eulerprompt}
>$showev('limit(t-sqrt(2-t),t,5,plus)) // Perhatikan hasilnya
\end{eulerprompt}
\begin{eulerformula}
\[
\lim_{t\downarrow 5}{t-\sqrt{2-t}}=5-\sqrt{3}\,i
\]
\end{eulerformula}
\begin{eulerprompt}
>plot2d("x-sqrt(2-x)",-2,5):
\end{eulerprompt}
\eulerimg{17}{images/Davina Safa Felisa 1-6-404.png}
\begin{eulerprompt}
>$showev('limit((x^2-9)/(2*x^2-5*x-3),x,3))
\end{eulerprompt}
\begin{eulerformula}
\[
\lim_{x\rightarrow 3}{\frac{x^2-9}{2\,x^2-5\,x-3}}=\frac{6}{7}
\]
\end{eulerformula}
\begin{eulerprompt}
>$showev('limit((1-cos(x))/x,x,0))
\end{eulerprompt}
\begin{eulerformula}
\[
\lim_{x\rightarrow 0}{\frac{1-\cos x}{x}}=0
\]
\end{eulerformula}
\begin{eulerprompt}
>$showev('limit((x^2+abs(x))/(x^2-abs(x)),x,0))
\end{eulerprompt}
\begin{eulerformula}
\[
\lim_{x\rightarrow 0}{\frac{\left| x\right| +x^2}{x^2-\left| x  \right| }}=-1
\]
\end{eulerformula}
\begin{eulerprompt}
>$showev('limit((1+1/x)^x,x,inf))
\end{eulerprompt}
\begin{eulerformula}
\[
\lim_{x\rightarrow \infty }{\left(\frac{1}{x}+1\right)^{x}}=e
\]
\end{eulerformula}
\begin{eulerprompt}
>$showev('limit((1+k/x)^x,x,inf))
\end{eulerprompt}
\begin{eulerformula}
\[
\lim_{x\rightarrow \infty }{\left(\frac{k}{x}+1\right)^{x}}=e^{k}
\]
\end{eulerformula}
\begin{eulerprompt}
>$showev('limit((1+x)^(1/x),x,0))
\end{eulerprompt}
\begin{eulerformula}
\[
\lim_{x\rightarrow 0}{\left(x+1\right)^{\frac{1}{x}}}=e
\]
\end{eulerformula}
\begin{eulerprompt}
>$showev('limit((x/(x+k))^x,x,inf))
\end{eulerprompt}
\begin{eulerformula}
\[
\lim_{x\rightarrow \infty }{\left(\frac{x}{x+k}\right)^{x}}=e^ {- k   }
\]
\end{eulerformula}
\begin{eulerprompt}
>$showev('limit(sin(1/x),x,0))
\end{eulerprompt}
\begin{eulerformula}
\[
\lim_{x\rightarrow 0}{\sin \left(\frac{1}{x}\right)}={\it ind}
\]
\end{eulerformula}
\begin{eulerprompt}
>$showev('limit(sin(1/x),x,inf))
\end{eulerprompt}
\begin{eulerformula}
\[
\lim_{x\rightarrow \infty }{\sin \left(\frac{1}{x}\right)}=0
\]
\end{eulerformula}
\begin{eulerprompt}
>plot2d("sin(1/x)",-5,5):
\end{eulerprompt}
\eulerimg{17}{images/Davina Safa Felisa 1-6-414.png}
\eulerheading{Latihan}
\begin{eulercomment}
Bukalah buku Kalkulus. Cari dan pilih beberapa (paling sedikit 5
fungsi berbeda tipe/bentuk/jenis) fungsi dari buku tersebut, kemudian
definisikan di EMT pada baris-baris perintah berikut (jika perlu
tambahkan lagi). Untuk setiap fungsi, hitung nilai limit fungsi
tersebut di beberapa nilai dan di tak hingga. Gambar grafik fungsi
tersebut untuk mengkonfirmasi nilai-nilai limit tersebut.

Jawab:\\
1. Fungsi 1\\
\end{eulercomment}
\begin{eulerformula}
\[
\text{$f(x)=\frac{3x-6}{x+2}$}
\]
\end{eulerformula}
\begin{eulerprompt}
>$showev('limit((3*x-6)/(x+2),x,2))
\end{eulerprompt}
\begin{eulerformula}
\[
\lim_{x\rightarrow 2}{\frac{3\,x-6}{x+2}}=0
\]
\end{eulerformula}
\begin{eulerprompt}
>plot2d("(3*x-6)/(x+2)",-2,3.5,-1,5):
\end{eulerprompt}
\eulerimg{17}{images/Davina Safa Felisa 1-6-417.png}
\begin{eulercomment}
2. Fungsi 2\\
\end{eulercomment}
\begin{eulerformula}
\[
\text{$f(x)=\frac{cos 2x}{sin x - cos x}$}
\]
\end{eulerformula}
\begin{eulerprompt}
>$showev('limit(cos(2*x)/(sin(x) - cos (x)),x,0))
\end{eulerprompt}
\begin{eulerformula}
\[
\lim_{x\rightarrow 0}{\frac{\cos \left(2\,x\right)}{\sin x-\cos x}}=  -1
\]
\end{eulerformula}
\begin{eulerprompt}
>plot2d("cos(2*x)/(sin(x) - cos (x))",-1,1):
\end{eulerprompt}
\eulerimg{17}{images/Davina Safa Felisa 1-6-420.png}
\begin{eulercomment}
3. Fungsi 3\\
\end{eulercomment}
\begin{eulerformula}
\[
\text{$f(x)=\frac{2x^2-2x+5}{3x^2+x-6}$}
\]
\end{eulerformula}
\begin{eulerprompt}
>$showev('limit(((2*x^2-2*x+5)/(3*x^2+x-6)),x,3))
\end{eulerprompt}
\begin{eulerformula}
\[
\lim_{x\rightarrow 3}{\frac{2\,x^2-2\,x+5}{3\,x^2+x-6}}=\frac{17}{  24}
\]
\end{eulerformula}
\begin{eulerprompt}
>plot2d("(2*x^2-2*x+5)/(3*x^2+x-6)",-2,10,-10,5):
\end{eulerprompt}
\eulerimg{17}{images/Davina Safa Felisa 1-6-423.png}
\begin{eulercomment}
4. Fungsi 4\\
\end{eulercomment}
\begin{eulerformula}
\[
\text{$f(x)=4x^2-3$}
\]
\end{eulerformula}
\begin{eulerprompt}
>$showev('limit((4*x^2-3),x,0))
\end{eulerprompt}
\begin{eulerformula}
\[
\lim_{x\rightarrow 0}{4\,x^2-3}=-3
\]
\end{eulerformula}
\begin{eulerprompt}
>plot2d("(4*x^2-3)"):
\end{eulerprompt}
\eulerimg{17}{images/Davina Safa Felisa 1-6-426.png}
\begin{eulercomment}
5. Fungsi 5\\
\end{eulercomment}
\begin{eulerformula}
\[
\text{$f(x)=x^{x^{x}}$}
\]
\end{eulerformula}
\begin{eulerprompt}
>$showev('limit((x^(x^(x))),x,0,plus))
\end{eulerprompt}
\begin{eulerformula}
\[
\lim_{x\downarrow 0}{x^{x^{x}}}=0
\]
\end{eulerformula}
\begin{eulerprompt}
>plot2d("(x^(x^(x)))",-3,3,-1,7):
\end{eulerprompt}
\eulerimg{17}{images/Davina Safa Felisa 1-6-429.png}
\begin{eulercomment}
6. Fungsi 6\\
\end{eulercomment}
\begin{eulerformula}
\[
\text{$f(x)=\frac{3xtanx}{1-cos4x}$}
\]
\end{eulerformula}
\begin{eulerprompt}
>$showev('limit((3*x*tan(x))/(1-cos(4*x)),x,0))
\end{eulerprompt}
\begin{eulerformula}
\[
3\,\left(\lim_{x\rightarrow 0}{\frac{x\,\tan x}{1-\cos \left(4\,x  \right)}}\right)=\frac{3}{8}
\]
\end{eulerformula}
\begin{eulerprompt}
>plot2d("(3*x*tan(x))/(1-cos(4*x))",-pi/2,2pi,0,2pi):
\end{eulerprompt}
\eulerimg{17}{images/Davina Safa Felisa 1-6-432.png}
\eulerheading{Turunan Fungsi}
\begin{eulercomment}
Definisi turunan:

\end{eulercomment}
\begin{eulerformula}
\[
f'(x) = \lim_{h\to 0} \frac{f(x+h)-f(x)}{h}
\]
\end{eulerformula}
\begin{eulercomment}
Berikut adalah contoh-contoh menentukan turunan fungsi dengan
menggunakan definisi turunan (limit).
\end{eulercomment}
\begin{eulerprompt}
>$showev('limit(((x+h)^n-x^n)/h,h,0)) // turunan x^n
\end{eulerprompt}
\begin{eulerformula}
\[
\lim_{h\rightarrow 0}{\frac{\left(x+h\right)^{n}-x^{n}}{h}}=n\,x^{n  -1}
\]
\end{eulerformula}
\begin{eulercomment}
Mengapa hasilnya seperti itu? Tuliskan atau tunjukkan bahwa hasil
limit tersebut benar, sehingga benar turunan fungsinya benar.  Tulis
penjelasan Anda di komentar ini.

Sebagai petunjuk, ekspansikan (x+h)\textasciicircum{}n dengan menggunakan teorema
binomial.\\
Jawab:\\
\end{eulercomment}
\begin{eulerformula}
\[
\text{Akan ditunjukkan bahwa \: $f'(x)=\lim_{h\to 0} \frac{(x+h)^n-x^n}{h}=nx^{n-1}$}
\]
\end{eulerformula}
\begin{eulercomment}
\end{eulercomment}
\begin{eulerformula}
\[
\text{Pertama, ekspansikan $(x+h)^n$, yakni: }
\]
\end{eulerformula}
\begin{eulercomment}
\end{eulercomment}
\begin{eulerformula}
\[
\text{$(x+h)^n=\sum_{k=0}^{n} \binom{n}{k}x^{n-k}h^k$}
\]
\end{eulerformula}
\begin{eulercomment}
\end{eulercomment}
\begin{eulerformula}
\[
\text{$\Leftrightarrow \: (x+h)^n=\binom{n}{0}x^{n}+\binom{n}{1}x^{n-1}h+\binom{n}{2}x^{n-2}h^2+ ...+\binom{n}{n}h^n$}
\]
\end{eulerformula}
\begin{eulercomment}
\end{eulercomment}
\begin{eulerformula}
\[
\text{$\Leftrightarrow \: (x+h)^n=x^{n}+nx^{n-1}h+\binom{n}{2}x^{n-2}h^2+\binom{n}{3}x^{n-3}h^3+ ...+h^n$}
\]
\end{eulerformula}
\begin{eulercomment}
\end{eulercomment}
\begin{eulerformula}
\[
\text{Sehingga, $f'(x)$ menjadi:\: $f'(x)=\lim_{h\to 0} \frac{(x+h)^n-x^n}{h}$}
\]
\end{eulerformula}
\begin{eulercomment}
\end{eulercomment}
\begin{eulerformula}
\[
\text{$\Leftrightarrow f'(x)=\lim_{h\to 0} \frac{x^{n}+nx^{n-1}h+\binom{n}{2}x^{n-2}h^2+\binom{n}{3}x^{n-3}h^3+ ...+h^n-x^n}{h}$}
\]
\end{eulerformula}
\begin{eulercomment}
\end{eulercomment}
\begin{eulerformula}
\[
\text{$\Leftrightarrow f'(x)=\lim_{h\to 0} nx^{n-1}+\binom{n}{2}x^{n-2}h+\binom{n}{3}x^{n-3}h^2+ ...+h^{n-1}$}
\]
\end{eulerformula}
\begin{eulercomment}
\end{eulercomment}
\begin{eulerformula}
\[
\text{$\Leftrightarrow f'(x)=nx^{n-1}$. Terbukti.}
\]
\end{eulerformula}
\begin{eulerprompt}
>$showev('limit((sin(x+h)-sin(x))/h,h,0)) // turunan sin(x)
\end{eulerprompt}
\begin{eulerformula}
\[
\lim_{h\rightarrow 0}{\frac{\sin \left(x+h\right)-\sin x}{h}}=\cos   x
\]
\end{eulerformula}
\begin{eulercomment}
Mengapa hasilnya seperti itu? Tuliskan atau tunjukkan bahwa hasil
limit tersebut\\
benar, sehingga benar turunan fungsinya benar.  Tulis penjelasan Anda
di komentar ini.

Sebagai petunjuk, ekspansikan sin(x+h) dengan menggunakan rumus jumlah
dua sudut.\\
Jawab:\\
\end{eulercomment}
\begin{eulerformula}
\[
\text{Akan ditunjukkan bahwa\: $\lim_{h\to 0} \frac{\sin(x+h)-\sin x}{h}=\cos x$}
\]
\end{eulerformula}
\begin{eulercomment}
\end{eulercomment}
\begin{eulerformula}
\[
\text{Diketahui bahwa:}
\]
\end{eulerformula}
\begin{eulercomment}
\end{eulercomment}
\begin{eulerformula}
\[
\text{$1).\: \sin(x+h)=\sin x\cos h+\cos x\sin h$}
\]
\end{eulerformula}
\begin{eulerformula}
\[
\text{$2).\: \lim_{h\to 0} \frac{1-\cos h}{h}=0$}
\]
\end{eulerformula}
\begin{eulerformula}
\[
\text{$3).\: \lim_{h\to 0} \frac{\sin h}{h}=1$}
\]
\end{eulerformula}
\begin{eulercomment}
\end{eulercomment}
\begin{eulerformula}
\[
\text{$\lim_{h\to 0} \frac{\sin(x+h)-\sin x}{h}$}
\]
\end{eulerformula}
\begin{eulercomment}
\end{eulercomment}
\begin{eulerformula}
\[
\text{$=\lim_{h\to 0} \frac{\sin x\cos h+\cos x\sin h-\sin x}{h}$}
\]
\end{eulerformula}
\begin{eulercomment}
\end{eulercomment}
\begin{eulerformula}
\[
\text{$=\lim_{h\to 0} \left[-\sin x\cdot\frac{1-\cos h}{h}+\cos x\cdot\frac{\sin h}{h}\right]$}
\]
\end{eulerformula}
\begin{eulercomment}
\end{eulercomment}
\begin{eulerformula}
\[
\text{$=(-\sin x)\left[\lim_{h\to 0} \frac{1-\cos h}{h}+(\cos x)\lim_{h\to 0} \frac{\sin h}{h}\right]$}
\]
\end{eulerformula}
\begin{eulercomment}
\end{eulercomment}
\begin{eulerformula}
\[
\text{$=(-\sin x)(0)+(\cos x)(1)=\cos x$. Terbukti.}
\]
\end{eulerformula}
\begin{eulerprompt}
>$showev('limit((log(x+h)-log(x))/h,h,0)) // turunan log(x)
\end{eulerprompt}
\begin{eulerformula}
\[
\lim_{h\rightarrow 0}{\frac{\log \left(x+h\right)-\log x}{h}}=  \frac{1}{x}
\]
\end{eulerformula}
\begin{eulercomment}
Mengapa hasilnya seperti itu? Tuliskan atau tunjukkan bahwa hasil
limit tersebut\\
benar, sehingga benar turunan fungsinya benar.  Tulis penjelasan Anda
di komentar ini.

Sebagai petunjuk, gunakan sifat-sifat logaritma dan hasil limit pada
bagian sebelumnya di atas.\\
Jawab:\\
Bukti:\\
\end{eulercomment}
\begin{eulerformula}
\[
\text{Ambil $f(x)=^a\log x$.}
\]
\end{eulerformula}
\begin{eulercomment}
\end{eulercomment}
\begin{eulerformula}
\[
\text{$\lim_{h\to 0} \frac{^a\log (x+h)-^a\log x}{h}$}
\]
\end{eulerformula}
\begin{eulercomment}
\end{eulercomment}
\begin{eulerformula}
\[
\text{$=\lim _{h\to 0} \frac{^a\log \frac{(x+h)}{x}}{h}$}
\]
\end{eulerformula}
\begin{eulercomment}
\end{eulercomment}
\begin{eulerformula}
\[
\text{$=\lim_{h\to 0} \frac{^a\log (1+\frac{h}{x})}{h}$}
\]
\end{eulerformula}
\begin{eulercomment}
\end{eulercomment}
\begin{eulerformula}
\[
\text{$=\lim_{h\to 0} \frac{^a\log (1+\frac{h}{x})}{\frac{h}{x}x}$}
\]
\end{eulerformula}
\begin{eulercomment}
\end{eulercomment}
\begin{eulerformula}
\[
\text{$=\lim_{h\to 0} \frac{\frac{x}{h}\cdot ^a\log (1+\frac{h}{x})}{x}$}
\]
\end{eulerformula}
\begin{eulercomment}
\end{eulercomment}
\begin{eulerformula}
\[
\text{$=\lim_{h\to 0} \frac{^a\log (1+\frac{h}{x})^\frac{x}{h}}{x}$}
\]
\end{eulerformula}
\begin{eulercomment}
\end{eulercomment}
\begin{eulerformula}
\[
\text{$=\frac{\lim_{h\to 0}\: ^a\log (1+\frac{h}{x})^\frac{x}{h}}{\lim _{h\to 0}\: x}$}
\]
\end{eulerformula}
\begin{eulercomment}
\end{eulercomment}
\begin{eulerformula}
\[
\text{$=\frac{1}{x\cdot ^e\log a}$}
\]
\end{eulerformula}
\begin{eulercomment}
\end{eulercomment}
\begin{eulerformula}
\[
\text{$=\frac{1}{x\cdot \ln a}$}
\]
\end{eulerformula}
\begin{eulercomment}
\end{eulercomment}
\begin{eulerformula}
\[
\text{Menggunakan hasil di atas, maka:}
\]
\end{eulerformula}
\begin{eulercomment}
\end{eulercomment}
\begin{eulerformula}
\[
\text{$\frac{d\: \ln x}{dx}=\frac{d\: ^e\log x}{dx}=\frac{1}{x\cdot \ln e}=\frac{1}{x}.$ Terbukti.}
\]
\end{eulerformula}
\begin{eulerprompt}
>$showev('limit((1/(x+h)-1/x)/h,h,0)) // turunan 1/x
\end{eulerprompt}
\begin{eulerformula}
\[
\lim_{h\rightarrow 0}{\frac{\frac{1}{x+h}-\frac{1}{x}}{h}}=-\frac{1  }{x^2}
\]
\end{eulerformula}
\begin{eulerprompt}
>$showev('limit((E^(x+h)-E^x)/h,h,0))// turunan f(x)=e^x
\end{eulerprompt}
\begin{euleroutput}
  Answering "Is x an integer?" with "integer"
  Answering "Is x an integer?" with "integer"
  Answering "Is x an integer?" with "integer"
  Answering "Is x an integer?" with "integer"
  Answering "Is x an integer?" with "integer"
  Maxima is asking
  Acceptable answers are: yes, y, Y, no, n, N, unknown, uk
  Is x an integer?
  
  Use assume!
  Error in:
   $showev('limit((E^(x+h)-E^x)/h,h,0))// turunan f(x)=e^x ...
                                      ^
\end{euleroutput}
\begin{eulercomment}
Maxima bermasalah dengan limit:

\end{eulercomment}
\begin{eulerformula}
\[
\lim_{h\to 0}\frac{e^{x+h}-e^x}{h}.
\]
\end{eulerformula}
\begin{eulercomment}
Oleh karena itu diperlukan trik khusus agar hasilnya benar.
\end{eulercomment}
\begin{eulerprompt}
>$showev('limit((E^h-1)/h,h,0))
\end{eulerprompt}
\begin{eulerformula}
\[
\lim_{h\rightarrow 0}{\frac{e^{h}-1}{h}}=1
\]
\end{eulerformula}
\begin{eulerprompt}
>$factor(E^(x+h)-E^x)
\end{eulerprompt}
\begin{eulerformula}
\[
\left(e^{h}-1\right)\,e^{x}
\]
\end{eulerformula}
\begin{eulerprompt}
>$showev('limit(factor((E^(x+h)-E^x)/h),h,0)) // turunan f(x)=e^x
\end{eulerprompt}
\begin{eulerformula}
\[
\left(\lim_{h\rightarrow 0}{\frac{e^{h}-1}{h}}\right)\,e^{x}=e^{x}
\]
\end{eulerformula}
\begin{eulerprompt}
>function f(x) &= x^x
\end{eulerprompt}
\begin{euleroutput}
  
                                     x
                                    x
  
\end{euleroutput}
\begin{eulerprompt}
>$showev('limit((f(x+h)-f(x))/h,h,0)) // turunan f(x)=x^x
\end{eulerprompt}
\begin{eulerformula}
\[
\lim_{h\rightarrow 0}{\frac{\left(x+h\right)^{x+h}-x^{x}}{h}}=  {\it infinity}
\]
\end{eulerformula}
\begin{eulercomment}
Di sini Maxima juga bermasalah terkait limit:

\end{eulercomment}
\begin{eulerformula}
\[
\lim_{h\to 0} \frac{(x+h)^{x+h}-x^x}{h}.
\]
\end{eulerformula}
\begin{eulercomment}
Dalam hal ini diperlukan asumsi nilai x.
\end{eulercomment}
\begin{eulerprompt}
>&assume(x>0); $showev('limit((f(x+h)-f(x))/h,h,0)) // turunan f(x)=x^x
\end{eulerprompt}
\begin{eulerformula}
\[
\lim_{h\rightarrow 0}{\frac{\left(x+h\right)^{x+h}-x^{x}}{h}}=x^{x}  \,\left(\log x+1\right)
\]
\end{eulerformula}
\begin{eulerprompt}
>&forget(x>0) // jangan lupa, lupakan asumsi untuk kembali ke semula
\end{eulerprompt}
\begin{euleroutput}
  
                                 [x > 0]
  
\end{euleroutput}
\begin{eulerprompt}
>&forget(x<0)
\end{eulerprompt}
\begin{euleroutput}
  
                                 [x < 0]
  
\end{euleroutput}
\begin{eulerprompt}
>&facts()
\end{eulerprompt}
\begin{euleroutput}
  
          [kind(sinh, one_to_one), kind(log, one_to_one), 
                          kind(tanh, one_to_one), kind(log, increasing)]
  
\end{euleroutput}
\begin{eulerprompt}
>$showev('limit((asin(x+h)-asin(x))/h,h,0)) // turunan arcsin(x)
\end{eulerprompt}
\begin{eulerformula}
\[
\lim_{h\rightarrow 0}{\frac{\arcsin \left(x+h\right)-\arcsin x}{h}}=  \frac{1}{\sqrt{1-x^2}}
\]
\end{eulerformula}
\begin{eulerprompt}
>$showev('limit((tan(x+h)-tan(x))/h,h,0)) // turunan tan(x)
\end{eulerprompt}
\begin{eulerformula}
\[
\lim_{h\rightarrow 0}{\frac{\tan \left(x+h\right)-\tan x}{h}}=  \frac{1}{\cos ^2x}
\]
\end{eulerformula}
\begin{eulerprompt}
>function f(x) &= sinh(x) // definisikan f(x)=sinh(x)
\end{eulerprompt}
\begin{euleroutput}
  
                                 sinh(x)
  
\end{euleroutput}
\begin{eulerprompt}
>function df(x) &= limit((f(x+h)-f(x))/h,h,0); $df(x) // df(x) = f'(x)
\end{eulerprompt}
\begin{eulerformula}
\[
\frac{e^ {- x }\,\left(e^{2\,x}+1\right)}{2}
\]
\end{eulerformula}
\begin{eulerprompt}
>plot2d(["f(x)","df(x)"],-pi,pi,color=[blue,red]):
\end{eulerprompt}
\eulerimg{17}{images/Davina Safa Felisa 1-6-479.png}
\begin{eulerprompt}
>function f(x) &= sin(3*x^5+7)^2
\end{eulerprompt}
\begin{euleroutput}
  
                                 2    5
                              sin (3 x  + 7)
  
\end{euleroutput}
\begin{eulerprompt}
>diff(f,3), diffc(f,3)
\end{eulerprompt}
\begin{euleroutput}
  1198.32948904
  1198.72863721
\end{euleroutput}
\begin{eulercomment}
diff(f,3) : Ini biasanya menunjukkan F mengenai X dievaluasi X=3.\\
Metode diff memberikan turunan yang tepat.

diffc(f,3) : Ini biasanya mengacu pada X=3.\\
Metode diffc biasanya memperkirakan.
\end{eulercomment}
\begin{eulerprompt}
>$showev('diff(f(x),x))
\end{eulerprompt}
\begin{eulerformula}
\[
\frac{d}{d\,x}\,\sin ^2\left(3\,x^5+7\right)=30\,x^4\,\cos \left(3  \,x^5+7\right)\,\sin \left(3\,x^5+7\right)
\]
\end{eulerformula}
\begin{eulerprompt}
>$% with x=3
\end{eulerprompt}
\begin{eulerformula}
\[
{\it \%at}\left(\frac{d}{d\,x}\,\sin ^2\left(3\,x^5+7\right) , x=3  \right)=2430\,\cos 736\,\sin 736
\]
\end{eulerformula}
\begin{eulerprompt}
>$float(%)
\end{eulerprompt}
\begin{eulerformula}
\[
{\it \%at}\left(\frac{d^{1.0}}{d\,x^{1.0}}\,\sin ^2\left(3.0\,x^5+  7.0\right) , x=3.0\right)=1198.728637211748
\]
\end{eulerformula}
\begin{eulerprompt}
>plot2d(f,0,3.1):
\end{eulerprompt}
\eulerimg{17}{images/Davina Safa Felisa 1-6-483.png}
\begin{eulerprompt}
>function f(x) &=5*cos(2*x)-2*x*sin(2*x) // mendifinisikan fungsi f
\end{eulerprompt}
\begin{euleroutput}
  
                        5 cos(2 x) - 2 x sin(2 x)
  
\end{euleroutput}
\begin{eulerprompt}
>function df(x) &=diff(f(x),x) // fd(x) = f’(x)
\end{eulerprompt}
\begin{euleroutput}
  
                       - 12 sin(2 x) - 4 x cos(2 x)
  
\end{euleroutput}
\begin{eulerprompt}
>$'f(1)=f(1), $float(f(1)), $'f(2)=f(2), $float(f(2)) // nilai f(1) dan f(2)
\end{eulerprompt}
\begin{eulerformula}
\[
-0.2410081230863468
\]
\end{eulerformula}
\eulerimg{0}{images/Davina Safa Felisa 1-6-485-large.png}
\eulerimg{0}{images/Davina Safa Felisa 1-6-486-large.png}
\eulerimg{0}{images/Davina Safa Felisa 1-6-487-large.png}
\begin{eulerprompt}
>xp=solve("df(x)",1,2,0) // solusi f’(x)=0 pada interval [1, 2]
\end{eulerprompt}
\begin{euleroutput}
  1.35822987384
\end{euleroutput}
\begin{eulerprompt}
>df(xp), f(xp) // cek bahwa f’(xp)=0 dan nilai ekstrim di titik tersebut
\end{eulerprompt}
\begin{euleroutput}
  0
  -5.67530133759
\end{euleroutput}
\begin{eulerprompt}
>plot2d(["f(x)","df(x)"],0,2*pi,color=[blue,red]): //grafik fungsi dan turunannya
\end{eulerprompt}
\eulerimg{17}{images/Davina Safa Felisa 1-6-488.png}
\eulerheading{Latihan}
\begin{eulercomment}
Bukalah buku Kalkulus. Cari dan pilih beberapa (paling sedikit 5
fungsi berbeda tipe/bentuk/jenis) fungsi dari buku tersebut, kemudian
definisikan di EMT pada baris-baris perintah berikut (jika perlu
tambahkan lagi). Untuk setiap fungsi, tentukan turunannya dengan
menggunakan definisi turunan (limit), seperti contoh-contoh tersebut.
Gambar grafik fungsi asli dan fungsi turunannya pada sumbu koordinat
yang sama.

Jawab:\\
1. Fungsi 1
\end{eulercomment}
\begin{eulerprompt}
>function f(x) := x^2
>$showev('limit((((x+h)^2-x^2)/h),h,0)) // turunan x^2
\end{eulerprompt}
\begin{eulerformula}
\[
\lim_{h\rightarrow 0}{\frac{\left(x+h\right)^2-x^2}{h}}=2\,x
\]
\end{eulerformula}
\begin{eulerprompt}
>function df(x) &= limit((((x+h)^2-x^2)/h),h,0);  $df(x)// df(x) = f'(x)
\end{eulerprompt}
\begin{eulerformula}
\[
2\,x
\]
\end{eulerformula}
\begin{eulerprompt}
>plot2d(["f(x)","df(x)"],-pi,pi,color=[blue,red]), label("f(x)",2,0.6), label("df(x)",2,0.17):
\end{eulerprompt}
\eulerimg{17}{images/Davina Safa Felisa 1-6-491.png}
\begin{eulercomment}
2. Fungsi 2
\end{eulercomment}
\begin{eulerprompt}
>function f(x) := sin(x)*cos(x)
>$showev('limit(((sin(x+h)*cos(x+h))-sin(x)*cos(x))/h,h,0)) // turunan sin(x)*cos(x)
\end{eulerprompt}
\begin{eulerformula}
\[
\lim_{h\rightarrow 0}{\frac{\cos \left(x+h\right)\,\sin \left(x+h  \right)-\cos x\,\sin x}{h}}=\cos ^2x-\sin ^2x
\]
\end{eulerformula}
\begin{eulerprompt}
>function df(x) &= limit(((sin(x+h)*cos(x+h))-sin(x)*cos(x))/h,h,0);  $df(x)// df(x) = f'(x)
\end{eulerprompt}
\begin{eulerformula}
\[
\cos ^2x-\sin ^2x
\]
\end{eulerformula}
\begin{eulerprompt}
>plot2d(["f(x)","df(x)"],-pi,pi,color=[blue,red]), label("f(x)",1,0), label("df(x)",2.3,1.2):
\end{eulerprompt}
\eulerimg{17}{images/Davina Safa Felisa 1-6-494.png}
\begin{eulercomment}
3. Fungsi 3
\end{eulercomment}
\begin{eulerprompt}
>function f(x) := sqrt(x)*4
>$showev('limit((sqrt(x+h)*4-sqrt(x)*4)/h,h,0)) // turunan sqrt(x)*4
\end{eulerprompt}
\begin{eulerformula}
\[
\lim_{h\rightarrow 0}{\frac{4\,\sqrt{x+h}-4\,\sqrt{x}}{h}}=\frac{2  }{\sqrt{x}}
\]
\end{eulerformula}
\begin{eulerprompt}
>function df(x) &= limit((sqrt(x+h)*4-sqrt(x)*4)/h,h,0);  $df(x)// df(x) = f'(x)
\end{eulerprompt}
\begin{eulerformula}
\[
\frac{2}{\sqrt{x}}
\]
\end{eulerformula}
\begin{eulerprompt}
>plot2d(["f(x)","df(x)"],-pi,pi,color=[blue,red]), label("f(x)",-2,11), label("df(x)",-2,-10):
\end{eulerprompt}
\eulerimg{17}{images/Davina Safa Felisa 1-6-497.png}
\begin{eulercomment}
4. Fungsi 4
\end{eulercomment}
\begin{eulerprompt}
>function f(x) := cos(1/x)
>$showev('limit((cos(1/(x+h))-cos(1/x))/h,h,0)) // turunan cos(1/x)
\end{eulerprompt}
\begin{eulerformula}
\[
\lim_{h\rightarrow 0}{\frac{\cos \left(\frac{1}{x+h}\right)-\cos   \left(\frac{1}{x}\right)}{h}}=\frac{\sin \left(\frac{1}{x}\right)}{x  ^2}
\]
\end{eulerformula}
\begin{eulerprompt}
>function df(x) &= limit((cos(1/(x+h))-cos(1/x))/h,h,0);  $df(x)// df(x) = f'(x)
\end{eulerprompt}
\begin{eulerformula}
\[
\frac{\sin \left(\frac{1}{x}\right)}{x^2}
\]
\end{eulerformula}
\begin{eulerprompt}
>plot2d(["f(x)","df(x)"],-pi,pi,color=[blue,red]), label("f(x)",2,0.4), label("df(x)",1,-0.5):
\end{eulerprompt}
\eulerimg{17}{images/Davina Safa Felisa 1-6-500.png}
\begin{eulercomment}
5. Fungsi 5
\end{eulercomment}
\begin{eulerprompt}
>function f(x) := (log(x))^5
>$showev('limit(((log(x+h))^5-(log(x))^5)/h,h,0)) // turunan (log(x))^5
\end{eulerprompt}
\begin{eulerformula}
\[
\lim_{h\rightarrow 0}{\frac{\log ^5\left(x+h\right)-\log ^5x}{h}}=  \frac{5\,\log ^4x}{x}
\]
\end{eulerformula}
\begin{eulerprompt}
>function df(x) &= limit(((log(x+h))^5-(log(x))^5)/h,h,0);  $df(x)// df(x) = f'(x)
\end{eulerprompt}
\begin{eulerformula}
\[
\frac{5\,\log ^4x}{x}
\]
\end{eulerformula}
\begin{eulerprompt}
>plot2d(["f(x)","df(x)"],-50,100,-10,50,color=[blue,red]), label("f(x)",25,35), label("df(x)",50,1):
\end{eulerprompt}
\eulerimg{17}{images/Davina Safa Felisa 1-6-503.png}
\begin{eulercomment}
6. Fungsi 6
\end{eulercomment}
\begin{eulerprompt}
>function f(x) := sqrt(tan(x))
>$showev('limit((sqrt(tan(x+h))-sqrt(tan(x)))/h,h,0)) // turunan exp(x)*cos(x)
\end{eulerprompt}
\begin{eulerformula}
\[
\lim_{h\rightarrow 0}{\frac{\sqrt{\tan \left(x+h\right)}-\sqrt{  \tan x}}{h}}=\frac{1}{2\,\cos ^2x\,\sqrt{\tan x}}
\]
\end{eulerformula}
\begin{eulerprompt}
>function df(x) &= limit((sqrt(tan(x+h))-sqrt(tan(x)))/h,h,0);  $df(x)// df(x) = f'(x)
\end{eulerprompt}
\begin{eulerformula}
\[
\frac{1}{2\,\cos ^2x\,\sqrt{\tan x}}
\]
\end{eulerformula}
\begin{eulerprompt}
>plot2d(["f(x)","df(x)"],-10,10,-10,10,color=[blue,red]), label("f(x)",4.5,0), label("df(x)",5.5,5):
\end{eulerprompt}
\eulerimg{17}{images/Davina Safa Felisa 1-6-506.png}
\eulerheading{Integral}
\begin{eulercomment}
EMT dapat digunakan untuk menghitung integral, baik integral tak tentu
maupun integral tentu. Untuk integral tak tentu (simbolik) sudah tentu
EMT menggunakan Maxima, sedangkan untuk perhitungan integral tentu EMT
sudah menyediakan beberapa fungsi yang mengimplementasikan algoritma
kuadratur (perhitungan integral tentu menggunakan metode numerik).

Pada notebook ini akan ditunjukkan perhitungan integral tentu dengan
menggunakan Teorema Dasar Kalkulus:

\end{eulercomment}
\begin{eulerformula}
\[
\int_a^b f(x)\ dx = F(b)-F(a), \quad \text{ dengan  } F'(x) = f(x).
\]
\end{eulerformula}
\begin{eulercomment}
Fungsi untuk menentukan integral adalah integrate. Fungsi ini dapat
digunakan untuk menentukan, baik integral tentu maupun tak tentu (jika
fungsinya memiliki antiderivatif). Untuk perhitungan integral tentu
fungsi integrate menggunakan metode numerik (kecuali fungsinya tidak
integrabel, kita tidak akan menggunakan metode ini).
\end{eulercomment}
\begin{eulerprompt}
>$showev('integrate(x^n,x))
\end{eulerprompt}
\begin{euleroutput}
  Answering "Is n equal to -1?" with "no"
\end{euleroutput}
\begin{eulerformula}
\[
\int {x^{n}}{\;dx}=\frac{x^{n+1}}{n+1}
\]
\end{eulerformula}
\begin{eulerprompt}
>$showev('integrate(1/(1+x),x))
\end{eulerprompt}
\begin{eulerformula}
\[
\int {\frac{1}{x+1}}{\;dx}=\log \left(x+1\right)
\]
\end{eulerformula}
\begin{eulerprompt}
>$showev('integrate(1/(1+x^2),x))
\end{eulerprompt}
\begin{eulerformula}
\[
\int {\frac{1}{x^2+1}}{\;dx}=\arctan x
\]
\end{eulerformula}
\begin{eulerprompt}
>$showev('integrate(1/sqrt(1-x^2),x))
\end{eulerprompt}
\begin{eulerformula}
\[
\int {\frac{1}{\sqrt{1-x^2}}}{\;dx}=\arcsin x
\]
\end{eulerformula}
\begin{eulerprompt}
>$showev('integrate(sin(x),x,0,pi))
\end{eulerprompt}
\begin{eulerformula}
\[
\int_{0}^{\pi}{\sin x\;dx}=2
\]
\end{eulerformula}
\begin{eulerprompt}
>$showev('integrate(sin(x),x,a,b))
\end{eulerprompt}
\begin{eulerformula}
\[
\int_{a}^{b}{\sin x\;dx}=\cos a-\cos b
\]
\end{eulerformula}
\begin{eulerprompt}
>$showev('integrate(x^n,x,a,b))
\end{eulerprompt}
\begin{euleroutput}
  Answering "Is n positive, negative or zero?" with "positive"
\end{euleroutput}
\begin{eulerformula}
\[
\int_{a}^{b}{x^{n}\;dx}=\frac{b^{n+1}}{n+1}-\frac{a^{n+1}}{n+1}
\]
\end{eulerformula}
\begin{eulerprompt}
>$showev('integrate(x^2*sqrt(2*x+1),x))
\end{eulerprompt}
\begin{eulerformula}
\[
\int {x^2\,\sqrt{2\,x+1}}{\;dx}=\frac{\left(2\,x+1\right)^{\frac{7  }{2}}}{28}-\frac{\left(2\,x+1\right)^{\frac{5}{2}}}{10}+\frac{\left(  2\,x+1\right)^{\frac{3}{2}}}{12}
\]
\end{eulerformula}
\begin{eulerprompt}
>$showev('integrate(x^2*sqrt(2*x+1),x,0,2))
\end{eulerprompt}
\begin{eulerformula}
\[
\int_{0}^{2}{x^2\,\sqrt{2\,x+1}\;dx}=\frac{2\,5^{\frac{5}{2}}}{21}-  \frac{2}{105}
\]
\end{eulerformula}
\begin{eulerprompt}
>$ratsimp(%)
\end{eulerprompt}
\begin{eulerformula}
\[
\int_{0}^{2}{x^2\,\sqrt{2\,x+1}\;dx}=\frac{2\,5^{\frac{7}{2}}-2}{  105}
\]
\end{eulerformula}
\begin{eulerprompt}
>$showev('integrate((sin(sqrt(x)+a)*E^sqrt(x))/sqrt(x),x,0,pi^2))
\end{eulerprompt}
\begin{eulerformula}
\[
\int_{0}^{\pi^2}{\frac{\sin \left(\sqrt{x}+a\right)\,e^{\sqrt{x}}}{  \sqrt{x}}\;dx}=\left(-e^{\pi}-1\right)\,\sin a+\left(e^{\pi}+1  \right)\,\cos a
\]
\end{eulerformula}
\begin{eulerprompt}
>$factor(%)
\end{eulerprompt}
\begin{eulerformula}
\[
\int_{0}^{\pi^2}{\frac{\sin \left(\sqrt{x}+a\right)\,e^{\sqrt{x}}}{  \sqrt{x}}\;dx}=\left(-e^{\pi}-1\right)\,\left(\sin a-\cos a\right)
\]
\end{eulerformula}
\begin{eulerprompt}
>function map f(x) &= E^(-x^2); $f(x)
\end{eulerprompt}
\begin{eulerformula}
\[
e^ {- x^2 }
\]
\end{eulerformula}
\begin{eulerprompt}
>$showev('integrate(f(x),x))
\end{eulerprompt}
\begin{eulerformula}
\[
\int {e^ {- x^2 }}{\;dx}=\frac{\sqrt{\pi}\,\mathrm{erf}\left(x  \right)}{2}
\]
\end{eulerformula}
\begin{eulercomment}
Fungsi f tidak memiliki antiturunan, integralnya masih memuat integral
lain.

\end{eulercomment}
\begin{eulerformula}
\[
erf(x) = \int \frac{e^{-x^2}}{\sqrt{\pi}} \ dx.
\]
\end{eulerformula}
\begin{eulercomment}
Kita tidak dapat menggunakan teorema Dasar kalkulus untuk menghitung
integral tentu fungsi tersebut jika semua batasnya berhingga. Dalam
hal ini dapat digunakan metode numerik (rumus kuadratur).

Misalkan kita akan menghitung:

\end{eulercomment}
\begin{eulerformula}
\[
\int_{0}^{\pi}{e^ {- x^2 }\;dx}
\]
\end{eulerformula}
\begin{eulerprompt}
>x=0:0.1:pi-0.1; plot2d(x,f(x+0.1),>bar); plot2d("f(x)",0,pi,>add):
\end{eulerprompt}
\eulerimg{17}{images/Davina Safa Felisa 1-6-524.png}
\begin{eulercomment}
Integral tentu

\end{eulercomment}
\begin{eulerformula}
\[
\int_{0}^{\pi}{e^ {- x^2 }\;dx}
\]
\end{eulerformula}
\begin{eulercomment}
dapat dihampiri dengan jumlah luas persegi-persegi panjang di bawah
kurva y=f(x) tersebut. Langkah-langkahnya adalah sebagai berikut.
\end{eulercomment}
\begin{eulerprompt}
>t &= makelist(a,a,0,pi-0.1,0.1); // t sebagai list untuk menyimpan nilai-nilai x
>fx &= makelist(f(t[i]+0.1),i,1,length(t)); // simpan nilai-nilai f(x)
>// jangan menggunakan x sebagai list, kecuali Anda pakar Maxima!
\end{eulerprompt}
\begin{eulercomment}
Hasilnya adalah:

\end{eulercomment}
\begin{eulerformula}
\[
\int_{0}^{\pi}{e^ {- x^2 }\;dx}=0.8362196102528469
\]
\end{eulerformula}
\begin{eulercomment}
Jumlah tersebut diperoleh dari hasil kali lebar sub-subinterval (=0.1)
dan jumlah nilai-nilai f(x) untuk x = 0.1, 0.2, 0.3, ..., 3.2.
\end{eulercomment}
\begin{eulerprompt}
>0.1*sum(f(x+0.1)) // cek langsung dengan perhitungan numerik EMT
\end{eulerprompt}
\begin{euleroutput}
  0.836219610253
\end{euleroutput}
\begin{eulercomment}
Untuk mendapatkan nilai integral tentu yang mendekati nilai
sebenarnya, lebar sub-intervalnya dapat diperkecil lagi, sehingga
daerah di bawah kurva tertutup semuanya, misalnya dapat digunakan
lebar subinterval 0.001. (Silakan dicoba!)

Meskipun Maxima tidak dapat menghitung integral tentu fungsi tersebut
untuk batas-batas yang berhingga, namun integral tersebut dapat
dihitung secara eksak jika batas-batasnya tak hingga. Ini adalah salah
satu keajaiban di dalam matematika, yang terbatas tidak dapat dihitung
secara eksak, namun yang tak hingga malah dapat dihitung secara eksak.
\end{eulercomment}
\begin{eulerprompt}
>$showev('integrate(f(x),x,0,inf))
\end{eulerprompt}
\begin{eulerformula}
\[
\int_{0}^{\infty }{e^ {- x^2 }\;dx}=\frac{\sqrt{\pi}}{2}
\]
\end{eulerformula}
\begin{eulercomment}
Berikut adalah contoh lain fungsi yang tidak memiliki antiderivatif, sehingga
integral tentunya hanya dapat dihitung dengan metode numerik.
\end{eulercomment}
\begin{eulerprompt}
>function f(x) &= x^x; $f(x)
\end{eulerprompt}
\begin{eulerformula}
\[
x^{x}
\]
\end{eulerformula}
\begin{eulerprompt}
>$showev('integrate(f(x),x,0,1))
\end{eulerprompt}
\begin{eulerformula}
\[
\int_{0}^{1}{x^{x}\;dx}=\int_{0}^{1}{x^{x}\;dx}
\]
\end{eulerformula}
\begin{eulerprompt}
>x=0:0.1:1-0.01; plot2d(x,f(x+0.01),>bar); plot2d("f(x)",0,1,>add):
\end{eulerprompt}
\eulerimg{17}{images/Davina Safa Felisa 1-6-530.png}
\begin{eulercomment}
Maxima gagal menghitung integral tentu tersebut secara langsung menggunakan perintah
integrate. Berikut kita lakukan seperti contoh sebelumnya untuk mendapat hasil atau
pendekatan nilai integral tentu tersebut.
\end{eulercomment}
\begin{eulerprompt}
>t &= makelist(a,a,0,1-0.01,0.01);
>fx &= makelist(f(t[i]+0.01),i,1,length(t));
\end{eulerprompt}
\begin{eulerformula}
\[
\int_{0}^{1}{x^{x}\;dx}=0.7834935879025506
\]
\end{eulerformula}
\begin{eulercomment}
Apakah hasil tersebut cukup baik? perhatikan gambarnya.
\end{eulercomment}
\begin{eulerprompt}
>function f(x) &= sin(3*x^5+7)^2
\end{eulerprompt}
\begin{euleroutput}
  
                                 2    5
                              sin (3 x  + 7)
  
\end{euleroutput}
\begin{eulerprompt}
>integrate(f,0,1)
\end{eulerprompt}
\begin{euleroutput}
  0.542581176074
\end{euleroutput}
\begin{eulerprompt}
>&showev(’integrate(f(x),x,0,1))
\end{eulerprompt}
\begin{euleroutput}
  
                        2    5
          ’integrate(sin (3 x  + 7), x, 0, 1) = 
                                                   2    5
                                     ’integrate(sin (3 x  + 7), x, 0, 1)
  
\end{euleroutput}
\begin{eulerprompt}
>&float(%)
\end{eulerprompt}
\begin{euleroutput}
  
                        2      5
          ’integrate(sin (3.0 x  + 7.0), x, 0.0, 1.0) = 
                                           2      5
                             ’integrate(sin (3.0 x  + 7.0), x, 0.0, 1.0)
  
\end{euleroutput}
\begin{eulerprompt}
>$showev('integrate(x*exp(-x),x,0,1)) // Integral tentu (eksak)
\end{eulerprompt}
\begin{eulerformula}
\[
\int_{0}^{1}{x\,e^ {- x }\;dx}=1-2\,e^ {- 1 }
\]
\end{eulerformula}
\eulerheading{Latihan}
\begin{eulercomment}
- Bukalah buku Kalkulus.\\
- Cari dan pilih beberapa (paling sedikit 5 fungsi berbeda
tipe/bentuk/jenis) fungsi dari buku tersebut, kemudian definisikan di
EMT pada baris-baris perintah berikut (jika perlu tambahkan lagi).\\
- Untuk setiap fungsi, tentukan anti turunannya (jika ada), hitunglah
integral tentu dengan batas-batas yang menarik (Anda tentukan
sendiri), seperti contoh-contoh tersebut.\\
- Lakukan hal yang sama untuk fungsi-fungsi yang tidak dapat
diintegralkan (cari sedikitnya 3 fungsi).\\
- Gambar grafik fungsi dan daerah integrasinya pada sumbu koordinat
yang sama.\\
- Gunakan integral tentu untuk mencari luas daerah yang dibatasi oleh
dua kurva yang berpotongan di dua titik. (Cari dan gambar kedua kurva
dan arsir (warnai) daerah yang dibatasi oleh keduanya.)\\
- Gunakan integral tentu untuk menghitung volume benda putar kurva y=
f(x) yang diputar mengelilingi sumbu x dari x=a sampai x=b, yakni

\end{eulercomment}
\begin{eulerformula}
\[
V = \int_a^b \pi (f(x))^2\ dx.
\]
\end{eulerformula}
\begin{eulercomment}
(Pilih fungsinya dan gambar kurva dan benda putar yang dihasilkan.
Anda dapat mencari contoh-contoh bagaimana cara menggambar benda hasil
perputaran suatu kurva.)\\
- Gunakan integral tentu untuk menghitung panjang kurva y=f(x) dari
x=a sampai x=b dengan menggunakan rumus:

\end{eulercomment}
\begin{eulerformula}
\[
S = \int_a^b \sqrt{1+(f'(x))^2} \ dx.
\]
\end{eulerformula}
\begin{eulercomment}
(Pilih fungsi dan gambar kurvanya.)

Jawab:\\
1. Fungsi 1
\end{eulercomment}
\begin{eulerprompt}
>function f(x) &= 5*x^2; $f(x)
\end{eulerprompt}
\begin{eulerformula}
\[
5\,x^2
\]
\end{eulerformula}
\begin{eulerprompt}
>$showev('integrate(f(x),x))
\end{eulerprompt}
\begin{eulerformula}
\[
5\,\int {x^2}{\;dx}=\frac{5\,x^3}{3}
\]
\end{eulerformula}
\begin{eulerprompt}
>$showev('integrate(f(x),x,2,3))
\end{eulerprompt}
\begin{eulerformula}
\[
5\,\int_{2}^{3}{x^2\;dx}=\frac{95}{3}
\]
\end{eulerformula}
\begin{eulerprompt}
>x=0.01:0.03:4; plot2d(x,f(x+0.01),>bar); plot2d("f(x)",2,3,>add):
\end{eulerprompt}
\eulerimg{17}{images/Davina Safa Felisa 1-6-538.png}
\begin{eulercomment}
2. Fungsi 2
\end{eulercomment}
\begin{eulerprompt}
>function f(x) &= cos(2*x+5); $f(x)
\end{eulerprompt}
\begin{eulerformula}
\[
\cos \left(2\,x+5\right)
\]
\end{eulerformula}
\begin{eulerprompt}
>$showev('integrate(f(x),x))
\end{eulerprompt}
\begin{eulerformula}
\[
\int {\cos \left(2\,x+5\right)}{\;dx}=\frac{\sin \left(2\,x+5  \right)}{2}
\]
\end{eulerformula}
\begin{eulerprompt}
>$showev('integrate(f(x),x,pi,2*pi))
\end{eulerprompt}
\begin{eulerformula}
\[
\int_{\pi}^{2\,\pi}{\cos \left(2\,x+5\right)\;dx}=0
\]
\end{eulerformula}
\begin{eulerprompt}
>x=0:0.05:pi-0.1; plot2d(x,f(x+0.03),>bar); plot2d("f(x)",pi,2*pi,>add):
\end{eulerprompt}
\eulerimg{17}{images/Davina Safa Felisa 1-6-542.png}
\begin{eulercomment}
3. Fungsi 3
\end{eulercomment}
\begin{eulerprompt}
>function f(x) &= (sin(x))*(cos((x)))^2; $f(x)
\end{eulerprompt}
\begin{eulerformula}
\[
\cos ^2x\,\sin x
\]
\end{eulerformula}
\begin{eulerprompt}
>$showev('integrate(f(x),x))
\end{eulerprompt}
\begin{eulerformula}
\[
\int {\cos ^2x\,\sin x}{\;dx}=-\frac{\cos ^3x}{3}
\]
\end{eulerformula}
\begin{eulerprompt}
>$showev('integrate(f(x),x,0,pi))
\end{eulerprompt}
\begin{eulerformula}
\[
\int_{0}^{\pi}{\cos ^2x\,\sin x\;dx}=\frac{2}{3}
\]
\end{eulerformula}
\begin{eulerprompt}
>x=-pi:0.04:pi; plot2d(x,f(x+0.01),>bar); plot2d("f(x)",0,pi,>add):
\end{eulerprompt}
\eulerimg{17}{images/Davina Safa Felisa 1-6-546.png}
\begin{eulercomment}
4. Fungsi 4
\end{eulercomment}
\begin{eulerprompt}
>function f(x) &= (x^2*(2-x^3)^(1/2)); $f(x)
\end{eulerprompt}
\begin{eulerformula}
\[
x^2\,\sqrt{2-x^3}
\]
\end{eulerformula}
\begin{eulerprompt}
>$showev('integrate(f(x),x))
\end{eulerprompt}
\begin{eulerformula}
\[
\int {x^2\,\sqrt{2-x^3}}{\;dx}=-\frac{2\,\left(2-x^3\right)^{\frac{  3}{2}}}{9}
\]
\end{eulerformula}
\begin{eulerprompt}
>$showev('integrate(f(x),x,0,1))
\end{eulerprompt}
\begin{eulerformula}
\[
\int_{0}^{1}{x^2\,\sqrt{2-x^3}\;dx}=\frac{2^{\frac{5}{2}}}{9}-  \frac{2}{9}
\]
\end{eulerformula}
\begin{eulerprompt}
>x=-1:0.04:1; plot2d(x,f(x+0.01),>bar); plot2d("f(x)",0,1,>add):
\end{eulerprompt}
\eulerimg{17}{images/Davina Safa Felisa 1-6-550.png}
\begin{eulercomment}
5. Fungsi 5
\end{eulercomment}
\begin{eulerprompt}
>function f(x) &= sqrt(24-x^2); $f(x)
\end{eulerprompt}
\begin{eulerformula}
\[
\sqrt{24-x^2}
\]
\end{eulerformula}
\begin{eulerprompt}
>$showev('integrate(f(x),x))
\end{eulerprompt}
\begin{eulerformula}
\[
\int {\sqrt{24-x^2}}{\;dx}=12\,\arcsin \left(\frac{x}{2\,\sqrt{6}}  \right)+\frac{x\,\sqrt{24-x^2}}{2}
\]
\end{eulerformula}
\begin{eulerprompt}
>$showev('integrate(f(x),x,1,2))
\end{eulerprompt}
\begin{eulerformula}
\[
\int_{1}^{2}{\sqrt{24-x^2}\;dx}=12\,\arcsin \left(\frac{1}{\sqrt{6}  }\right)-\frac{24\,\arcsin \left(\frac{1}{2\,\sqrt{6}}\right)+\sqrt{  23}}{2}+2\,\sqrt{5}
\]
\end{eulerformula}
\begin{eulerprompt}
>x=-2:0.04:1; plot2d(x,f(x+0.01),>bar); plot2d("f(x)",1,2,>add):
\end{eulerprompt}
\eulerimg{17}{images/Davina Safa Felisa 1-6-554.png}
\begin{eulercomment}
6. Fungsi 6
\end{eulercomment}
\begin{eulerprompt}
>t &= makelist(a,a,0,1-0.01,0.01);
>fx &= makelist(f(t[i]+0.01),i,1,length(t));
>function f(x) &= x^2+50; $f(x)
\end{eulerprompt}
\begin{eulerformula}
\[
x^2+50
\]
\end{eulerformula}
\begin{eulerprompt}
>x=0:0.1:pi-0.01; plot2d(x,f(x+0.01),>bar); plot2d("f(x)",0,pi,>add):
\end{eulerprompt}
\eulerimg{17}{images/Davina Safa Felisa 1-6-556.png}
\begin{eulerprompt}
>0.01*sum(f(x+0.01))
\end{eulerprompt}
\begin{euleroutput}
  17.051552
\end{euleroutput}
\begin{eulercomment}
7. Fungsi 7
\end{eulercomment}
\begin{eulerprompt}
>t &= makelist(a,a,0,1-0.01,0.01);
>fx &= makelist(f(t[i]+0.01),i,1,length(t));
>function f(x) &= cos(x)/x; $f(x)
\end{eulerprompt}
\begin{eulerformula}
\[
\frac{\cos x}{x}
\]
\end{eulerformula}
\begin{eulerprompt}
>x=-pi:0.07:pi-0.01; plot2d(x,f(x+0.01),>bar); plot2d("f(x)",0,pi,>add):
\end{eulerprompt}
\eulerimg{17}{images/Davina Safa Felisa 1-6-558.png}
\begin{eulerprompt}
>0.01*sum(f(x+0.01))
\end{eulerprompt}
\begin{euleroutput}
  0.415163991256
\end{euleroutput}
\begin{eulercomment}
8. Fungsi 8
\end{eulercomment}
\begin{eulerprompt}
>t &= makelist(a,a,0,1-0.01,0.01);
>fx &= makelist(f(t[i]+0.01),i,1,length(t));
>function f(x) &= sqrt(x^2-1); $f(x)
\end{eulerprompt}
\begin{eulerformula}
\[
\sqrt{x^2-1}
\]
\end{eulerformula}
\begin{eulerprompt}
>x=3:0.04:pi-0.01; plot2d(x,f(x+0.01),>bar); plot2d("f(x)",0,2,>add):
\end{eulerprompt}
\eulerimg{17}{images/Davina Safa Felisa 1-6-560.png}
\begin{eulerprompt}
>0.01*sum(f(x+0.01))
\end{eulerprompt}
\begin{euleroutput}
  0.11610107668
\end{euleroutput}
\eulersubheading{Luas daerah dibatasi 2 kurva}
\begin{eulercomment}
1). Fungsi 1
\end{eulercomment}
\begin{eulerprompt}
>function f(x) &= x^3; $f(x)
\end{eulerprompt}
\begin{eulerformula}
\[
x^3
\]
\end{eulerformula}
\begin{eulerprompt}
>function g(x) &= x; $g(x)
\end{eulerprompt}
\begin{eulerformula}
\[
x
\]
\end{eulerformula}
\begin{eulerprompt}
>plot2d(["x^4","x^3"],-2,2,-1,2):
\end{eulerprompt}
\eulerimg{17}{images/Davina Safa Felisa 1-6-563.png}
\begin{eulerprompt}
>function h(x) &= f(x)-g(x); $h(x)
\end{eulerprompt}
\begin{eulerformula}
\[
x^3-x
\]
\end{eulerformula}
\begin{eulerprompt}
>$showev('integrate(h(x),x))
\end{eulerprompt}
\begin{eulerformula}
\[
\int {x^3-x}{\;dx}=\frac{x^4}{4}-\frac{x^2}{2}
\]
\end{eulerformula}
\begin{eulerprompt}
>$&solve(f(x)=g(x))
\end{eulerprompt}
\begin{eulerformula}
\[
\left[ x=-1 , x=1 , x=0 \right] 
\]
\end{eulerformula}
\begin{eulerprompt}
>$showev('integrate(h(x),x,0,1)) // menghitung luas daerah yang dibatasi 2 kurva
\end{eulerprompt}
\begin{eulerformula}
\[
\int_{0}^{1}{x^3-x\;dx}=-\frac{1}{4}
\]
\end{eulerformula}
\begin{eulercomment}
\end{eulercomment}
\begin{eulerformula}
\[
\text{Arsiran daerah yang dibatasi kurva $f(x)$ dan $g(x)$ sebagai berikut:}
\]
\end{eulerformula}
\begin{eulerprompt}
>x=-1:0.01:1; plot2d(x,f(x),>bar,>filled,style="-",fillcolor=orange,>grid); plot2d(x,g(x),>bar,>add,>filled,style="-",fillcolor=white); label("f(x)",0,2.1); label("g(x)",0.5,0.3):
\end{eulerprompt}
\eulerimg{17}{images/Davina Safa Felisa 1-6-569.png}
\begin{eulercomment}
2). Fungsi 2
\end{eulercomment}
\begin{eulerprompt}
>function f(x) &= x^3+1; $f(x)
\end{eulerprompt}
\begin{eulerformula}
\[
x^3+1
\]
\end{eulerformula}
\begin{eulerprompt}
>function g(x) &= x^2; $g(x)
\end{eulerprompt}
\begin{eulerformula}
\[
x^2
\]
\end{eulerformula}
\begin{eulerprompt}
>plot2d(["-x^2+2","x^2"],-2,2,-1,2):
\end{eulerprompt}
\eulerimg{17}{images/Davina Safa Felisa 1-6-572.png}
\begin{eulerprompt}
>function h(x) &= f(x)-g(x); $h(x)
\end{eulerprompt}
\begin{eulerformula}
\[
x^3-x^2+1
\]
\end{eulerformula}
\begin{eulerprompt}
>$&solve(f(x)=g(x))
\end{eulerprompt}
\begin{eulerformula}
\[
\left[ x=\frac{\frac{\sqrt{3}\,i}{2}-\frac{1}{2}}{9\,\left(\frac{  \sqrt{23}}{2\,3^{\frac{3}{2}}}-\frac{25}{54}\right)^{\frac{1}{3}}}+  \left(\frac{\sqrt{23}}{2\,3^{\frac{3}{2}}}-\frac{25}{54}\right)^{  \frac{1}{3}}\,\left(-\frac{\sqrt{3}\,i}{2}-\frac{1}{2}\right)+\frac{  1}{3} , x=\left(\frac{\sqrt{23}}{2\,3^{\frac{3}{2}}}-\frac{25}{54}  \right)^{\frac{1}{3}}\,\left(\frac{\sqrt{3}\,i}{2}-\frac{1}{2}  \right)+\frac{-\frac{\sqrt{3}\,i}{2}-\frac{1}{2}}{9\,\left(\frac{  \sqrt{23}}{2\,3^{\frac{3}{2}}}-\frac{25}{54}\right)^{\frac{1}{3}}}+  \frac{1}{3} , x=\left(\frac{\sqrt{23}}{2\,3^{\frac{3}{2}}}-\frac{25  }{54}\right)^{\frac{1}{3}}+\frac{1}{9\,\left(\frac{\sqrt{23}}{2\,3^{  \frac{3}{2}}}-\frac{25}{54}\right)^{\frac{1}{3}}}+\frac{1}{3}   \right] 
\]
\end{eulerformula}
\begin{eulerprompt}
>$showev('integrate(h(x),x,-1,1)) // menghitung luas daerah yang dibatasi 2 kurva
\end{eulerprompt}
\begin{eulerformula}
\[
\int_{-1}^{1}{x^3-x^2+1\;dx}=\frac{4}{3}
\]
\end{eulerformula}
\begin{eulercomment}
\end{eulercomment}
\begin{eulerformula}
\[
\text{Arsiran daerah yang dibatasi kurva $f(x)$ dan $g(x)$ sebagai berikut:}
\]
\end{eulerformula}
\begin{eulerprompt}
>x=-1:0.01:1; plot2d(x,f(x),>bar,>filled,style="-",fillcolor=orange,>grid); plot2d(x,g(x),>bar,>add,>filled,style="-",fillcolor=white); label("f(x)",0,2.1); label("g(x)",0.5,0.3):
\end{eulerprompt}
\eulerimg{17}{images/Davina Safa Felisa 1-6-577.png}
\eulersubheading{Volume benda putar}
\begin{eulercomment}
Menghitung volume hasil perputaran kurva\\
\end{eulercomment}
\begin{eulerformula}
\[
m(x)=x^3+1
\]
\end{eulerformula}
\begin{eulercomment}
dari x=-1 sampai x=0. Diputar terhadap sumbu-x.\\
Jawab:
\end{eulercomment}
\begin{eulerprompt}
>function m(x) &= x^4+3; $m(x)
\end{eulerprompt}
\begin{eulerformula}
\[
x^4+3
\]
\end{eulerformula}
\begin{eulerprompt}
>$showev('integrate(pi*(m(x))^2,x,-1,0)) // Menghitung volume hasil perputaran m(x)
\end{eulerprompt}
\begin{eulerformula}
\[
\pi\,\int_{-1}^{0}{\left(x^4+3\right)^2\;dx}=\frac{464\,\pi}{45}
\]
\end{eulerformula}
\begin{eulercomment}
Daerah di bawah kurva yang akan dirotasi terhadap sumbu x sebagai
berikut:
\end{eulercomment}
\begin{eulerprompt}
>plot2d("m(x)",-1,0,-1,2,grid=7,>filled, style="/\(\backslash\)"): 
\end{eulerprompt}
\eulerimg{17}{images/Davina Safa Felisa 1-6-581.png}
\begin{eulercomment}
Hasil perputaran m(x) terhadap sumbu x sebagai berikut:
\end{eulercomment}
\begin{eulerprompt}
>plot3d("m(x)",-1,0,-1,1,>rotate,angle=6.3,>hue,>contour,color=redgreen,height=11):
\end{eulerprompt}
\eulerimg{17}{images/Davina Safa Felisa 1-6-582.png}
\begin{eulercomment}
\end{eulercomment}
\eulersubheading{Menghitung panjang kurva}
\begin{eulercomment}
Menghitung panjang kurva\\
\end{eulercomment}
\begin{eulerformula}
\[
\text{$y=x^2-x+1$}
\]
\end{eulerformula}
\begin{eulercomment}
dari x=1 sampai x=3.
\end{eulercomment}
\begin{eulerprompt}
>function d(x) &= x^2-x+1; $d(x)
\end{eulerprompt}
\begin{eulerformula}
\[
x^2-x+1
\]
\end{eulerformula}
\begin{eulerprompt}
>plot2d("d(x)",-5,6): // gambar kurva d(x)
\end{eulerprompt}
\eulerimg{17}{images/Davina Safa Felisa 1-6-585.png}
\begin{eulerprompt}
>$showev('limit((d(x+h)-d(x))/h,h,0))
\end{eulerprompt}
\begin{eulerformula}
\[
\lim_{h\rightarrow 0}{\frac{\left(x+h\right)^2-x^2-h}{h}}=2\,x-1
\]
\end{eulerformula}
\begin{eulerprompt}
>function dd(x) &= limit((d(x+h)-d(x))/h,h,0); $dd(x)
\end{eulerprompt}
\begin{eulerformula}
\[
2\,x-1
\]
\end{eulerformula}
\begin{eulerprompt}
>function q(x) &= ((dd(x))^2); $q(x)
\end{eulerprompt}
\begin{eulerformula}
\[
\left(2\,x-1\right)^2
\]
\end{eulerformula}
\begin{eulerprompt}
>$showev('integrate(sqrt(1+q(x)),x,1,3)) // menghitung panjang kurva
\end{eulerprompt}
\begin{eulerformula}
\[
\int_{1}^{3}{\sqrt{\left(2\,x-1\right)^2+1}\;dx}=\frac{  {\rm asinh}\; 5+5\,\sqrt{26}}{4}-\frac{{\rm asinh}\; 1+\sqrt{2}}{4}
\]
\end{eulerformula}
\begin{eulercomment}
Jadi, panjang kurva\\
\end{eulercomment}
\begin{eulerformula}
\[
\text{$y=x^2-x+1$}
\]
\end{eulerformula}
\begin{eulercomment}
dari x=0 sampai x=4 adalah\\
\end{eulercomment}
\begin{eulerformula}
\[
\text{$S=\frac{asinh 5+5sqrt(26)}{4}-\frac{asinh(1)+sqrt(2)}{4}$}.
\]
\end{eulerformula}
\begin{eulercomment}
\begin{eulercomment}
\eulerheading{Barisan dan Deret}
\begin{eulercomment}
(Catatan: bagian ini belum lengkap. Anda dapat membaca contoh-contoh
pengguanaan EMT dan Maxima untuk menghitung limit barisan, rumus
jumlah parsial suatu deret, jumlah tak hingga suatu deret konvergen,
dan sebagainya. Anda dapat mengeksplor contoh-contoh di EMT atau
perbagai panduan penggunaan Maxima di software Maxima atau dari
Internet.)

Barisan dapat didefinisikan dengan beberapa cara di dalam EMT, di
antaranya:

- dengan cara yang sama seperti mendefinisikan vektor dengan
elemen-elemen beraturan (menggunakan titik dua ":");\\
- menggunakan perintah "sequence" dan rumus barisan (suku ke -n);\\
- menggunakan perintah "iterate" atau "niterate";\\
- menggunakan fungsi Maxima "create\_list" atau "makelist" untuk
menghasilkan barisan simbolik;\\
- menggunakan fungsi biasa yang inputnya vektor atau barisan;\\
- menggunakan fungsi rekursif.

EMT menyediakan beberapa perintah (fungsi) terkait barisan, yakni:

- sum: menghitung jumlah semua elemen suatu barisan\\
- cumsum: jumlah kumulatif suatu barisan\\
- differences: selisih antar elemen-elemen berturutan

EMT juga dapat digunakan untuk menghitung jumlah deret berhingga
maupun deret tak hingga, dengan menggunakan perintah (fungsi) "sum".
Perhitungan dapat dilakukan secara numerik maupun simbolik dan eksak.

Berikut adalah beberapa contoh perhitungan barisan dan deret
menggunakan EMT.
\end{eulercomment}
\begin{eulerprompt}
>1:10 // barisan sederhana
\end{eulerprompt}
\begin{euleroutput}
  [1,  2,  3,  4,  5,  6,  7,  8,  9,  10]
\end{euleroutput}
\begin{eulerprompt}
>1:2:30
\end{eulerprompt}
\begin{euleroutput}
  [1,  3,  5,  7,  9,  11,  13,  15,  17,  19,  21,  23,  25,  27,  29]
\end{euleroutput}
\begin{eulerprompt}
>sum(1:2:30), sum(1/(1:2:30))
\end{eulerprompt}
\begin{euleroutput}
  225
  2.33587263431
\end{euleroutput}
\begin{eulerprompt}
>$'sum(k, k, 1, n) = factor(ev(sum(k, k, 1, n),simpsum=true)) // simpsum:menghitung deret secara simbolik
\end{eulerprompt}
\begin{eulerformula}
\[
\sum_{k=1}^{n}{k}=\frac{n\,\left(n+1\right)}{2}
\]
\end{eulerformula}
\begin{eulerprompt}
>$'sum(1/(3^k+k), k, 0, inf) = factor(ev(sum(1/(3^k+k), k, 0, inf),simpsum=true))
\end{eulerprompt}
\begin{eulerformula}
\[
\sum_{k=0}^{\infty }{\frac{1}{3^{k}+k}}=\sum_{k=0}^{\infty }{\frac{  1}{3^{k}+k}}
\]
\end{eulerformula}
\begin{eulercomment}
Di sini masih gagal, hasilnya tidak dihitung.
\end{eulercomment}
\begin{eulerprompt}
>$'sum(1/x^2, x, 1, inf)= ev(sum(1/x^2, x, 1, inf),simpsum=true) // ev: menghitung nilai ekspresi
\end{eulerprompt}
\begin{eulerformula}
\[
\sum_{x=1}^{\infty }{\frac{1}{x^2}}=\frac{\pi^2}{6}
\]
\end{eulerformula}
\begin{eulerprompt}
>$'sum((-1)^(k-1)/k, k, 1, inf) = factor(ev(sum((-1)^(x-1)/x, x, 1, inf),simpsum=true))
\end{eulerprompt}
\begin{eulerformula}
\[
\sum_{k=1}^{\infty }{\frac{\left(-1\right)^{k-1}}{k}}=-\sum_{x=1}^{  \infty }{\frac{\left(-1\right)^{x}}{x}}
\]
\end{eulerformula}
\begin{eulercomment}
Di sini masih gagal, hasilnya tidak dihitung.
\end{eulercomment}
\begin{eulerprompt}
>$'sum((-1)^k/(2*k-1), k, 1, inf) = factor(ev(sum((-1)^k/(2*k-1), k, 1, inf),simpsum=true))
\end{eulerprompt}
\begin{eulerformula}
\[
\sum_{k=1}^{\infty }{\frac{\left(-1\right)^{k}}{2\,k-1}}=\sum_{k=1  }^{\infty }{\frac{\left(-1\right)^{k}}{2\,k-1}}
\]
\end{eulerformula}
\begin{eulerprompt}
>$ev(sum(1/n!, n, 0, inf),simpsum=true)
\end{eulerprompt}
\begin{eulerformula}
\[
\sum_{n=0}^{\infty }{\frac{1}{n!}}
\]
\end{eulerformula}
\begin{eulercomment}
Di sini masih gagal, hasilnya tidak dihitung, harusnya hasilnya e.
\end{eulercomment}
\begin{eulerprompt}
>&assume(abs(x)<1); $'sum(a*x^k, k, 0, inf)=ev(sum(a*x^k, k, 0, inf),simpsum=true), &forget(abs(x)<1);
\end{eulerprompt}
\begin{eulerformula}
\[
a\,\sum_{k=0}^{\infty }{x^{k}}=\frac{a}{1-x}
\]
\end{eulerformula}
\begin{eulercomment}
Deret geometri tak hingga, dengan asumsi rasional antara -1 dan 1.
\end{eulercomment}
\eulerheading{Aplikasi Integral Tentu}
\begin{eulerprompt}
>plot2d("x^3-x",-0.1,1.1); plot2d("-x^2",>add);  ...
>b=solve("x^3-x+x^2",0.5); x=linspace(0,b,200); xi=flipx(x); ...
>plot2d(x|xi,x^3-x|-xi^2,>filled,style="|",fillcolor=1,>add): // Plot daerah antara 2 kurva
\end{eulerprompt}
\eulerimg{17}{images/Davina Safa Felisa 1-6-599.png}
\begin{eulerprompt}
>a=solve("x^3-x+x^2",0), b=solve("x^3-x+x^2",1) // absis titik-titik potong kedua kurva
\end{eulerprompt}
\begin{euleroutput}
  0
  0.61803398875
\end{euleroutput}
\begin{eulerprompt}
>integrate("(-x^2)-(x^3-x)",a,b) // luas daerah yang diarsir
\end{eulerprompt}
\begin{euleroutput}
  0.0758191713542
\end{euleroutput}
\begin{eulercomment}
Hasil tersebut akan kita bandngkan dengan perhitungan secara analitik.
\end{eulercomment}
\begin{eulerprompt}
>a &= solve((-x^2)-(x^3-x),x); $a // menentukan absis titik potong kedua kurva secara eksak
\end{eulerprompt}
\begin{eulerformula}
\[
\left[ x=\frac{-\sqrt{5}-1}{2} , x=\frac{\sqrt{5}-1}{2} , x=0   \right] 
\]
\end{eulerformula}
\begin{eulerprompt}
>$showev('integrate(-x^2-x^3+x,x,0,(sqrt(5)-1)/2)) // Nilai integral secara eksak
\end{eulerprompt}
\begin{eulerformula}
\[
\int_{0}^{\frac{\sqrt{5}-1}{2}}{-x^3-x^2+x\;dx}=\frac{13-5^{\frac{3  }{2}}}{24}
\]
\end{eulerformula}
\begin{eulerprompt}
>$float(%)
\end{eulerprompt}
\begin{eulerformula}
\[
\int_{0.0}^{0.6180339887498949}{-1.0\,x^3-1.0\,x^2+x\;dx}=  0.07581917135421037
\]
\end{eulerformula}
\eulerheading{Panjang Kurva}
\begin{eulercomment}
Hitunglah panjang kurva berikut ini dan luas daerah di dalam kurva
tersebut.

\end{eulercomment}
\begin{eulerformula}
\[
\gamma(t) = (r(t) \cos(t), r(t) \sin(t))
\]
\end{eulerformula}
\begin{eulercomment}
dengan

\end{eulercomment}
\begin{eulerformula}
\[
r(t) = 1 + \dfrac{\sin(3t)}{2},\quad 0\le t\le 2\pi.
\]
\end{eulerformula}
\begin{eulerprompt}
>t=linspace(0,2pi,1000); r=1+sin(3*t)/2; x=r*cos(t); y=r*sin(t); ...
>plot2d(x,y,>filled,fillcolor=red,style="/",r=1.5): // Kita gambar kurvanya terlebih dahulu
\end{eulerprompt}
\eulerimg{17}{images/Davina Safa Felisa 1-6-605.png}
\begin{eulerprompt}
>function r(t) &= 1+sin(3*t)/2; $'r(t)=r(t)
\end{eulerprompt}
\begin{eulerformula}
\[
r\left(\left[ 0 , 0.01 , 0.02 , 0.03 , 0.04 , 0.05 , 0.06 , 0.07 ,   0.08 , 0.09 , 0.1 , 0.11 , 0.12 , 0.13 , 0.14 , 0.15 , 0.16 , 0.17   , 0.18 , 0.19 , 0.2 , 0.21 , 0.2200000000000001 ,   0.2300000000000001 , 0.2400000000000001 , 0.2500000000000001 ,   0.2600000000000001 , 0.2700000000000001 , 0.2800000000000001 ,   0.2900000000000001 , 0.3000000000000001 , 0.3100000000000001 ,   0.3200000000000001 , 0.3300000000000001 , 0.3400000000000001 ,   0.3500000000000001 , 0.3600000000000002 , 0.3700000000000002 ,   0.3800000000000002 , 0.3900000000000002 , 0.4000000000000002 ,   0.4100000000000002 , 0.4200000000000002 , 0.4300000000000002 ,   0.4400000000000002 , 0.4500000000000002 , 0.4600000000000002 ,   0.4700000000000003 , 0.4800000000000003 , 0.4900000000000003 ,   0.5000000000000002 , 0.5100000000000002 , 0.5200000000000002 ,   0.5300000000000002 , 0.5400000000000003 , 0.5500000000000003 ,   0.5600000000000003 , 0.5700000000000003 , 0.5800000000000003 ,   0.5900000000000003 , 0.6000000000000003 , 0.6100000000000003 ,   0.6200000000000003 , 0.6300000000000003 , 0.6400000000000003 ,   0.6500000000000004 , 0.6600000000000004 , 0.6700000000000004 ,   0.6800000000000004 , 0.6900000000000004 , 0.7000000000000004 ,   0.7100000000000004 , 0.7200000000000004 , 0.7300000000000004 ,   0.7400000000000004 , 0.7500000000000004 , 0.7600000000000005 ,   0.7700000000000005 , 0.7800000000000005 , 0.7900000000000005 ,   0.8000000000000005 , 0.8100000000000005 , 0.8200000000000005 ,   0.8300000000000005 , 0.8400000000000005 , 0.8500000000000005 ,   0.8600000000000005 , 0.8700000000000006 , 0.8800000000000006 ,   0.8900000000000006 , 0.9000000000000006 , 0.9100000000000006 ,   0.9200000000000006 , 0.9300000000000006 , 0.9400000000000006 ,   0.9500000000000006 , 0.9600000000000006 , 0.9700000000000006 ,   0.9800000000000006 , 0.9900000000000007 \right] \right)=\left[ 1 ,   1.014997750101248 , 1.029982003239722 , 1.044939274599006 ,   1.05985610364446 , 1.0747190662368 , 1.089514786712912 ,   1.10422994992305 , 1.118851313213567 , 1.133365718344415 ,   1.14776010333067 , 1.162021514197434 , 1.176137116637545 ,   1.190094207561581 , 1.203880226529785 , 1.217482767055615 ,   1.230889587770742 , 1.244088623441454 , 1.257067995826556 ,   1.269816024366985 , 1.282321236697518 , 1.294572378971135 ,   1.306558425986717 , 1.318268591110984 , 1.329692335985737 ,   1.340819380011667 , 1.351639709600205 , 1.362143587185071 ,   1.37232155998543 , 1.382164468512753 , 1.391663454813742 ,   1.400809970441889 , 1.409595784150499 , 1.41801298930026 ,   1.426054010974682 , 1.433711612797009 , 1.440978903442474 ,   1.447849342840024 , 1.454316748057942 , 1.460375298868068 ,   1.466019542983613 , 1.471244400965849 , 1.476045170795258 ,   1.480417532103036 , 1.484357550059133 , 1.48786167891333 ,   1.49092676518618 , 1.493550050506925 , 1.495729174095843 ,   1.49746217488879 , 1.498747493302027 , 1.499583972635738 ,   1.499970860114983 , 1.499907807567145 , 1.499394871735262 ,   1.498432514226959 , 1.497021601099038 , 1.495163402078079 ,   1.492859589417777 , 1.490112236394023 , 1.486923815439098 ,   1.483297195916649 , 1.479235641539457 , 1.474742807432315 ,   1.469822736842662 , 1.464479857501934 , 1.458718977640905 ,   1.4525452816626 , 1.44596432547669 , 1.438982031499539 ,   1.431604683324436 , 1.423838920066784 , 1.415691730389341 ,   1.407170446212898 , 1.398282736118043 , 1.38903659844396 ,   1.379440354090461 , 1.369502639029735 , 1.359232396534563 ,   1.348638869129968 , 1.337731590275575 , 1.326520375786132 ,   1.315015314997945 , 1.303226761689157 , 1.29116532476204 ,   1.278841858695708 , 1.26626745377781 , 1.253453426124026 ,   1.240411307494323 , 1.227152834915152 , 1.213689940116914 ,   1.200034738796209 , 1.186199519712527 , 1.172196733629194 ,   1.158038982108526 , 1.143739006171271 , 1.129309674830555 ,   1.114763973510631 , 1.100114992360884 , 1.085375914475572 \right] 
\]
\end{eulerformula}
\begin{eulerprompt}
>function fx(t) &= r(t)*cos(t); $'fx(t)=fx(t)
\end{eulerprompt}
\begin{eulerformula}
\[
{\it fx}\left(\left[ 0 , 0.01 , 0.02 , 0.03 , 0.04 , 0.05 , 0.06 ,   0.07 , 0.08 , 0.09 , 0.1 , 0.11 , 0.12 , 0.13 , 0.14 , 0.15 , 0.16   , 0.17 , 0.18 , 0.19 , 0.2 , 0.21 , 0.2200000000000001 ,   0.2300000000000001 , 0.2400000000000001 , 0.2500000000000001 ,   0.2600000000000001 , 0.2700000000000001 , 0.2800000000000001 ,   0.2900000000000001 , 0.3000000000000001 , 0.3100000000000001 ,   0.3200000000000001 , 0.3300000000000001 , 0.3400000000000001 ,   0.3500000000000001 , 0.3600000000000002 , 0.3700000000000002 ,   0.3800000000000002 , 0.3900000000000002 , 0.4000000000000002 ,   0.4100000000000002 , 0.4200000000000002 , 0.4300000000000002 ,   0.4400000000000002 , 0.4500000000000002 , 0.4600000000000002 ,   0.4700000000000003 , 0.4800000000000003 , 0.4900000000000003 ,   0.5000000000000002 , 0.5100000000000002 , 0.5200000000000002 ,   0.5300000000000002 , 0.5400000000000003 , 0.5500000000000003 ,   0.5600000000000003 , 0.5700000000000003 , 0.5800000000000003 ,   0.5900000000000003 , 0.6000000000000003 , 0.6100000000000003 ,   0.6200000000000003 , 0.6300000000000003 , 0.6400000000000003 ,   0.6500000000000004 , 0.6600000000000004 , 0.6700000000000004 ,   0.6800000000000004 , 0.6900000000000004 , 0.7000000000000004 ,   0.7100000000000004 , 0.7200000000000004 , 0.7300000000000004 ,   0.7400000000000004 , 0.7500000000000004 , 0.7600000000000005 ,   0.7700000000000005 , 0.7800000000000005 , 0.7900000000000005 ,   0.8000000000000005 , 0.8100000000000005 , 0.8200000000000005 ,   0.8300000000000005 , 0.8400000000000005 , 0.8500000000000005 ,   0.8600000000000005 , 0.8700000000000006 , 0.8800000000000006 ,   0.8900000000000006 , 0.9000000000000006 , 0.9100000000000006 ,   0.9200000000000006 , 0.9300000000000006 , 0.9400000000000006 ,   0.9500000000000006 , 0.9600000000000006 , 0.9700000000000006 ,   0.9800000000000006 , 0.9900000000000007 \right] \right)=\left[ 1 ,   1.014947000636657 , 1.029776013705529 , 1.044469087191079 ,   1.059008331806833 , 1.073375947255439 , 1.087554248364218 ,   1.101525691055367 , 1.11527289811021 , 1.128778684687222 ,   1.142026083553954 , 1.154998369993414 , 1.16767908634602 ,   1.180052066148761 , 1.192101457833886 , 1.203811747950136 ,   1.215167783870255 , 1.226154795949382 , 1.236758419099762 ,   1.246964713748154 , 1.256760186143285 , 1.266131807981756 ,   1.275067035321848 , 1.283553826755846 , 1.29158066081265 ,   1.29913655256367 , 1.306211069406282 , 1.312794346000405 ,   1.318877098335118 , 1.324450636903608 , 1.329506878966172 ,   1.334038359882425 , 1.338038243495345 , 1.341500331551311 ,   1.344419072141793 , 1.346789567153917 , 1.348607578718725 ,   1.349869534647481 , 1.350572532848044 , 1.350714344714907 ,   1.350293417488142 , 1.349308875578123 , 1.347760520854542 ,   1.345648831899879 , 1.342974962229111 , 1.339740737479097 ,   1.335948651572729 , 1.331601861864506 , 1.326704183275865 ,   1.321260081430156 , 1.315274664798767 , 1.308753675871437 ,   1.301703481365363 , 1.294131061489226 , 1.286043998279732 ,   1.277450463029762 , 1.268359202828647 , 1.25877952623647 ,   1.248721288115691 , 1.238194873644713 , 1.227211181539273 ,   1.215781606508839 , 1.203918020976346 , 1.191632756090801 ,   1.17893858206338 , 1.165848687858719 , 1.152376660274093 ,   1.138536462440146 , 1.124342411777761 , 1.10980915744646 ,   1.094951657320579 , 1.079785154530145 , 1.064325153604093 ,   1.04858739625406 , 1.032587836837555 , 1.0163426175398 ,   0.999868043313951 , 0.9831805566197906 , 0.9662967120012925 ,   0.9492331505436565 , 0.932006574250646 , 0.9146337203831 ,   0.897131335799599 , 0.8795161513401855 , 0.8618048562939812 ,   0.8440140729913906 , 0.8261603315613344 , 0.8082600448937051 ,   0.7903294838468643 , 0.7723847527396025 , 0.754441765166499 ,   0.7365162201750889 , 0.7186235788426429 , 0.7007790412897039 ,   0.6829975241668103 , 0.6652936386500562 , 0.6476816689803099 ,   0.6301755515800127 , 0.6127888547805567 , 0.595534759192214 \right] 
\]
\end{eulerformula}
\begin{eulerprompt}
>function fy(t) &= r(t)*sin(t); $'fy(t)=fy(t)
\end{eulerprompt}
\begin{eulerformula}
\[
{\it fy}\left(\left[ 0 , 0.01 , 0.02 , 0.03 , 0.04 , 0.05 , 0.06 ,   0.07 , 0.08 , 0.09 , 0.1 , 0.11 , 0.12 , 0.13 , 0.14 , 0.15 , 0.16   , 0.17 , 0.18 , 0.19 , 0.2 , 0.21 , 0.2200000000000001 ,   0.2300000000000001 , 0.2400000000000001 , 0.2500000000000001 ,   0.2600000000000001 , 0.2700000000000001 , 0.2800000000000001 ,   0.2900000000000001 , 0.3000000000000001 , 0.3100000000000001 ,   0.3200000000000001 , 0.3300000000000001 , 0.3400000000000001 ,   0.3500000000000001 , 0.3600000000000002 , 0.3700000000000002 ,   0.3800000000000002 , 0.3900000000000002 , 0.4000000000000002 ,   0.4100000000000002 , 0.4200000000000002 , 0.4300000000000002 ,   0.4400000000000002 , 0.4500000000000002 , 0.4600000000000002 ,   0.4700000000000003 , 0.4800000000000003 , 0.4900000000000003 ,   0.5000000000000002 , 0.5100000000000002 , 0.5200000000000002 ,   0.5300000000000002 , 0.5400000000000003 , 0.5500000000000003 ,   0.5600000000000003 , 0.5700000000000003 , 0.5800000000000003 ,   0.5900000000000003 , 0.6000000000000003 , 0.6100000000000003 ,   0.6200000000000003 , 0.6300000000000003 , 0.6400000000000003 ,   0.6500000000000004 , 0.6600000000000004 , 0.6700000000000004 ,   0.6800000000000004 , 0.6900000000000004 , 0.7000000000000004 ,   0.7100000000000004 , 0.7200000000000004 , 0.7300000000000004 ,   0.7400000000000004 , 0.7500000000000004 , 0.7600000000000005 ,   0.7700000000000005 , 0.7800000000000005 , 0.7900000000000005 ,   0.8000000000000005 , 0.8100000000000005 , 0.8200000000000005 ,   0.8300000000000005 , 0.8400000000000005 , 0.8500000000000005 ,   0.8600000000000005 , 0.8700000000000006 , 0.8800000000000006 ,   0.8900000000000006 , 0.9000000000000006 , 0.9100000000000006 ,   0.9200000000000006 , 0.9300000000000006 , 0.9400000000000006 ,   0.9500000000000006 , 0.9600000000000006 , 0.9700000000000006 ,   0.9800000000000006 , 0.9900000000000007 \right] \right)=\left[ 0 ,   0.01014980833556662 , 0.02059826678292271 , 0.03134347622283015 ,   0.04238293991838228 , 0.05371356612987439 , 0.06533167172990376 ,   0.07723298681299934 , 0.08941266029246918 , 0.1018652664755576 ,   0.1145848126064173 , 0.1275647473648353 , 0.1407979703071057 ,   0.1542768422339107 , 0.1679931964685752 , 0.1819383510275811 ,   0.1961031216637831 , 0.2104778357613507 , 0.2250523470600841 ,   0.2398160511854019 , 0.2547579019589912 , 0.2698664284638497 ,   0.2851297528362152 , 0.3005356087557041 , 0.3160713606038417 ,   0.3317240232600813 , 0.3474802825033731 , 0.3633265159863522 ,   0.3792488147482899 , 0.3952330052320643 , 0.411264671769591 ,   0.4273291794993832 , 0.4434116976792021 , 0.4594972233561165 ,   0.4755706053556919 , 0.4916165685515136 , 0.5076197383757777 ,   0.5235646655312819 , 0.5394358508648145 , 0.5552177703616642 ,   0.5708949002207642 , 0.5864517419698421 , 0.6018728475798654 ,   0.6171428445380648 , 0.6322464608388652 , 0.6471685498521687 ,   0.6618941150286309 , 0.6764083344018014 , 0.6906965848473219 ,   0.704744466059751 , 0.7185378242080237 , 0.7320627752310482 ,   0.7453057277355214 , 0.7582534054586558 , 0.7708928692592016 ,   0.7832115386008901 , 0.7951972124932317 , 0.8068380898554457 ,   0.8181227892702304 , 0.8290403680950348 , 0.8395803408995157 ,   0.8497326971989371 , 0.8594879184543822 , 0.8688369943118147 ,   0.877771438053233 , 0.8862833012344233 , 0.894365187485098 ,   0.9020102654485477 , 0.9092122808393135 , 0.91596556759876 ,   0.9222650581299157 , 0.9281062925943645 , 0.9334854272555032 ,   0.9383992418539865 , 0.9428451460027243 , 0.9468211845903713 ,   0.9503260421838114 , 0.9533590464217597 , 0.9559201703932094 ,   0.9580100339960551 , 0.9596299042728891 , 0.9607816947225576 ,   0.9614679635877484 , 0.9616919111204768 , 0.9614573758289937 ,   0.9607688297112769 , 0.9596313724818526 , 0.9580507248003547 ,   0.9560332205117796 , 0.9535857979100135 , 0.950715990037748 ,   0.9474319140374602 , 0.9437422595696462 , 0.9396562763159917 ,   0.9351837605866338 , 0.9303350410521015 , 0.9251209636219332 ,   0.9195528754933222 , 0.9136426083945087 , 0.9074024610488752   \right] 
\]
\end{eulerformula}
\begin{eulerprompt}
>function ds(t) &= trigreduce(radcan(sqrt(diff(fx(t),t)^2+diff(fy(t),t)^2))); $'ds(t)=ds(t)
\end{eulerprompt}
\begin{euleroutput}
  Maxima said:
  diff: second argument must be a variable; found errexp1
   -- an error. To debug this try: debugmode(true);
  
  Error in:
  ... e(radcan(sqrt(diff(fx(t),t)^2+diff(fy(t),t)^2))); $'ds(t)=ds(t ...
                                                       ^
\end{euleroutput}
\begin{eulerprompt}
>$integrate(ds(x),x,0,2*pi) //panjang (keliling) kurva
\end{eulerprompt}
\begin{eulerformula}
\[
\int_{0}^{2\,\pi}{{\it ds}\left(x\right)\;dx}
\]
\end{eulerformula}
\begin{eulercomment}
Maxima gagal melakukan perhitungan eksak integral tersebut.

Berikut kita hitung integralnya secara umerik dengan perintah EMT.
\end{eulercomment}
\begin{eulerprompt}
>integrate("ds(x)",0,2*pi)
\end{eulerprompt}
\begin{euleroutput}
  Function ds not found.
  Try list ... to find functions!
  Error in expression: ds(x)
   %mapexpression1:
      return expr(x,args());
  Error in map.
   %evalexpression:
      if maps then return %mapexpression1(x,f$;args());
  gauss:
      if maps then y=%evalexpression(f$,a+h-(h*xn)',maps;args());
  adaptivegauss:
      t1=gauss(f$,c,c+h;args(),=maps);
  Try "trace errors" to inspect local variables after errors.
  integrate:
      return adaptivegauss(f$,a,b,eps*1000;args(),=maps);
\end{euleroutput}
\begin{eulercomment}
Spiral Logaritmik

\end{eulercomment}
\begin{eulerformula}
\[
x=e^{ax}\cos x,\ y=e^{ax}\sin x.
\]
\end{eulerformula}
\begin{eulerprompt}
>a=0.1; plot2d("exp(a*x)*cos(x)","exp(a*x)*sin(x)",r=2,xmin=0,xmax=2*pi):
\end{eulerprompt}
\eulerimg{17}{images/Davina Safa Felisa 1-6-611.png}
\begin{eulerprompt}
>&kill(a) // hapus expresi a
\end{eulerprompt}
\begin{euleroutput}
  
                                   done
  
\end{euleroutput}
\begin{eulerprompt}
>function fx(t) &= exp(a*t)*cos(t); $'fx(t)=fx(t)
\end{eulerprompt}
\begin{eulerformula}
\[
{\it fx}\left(\left[ 0 , 0.01 , 0.02 , 0.03 , 0.04 , 0.05 , 0.06 ,   0.07 , 0.08 , 0.09 , 0.1 , 0.11 , 0.12 , 0.13 , 0.14 , 0.15 , 0.16   , 0.17 , 0.18 , 0.19 , 0.2 , 0.21 , 0.2200000000000001 ,   0.2300000000000001 , 0.2400000000000001 , 0.2500000000000001 ,   0.2600000000000001 , 0.2700000000000001 , 0.2800000000000001 ,   0.2900000000000001 , 0.3000000000000001 , 0.3100000000000001 ,   0.3200000000000001 , 0.3300000000000001 , 0.3400000000000001 ,   0.3500000000000001 , 0.3600000000000002 , 0.3700000000000002 ,   0.3800000000000002 , 0.3900000000000002 , 0.4000000000000002 ,   0.4100000000000002 , 0.4200000000000002 , 0.4300000000000002 ,   0.4400000000000002 , 0.4500000000000002 , 0.4600000000000002 ,   0.4700000000000003 , 0.4800000000000003 , 0.4900000000000003 ,   0.5000000000000002 , 0.5100000000000002 , 0.5200000000000002 ,   0.5300000000000002 , 0.5400000000000003 , 0.5500000000000003 ,   0.5600000000000003 , 0.5700000000000003 , 0.5800000000000003 ,   0.5900000000000003 , 0.6000000000000003 , 0.6100000000000003 ,   0.6200000000000003 , 0.6300000000000003 , 0.6400000000000003 ,   0.6500000000000004 , 0.6600000000000004 , 0.6700000000000004 ,   0.6800000000000004 , 0.6900000000000004 , 0.7000000000000004 ,   0.7100000000000004 , 0.7200000000000004 , 0.7300000000000004 ,   0.7400000000000004 , 0.7500000000000004 , 0.7600000000000005 ,   0.7700000000000005 , 0.7800000000000005 , 0.7900000000000005 ,   0.8000000000000005 , 0.8100000000000005 , 0.8200000000000005 ,   0.8300000000000005 , 0.8400000000000005 , 0.8500000000000005 ,   0.8600000000000005 , 0.8700000000000006 , 0.8800000000000006 ,   0.8900000000000006 , 0.9000000000000006 , 0.9100000000000006 ,   0.9200000000000006 , 0.9300000000000006 , 0.9400000000000006 ,   0.9500000000000006 , 0.9600000000000006 , 0.9700000000000006 ,   0.9800000000000006 , 0.9900000000000007 \right] \right)=\left[ 1 ,   0.9999500004166653\,e^{0.01\,a} , 0.9998000066665778\,e^{0.02\,a} ,   0.9995500337489875\,e^{0.03\,a} , 0.9992001066609779\,e^{0.04\,a} ,   0.9987502603949663\,e^{0.05\,a} , 0.9982005399352042\,e^{0.06\,a} ,   0.9975510002532796\,e^{0.07\,a} , 0.9968017063026194\,e^{0.08\,a} ,   0.9959527330119943\,e^{0.09\,a} , 0.9950041652780258\,e^{0.1\,a} ,   0.9939560979566968\,e^{0.11\,a} , 0.9928086358538663\,e^{0.12\,a} ,   0.9915618937147881\,e^{0.13\,a} , 0.9902159962126372\,e^{0.14\,a} ,   0.9887710779360422\,e^{0.15\,a} , 0.9872272833756269\,e^{0.16\,a} ,   0.9855847669095608\,e^{0.17\,a} , 0.9838436927881214\,e^{0.18\,a} ,   0.9820042351172703\,e^{0.19\,a} , 0.9800665778412416\,e^{0.2\,a} ,   0.9780309147241483\,e^{0.21\,a} , 0.9758974493306055\,e^{  0.2200000000000001\,a} , 0.9736663950053748\,e^{0.2300000000000001\,  a} , 0.9713379748520296\,e^{0.2400000000000001\,a} ,   0.9689124217106447\,e^{0.2500000000000001\,a} , 0.9663899781345132\,  e^{0.2600000000000001\,a} , 0.9637708963658905\,e^{  0.2700000000000001\,a} , 0.9610554383107709\,e^{0.2800000000000001\,  a} , 0.9582438755126972\,e^{0.2900000000000001\,a} ,   0.955336489125606\,e^{0.3000000000000001\,a} , 0.9523335698857134\,e  ^{0.3100000000000001\,a} , 0.9492354180824408\,e^{0.3200000000000001  \,a} , 0.9460423435283869\,e^{0.3300000000000001\,a} ,   0.9427546655283462\,e^{0.3400000000000001\,a} , 0.9393727128473789\,  e^{0.3500000000000001\,a} , 0.9358968236779348\,e^{  0.3600000000000002\,a} , 0.9323273456060344\,e^{0.3700000000000002\,  a} , 0.9286646355765101\,e^{0.3800000000000002\,a} ,   0.924909059857313\,e^{0.3900000000000002\,a} , 0.921060994002885\,e  ^{0.4000000000000002\,a} , 0.917120822816605\,e^{0.4100000000000002  \,a} , 0.9130889403123081\,e^{0.4200000000000002\,a} ,   0.9089657496748851\,e^{0.4300000000000002\,a} , 0.9047516632199634\,  e^{0.4400000000000002\,a} , 0.9004471023526768\,e^{  0.4500000000000002\,a} , 0.8960524975255252\,e^{0.4600000000000002\,  a} , 0.8915682881953289\,e^{0.4700000000000003\,a} ,   0.886994922779284\,e^{0.4800000000000003\,a} , 0.8823328586101213\,e  ^{0.4900000000000003\,a} , 0.8775825618903726\,e^{0.5000000000000002  \,a} , 0.8727445076457512\,e^{0.5100000000000002\,a} ,   0.8678191796776498\,e^{0.5200000000000002\,a} , 0.8628070705147609\,  e^{0.5300000000000002\,a} , 0.857708681363824\,e^{0.5400000000000003  \,a} , 0.8525245220595056\,e^{0.5500000000000003\,a} ,   0.847255111013416\,e^{0.5600000000000003\,a} , 0.8419009751622686\,e  ^{0.5700000000000003\,a} , 0.8364626499151868\,e^{0.5800000000000003  \,a} , 0.8309406791001633\,e^{0.5900000000000003\,a} ,   0.8253356149096781\,e^{0.6000000000000003\,a} , 0.8196480178454794\,  e^{0.6100000000000003\,a} , 0.8138784566625338\,e^{  0.6200000000000003\,a} , 0.8080275083121516\,e^{0.6300000000000003\,  a} , 0.8020957578842924\,e^{0.6400000000000003\,a} ,   0.7960837985490556\,e^{0.6500000000000004\,a} , 0.7899922314973649\,  e^{0.6600000000000004\,a} , 0.783821665880849\,e^{0.6700000000000004  \,a} , 0.7775727187509277\,e^{0.6800000000000004\,a} ,   0.7712460149971063\,e^{0.6900000000000004\,a} , 0.7648421872844882\,  e^{0.7000000000000004\,a} , 0.7583618759905079\,e^{  0.7100000000000004\,a} , 0.7518057291408947\,e^{0.7200000000000004\,  a} , 0.7451744023448701\,e^{0.7300000000000004\,a} ,   0.7384685587295876\,e^{0.7400000000000004\,a} , 0.7316888688738206\,  e^{0.7500000000000004\,a} , 0.7248360107409049\,e^{  0.7600000000000005\,a} , 0.7179106696109431\,e^{0.7700000000000005\,  a} , 0.7109135380122771\,e^{0.7800000000000005\,a} ,   0.7038453156522357\,e^{0.7900000000000005\,a} , 0.696706709347165\,e  ^{0.8000000000000005\,a} , 0.6894984329517466\,e^{0.8100000000000005  \,a} , 0.6822212072876132\,e^{0.8200000000000005\,a} ,   0.6748757600712667\,e^{0.8300000000000005\,a} , 0.6674628258413078\,  e^{0.8400000000000005\,a} , 0.6599831458849817\,e^{  0.8500000000000005\,a} , 0.6524374681640515\,e^{0.8600000000000005\,  a} , 0.6448265472400008\,e^{0.8700000000000006\,a} ,   0.6371511441985798\,e^{0.8800000000000006\,a} , 0.6294120265736964\,  e^{0.8900000000000006\,a} , 0.6216099682706641\,e^{  0.9000000000000006\,a} , 0.6137457494888111\,e^{0.9100000000000006\,  a} , 0.6058201566434623\,e^{0.9200000000000006\,a} ,   0.5978339822872978\,e^{0.9300000000000006\,a} , 0.5897880250310977\,  e^{0.9400000000000006\,a} , 0.581683089463883\,e^{0.9500000000000006  \,a} , 0.5735199860724561\,e^{0.9600000000000006\,a} ,   0.5652995311603538\,e^{0.9700000000000006\,a} , 0.5570225467662168\,  e^{0.9800000000000006\,a} , 0.548689860581587\,e^{0.9900000000000007  \,a} \right] 
\]
\end{eulerformula}
\begin{eulerprompt}
>function fy(t) &= exp(a*t)*sin(t); $'fy(t)=fy(t)
\end{eulerprompt}
\begin{eulerformula}
\[
{\it fy}\left(\left[ 0 , 0.01 , 0.02 , 0.03 , 0.04 , 0.05 , 0.06 ,   0.07 , 0.08 , 0.09 , 0.1 , 0.11 , 0.12 , 0.13 , 0.14 , 0.15 , 0.16   , 0.17 , 0.18 , 0.19 , 0.2 , 0.21 , 0.2200000000000001 ,   0.2300000000000001 , 0.2400000000000001 , 0.2500000000000001 ,   0.2600000000000001 , 0.2700000000000001 , 0.2800000000000001 ,   0.2900000000000001 , 0.3000000000000001 , 0.3100000000000001 ,   0.3200000000000001 , 0.3300000000000001 , 0.3400000000000001 ,   0.3500000000000001 , 0.3600000000000002 , 0.3700000000000002 ,   0.3800000000000002 , 0.3900000000000002 , 0.4000000000000002 ,   0.4100000000000002 , 0.4200000000000002 , 0.4300000000000002 ,   0.4400000000000002 , 0.4500000000000002 , 0.4600000000000002 ,   0.4700000000000003 , 0.4800000000000003 , 0.4900000000000003 ,   0.5000000000000002 , 0.5100000000000002 , 0.5200000000000002 ,   0.5300000000000002 , 0.5400000000000003 , 0.5500000000000003 ,   0.5600000000000003 , 0.5700000000000003 , 0.5800000000000003 ,   0.5900000000000003 , 0.6000000000000003 , 0.6100000000000003 ,   0.6200000000000003 , 0.6300000000000003 , 0.6400000000000003 ,   0.6500000000000004 , 0.6600000000000004 , 0.6700000000000004 ,   0.6800000000000004 , 0.6900000000000004 , 0.7000000000000004 ,   0.7100000000000004 , 0.7200000000000004 , 0.7300000000000004 ,   0.7400000000000004 , 0.7500000000000004 , 0.7600000000000005 ,   0.7700000000000005 , 0.7800000000000005 , 0.7900000000000005 ,   0.8000000000000005 , 0.8100000000000005 , 0.8200000000000005 ,   0.8300000000000005 , 0.8400000000000005 , 0.8500000000000005 ,   0.8600000000000005 , 0.8700000000000006 , 0.8800000000000006 ,   0.8900000000000006 , 0.9000000000000006 , 0.9100000000000006 ,   0.9200000000000006 , 0.9300000000000006 , 0.9400000000000006 ,   0.9500000000000006 , 0.9600000000000006 , 0.9700000000000006 ,   0.9800000000000006 , 0.9900000000000007 \right] \right)=\left[ 0 ,   0.009999833334166664\,e^{0.01\,a} , 0.01999866669333308\,e^{0.02\,a}   , 0.02999550020249566\,e^{0.03\,a} , 0.03998933418663416\,e^{0.04\,  a} , 0.04997916927067833\,e^{0.05\,a} , 0.0599640064794446\,e^{0.06  \,a} , 0.06994284733753277\,e^{0.07\,a} , 0.0799146939691727\,e^{  0.08\,a} , 0.08987854919801104\,e^{0.09\,a} , 0.09983341664682814\,e  ^{0.1\,a} , 0.1097783008371748\,e^{0.11\,a} , 0.1197122072889193\,e  ^{0.12\,a} , 0.1296341426196948\,e^{0.13\,a} , 0.1395431146442365\,e  ^{0.14\,a} , 0.1494381324735992\,e^{0.15\,a} , 0.159318206614246\,e  ^{0.16\,a} , 0.169182349066996\,e^{0.17\,a} , 0.1790295734258242\,e  ^{0.18\,a} , 0.1888588949765006\,e^{0.19\,a} , 0.1986693307950612\,e  ^{0.2\,a} , 0.2084598998460996\,e^{0.21\,a} , 0.2182296230808694\,e  ^{0.2200000000000001\,a} , 0.2279775235351885\,e^{0.2300000000000001  \,a} , 0.2377026264271347\,e^{0.2400000000000001\,a} ,   0.247403959254523\,e^{0.2500000000000001\,a} , 0.2570805518921552\,e  ^{0.2600000000000001\,a} , 0.2667314366888312\,e^{0.2700000000000001  \,a} , 0.2763556485641138\,e^{0.2800000000000001\,a} ,   0.2859522251048356\,e^{0.2900000000000001\,a} , 0.2955202066613397\,  e^{0.3000000000000001\,a} , 0.3050586364434436\,e^{  0.3100000000000001\,a} , 0.3145665606161179\,e^{0.3200000000000001\,  a} , 0.3240430283948685\,e^{0.3300000000000001\,a} ,   0.3334870921408145\,e^{0.3400000000000001\,a} , 0.3428978074554515\,  e^{0.3500000000000001\,a} , 0.3522742332750901\,e^{  0.3600000000000002\,a} , 0.3616154319649622\,e^{0.3700000000000002\,  a} , 0.3709204694129828\,e^{0.3800000000000002\,a} ,   0.3801884151231616\,e^{0.3900000000000002\,a} , 0.3894183423086507\,  e^{0.4000000000000002\,a} , 0.3986093279844231\,e^{  0.4100000000000002\,a} , 0.4077604530595704\,e^{0.4200000000000002\,  a} , 0.416870802429211\,e^{0.4300000000000002\,a} ,   0.4259394650659998\,e^{0.4400000000000002\,a} , 0.4349655341112304\,  e^{0.4500000000000002\,a} , 0.44394810696552\,e^{0.4600000000000002  \,a} , 0.4528862853790685\,e^{0.4700000000000003\,a} ,   0.4617791755414831\,e^{0.4800000000000003\,a} , 0.4706258881711582\,  e^{0.4900000000000003\,a} , 0.4794255386042032\,e^{  0.5000000000000002\,a} , 0.4881772468829077\,e^{0.5100000000000002\,  a} , 0.4968801378437369\,e^{0.5200000000000002\,a} ,   0.5055333412048472\,e^{0.5300000000000002\,a} , 0.5141359916531133\,  e^{0.5400000000000003\,a} , 0.5226872289306594\,e^{  0.5500000000000003\,a} , 0.5311861979208836\,e^{0.5600000000000003\,  a} , 0.5396320487339695\,e^{0.5700000000000003\,a} ,   0.5480239367918738\,e^{0.5800000000000003\,a} , 0.556361022912784\,e  ^{0.5900000000000003\,a} , 0.5646424733950356\,e^{0.6000000000000003  \,a} , 0.5728674601004815\,e^{0.6100000000000003\,a} ,   0.5810351605373053\,e^{0.6200000000000003\,a} , 0.5891447579422698\,  e^{0.6300000000000003\,a} , 0.5971954413623923\,e^{  0.6400000000000003\,a} , 0.6051864057360399\,e^{0.6500000000000004\,  a} , 0.6131168519734341\,e^{0.6600000000000004\,a} ,   0.6209859870365599\,e^{0.6700000000000004\,a} , 0.6287930240184688\,  e^{0.6800000000000004\,a} , 0.6365371822219682\,e^{  0.6900000000000004\,a} , 0.6442176872376913\,e^{0.7000000000000004\,  a} , 0.651833771021537\,e^{0.7100000000000004\,a} ,   0.6593846719714734\,e^{0.7200000000000004\,a} , 0.6668696350036982\,  e^{0.7300000000000004\,a} , 0.6742879116281454\,e^{  0.7400000000000004\,a} , 0.6816387600233345\,e^{0.7500000000000004\,  a} , 0.6889214451105516\,e^{0.7600000000000005\,a} ,   0.696135238627357\,e^{0.7700000000000005\,a} , 0.7032794192004105\,e  ^{0.7800000000000005\,a} , 0.7103532724176082\,e^{0.7900000000000005  \,a} , 0.7173560908995231\,e^{0.8000000000000005\,a} ,   0.7242871743701429\,e^{0.8100000000000005\,a} , 0.7311458297268962\,  e^{0.8200000000000005\,a} , 0.7379313711099631\,e^{  0.8300000000000005\,a} , 0.7446431199708596\,e^{0.8400000000000005\,  a} , 0.751280405140293\,e^{0.8500000000000005\,a} ,   0.7578425628952773\,e^{0.8600000000000005\,a} , 0.7643289370255054\,  e^{0.8700000000000006\,a} , 0.7707388788989696\,e^{  0.8800000000000006\,a} , 0.7770717475268242\,e^{0.8900000000000006\,  a} , 0.7833269096274837\,e^{0.9000000000000006\,a} ,   0.7895037396899508\,e^{0.9100000000000006\,a} , 0.7956016200363664\,  e^{0.9200000000000006\,a} , 0.8016199408837775\,e^{  0.9300000000000006\,a} , 0.8075581004051147\,e^{0.9400000000000006\,  a} , 0.8134155047893741\,e^{0.9500000000000006\,a} ,   0.8191915683009986\,e^{0.9600000000000006\,a} , 0.8248857133384504\,  e^{0.9700000000000006\,a} , 0.8304973704919708\,e^{  0.9800000000000006\,a} , 0.8360259786005209\,e^{0.9900000000000007\,  a} \right] 
\]
\end{eulerformula}
\begin{eulerprompt}
>function df(t) &= trigreduce(radcan(sqrt(diff(fx(t),t)^2+diff(fy(t),t)^2))); $'df(t)=df(t)
\end{eulerprompt}
\begin{euleroutput}
  Maxima said:
  diff: second argument must be a variable; found errexp1
   -- an error. To debug this try: debugmode(true);
  
  Error in:
  ... e(radcan(sqrt(diff(fx(t),t)^2+diff(fy(t),t)^2))); $'df(t)=df(t ...
                                                       ^
\end{euleroutput}
\begin{eulerprompt}
>S &=integrate(df(t),t,0,2*%pi); $S // panjang kurva (spiral)
\end{eulerprompt}
\begin{euleroutput}
  Maxima said:
  expt: undefined: 0 to a negative exponent.
  #0: df(x=[0,0.01,0.02,0.03,0.04,0.05,0.06,0.07,0.08,0.09,0.1,0.11,0.12,0.13,0.14,0.15,0.16,0.17,0.18,0.19,0.2...)
   -- an error. To debug this try: debugmode(true);
  
  Error in:
  S &=integrate(df(t),t,0,2*%pi); $S // panjang kurva (spiral) ...
                                ^
\end{euleroutput}
\begin{eulerprompt}
>S(a=0.1) // Panjang kurva untuk a=0.1
\end{eulerprompt}
\begin{euleroutput}
  Function S not found.
  Try list ... to find functions!
  Error in:
  S(a=0.1) // Panjang kurva untuk a=0.1 ...
          ^
\end{euleroutput}
\begin{eulercomment}
Berikut adalah contoh menghitung panjang parabola
\end{eulercomment}
\begin{eulerprompt}
>plot2d("x^2",xmin=-1,xmax=1):
\end{eulerprompt}
\eulerimg{17}{images/Davina Safa Felisa 1-6-614.png}
\begin{eulerprompt}
>$showev('integrate(sqrt(1+diff(x^2,x)^2),x,-1,1))
\end{eulerprompt}
\begin{eulerformula}
\[
\int_{-1}^{1}{\sqrt{4\,x^2+1}\;dx}=\frac{{\rm asinh}\; 2+2\,\sqrt{5  }}{2}
\]
\end{eulerformula}
\begin{eulerprompt}
>$float(%)
\end{eulerprompt}
\begin{eulerformula}
\[
\int_{-1.0}^{1.0}{\sqrt{4.0\,x^2+1.0}\;dx}=2.957885715089195
\]
\end{eulerformula}
\begin{eulerprompt}
>x=-1:0.2:1; y=x^2; plot2d(x,y);  ...
>plot2d(x,y,points=1,style="o#",add=1):
\end{eulerprompt}
\eulerimg{17}{images/Davina Safa Felisa 1-6-617.png}
\begin{eulerprompt}
>i=1:cols(x)-1; sum(sqrt((x[i+1]-x[i])^2+(y[i+1]-y[i])^2))
\end{eulerprompt}
\begin{euleroutput}
  2.95191957027
\end{euleroutput}
\begin{eulercomment}
Hasilnya mendekati panjang yang dihitung secara eksak. Untuk
mendapatkan hampiran yang cukup akurat, jarak antar titik dapat
diperkecil, misalnya 0.1, 0.05, 0.01, dan seterusnya. Cobalah Anda
ulangi perhitungannya dengan nilai-nilai tersebut.

\begin{eulercomment}
\eulerheading{Koordinat Kartesius}
\begin{eulercomment}
Berikut diberikan contoh perhitungan panjang kurva menggunakan
koordinat Kartesius. Kita akan hitung panjang kurva dengan persamaan
implisit:

\end{eulercomment}
\begin{eulerformula}
\[
x^3+y^3-3xy=0.
\]
\end{eulerformula}
\begin{eulerprompt}
>z &= x^3+y^3-3*x*y; $z
\end{eulerprompt}
\begin{eulerformula}
\[
y^3-3\,x\,y+x^3
\]
\end{eulerformula}
\begin{eulerprompt}
>plot2d(z,r=2,level=0,n=100):
\end{eulerprompt}
\eulerimg{17}{images/Davina Safa Felisa 1-6-620.png}
\begin{eulercomment}
Kita tertarik pada kurva di kuadran pertama.
\end{eulercomment}
\begin{eulerprompt}
>plot2d(z,a=0,b=2,c=0,d=2,level=[-10;0],n=100,contourwidth=3,style="/"):
\end{eulerprompt}
\eulerimg{17}{images/Davina Safa Felisa 1-6-621.png}
\begin{eulercomment}
Kita selesaikan persamaannya untuk x.
\end{eulercomment}
\begin{eulerprompt}
>$z with y=l*x, sol &= solve(%,x); $sol
\end{eulerprompt}
\begin{eulerformula}
\[
\left[ x=\frac{3\,l}{l^3+1} , x=0 \right] 
\]
\end{eulerformula}
\eulerimg{1}{images/Davina Safa Felisa 1-6-623-large.png}
\begin{eulercomment}
Kita gunakan solusi tersebut untuk mendefinisikan fungsi dengan
Maxima.
\end{eulercomment}
\begin{eulerprompt}
>function f(l) &= rhs(sol[1]); $'f(l)=f(l)
\end{eulerprompt}
\begin{eulerformula}
\[
f\left(l\right)=\frac{3\,l}{l^3+1}
\]
\end{eulerformula}
\begin{eulercomment}
Fungsi tersebut juga dapat digunaka untuk menggambar kurvanya. Ingat,
bahwa fungsi tersebut adalah nilai x dan nilai y=l*x, yakni x=f(l) dan
y=l*f(l).
\end{eulercomment}
\begin{eulerprompt}
>plot2d(&f(x),&x*f(x),xmin=-0.5,xmax=2,a=0,b=2,c=0,d=2,r=1.5):
\end{eulerprompt}
\begin{euleroutput}
  Unexpected "(". Index () not allowed in strict mode!
  In Euler files, use relax to avoid this.
  Error in expression: f(x)
  adaptiveeval:
      sx=f$(t;args());
  Try "trace errors" to inspect local variables after errors.
  plot2d:
      dw/n,dw/n^2,dw/n;args());
\end{euleroutput}
\begin{eulercomment}
Elemen panjang kurva adalah:

\end{eulercomment}
\begin{eulerformula}
\[
ds=\sqrt{f'(l)^2+(lf'(l)+f(l))^2}.
\]
\end{eulerformula}
\begin{eulerprompt}
>function ds(l) &= ratsimp(sqrt(diff(f(l),l)^2+diff(l*f(l),l)^2)); $'ds(l)=ds(l)
\end{eulerprompt}
\begin{eulerformula}
\[
{\it ds}\left(l\right)=\sqrt{\left(l^2+1\right)\,\left(\frac{d}{d\,
 l}\,f\left(l\right)\right)^2+2\,l\,f\left(l\right)\,\left(\frac{d}{d
 \,l}\,f\left(l\right)\right)+f^2\left(l\right)}
\]
\end{eulerformula}
\begin{eulerprompt}
>$integrate(ds(l),l,0,1)
\end{eulerprompt}
\begin{eulerformula}
\[
\int_{0}^{1}{\sqrt{\left(l^2+1\right)\,\left(\frac{d}{d\,l}\,f
 \left(l\right)\right)^2+2\,l\,f\left(l\right)\,\left(\frac{d}{d\,l}
 \,f\left(l\right)\right)+f^2\left(l\right)}\;dl}
\]
\end{eulerformula}
\begin{eulercomment}
Integral tersebut tidak dapat dihitung secara eksak menggunakan
Maxima. Kita hitung integral etrsebut secara numerik dengan Euler.
Karena kurva simetris, kita hitung untuk nilai variabel integrasi dari
0 sampai 1, kemudian hasilnya dikalikan 2.
\end{eulercomment}
\begin{eulerprompt}
>2*integrate("ds(x)",0,1)
\end{eulerprompt}
\begin{euleroutput}
  Syntax error in expression, or unfinished expression!
  ds:
      useglobal; return sqrt((l^2+1)*('diff(f(l),l,1))^2+2*l*f(l)*' ...
  Error in expression: ds(x)
  %mapexpression1:
      return expr(x,args());
  Error in map.
  %evalexpression:
      if maps then return %mapexpression1(x,f$;args());
  gauss:
      if maps then y=%evalexpression(f$,a+h-(h*xn)',maps;args());
  adaptivegauss:
      t1=gauss(f$,c,c+h;args(),=maps);
  Try "trace errors" to inspect local variables after errors.
  integrate:
      return adaptivegauss(f$,a,b,eps*1000;args(),=maps);
\end{euleroutput}
\begin{eulerprompt}
>2*romberg(&ds(x),0,1)// perintah Euler lain untuk menghitung nilai hampiran integral yang sama
\end{eulerprompt}
\begin{euleroutput}
  Syntax error in expression, or unfinished expression!
  Error in expression: sqrt((x^2+1)*('diff(f(x),x,1))^2+2*x*f(x)*'diff(f(x),x,1)+f(x)^2)
  %evalexpression:
      else return f$(x,args());
  Try "trace errors" to inspect local variables after errors.
  romberg:
      y=%evalexpression(f$,linspace(a,b,m),maps;args());
\end{euleroutput}
\begin{eulercomment}
Perhitungan di datas dapat dilakukan untuk sebarang fungsi x dan y
dengan mendefinisikan fungsi EMT, misalnya kita beri nama
panjangkurva. Fungsi ini selalu memanggil Maxima untuk menurunkan
fungsi yang diberikan.
\end{eulercomment}
\begin{eulerprompt}
>function panjangkurva(fx,fy,a,b) ...
\end{eulerprompt}
\begin{eulerudf}
  ds=mxm("sqrt(diff(@fx,x)^2+diff(@fy,x)^2)");
  return romberg(ds,a,b);
  endfunction
\end{eulerudf}
\begin{eulerprompt}
>panjangkurva("x","x^2",-1,1) // cek untuk menghitung panjang kurva parabola sebelumnya
\end{eulerprompt}
\begin{euleroutput}
  2.9579
\end{euleroutput}
\begin{eulercomment}
Bandingkan dengan nilai eksak di atas.
\end{eulercomment}
\begin{eulerprompt}
>2*panjangkurva(mxm("f(x)"),mxm("x*f(x)"),0,1) // cek contoh terakhir, bandingkan hasilnya!
\end{eulerprompt}
\begin{euleroutput}
  Syntax error in expression, or unfinished expression!
  Error in expression: sqrt((x*'diff(f(x),x,1)+f(x))^2+('diff(f(x),x,1))^2)
  %evalexpression:
      if maps then return %mapexpression1(x,f$;args());
  romberg:
      y=%evalexpression(f$,linspace(a,b,m),maps;args());
  Try "trace errors" to inspect local variables after errors.
  panjangkurva:
      return romberg(ds,a,b);
\end{euleroutput}
\begin{eulercomment}
Kita hitung panjang spiral Archimides berikut ini dengan fungsi
tersebut.
\end{eulercomment}
\begin{eulerprompt}
>plot2d("x*cos(x)","x*sin(x)",xmin=0,xmax=2*pi,square=1):
\end{eulerprompt}
\eulerimg{15}{images/Davina Safa Felisa 1-6-628.png}
\begin{eulerprompt}
>panjangkurva("x*cos(x)","x*sin(x)",0,2*pi)
\end{eulerprompt}
\begin{euleroutput}
  21.256
\end{euleroutput}
\begin{eulercomment}
Berikut kita definisikan fungsi yang sama namun dengan Maxima, untuk
perhitungan eksak.
\end{eulercomment}
\begin{eulerprompt}
>&kill(ds,x,fx,fy)
\end{eulerprompt}
\begin{euleroutput}
  
                                   done
  
\end{euleroutput}
\begin{eulerprompt}
>function ds(fx,fy) &&= sqrt(diff(fx,x)^2+diff(fy,x)^2)
\end{eulerprompt}
\begin{euleroutput}
  
                             2              2
                    sqrt(diff (fy, x) + diff (fx, x))
  
\end{euleroutput}
\begin{eulerprompt}
>sol &= ds(x*cos(x),x*sin(x)); $sol // // Kita gunakan untuk menghitung panjang kurva terakhir di atas
\end{eulerprompt}
\begin{eulerformula}
\[
\sqrt{\left(\cos x-x\,\sin x\right)^2+\left(\sin x+x\,\cos x\right)
 ^2}
\]
\end{eulerformula}
\begin{eulerprompt}
>$sol | trigreduce | expand, $integrate(%,x,0,2*pi), %()
\end{eulerprompt}
\begin{eulerformula}
\[
\sqrt{x^2+1}
\]
\end{eulerformula}
\begin{eulerformula}
\[
\frac{{\rm asinh}\; \left(2\,\pi\right)+2\,\pi\,\sqrt{4\,\pi^2+1}}{
 2}
\]
\end{eulerformula}
\begin{euleroutput}
  21.256
\end{euleroutput}
\begin{eulercomment}
Hasilnya sama dengan perhitungan menggunakan fungsi EMT.

Berikut adalah contoh lain penggunaan fungsi Maxima tersebut.
\end{eulercomment}
\begin{eulerprompt}
>plot2d("3*x^2-1","3*x^3-1",xmin=-1/sqrt(3),xmax=1/sqrt(3),square=1):
\end{eulerprompt}
\eulerimg{15}{images/Davina Safa Felisa 1-6-632.png}
\begin{eulerprompt}
>sol &= radcan(ds(3*x^2-1,3*x^3-1)); $sol
\end{eulerprompt}
\begin{eulerformula}
\[
3\,x\,\sqrt{9\,x^2+4}
\]
\end{eulerformula}
\begin{eulerprompt}
>$showev('integrate(sol,x,0,1/sqrt(3))), $2*float(%) // panjang kurva di atas
\end{eulerprompt}
\begin{eulerformula}
\[
3\,\int_{0}^{\frac{1}{\sqrt{3}}}{x\,\sqrt{9\,x^2+4}\;dx}=3\,\left(
 \frac{7^{\frac{3}{2}}}{27}-\frac{8}{27}\right)
\]
\end{eulerformula}
\begin{eulerformula}
\[
6.0\,\int_{0.0}^{0.5773502691896258}{x\,\sqrt{9.0\,x^2+4.0}\;dx}=
 2.337835372767141
\]
\end{eulerformula}
\eulerheading{Sikloid}
\begin{eulercomment}
Berikut kita akan menghitung panjang kurva lintasan (sikloid) suatu
titik pada lingkaran yang berputar ke kanan pada permukaan datar.
Misalkan jari-jari lingkaran tersebut adalah r. Posisi titik pusat
lingkaran pada saat t adalah:

\end{eulercomment}
\begin{eulerformula}
\[
(rt,r).
\]
\end{eulerformula}
\begin{eulercomment}
Misalkan posisi titik pada lingkaran tersebut mula-mula (0,0) dan
posisinya pada saat t adalah:

\end{eulercomment}
\begin{eulerformula}
\[
(r(t-\sin(t)),r(1-\cos(t))).
\]
\end{eulerformula}
\begin{eulercomment}
Berikut kita plot lintasan tersebut dan beberapa posisi lingkaran
ketika t=0, t=pi/2, t=r*pi.
\end{eulercomment}
\begin{eulerprompt}
>x &= r*(t-sin(t))
\end{eulerprompt}
\begin{euleroutput}
  
                              r (t - sin(t))
  
\end{euleroutput}
\begin{eulerprompt}
>y &= r*(1-cos(t))
\end{eulerprompt}
\begin{euleroutput}
  
                              r (1 - cos(t))
  
\end{euleroutput}
\begin{eulercomment}
Berikut kita gambar sikloid untuk r=1.
\end{eulercomment}
\begin{eulerprompt}
>ex &= x-sin(x); ey &= 1-cos(x); aspect(1);
>plot2d(ex,ey,xmin=0,xmax=4pi,square=1); ...
>  plot2d("2+cos(x)","1+sin(x)",xmin=0,xmax=2pi,>add,color=blue); ...
>  plot2d([2,ex(2)],[1,ey(2)],color=red,>add); ...
>  plot2d(ex(2),ey(2),>points,>add,color=red); ...
>  plot2d("2pi+cos(x)","1+sin(x)",xmin=0,xmax=2pi,>add,color=blue); ...
>  plot2d([2pi,ex(2pi)],[1,ey(2pi)],color=red,>add);  ...
>  plot2d(ex(2pi),ey(2pi),>points,>add,color=red):
\end{eulerprompt}
\begin{euleroutput}
  Cannot combine a 1x10000000 and a 1x21 matrix for *!
  Error in expression: r*(t-sin(t))-sin(r*(t-sin(t)))
  adaptiveeval:
      sx=f$(t;args());
  Try "trace errors" to inspect local variables after errors.
  plot2d:
      dw/n,dw/n^2,dw/n;args());
\end{euleroutput}
\begin{eulercomment}
Berikut dihitung panjang lintasan untuk 1 putaran penuh. (Jangan salah
menduga bahwa panjang lintasan 1 putaran penuh sama dengan keliling
lingkaran!)
\end{eulercomment}
\begin{eulerprompt}
>ds &= radcan(sqrt(diff(ex,x)^2+diff(ey,x)^2)); $ds=trigsimp(ds) // elemen panjang kurva sikloid
\end{eulerprompt}
\begin{euleroutput}
  Maxima said:
  diff: second argument must be a variable; found r*(t-sin(t))
   -- an error. To debug this try: debugmode(true);
  
  Error in:
  ds &= radcan(sqrt(diff(ex,x)^2+diff(ey,x)^2)); $ds=trigsimp(ds ...
                                               ^
\end{euleroutput}
\begin{eulerprompt}
>ds &= trigsimp(ds); $ds
>$showev('integrate(ds,x,0,2*pi)) // hitung panjang sikloid satu putaran penuh
\end{eulerprompt}
\begin{euleroutput}
  Maxima said:
  defint: variable of integration must be a simple or subscripted variable.
  defint: found r*(t-sin(t))
  #0: showev(f='integrate(ds,r*(t-sin(t)),0,2*pi))
   -- an error. To debug this try: debugmode(true);
  
  Error in:
  $showev('integrate(ds,x,0,2*pi)) // hitung panjang sikloid sat ...
                                   ^
\end{euleroutput}
\begin{eulerprompt}
>integrate(mxm("ds"),0,2*pi) // hitung secara numerik
\end{eulerprompt}
\begin{euleroutput}
  Illegal function result in map.
  %evalexpression:
      if maps then return %mapexpression1(x,f$;args());
  gauss:
      if maps then y=%evalexpression(f$,a+h-(h*xn)',maps;args());
  adaptivegauss:
      t1=gauss(f$,c,c+h;args(),=maps);
  Try "trace errors" to inspect local variables after errors.
  integrate:
      return adaptivegauss(f$,a,b,eps*1000;args(),=maps);
\end{euleroutput}
\begin{eulerprompt}
>romberg(mxm("ds"),0,2*pi) // cara lain hitung secara numerik
\end{eulerprompt}
\begin{euleroutput}
  Wrong argument!
  
  Cannot combine a symbolic expression here.
  Did you want to create a symbolic expression?
  Then start with &.
  
  Try "trace errors" to inspect local variables after errors.
  romberg:
      if cols(y)==1 then return y*(b-a); endif;
  Error in:
  romberg(mxm("ds"),0,2*pi) // cara lain hitung secara numerik ...
                           ^
\end{euleroutput}
\begin{eulercomment}
Perhatikan, seperti terlihat pada gambar, panjang sikloid lebih besar
daripada keliling lingkarannya, yakni:

\end{eulercomment}
\begin{eulerformula}
\[
2\pi.
\]
\end{eulerformula}
\eulerheading{Kurvatur (Kelengkungan) Kurva}
\begin{eulercomment}
image: Osculating.png

Aslinya, kelengkungan kurva diferensiabel (yakni, kurva mulus yang
tidak lancip) di titik P didefinisikan melalui lingkaran oskulasi
(yaitu, lingkaran yang melalui titik P dan terbaik memperkirakan,
paling banyak menyinggung kurva di sekitar P). Pusat dan radius
kelengkungan kurva di P adalah pusat dan radius lingkaran oskulasi.
Kelengkungan adalah kebalikan dari radius kelengkungan:

\end{eulercomment}
\begin{eulerformula}
\[
\kappa =\frac {1}{R}
\]
\end{eulerformula}
\begin{eulercomment}
dengan R adalah radius kelengkungan. (Setiap lingkaran memiliki
kelengkungan ini pada setiap titiknya, dapat diartikan, setiap
lingkaran berputar 2pi sejauh 2piR.)\\
Definisi ini sulit dimanipulasi dan dinyatakan ke dalam rumus untuk
kurva umum. Oleh karena itu digunakan definisi lain yang ekivalen.

\end{eulercomment}
\eulersubheading{Definisi Kurvatur dengan Fungsi Parametrik Panjang Kurva Setiap}
\begin{eulercomment}
kurva diferensiabel dapat dinyatakan dengan persamaan parametrik
terhadap panjang kurva s:

\end{eulercomment}
\begin{eulerformula}
\[
\gamma(s) = (x(s),\ y(s)),
\]
\end{eulerformula}
\begin{eulercomment}
dengan x dan y adalah fungsi riil yang diferensiabel, yang memenuhi:

\end{eulercomment}
\begin{eulerformula}
\[
\|\gamma'(s)\|=\sqrt{x'(s)^2+y'(s)^2}=1.
\]
\end{eulerformula}
\begin{eulercomment}
Ini berarti bahwa vektor singgung


\end{eulercomment}
\begin{eulerformula}
\[
\mathbf{T}(s)=(x'(s),\ y'(s))
\]
\end{eulerformula}
\begin{eulercomment}
memiliki norm 1 dan merupakan vektor singgung satuan.

Apabila kurvanya memiliki turunan kedua, artinya turunan kedua x dan y
ada, maka T'(s) ada. Vektor ini merupakan normal kurva yang arahnya
menuju pusat kurvatur, norm-nya merupakan nilai kurvatur
(kelengkungan):

\end{eulercomment}
\begin{eulerformula}
\[
 \begin{aligned}\mathbf{T}(s) &= \mathbf{\gamma}'(s),\\ \mathbf{T}^{2}(s) &=1\ \text{(konstanta)}\Rightarrow \mathbf{T}'(s)\cdot \mathbf{T}(s)=0\\ \kappa(s) &=\|\mathbf {T}'(s)\|= \|\mathbf{\gamma}''(s)\|=\sqrt{x''(s)^{2}+y''(s)^{2}}.\end{aligned}
\]
\end{eulerformula}
\begin{eulercomment}
Nilai

\end{eulercomment}
\begin{eulerformula}
\[
R(s)=\frac{1}{\kappa(s)}
\]
\end{eulerformula}
\begin{eulercomment}
disebut jari-jari (radius) kelengkungan kurva.

Bilangan riil

\end{eulercomment}
\begin{eulerformula}
\[
 k(s) = \pm\kappa(s)
\]
\end{eulerformula}
\begin{eulercomment}
disebut nilai kelengkungan bertanda.

Contoh:\\
Akan ditentukan kurvatur lingkaran

\end{eulercomment}
\begin{eulerformula}
\[
x=r\cos t,\ y= r\sin t.
\]
\end{eulerformula}
\begin{eulerprompt}
>fx &= r*cos(t); fy &=r*sin(t);
>&assume(t>0,r>0); s &=integrate(sqrt(diff(fx,t)^2+diff(fy,t)^2),t,0,t); s // elemen panjang kurva, panjang busur lingkaran (s)
\end{eulerprompt}
\begin{euleroutput}
  
                                   r t
  
\end{euleroutput}
\begin{eulerprompt}
>&kill(s); fx &= r*cos(s/r); fy &=r*sin(s/r); // definisi ulang persamaan parametrik terhadap s dengan substitusi t=s/r
>k &= trigsimp(sqrt(diff(fx,s,2)^2+diff(fy,s,2)^2)); $k // nilai kurvatur lingkaran dengan menggunakan definisi di atas
\end{eulerprompt}
\begin{eulerformula}
\[
\frac{1}{r}
\]
\end{eulerformula}
\begin{eulercomment}
Untuk representasi parametrik umum, misalkan

\end{eulercomment}
\begin{eulerformula}
\[
x = x(t),\ y= y(t)
\]
\end{eulerformula}
\begin{eulercomment}
merupakan persamaan parametrik untuk kurva bidang yang
terdiferensialkan dua kali. Kurvatur untuk kurva tersebut
didefinisikan sebagai

\end{eulercomment}
\begin{eulerformula}
\[
\begin{aligned}\kappa &= \frac{d\phi}{ds}=\frac{\frac{d\phi}{dt}}{\frac{ds}{dt}}\quad (\phi \text{ adalah sudut kemiringan garis singgung dan }s \text{ adalah panjang kurva})\\ &=\frac{\frac{d\phi}{dt}}{\sqrt{(\frac{dx}{dt})^2+(\frac{dy}{dt})^2}}= \frac{\frac{d\phi}{dt}}{\sqrt{x'(t)^2+y'(t)^2}}.\end{aligned}.
\]
\end{eulerformula}
\begin{eulercomment}
Selanjutnya, pembilang pada persamaan di atas dapat dicari sebagai
berikut.

\end{eulercomment}
\begin{eulerformula}
\[
\begin{aligned}\sec^2\phi\frac{d\phi}{dt} &= \frac{d}{dt}\left(\tan\phi\right)= \frac{d}{dt}\left(\frac{dy}{dx}\right)= \frac{d}{dt}\left(\frac{dy/dt}{dx/dt}\right)= \frac{d}{dt}\left(\frac{y'(t)}{x'(t)}\right)=\frac{x'(t)y''(t)-x''(t)y'(t)}{x'(t)^2}.\\ & \\ \frac{d\phi}{dt} &= \frac{1}{\sec^2\phi}\frac{x'(t)y''(t)-x''(t)y'(t)}{x'(t)^2}\\ &= \frac{1}{1+\tan^2\phi}\frac{x'(t)y''(t)-x''(t)y'(t)}{x'(t)^2}\\ &= \frac{1}{1+\left(\frac{y'(t)}{x'(t)}\right)^2}\frac{x'(t)y''(t)-x''(t)y'(t)}{x'(t)^2}\\ &= \frac{x'(t)y''(t)-x''(t)y'(t)}{x'(t)^2+y'(t)^2}.\end{aligned}
\]
\end{eulerformula}
\begin{eulercomment}
Jadi, rumus kurvatur untuk kurva parametrik

\end{eulercomment}
\begin{eulerformula}
\[
x=x(t),\ y=y(t)
\]
\end{eulerformula}
\begin{eulercomment}
adalah

\end{eulercomment}
\begin{eulerformula}
\[
\kappa(t) = \frac{x'(t)y''(t)-x''(t)y'(t)}{\left(x'(t)^2+y'(t)^2\right)^{3/2}}.
\]
\end{eulerformula}
\begin{eulercomment}
Jika kurvanya dinyatakan dengan persamaan parametrik pada koordinat
kutub

\end{eulercomment}
\begin{eulerformula}
\[
x=r(\theta)\cos\theta,\ y=r(\theta)\sin\theta,
\]
\end{eulerformula}
\begin{eulercomment}
maka rumus kurvaturnya adalah

\end{eulercomment}
\begin{eulerformula}
\[
\kappa(\theta) = \frac{r(\theta)^2+2r'(\theta)^2-r(\theta)r''(\theta)}{\left(r'(\theta)^2+r'(\theta)^2\right)^{3/2}}.
\]
\end{eulerformula}
\begin{eulercomment}
(Silakan Anda turunkan rumus tersebut!)

Contoh:\\
Lingkaran dengan pusat (0,0) dan jari-jari r dapat dinyatakan dengan
persamaan parametrik

\end{eulercomment}
\begin{eulerformula}
\[
x=r\cos t,\ y=r\sin t.
\]
\end{eulerformula}
\begin{eulercomment}
Nilai kelengkungan lingkaran tersebut adalah

\end{eulercomment}
\begin{eulerformula}
\[
\kappa(t)=\frac{x'(t)y''(t)-x''(t)y'(t)}{\left(x'(t)^2+y'(t)^2\right)^{3/2}}=\frac{r^2}{r^3}=\frac 1 r.
\]
\end{eulerformula}
\begin{eulercomment}
Hasil cocok dengan definisi kurvatur suatu kelengkungan.
\end{eulercomment}
\begin{eulercomment}
Kurva

\end{eulercomment}
\begin{eulerformula}
\[
y=f(x)
\]
\end{eulerformula}
\begin{eulercomment}
dapat dinyatakan ke dalam persamaan parametrik

\end{eulercomment}
\begin{eulerformula}
\[
x=t,\ y=f(t),\ \text{ dengan } x'(t)=1,\ x''(t)=0,
\]
\end{eulerformula}
\begin{eulercomment}
sehingga kurvaturnya adalah

\end{eulercomment}
\begin{eulerformula}
\[
\kappa(t) = \frac{y''(t)}{\left(1+y'(t)^2\right)^{3/2}}.
\]
\end{eulerformula}
\begin{eulercomment}
Contoh:\\
Akan ditentukan kurvatur parabola

\end{eulercomment}
\begin{eulerformula}
\[
y=ax^2+bx+c.
\]
\end{eulerformula}
\begin{eulerprompt}
>function f(x) &= a*x^2+b*x+c; $y=f(x)
\end{eulerprompt}
\begin{eulerformula}
\[
r\,\left(1-\cos t\right)=b\,r\,\left(t-\sin t\right)+a\,r^2\,\left(
 t-\sin t\right)^2+c
\]
\end{eulerformula}
\begin{eulerprompt}
>function k(x) &= (diff(f(x),x,2))/(1+diff(f(x),x)^2)^(3/2); $'k(x)=k(x) // kelengkungan parabola 
\end{eulerprompt}
\begin{euleroutput}
  Maxima said:
  diff: second argument must be a variable; found r*(t-sin(t))
   -- an error. To debug this try: debugmode(true);
  
  Error in:
  ... (x) &= (diff(f(x),x,2))/(1+diff(f(x),x)^2)^(3/2); $'k(x)=k(x)  ...
                                                       ^
\end{euleroutput}
\begin{eulerprompt}
>function f(x) &= x^2+x+1; $y=f(x) // akan kita plot kelengkungan parabola untuk a=b=c=1
\end{eulerprompt}
\begin{eulerformula}
\[
r\,\left(1-\cos t\right)=r\,\left(t-\sin t\right)+r^2\,\left(t-
 \sin t\right)^2+1
\]
\end{eulerformula}
\begin{eulerprompt}
>function k(x) &= (diff(f(x),x,2))/(1+diff(f(x),x)^2)^(3/2); $'k(x)=k(x) // kelengkungan parabola
\end{eulerprompt}
\begin{euleroutput}
  Maxima said:
  diff: second argument must be a variable; found r*(t-sin(t))
   -- an error. To debug this try: debugmode(true);
  
  Error in:
  ... (x) &= (diff(f(x),x,2))/(1+diff(f(x),x)^2)^(3/2); $'k(x)=k(x)  ...
                                                       ^
\end{euleroutput}
\begin{eulercomment}
Berikut kita gambar parabola tersebut beserta kurva kelengkungan,
kurva jari-jari kelengkungan dan salah satu lingkaran oskulasi di
titik puncak parabola. Perhatikan, puncak parabola dan jari-jari
lingkaran oskulasi di puncak parabola adalah

\end{eulercomment}
\begin{eulerformula}
\[
(-1/2,3/4),\ 1/k(2)=1/2,
\]
\end{eulerformula}
\begin{eulercomment}
sehingga pusat lingkaran oskulasi adalah (-1/2, 5/4).
\end{eulercomment}
\begin{eulerprompt}
>plot2d(["f(x)", "k(x)"],-2,1, color=[blue,red]); plot2d("1/k(x)",-1.5,1,color=green,>add); ...
>plot2d("-1/2+1/k(-1/2)*cos(x)","5/4+1/k(-1/2)*sin(x)",xmin=0,xmax=2pi,>add,color=blue):
\end{eulerprompt}
\begin{euleroutput}
  Cannot combine a 1x10000000 and a 1x21 matrix for *!
  f:
      useglobal; return r*(t-sin(t))+r^2*(t-sin(t))^2+1 
  Error in expression: f(x)
  %ploteval:
      y0=f$(x[1],args());
  adaptiveevalone:
      s=%ploteval(g$,t;args());
  Try "trace errors" to inspect local variables after errors.
  plot2d:
      dw/n,dw/n^2,dw/n,auto;args());
\end{euleroutput}
\begin{eulercomment}
Untuk kurva yang dinyatakan dengan fungsi implisit

\end{eulercomment}
\begin{eulerformula}
\[
F(x,y)=0
\]
\end{eulerformula}
\begin{eulercomment}
dengan turunan-turunan parsial

\end{eulercomment}
\begin{eulerformula}
\[
F_x=\frac{\partial F}{\partial x},\ F_y=\frac{\partial F}{\partial y},\ F_{xy}=\frac{\partial}{\partial y}\left(\frac{\partial F}{\partial x}\right),\ F_{xx}=\frac{\partial}{\partial x}\left(\frac{\partial F}{\partial x}\right),\ F_{yy}=\frac{\partial}{\partial y}\left(\frac{\partial F}{\partial y}\right),
\]
\end{eulerformula}
\begin{eulercomment}
berlaku

\end{eulercomment}
\begin{eulerformula}
\[
F_x dx+ F_y dy = 0\text{ atau } \frac{dy}{dx}=-\frac{F_x}{F_y},
\]
\end{eulerformula}
\begin{eulercomment}
sehingga kurvaturnya adalah

\end{eulercomment}
\begin{eulerformula}
\[
\kappa =\frac {F_y^2F_{xx}-2F_xF_yF_{xy}+F_x^2F_{yy}}{\left(F_x^2+F_y^2\right)^{3/2}}.
\]
\end{eulerformula}
\begin{eulercomment}
(Silakan Anda turunkan sendiri!)

Contoh 1:\\
Parabola

\end{eulercomment}
\begin{eulerformula}
\[
y=ax^2+bx+c
\]
\end{eulerformula}
\begin{eulercomment}
dapat dinyatakan ke dalam persamaan implisit

\end{eulercomment}
\begin{eulerformula}
\[
ax^2+bx+c-y=0.
\]
\end{eulerformula}
\begin{eulerprompt}
>function F(x,y) &=a*x^2+b*x+c-y; $F(x,y)
\end{eulerprompt}
\begin{eulerformula}
\[
b\,r\,\left(t-\sin t\right)+a\,r^2\,\left(t-\sin t\right)^2-r\,
 \left(1-\cos t\right)+c
\]
\end{eulerformula}
\begin{eulerprompt}
>Fx &= diff(F(x,y),x), Fxx &=diff(F(x,y),x,2), Fy &=diff(F(x,y),y), Fxy &=diff(diff(F(x,y),x),y), Fyy &=diff(F(x,y),y,2) 
\end{eulerprompt}
\begin{euleroutput}
  Maxima said:
  diff: second argument must be a variable; found errexp1
   -- an error. To debug this try: debugmode(true);
  
  Error in:
  Fx &= diff(F(x,y),x), Fxx &=diff(F(x,y),x,2), Fy &=diff(F(x,y) ...
                      ^
\end{euleroutput}
\begin{eulerprompt}
>function k(x) &= (Fy^2*Fxx-2*Fx*Fy*Fxy+Fx^2*Fyy)/(Fx^2+Fy^2)^(3/2); $'k(x)=k(x) // kurvatur parabola tersebut
\end{eulerprompt}
\begin{eulercomment}
Hasilnya sama dengan sebelumnya yang menggunakan persamaan parabola
biasa.
\end{eulercomment}
\eulerheading{Latihan}
\begin{eulercomment}
- Bukalah buku Kalkulus.\\
- Cari dan pilih beberapa (paling sedikit 5 fungsi berbeda
tipe/bentuk/jenis) fungsi dari buku tersebut, kemudian definisikan di
EMT pada baris-baris perintah berikut (jika perlu tambahkan lagi).\\
- Untuk setiap fungsi, tentukan anti turunannya (jika ada), hitunglah
integral tentu dengan batas-batas yang menarik (Anda tentukan
sendiri), seperti contoh-contoh tersebut.\\
- Lakukan hal yang sama untuk fungsi-fungsi yang tidak dapat
diintegralkan (cari sedikitnya 3 fungsi).\\
- Gambar grafik fungsi dan daerah integrasinya pada sumbu koordinat
yang sama.\\
- Gunakan integral tentu untuk mencari luas daerah yang dibatasi oleh
dua kurva yang berpotongan di dua titik. (Cari dan gambar kedua kurva
dan arsir (warnai) daerah yang dibatasi oleh keduanya.)\\
- Gunakan integral tentu untuk menghitung volume benda putar kurva y=
f(x) yang diputar mengelilingi sumbu x dari x=a sampai x=b, yakni

\end{eulercomment}
\begin{eulerformula}
\[
V = \int_a^b \pi (f(x)^2\ dx.
\]
\end{eulerformula}
\begin{eulercomment}
(Pilih fungsinya dan gambar kurva dan benda putar yang dihasilkan.
Anda dapat mencari contoh-contoh bagaimana cara menggambar benda hasil
perputaran suatu kurva.)\\
- Gunakan integral tentu untuk menghitung panjang kurva y=f(x) dari
x=a sampai x=b dengan menggunakan rumus:

\end{eulercomment}
\begin{eulerformula}
\[
S = \int_a^b \sqrt{1+(f'(x))^2} \ dx.
\]
\end{eulerformula}
\begin{eulercomment}
(Pilih fungsi dan gambar kurvanya.)\\
- Apabila fungsi dinyatakan dalam koordinat kutub x=f(r,t), y=g(r,t),
r=h(t), x=a bersesuaian dengan t=t0 dan x=b bersesuian dengan t=t1,
maka rumus di atas akan menjadi:

\end{eulercomment}
\begin{eulerformula}
\[
S=\int_{t_0}^{t_1} \sqrt{x'(t)^2+y'(t)^2}\ dt.
\]
\end{eulerformula}
\begin{eulercomment}
- Pilih beberapa kurva menarik (selain lingkaran dan parabola) dari
buku  kalkulus. Nyatakan setiap kurva tersebut dalam bentuk:\\
\end{eulercomment}
\begin{eulerttcomment}
  a. koordinat Kartesius (persamaan y=f(x))
  b. koordinat kutub ( r=r(theta))
  c. persamaan parametrik x=x(t), y=y(t)
  d. persamaan implit F(x,y)=0
\end{eulerttcomment}
\begin{eulercomment}
- Tentukan kurvatur masing-masing kurva dengan menggunakan keempat
representasi tersebut (hasilnya harus sama).\\
- Gambarlah kurva asli, kurva kurvatur, kurva jari-jari lingkaran
oskulasi, dan salah satu lingkaran oskulasinya.
\end{eulercomment}
\eulerheading{Barisan dan Deret}
\begin{eulercomment}
(Catatan: bagian ini belum lengkap. Anda dapat membaca contoh-contoh
pengguanaan EMT dan Maxima untuk menghitung limit barisan, rumus
jumlah parsial suatu deret, jumlah tak hingga suatu deret konvergen,
dan sebagainya. Anda dapat mengeksplor contoh-contoh di EMT atau
perbagai panduan penggunaan Maxima di software Maxima atau dari
Internet.)

Barisan dapat didefinisikan dengan beberapa cara di dalam EMT, di
antaranya:

- dengan cara yang sama seperti mendefinisikan vektor dengan
elemen-elemen beraturan (menggunakan titik dua ":");\\
- menggunakan perintah "sequence" dan rumus barisan (suku ke -n);\\
- menggunakan perintah "iterate" atau "niterate";\\
- menggunakan fungsi Maxima "create\_list" atau "makelist" untuk
menghasilkan barisan simbolik;\\
- menggunakan fungsi biasa yang inputnya vektor atau barisan;\\
- menggunakan fungsi rekursif.

EMT menyediakan beberapa perintah (fungsi) terkait barisan, yakni:

- sum: menghitung jumlah semua elemen suatu barisan\\
- cumsum: jumlah kumulatif suatu barisan\\
- differences: selisih antar elemen-elemen berturutan

EMT juga dapat digunakan untuk menghitung jumlah deret berhingga
maupun deret tak hingga, dengan menggunakan perintah (fungsi) "sum".
Perhitungan dapat dilakukan secara numerik maupun simbolik dan eksak.

Berikut adalah beberapa contoh perhitungan barisan dan deret
menggunakan EMT.
\end{eulercomment}
\begin{eulerprompt}
>1:10 // barisan sederhana
\end{eulerprompt}
\begin{euleroutput}
  [1,  2,  3,  4,  5,  6,  7,  8,  9,  10]
\end{euleroutput}
\begin{eulerprompt}
>1:2:30
\end{eulerprompt}
\begin{euleroutput}
  [1,  3,  5,  7,  9,  11,  13,  15,  17,  19,  21,  23,  25,  27,  29]
\end{euleroutput}
\eulerheading{Iterasi dan Barisan}
\begin{eulercomment}
EMT menyediakan fungsi iterate("g(x)", x0, n) untuk melakukan iterasi

\end{eulercomment}
\begin{eulerformula}
\[
x_{k+1}=g(x_k), \ x_0=x_0, k= 1, 2, 3, ..., n.
\]
\end{eulerformula}
\begin{eulercomment}
Berikut ini disajikan contoh-contoh penggunaan iterasi dan rekursi
dengan EMT. Contoh pertama menunjukkan pertumbuhan dari nilai awal
1000 dengan laju pertambahan 5\%, selama 10 periode.
\end{eulercomment}
\begin{eulerprompt}
>q=1.05; iterate("x*q",1000,n=10)'
\end{eulerprompt}
\begin{euleroutput}
           1000 
           1050 
         1102.5 
        1157.63 
        1215.51 
        1276.28 
         1340.1 
         1407.1 
        1477.46 
        1551.33 
        1628.89 
\end{euleroutput}
\begin{eulercomment}
Contoh berikutnya memperlihatkan bahaya menabung di bank pada masa
sekarang! Dengan bunga tabungan sebesar 6\% per tahun atau 0.5\% per
bulan dipotong pajak 20\%, dan biaya administrasi 10000 per bulan,
tabungan sebesar 1 juta tanpa diambil selama sekitar 10 tahunan akan
habis diambil oleh bank!
\end{eulercomment}
\begin{eulerprompt}
>r=0.005; plot2d(iterate("(1+0.8*r)*x-10000",1000000,n=130)):
\end{eulerprompt}
\begin{eulercomment}
Silakan Anda coba-coba, dengan tabungan minimal berapa agar tidak akan
habis diambil oleh bank dengan ketentuan bunga dan biaya administrasi
seperti di atas.

Berikut adalah perhitungan minimal tabungan agar aman di bank dengan
bunga sebesar r dan biaya administrasi a, pajak bunga 20\%.
\end{eulercomment}
\begin{eulerprompt}
>$solve(0.8*r*A-a,A), $% with [r=0.005, a=10] 
\end{eulerprompt}
\begin{eulercomment}
Berikut didefinisikan fungsi untuk menghitung saldo tabungan, kemudian
dilakukan iterasi.
\end{eulercomment}
\begin{eulerprompt}
>function saldo(x,r,a) := round((1+0.8*r)*x-a,2);
>iterate(\{\{"saldo",0.005,10\}\},1000,n=6)
\end{eulerprompt}
\begin{euleroutput}
  [1000,  994,  987.98,  981.93,  975.86,  969.76,  963.64]
\end{euleroutput}
\begin{eulerprompt}
>iterate(\{\{"saldo",0.005,10\}\},2000,n=6)
\end{eulerprompt}
\begin{euleroutput}
  [2000,  1998,  1995.99,  1993.97,  1991.95,  1989.92,  1987.88]
\end{euleroutput}
\begin{eulerprompt}
>iterate(\{\{"saldo",0.005,10\}\},2500,n=6)
\end{eulerprompt}
\begin{euleroutput}
  [2500,  2500,  2500,  2500,  2500,  2500,  2500]
\end{euleroutput}
\begin{eulercomment}
Tabungan senilai 2,5 juta akan aman dan tidak akan berubah nilai (jika
tidak ada penarikan), sedangkan jika tabungan awal kurang dari 2,5
juta, lama kelamaan akan berkurang meskipun tidak pernah dilakukan
penarikan uang tabungan.
\end{eulercomment}
\begin{eulerprompt}
>iterate(\{\{"saldo",0.005,10\}\},3000,n=6)
\end{eulerprompt}
\begin{euleroutput}
  [3000,  3002,  3004.01,  3006.03,  3008.05,  3010.08,  3012.12]
\end{euleroutput}
\begin{eulercomment}
Tabungan yang lebih dari 2,5 juta baru akan bertambah jika tidak ada
penarikan.

Untuk barisan yang lebih kompleks dapat digunakan fungsi "sequence()".
Fungsi ini menghitung nilai-nilai x[n] dari semua nilai sebelumnya,
x[1],...,x[n-1] yang diketahui.\\
Berikut adalah contoh barisan Fibonacci.

\end{eulercomment}
\begin{eulerformula}
\[
x_n = x_{n-1}+x_{n-2}, \quad x_1=1, \quad x_2 =1
\]
\end{eulerformula}
\begin{eulerprompt}
>sequence("x[n-1]+x[n-2]",[1,1],15)
\end{eulerprompt}
\begin{euleroutput}
  [1,  1,  2,  3,  5,  8,  13,  21,  34,  55,  89,  144,  233,  377,  610]
\end{euleroutput}
\begin{eulercomment}
Barisan Fibonacci memiliki banyak sifat menarik, salah satunya adalah
akar pangkat ke-n suku ke-n akan konvergen ke pecahan emas:
\end{eulercomment}
\begin{eulerprompt}
>$'(1+sqrt(5))/2=float((1+sqrt(5))/2)
>plot2d(sequence("x[n-1]+x[n-2]",[1,1],250)^(1/(1:250))):
\end{eulerprompt}
\begin{eulercomment}
Barisan yang sama juga dapat dihasilkan dengan menggunakan loop.
\end{eulercomment}
\begin{eulerprompt}
>x=ones(500); for k=3 to 500; x[k]=x[k-1]+x[k-2]; end;
\end{eulerprompt}
\begin{eulercomment}
Rekursi dapat dilakukan dengan menggunakan rumus yang tergantung pada
semua elemen sebelumnya. Pada contoh berikut, elemen ke-n merupakan
jumlah (n-1) elemen sebelumnya, dimulai dengan 1 (elemen ke-1). Jelas,
nilai elemen ke-n adalah 2\textasciicircum{}(n-2), untuk n=2, 4, 5, ....
\end{eulercomment}
\begin{eulerprompt}
>sequence("sum(x)",1,10)
\end{eulerprompt}
\begin{euleroutput}
  [1,  1,  2,  4,  8,  16,  32,  64,  128,  256]
\end{euleroutput}
\begin{eulercomment}
Selain menggunakan ekspresi dalam x dan n, kita juga dapat menggunakan
fungsi.

Pada contoh berikut, digunakan iterasi

\end{eulercomment}
\begin{eulerformula}
\[
x_n =A \cdot x_{n-1},
\]
\end{eulerformula}
\begin{eulercomment}
dengan A suatu matriks 2x2, dan setiap x[n] merupakan matriks/vektor
2x1.
\end{eulercomment}
\begin{eulerprompt}
>A=[1,1;1,2]; function suku(x,n) := A.x[,n-1]
>sequence("suku",[1;1],6)
\end{eulerprompt}
\begin{euleroutput}
  Real 2 x 6 matrix
  
              1             2             5            13     ...
              1             3             8            21     ...
\end{euleroutput}
\begin{eulercomment}
Hasil yang sama juga dapat diperoleh dengan menggunakan fungsi
perpangkatan matriks "matrixpower()". Cara ini lebih cepat, karena
hanya menggunakan perkalian matriks sebanyak log\_2(n).

\end{eulercomment}
\begin{eulerformula}
\[
x_n=A.x_{n-1}=A^2.x_{n-2}=A^3.x_{n-3}= ... = A^{n-1}.x_1.
\]
\end{eulerformula}
\begin{eulerprompt}
>sequence("matrixpower(A,n).[1;1]",1,6)
\end{eulerprompt}
\begin{euleroutput}
  Real 2 x 6 matrix
  
              1             5            13            34     ...
              1             8            21            55     ...
\end{euleroutput}
\eulerheading{Spiral Theodorus}
\begin{eulercomment}
image: Spiral\_of\_Theodorus.png\\
Spiral Theodorus (spiral segitiga siku-siku) dapat digambar secara
rekursif. Rumus rekursifnya adalah:

\end{eulercomment}
\begin{eulerformula}
\[
x_n = \left( 1 + \frac{i}{\sqrt{n-1}} \right) \, x_{n-1}, \quad x_1=1,
\]
\end{eulerformula}
\begin{eulercomment}
yang menghasilkan barisan bilangan kompleks.
\end{eulercomment}
\begin{eulerprompt}
>function g(n) := 1+I/sqrt(n)
\end{eulerprompt}
\begin{eulercomment}
Rekursinya dapat dijalankan sebanyak 17 untuk menghasilkan barisan 17
bilangan kompleks, kemudian digambar bilangan-bilangan kompleksnya.
\end{eulercomment}
\begin{eulerprompt}
>x=sequence("g(n-1)*x[n-1]",1,17); plot2d(x,r=3.5); textbox(latex("Spiral\(\backslash\) Theodorus"),0.4):
\end{eulerprompt}
\begin{eulercomment}
Selanjutnya dihubungan titik 0 dengan titik-titik kompleks tersebut
menggunakan loop.
\end{eulercomment}
\begin{eulerprompt}
>for i=1:cols(x); plot2d([0,x[i]],>add); end:
\end{eulerprompt}
\begin{eulercomment}
Spiral tersebut juga dapat didefinisikan menggunakan fungsi rekursif,
yang tidak memmerlukan indeks dan bilangan kompleks. Dalam hal ini
diigunakan vektor kolom pada bidang.
\end{eulercomment}
\begin{eulerprompt}
>function gstep (v) ...
\end{eulerprompt}
\begin{eulerudf}
  w=[-v[2];v[1]];
  return v+w/norm(w);
  endfunction
\end{eulerudf}
\begin{eulercomment}
Jika dilakukan iterasi 16 kali dimulai dari [1;0] akan didapatkan
matriks yang memuat vektor-vektor dari setiap iterasi.
\end{eulercomment}
\begin{eulerprompt}
>x=iterate("gstep",[1;0],16); plot2d(x[1],x[2],r=3.5,>points):
\end{eulerprompt}
\begin{eulercomment}
\begin{eulercomment}
\eulerheading{Kekonvergenan}
\begin{eulercomment}
Terkadang kita ingin melakukan iterasi sampai konvergen. Apabila
iterasinya tidak konvergen setelah ditunggu lama, Anda dapat
menghentikannya dengan menekan tombol [ESC].
\end{eulercomment}
\begin{eulerprompt}
>iterate("cos(x)",1) // iterasi x(n+1)=cos(x(n)), dengan x(0)=1.
\end{eulerprompt}
\begin{euleroutput}
  0.739085133216
\end{euleroutput}
\begin{eulercomment}
Iterasi tersebut konvergen ke penyelesaian persamaan

\end{eulercomment}
\begin{eulerformula}
\[
x = \cos(x).
\]
\end{eulerformula}
\begin{eulercomment}
Iterasi ini juga dapat dilakukan pada interval, hasilnya adalah
barisan interval yang memuat akar tersebut.
\end{eulercomment}
\begin{eulerprompt}
>hasil := iterate("cos(x)",~1,2~) //iterasi x(n+1)=cos(x(n)), dengan interval awal (1, 2)
\end{eulerprompt}
\begin{euleroutput}
  ~0.739085133211,0.7390851332133~
\end{euleroutput}
\begin{eulercomment}
Jika interval hasil tersebut sedikit diperlebar, akan terlihat bahwa
interval tersebut memuat akar persamaan x=cos(x).
\end{eulercomment}
\begin{eulerprompt}
>h=expand(hasil,100), cos(h) << h
\end{eulerprompt}
\begin{euleroutput}
  ~0.73908513309,0.73908513333~
  1
\end{euleroutput}
\begin{eulercomment}
Iterasi juga dapat digunakan pada fungsi yang didefinisikan.
\end{eulercomment}
\begin{eulerprompt}
>function f(x) := (x+2/x)/2
\end{eulerprompt}
\begin{eulercomment}
Iterasi x(n+1)=f(x(n)) akan konvergen ke akar kuadrat 2.
\end{eulercomment}
\begin{eulerprompt}
>iterate("f",2), sqrt(2)
\end{eulerprompt}
\begin{euleroutput}
  1.41421356237
  1.41421356237
\end{euleroutput}
\begin{eulercomment}
Jika pada perintah iterate diberikan tambahan parameter n, maka hasil
iterasinya akan ditampilkan mulai dari iterasi pertama sampai ke-n.
\end{eulercomment}
\begin{eulerprompt}
>iterate("f",2,5)
\end{eulerprompt}
\begin{euleroutput}
  [2,  1.5,  1.41667,  1.41422,  1.41421,  1.41421]
\end{euleroutput}
\begin{eulercomment}
Untuk iterasi ini tidak dapat dilakukan terhadap interval.
\end{eulercomment}
\begin{eulerprompt}
>niterate("f",~1,2~,5)
\end{eulerprompt}
\begin{euleroutput}
  [ ~1,2~,  ~1,2~,  ~1,2~,  ~1,2~,  ~1,2~,  ~1,2~ ]
\end{euleroutput}
\begin{eulercomment}
Perhatikan, hasil iterasinya sama dengan interval awal. Alasannya
adalah perhitungan dengan interval bersifat terlalu longgar. Untuk
meingkatkan perhitungan pada ekspresi dapat digunakan pembagian
intervalnya, menggunakan fungsi ieval().
\end{eulercomment}
\begin{eulerprompt}
>function s(x) := ieval("(x+2/x)/2",x,10)
\end{eulerprompt}
\begin{eulercomment}
Selanjutnya dapat dilakukan iterasi hingga diperoleh hasil optimal,
dan intervalnya tidak semakin mengecil. Hasilnya berupa interval yang
memuat akar persamaan:

\end{eulercomment}
\begin{eulerformula}
\[
x = \frac{1}{2} \left( x + \frac{2}{x} \right).
\]
\end{eulerformula}
\begin{eulercomment}
Satu-satunya solusi adalah\\
\end{eulercomment}
\begin{eulerformula}
\[
x = \sqrt2.
\]
\end{eulerformula}
\begin{eulerprompt}
>iterate("s",~1,2~)
\end{eulerprompt}
\begin{euleroutput}
  ~1.41421356236,1.41421356239~
\end{euleroutput}
\begin{eulercomment}
Fungsi "iterate()" juga dapat bekerja pada vektor. Berikut adalah
contoh fungsi vektor, yang menghasilkan rata-rata aritmetika dan
rata-rata geometri.

\end{eulercomment}
\begin{eulerformula}
\[
(a_{n+1},b_{n+1}) = \left( \frac{a_n+b_n}{2}, \sqrt{a_nb_n} \right)
\]
\end{eulerformula}
\begin{eulercomment}
Iterasi ke-n disimpan pada vektor kolom x[n].
\end{eulercomment}
\begin{eulerprompt}
>function g(x) := [(x[1]+x[2])/2;sqrt(x[1]*x[2])]
\end{eulerprompt}
\begin{eulercomment}
Iterasi dengan menggunakan fungsi tersebut akan konvergen ke rata-rata
aritmetika dan geometri dari nilai-nilai awal.
\end{eulercomment}
\begin{eulerprompt}
>iterate("g",[1;5])
\end{eulerprompt}
\begin{euleroutput}
        2.60401 
        2.60401 
\end{euleroutput}
\begin{eulercomment}
Hasil tersebut konvergen agak cepat, seperti kita cek sebagai berikut.
\end{eulercomment}
\begin{eulerprompt}
>iterate("g",[1;5],4)
\end{eulerprompt}
\begin{euleroutput}
              1             3       2.61803       2.60403       2.60401 
              5       2.23607       2.59002       2.60399       2.60401 
\end{euleroutput}
\begin{eulercomment}
Iterasi pada interval dapat dilakukan dan stabil, namun tidak
menunjukkan bahwa limitnya pada batas-batas yang dihitung.
\end{eulercomment}
\begin{eulerprompt}
>iterate("g",[~1~;~5~],4)
\end{eulerprompt}
\begin{euleroutput}
  Interval 2 x 5 matrix
  
  ~0.999999999999999778,1.00000000000000022~     ...
  ~4.99999999999999911,5.00000000000000089~     ...
\end{euleroutput}
\begin{eulercomment}
Iterasi berikut konvergen sangat lambat.

\end{eulercomment}
\begin{eulerformula}
\[
x_{n+1} = \sqrt{x_n}.
\]
\end{eulerformula}
\begin{eulerprompt}
>iterate("sqrt(x)",2,10)
\end{eulerprompt}
\begin{euleroutput}
  [2,  1.41421,  1.18921,  1.09051,  1.04427,  1.0219,  1.01089,
  1.00543,  1.00271,  1.00135,  1.00068]
\end{euleroutput}
\begin{eulercomment}
Kekonvergenan iterasi tersebut dapat dipercepatdengan percepatan
Steffenson:
\end{eulercomment}
\begin{eulerprompt}
>steffenson("sqrt(x)",2,10)
\end{eulerprompt}
\begin{euleroutput}
  [1.04888,  1.00028,  1,  1]
\end{euleroutput}
\eulerheading{Iterasi menggunakan Loop yang ditulis Langsung}
\begin{eulercomment}
Berikut adalah beberapa contoh penggunaan loop untuk melakukan iterasi
yang ditulis langsung pada baris perintah.
\end{eulercomment}
\begin{eulerprompt}
>x=2; repeat x=(x+2/x)/2; until x^2~=2; end; x,
\end{eulerprompt}
\begin{euleroutput}
  1.41421356237
\end{euleroutput}
\begin{eulercomment}
Penggabungan matriks menggunakan tanda "\textbar{}" dapat digunakan untuk
menyimpan semua hasil iterasi.
\end{eulercomment}
\begin{eulerprompt}
>v=[1]; for i=2 to 8; v=v|(v[i-1]*i); end; v,
\end{eulerprompt}
\begin{euleroutput}
  [1,  2,  6,  24,  120,  720,  5040,  40320]
\end{euleroutput}
\begin{eulercomment}
hasil iterasi juga dapat disimpan pada vektor yang sudah ada.
\end{eulercomment}
\begin{eulerprompt}
>v=ones(1,100); for i=2 to cols(v); v[i]=v[i-1]*i; end; ...
>plot2d(v,logplot=1); textbox(latex(&log(n)),x=0.5):
>A =[0.5,0.2;0.7,0.1]; b=[2;2]; ...
>x=[1;1]; repeat xnew=A.x-b; until all(xnew~=x); x=xnew; end; ...
>x
\end{eulerprompt}
\begin{euleroutput}
       -7.09677 
       -7.74194 
\end{euleroutput}
\eulerheading{Iterasi di dalam Fungsi}
\begin{eulercomment}
Fungsi atau program juga dapat menggunakan iterasi dan dapat digunakan
untuk melakukan iterasi. Berikut adalah beberapa contoh iterasi di
dalam fungsi.

Contoh berikut adalah suatu fungsi untuk menghitung berapa lama suatu
iterasi konvergen. Nilai fungsi tersebut adalah hasil akhir iterasi
dan banyak iterasi sampai konvergen.
\end{eulercomment}
\begin{eulerprompt}
>function map hiter(f$,x0) ...
\end{eulerprompt}
\begin{eulerudf}
  x=x0;
  maxiter=0;
  repeat
    xnew=f$(x);
    maxiter=maxiter+1;
    until xnew~=x;
    x=xnew;
  end;
  return maxiter;
  endfunction
\end{eulerudf}
\begin{eulercomment}
Misalnya, berikut adalah iterasi untuk mendapatkan hampiran akar
kuadrat 2, cukup cepat, konvergen pada iterasi ke-5, jika dimulai dari
hampiran awal 2.
\end{eulercomment}
\begin{eulerprompt}
>hiter("(x+2/x)/2",2)
\end{eulerprompt}
\begin{euleroutput}
  5
\end{euleroutput}
\begin{eulercomment}
Karena fungsinya didefinisikan menggunakan "map". maka nilai awalnya
dapat berupa vektor.
\end{eulercomment}
\begin{eulerprompt}
>x=1.5:0.1:10; hasil=hiter("(x+2/x)/2",x); ...
>  plot2d(x,hasil):
\end{eulerprompt}
\begin{eulercomment}
Dari gambar di atas terlihat bahwa kekonvergenan iterasinya semakin
lambat, untuk nilai awal semakin besar, namun penambahnnya tidak
kontinu. Kita dapat menemukan kapan maksimum iterasinya bertambah.
\end{eulercomment}
\begin{eulerprompt}
>hasil[1:10]
\end{eulerprompt}
\begin{euleroutput}
  [4,  5,  5,  5,  5,  5,  6,  6,  6,  6]
\end{euleroutput}
\begin{eulerprompt}
>x[nonzeros(differences(hasil))]
\end{eulerprompt}
\begin{euleroutput}
  [1.5,  2,  3.4,  6.6]
\end{euleroutput}
\begin{eulercomment}
maksimum iterasi sampai konvergen meningkat pada saat nilai awalnya
1.5, 2, 3.4, dan 6.6.

Contoh berikutnya adalah metode Newton pada polinomial kompleks
berderajat 3.
\end{eulercomment}
\begin{eulerprompt}
>p &= x^3-1; newton &= x-p/diff(p,x); $newton
\end{eulerprompt}
\begin{euleroutput}
  Maxima said:
  diff: second argument must be a variable; found errexp1
   -- an error. To debug this try: debugmode(true);
  
  Error in:
  p &= x^3-1; newton &= x-p/diff(p,x); $newton ...
                                     ^
\end{euleroutput}
\begin{eulercomment}
Selanjutnya didefinisikan fungsi untuk melakukan iterasi (aslinya 10
kali).
\end{eulercomment}
\begin{eulerprompt}
>function iterasi(f$,x,n=10) ...
\end{eulerprompt}
\begin{eulerudf}
  
  loop 1 to n; x=f$(x); end;
  return x;
  endfunction
\end{eulerudf}
\begin{eulercomment}
Kita mulai dengan menentukan titik-titik grid pada bidang kompleksnya.
\end{eulercomment}
\begin{eulerprompt}
>r=1.5; x=linspace(-r,r,501); Z=x+I*x'; W=iterasi(newton,Z);
\end{eulerprompt}
\begin{euleroutput}
  Function newton needs at least 3 arguments!
  Use: newton (f$: call, df$: call, x: scalar complex \{, y: number, eps: none\}) 
  Error in:
  ...  x=linspace(-r,r,501); Z=x+I*x'; W=iterasi(newton,Z); ...
                                                       ^
\end{euleroutput}
\begin{eulercomment}
Berikut adalah akar-akar polinomial di atas.
\end{eulercomment}
\begin{eulerprompt}
>z=&solve(p)()
\end{eulerprompt}
\begin{euleroutput}
  Maxima said:
  solve: more equations than unknowns.
  Unknowns given :  
  [r]
  Equations given:  
  errexp1
   -- an error. To debug this try: debugmode(true);
  
  Error in:
  z=&solve(p)() ...
             ^
\end{euleroutput}
\begin{eulercomment}
Untuk menggambar hasil iterasinya, dihitung jarak dari hasil iterasi
ke-10 ke masing-masing akar, kemudian digunakan untuk menghitung warna
yang akan digambar, yang menunjukkan limit untuk masing-masing nilai
awal.

Fungsi plotrgb() menggunakan jendela gambar terkini untuk menggambar
warna RGB sebagai matriks.
\end{eulercomment}
\begin{eulerprompt}
>C=rgb(max(abs(W-z[1]),1),max(abs(W-z[2]),1),max(abs(W-z[3]),1)); ...
>  plot2d(none,-r,r,-r,r); plotrgb(C):
\end{eulerprompt}
\begin{euleroutput}
  Variable W not found!
  Error in:
  C=rgb(max(abs(W-z[1]),1),max(abs(W-z[2]),1),max(abs(W-z[3]),1) ...
                      ^
\end{euleroutput}
\eulerheading{Iterasi Simbolik}
\begin{eulercomment}
Seperti sudah dibahas sebelumnya, untuk menghasilkan barisan ekspresi
simbolik dengan Maxima dapat digunakan fungsi makelist().
\end{eulercomment}
\begin{eulerprompt}
>&powerdisp:true // untuk menampilkan deret pangkat mulai dari suku berpangkat terkecil
\end{eulerprompt}
\begin{euleroutput}
  
                                   true
  
\end{euleroutput}
\begin{eulerprompt}
>deret &= makelist(taylor(exp(x),x,0,k),k,1,3); $deret // barisan deret Taylor untuk e^x
\end{eulerprompt}
\begin{euleroutput}
  Maxima said:
  taylor: 0.1539740213994798*r cannot be a variable.
   -- an error. To debug this try: debugmode(true);
  
  Error in:
  deret &= makelist(taylor(exp(x),x,0,k),k,1,3); $deret // baris ...
                                               ^
\end{euleroutput}
\begin{eulercomment}
Untuk mengubah barisan deret tersebut menjadi vektor string di EMT
digunakan fungsi mxm2str(). Selanjutnya, vektor string/ekspresi
hasilnya dapat digambar seperti menggambar vektor eskpresi pada EMT.
\end{eulercomment}
\begin{eulerprompt}
>plot2d("exp(x)",0,3); // plot fungsi aslinya, e^x
>plot2d(mxm2str("deret"),>add,color=4:6): // plot ketiga deret taylor hampiran fungsi tersebut
\end{eulerprompt}
\begin{euleroutput}
  Maxima said:
  length: argument cannot be a symbol; found deret
   -- an error. To debug this try: debugmode(true);
  
  mxmeval:
      return evaluate(mxm(s));
  Try "trace errors" to inspect local variables after errors.
  mxm2str:
      n=mxmeval("length(VVV)");
\end{euleroutput}
\begin{eulercomment}
Selain cara di atas dapat juga dengan cara menggunakan indeks pada
vektor/list yang dihasilkan.
\end{eulercomment}
\begin{eulerprompt}
>$deret[3]
>plot2d(["exp(x)",&deret[1],&deret[2],&deret[3]],0,3,color=1:4):
\end{eulerprompt}
\begin{euleroutput}
  deret is not a variable!
  Error in expression: deret[1]
   %ploteval:
      y0=f$(x[1],args());
  Try "trace errors" to inspect local variables after errors.
  plot2d:
      u=u_(%ploteval(xx[#],t;args()));
\end{euleroutput}
\begin{eulerprompt}
>$sum(sin(k*x)/k,k,1,5)
\end{eulerprompt}
\begin{eulercomment}
Berikut adalah cara menggambar kurva

\end{eulercomment}
\begin{eulerformula}
\[
y=\sin(x) + \dfrac{\sin 3x}{3} + \dfrac{\sin 5x}{5} + \ldots.
\]
\end{eulerformula}
\begin{eulerprompt}
>plot2d(&sum(sin((2*k+1)*x)/(2*k+1),k,0,20),0,2pi):
\end{eulerprompt}
\begin{euleroutput}
  
  Maxima output too long!
  Error in:
  plot2d(&sum(sin((2*k+1)*x)/(2*k+1),k,0,20),0,2pi): ...
                                            ^
\end{euleroutput}
\begin{eulercomment}
Hal serupa juga dapat dilakukan dengan menggunakan matriks, misalkan
kita akan menggambar kurva

\end{eulercomment}
\begin{eulerformula}
\[
y = \sum_{k=1}^{100} \dfrac{\sin(kx)}{k},\quad 0\le x\le 2\pi.
\]
\end{eulerformula}
\begin{eulercomment}
\end{eulercomment}
\begin{eulerprompt}
>x=linspace(0,2pi,1000); k=1:100; y=sum(sin(k*x')/k)'; plot2d(x,y):
\end{eulerprompt}
\eulerheading{Tabel Fungsi}
\begin{eulercomment}
Terdapat cara menarik untuk menghasilkan barisan dengan ekspresi
Maxima. Perintah mxmtable() berguna untuk menampilkan dan menggambar
barisan dan menghasilkan barisan sebagai vektor kolom.

Sebagai contoh berikut adalah barisan turunan ke-n x\textasciicircum{}x di x=1.
\end{eulercomment}
\begin{eulerprompt}
>mxmtable("diffat(x^x,x=1,n)","n",1,8,frac=1);
\end{eulerprompt}
\begin{euleroutput}
  Maxima said:
  diff: second argument must be a variable; found errexp1
  #0: diffat(expr=[0,1.66665833335744e-7*r,1.33330666692022e-6*r,4.499797504338432e-6*r,1.066581336583994e-5*r,2.08307...,x=[[0,1.66665833335744e-7*r,1.33330666692022e-6*r,4.499797504338432e-6*r,1.066581336583994e-5*r,2.0830...)
   -- an error. To debug this try: debugmode(true);
  
   %mxmevtable:
      return mxm("@expr,@var=@value")();
  Try "trace errors" to inspect local variables after errors.
  mxmtable:
      y[#,1]=%mxmevtable(expr,var,x[#]);
\end{euleroutput}
\begin{eulerprompt}
>$'sum(k, k, 1, n) = factor(ev(sum(k, k, 1, n),simpsum=true)) // simpsum:menghitung deret secara simbolik
>$'sum(1/(3^k+k), k, 0, inf) = factor(ev(sum(1/(3^k+k), k, 0, inf),simpsum=true))
\end{eulerprompt}
\begin{eulercomment}
Di sini masih gagal, hasilnya tidak dihitung.
\end{eulercomment}
\begin{eulerprompt}
>$'sum(1/x^2, x, 1, inf)= ev(sum(1/x^2, x, 1, inf),simpsum=true) // ev: menghitung nilai ekspresi
>$'sum((-1)^(k-1)/k, k, 1, inf) = factor(ev(sum((-1)^(x-1)/x, x, 1, inf),simpsum=true))
\end{eulerprompt}
\begin{eulercomment}
Di sini masih gagal, hasilnya tidak dihitung.
\end{eulercomment}
\begin{eulerprompt}
>$'sum((-1)^k/(2*k-1), k, 1, inf) = factor(ev(sum((-1)^k/(2*k-1), k, 1, inf),simpsum=true))
>$ev(sum(1/n!, n, 0, inf),simpsum=true)
\end{eulerprompt}
\begin{eulercomment}
Di sini masih gagal, hasilnya tidak dihitung, harusnya hasilnya e.
\end{eulercomment}
\begin{eulerprompt}
>&assume(abs(x)<1); $'sum(a*x^k, k, 0, inf)=ev(sum(a*x^k, k, 0, inf),simpsum=true), &forget(abs(x)<1);
\end{eulerprompt}
\begin{euleroutput}
  Answering "Is -94914474571+15819*r positive, negative or zero?" with "positive"
  Maxima said:
  sum: sum is divergent.
   -- an error. To debug this try: debugmode(true);
  
  Error in:
  ... k, 0, inf)=ev(sum(a*x^k, k, 0, inf),simpsum=true), &forget(abs ...
                                                       ^
\end{euleroutput}
\begin{eulercomment}
Deret geometri tak hingga, dengan asumsi rasional antara -1 dan 1.
\end{eulercomment}
\begin{eulerprompt}
>$'sum(x^k/k!,k,0,inf)=ev(sum(x^k/k!,k,0,inf),simpsum=true)
>$limit(sum(x^k/k!,k,0,n),n,inf)
>function d(n) &= sum(1/(k^2-k),k,2,n); $'d(n)=d(n)
>$d(10)=ev(d(10),simpsum=true)
>$d(100)=ev(d(100),simpsum=true)
\end{eulerprompt}
\eulerheading{Deret Taylor}
\begin{eulercomment}
Deret Taylor suatu fungsi f yang diferensiabel sampai tak hingga di
sekitar x=a adalah:

\end{eulercomment}
\begin{eulerformula}
\[
f(x) = \sum_{k=0}^\infty \frac{(x-a)^k f^{(k)}(a)}{k!}.
\]
\end{eulerformula}
\begin{eulerprompt}
>$'e^x =taylor(exp(x),x,0,10) // deret Taylor e^x di sekitar x=0, sampai suku ke-11
\end{eulerprompt}
\begin{euleroutput}
  Maxima said:
  taylor: 0.1539740213994798*r cannot be a variable.
   -- an error. To debug this try: debugmode(true);
  
  Error in:
   $'e^x =taylor(exp(x),x,0,10) // deret Taylor e^x di sekitar x= ...
                               ^
\end{euleroutput}
\begin{eulerprompt}
>$'log(x)=taylor(log(x),x,1,10)// deret log(x) di sekitar x=1
\end{eulerprompt}
\begin{euleroutput}
  Maxima said:
  log: encountered log(0).
   -- an error. To debug this try: debugmode(true);
  
  Error in:
   $'log(x)=taylor(log(x),x,1,10)// deret log(x) di sekitar x=1 ...
                                ^
\end{euleroutput}
\eulerheading{Visualisasi dan Perhitungan Geometri dengan EMT}
\begin{eulercomment}
Euler menyediakan beberapa fungsi untuk melakukan visualisasi dan
perhitungan geometri, baik secara numerik maupun analitik (seperti
biasanya tentunya, menggunakan Maxima). Fungsi-fungsi untuk
visualisasi dan perhitungan geometeri tersebut disimpan di dalam file
program "geometry.e", sehingga file tersebut harus dipanggil sebelum
menggunakan fungsi-fungsi atau perintah-perintah untuk geometri.
\end{eulercomment}
\begin{eulerprompt}
>load geometry
\end{eulerprompt}
\begin{euleroutput}
  Numerical and symbolic geometry.
\end{euleroutput}
\eulersubheading{Fungsi-fungsi Geometri}
\begin{eulercomment}
Fungsi-fungsi untuk Menggambar Objek Geometri:

\end{eulercomment}
\begin{eulerttcomment}
  defaultd:=textheight()*1.5: nilai asli untuk parameter d
  setPlotrange(x1,x2,y1,y2): menentukan rentang x dan y pada bidang
\end{eulerttcomment}
\begin{eulercomment}
koordinat\\
\end{eulercomment}
\begin{eulerttcomment}
  setPlotRange(r): pusat bidang koordinat (0,0) dan batas-batas
\end{eulerttcomment}
\begin{eulercomment}
sumbu-x dan y adalah -r sd r\\
\end{eulercomment}
\begin{eulerttcomment}
  plotPoint (P, "P"): menggambar titik P dan diberi label "P"
  plotSegment (A,B, "AB", d): menggambar ruas garis AB, diberi label
\end{eulerttcomment}
\begin{eulercomment}
"AB" sejauh d\\
\end{eulercomment}
\begin{eulerttcomment}
  plotLine (g, "g", d): menggambar garis g diberi label "g" sejauh d
  plotCircle (c,"c",v,d): Menggambar lingkaran c dan diberi label "c"
  plotLabel (label, P, V, d): menuliskan label pada posisi P
\end{eulerttcomment}
\begin{eulercomment}

Fungsi-fungsi Geometri Analitik (numerik maupun simbolik):

\end{eulercomment}
\begin{eulerttcomment}
  turn(v, phi): memutar vektor v sejauh phi
  turnLeft(v):   memutar vektor v ke kiri
  turnRight(v):  memutar vektor v ke kanan
  normalize(v): normal vektor v
  crossProduct(v, w): hasil kali silang vektorv dan w.
  lineThrough(A, B): garis melalui A dan B, hasilnya [a,b,c] sdh.
\end{eulerttcomment}
\begin{eulercomment}
ax+by=c.\\
\end{eulercomment}
\begin{eulerttcomment}
  lineWithDirection(A,v): garis melalui A searah vektor v
  getLineDirection(g): vektor arah (gradien) garis g
  getNormal(g): vektor normal (tegak lurus) garis g
  getPointOnLine(g):  titik pada garis g
  perpendicular(A, g):  garis melalui A tegak lurus garis g
  parallel (A, g):  garis melalui A sejajar garis g
  lineIntersection(g, h):  titik potong garis g dan h
  projectToLine(A, g):   proyeksi titik A pada garis g
  distance(A, B):  jarak titik A dan B
  distanceSquared(A, B):  kuadrat jarak A dan B
  quadrance(A, B): kuadrat jarak A dan B
  areaTriangle(A, B, C):  luas segitiga ABC
  computeAngle(A, B, C):   besar sudut <ABC
  angleBisector(A, B, C): garis bagi sudut <ABC
  circleWithCenter (A, r): lingkaran dengan pusat A dan jari-jari r
  getCircleCenter(c):  pusat lingkaran c
  getCircleRadius(c):  jari-jari lingkaran c
  circleThrough(A,B,C):  lingkaran melalui A, B, C
  middlePerpendicular(A, B): titik tengah AB
  lineCircleIntersections(g, c): titik potong garis g dan lingkran c
  circleCircleIntersections (c1, c2):  titik potong lingkaran c1 dan
\end{eulerttcomment}
\begin{eulercomment}
c2\\
\end{eulercomment}
\begin{eulerttcomment}
  planeThrough(A, B, C):  bidang melalui titik A, B, C
\end{eulerttcomment}
\begin{eulercomment}

Fungsi-fungsi Khusus Untuk Geometri Simbolik:

\end{eulercomment}
\begin{eulerttcomment}
  getLineEquation (g,x,y): persamaan garis g dinyatakan dalam x dan y
  getHesseForm (g,x,y,A): bentuk Hesse garis g dinyatakan dalam x dan
\end{eulerttcomment}
\begin{eulercomment}
y dengan titik A pada\\
\end{eulercomment}
\begin{eulerttcomment}
  sisi positif (kanan/atas) garis
  quad(A,B): kuadrat jarak AB
  spread(a,b,c): Spread segitiga dengan panjang sisi-sisi a,b,c, yakni
\end{eulerttcomment}
\begin{eulercomment}
sin(alpha)\textasciicircum{}2 dengan\\
\end{eulercomment}
\begin{eulerttcomment}
  alpha sudut yang menghadap sisi a.
  crosslaw(a,b,c,sa): persamaan 3 quads dan 1 spread pada segitiga
\end{eulerttcomment}
\begin{eulercomment}
dengan panjang sisi a, b, c.\\
\end{eulercomment}
\begin{eulerttcomment}
  triplespread(sa,sb,sc): persamaan 3 spread sa,sb,sc yang memebntuk
\end{eulerttcomment}
\begin{eulercomment}
suatu segitiga\\
\end{eulercomment}
\begin{eulerttcomment}
  doublespread(sa): Spread sudut rangkap Spread 2*phi, dengan
\end{eulerttcomment}
\begin{eulercomment}
sa=sin(phi)\textasciicircum{}2 spread a.

\end{eulercomment}
\eulersubheading{Contoh 1: Luas, Lingkaran Luar, Lingkaran Dalam Segitiga}
\begin{eulercomment}
Untuk menggambar objek-objek geometri, langkah pertama adalah
menentukan rentang sumbu-sumbu koordinat. Semua objek geometri akan
digambar pada satu bidang koordinat, sampai didefinisikan bidang
koordinat yang baru.
\end{eulercomment}
\begin{eulerprompt}
>setPlotRange(-0.5,2.5,-0.5,2.5); // mendefinisikan bidang koordinat baru 
\end{eulerprompt}
\begin{eulercomment}
Sekarang tetapkan tiga poin dan plot mereka.
\end{eulercomment}
\begin{eulerprompt}
>A=[1,0]; plotPoint(A,"A"); // definisi dan gambar tiga titik
>B=[0,1]; plotPoint(B,"B");
>C=[2,2]; plotPoint(C,"C");
\end{eulerprompt}
\begin{eulercomment}
Kemudian tiga segmen.
\end{eulercomment}
\begin{eulerprompt}
>plotSegment(A,B,"c"); // c=AB
>plotSegment(B,C,"a"); // a=BC
>plotSegment(A,C,"b"); // b=AC
\end{eulerprompt}
\begin{eulercomment}
Fungsi geometri meliputi fungsi untuk membuat garis dan lingkaran.
Format garis adalah [a,b,c], yang mewakili garis dengan persamaan
ax+by=c.
\end{eulercomment}
\begin{eulerprompt}
>lineThrough(B,C) // garis yang melalui B dan C
\end{eulerprompt}
\begin{euleroutput}
  [-1,  2,  2]
\end{euleroutput}
\begin{eulercomment}
Hitunglah garis tegak lurus yang melalui A pada BC.
\end{eulercomment}
\begin{eulerprompt}
>h=perpendicular(A,lineThrough(B,C)); // garis h tegak lurus BC melalui A
\end{eulerprompt}
\begin{eulercomment}
Dan persimpangannya dengan BC.
\end{eulercomment}
\begin{eulerprompt}
>D=lineIntersection(h,lineThrough(B,C)); // D adalah titik potong h dan BC
\end{eulerprompt}
\begin{eulercomment}
Plot itu.
\end{eulercomment}
\begin{eulerprompt}
>plotPoint(D,value=1); // koordinat D ditampilkan
>aspect(1); plotSegment(A,D): // tampilkan semua gambar hasil plot...()
\end{eulerprompt}
\begin{eulercomment}
Hitung luas ABC:

\end{eulercomment}
\begin{eulerformula}
\[
L_{\triangle ABC}= \frac{1}{2}AD.BC.
\]
\end{eulerformula}
\begin{eulerprompt}
>norm(A-D)*norm(B-C)/2 // AD=norm(A-D), BC=norm(B-C)
\end{eulerprompt}
\begin{euleroutput}
  1.5
\end{euleroutput}
\begin{eulercomment}
Bandingkan dengan rumus determinan.
\end{eulercomment}
\begin{eulerprompt}
>areaTriangle(A,B,C) // hitung luas segitiga langusng dengan fungsi
\end{eulerprompt}
\begin{euleroutput}
  1.5
\end{euleroutput}
\begin{eulercomment}
Cara lain menghitung luas segitigas ABC:
\end{eulercomment}
\begin{eulerprompt}
>distance(A,D)*distance(B,C)/2
\end{eulerprompt}
\begin{euleroutput}
  1.5
\end{euleroutput}
\begin{eulercomment}
Sudut di C
\end{eulercomment}
\begin{eulerprompt}
>degprint(computeAngle(B,C,A))
\end{eulerprompt}
\begin{euleroutput}
  36°52'11.63''
\end{euleroutput}
\begin{eulercomment}
Sekarang lingkaran luar segitiga.
\end{eulercomment}
\begin{eulerprompt}
>c=circleThrough(A,B,C); // lingkaran luar segitiga ABC
>R=getCircleRadius(c); // jari2 lingkaran luar 
>O=getCircleCenter(c); // titik pusat lingkaran c 
>plotPoint(O,"O"); // gambar titik "O"
>plotCircle(c,"Lingkaran luar segitiga ABC"):
\end{eulerprompt}
\begin{eulercomment}
Tampilkan koordinat titik pusat dan jari-jari lingkaran luar.
\end{eulercomment}
\begin{eulerprompt}
>O, R
\end{eulerprompt}
\begin{euleroutput}
  [1.16667,  1.16667]
  1.17851130198
\end{euleroutput}
\begin{eulercomment}
Sekarang akan digambar lingkaran dalam segitiga ABC. Titik pusat lingkaran dalam adalah
titik potong garis-garis bagi sudut.
\end{eulercomment}
\begin{eulerprompt}
>l=angleBisector(A,C,B); // garis bagi <ACB
>g=angleBisector(C,A,B); // garis bagi <CAB
>P=lineIntersection(l,g) // titik potong kedua garis bagi sudut
\end{eulerprompt}
\begin{euleroutput}
  [0.86038,  0.86038]
\end{euleroutput}
\begin{eulercomment}
Tambahkan semuanya ke plot.
\end{eulercomment}
\begin{eulerprompt}
>color(5); plotLine(l); plotLine(g); color(1); // gambar kedua garis bagi sudut
>plotPoint(P,"P"); // gambar titik potongnya
>r=norm(P-projectToLine(P,lineThrough(A,B))) // jari-jari lingkaran dalam
\end{eulerprompt}
\begin{euleroutput}
  0.509653732104
\end{euleroutput}
\begin{eulerprompt}
>plotCircle(circleWithCenter(P,r),"Lingkaran dalam segitiga ABC"): // gambar lingkaran dalam
\end{eulerprompt}
\eulersubheading{Latihan}
\begin{eulercomment}
1. Tentukan ketiga titik singgung lingkaran dalam dengan sisi-sisi
segitiga ABC.
\end{eulercomment}
\begin{eulerprompt}
>setPlotRange(-2.5,4.5,-2.5,4.5);
>A=[-2,1]; plotPoint(A,"A");
>B=[1,-2]; plotPoint(B,"B");
>C=[4,4]; plotPoint(C,"C"):
\end{eulerprompt}
\begin{eulercomment}
2. Gambar segitiga dengan titik-titik sudut ketiga titik singgung
tersebut.
\end{eulercomment}
\begin{eulerprompt}
>plotSegment(A,B,"c")
>plotSegment(B,C,"a")
>plotSegment(A,C,"b")
>aspect(1):
\end{eulerprompt}
\begin{eulercomment}
3. Tunjukkan bahwa garis bagi sudut yang ke tiga juga melalui titik
pusat lingkaran dalam.
\end{eulercomment}
\begin{eulerprompt}
>l=angleBisector(A,C,B);
>g=angleBisector(C,A,B);
>P=lineIntersection(l,g)
\end{eulerprompt}
\begin{euleroutput}
  [0.581139,  0.581139]
\end{euleroutput}
\begin{eulerprompt}
>color(5); plotLine(l); plotLine(g); color(1);
>plotPoint(P,"P");
>r=norm(P-projectToLine(P,lineThrough(A,B)))
\end{eulerprompt}
\begin{euleroutput}
  1.52896119631
\end{euleroutput}
\begin{eulerprompt}
>plotCircle(circleWithCenter(P,r),"Lingkaran dalam segitiga ABC"):
\end{eulerprompt}
\begin{eulercomment}
Jadi, terbukti bahwa garis bagi sudut yang ketiga juga melalui titik
pusat lingkaran dalam.

4. Gambar jari-jari lingkaran dalam.
\end{eulercomment}
\begin{eulerprompt}
>r=norm(P-projectToLine(P,lineThrough(A,B)))
\end{eulerprompt}
\begin{euleroutput}
  1.52896119631
\end{euleroutput}
\begin{eulerprompt}
>plotCircle(circleWithCenter(P,r),"Lingkaran dalam segitiga ABC"):
\end{eulerprompt}
\eulersubheading{Contoh 2: Geometri Simbolik}
\begin{eulercomment}
Kita dapat menghitung geometri eksak dan simbolik menggunakan Maxima.

File geometri.e menyediakan fungsi yang sama (dan lebih banyak lagi)
di Maxima. Namun, kita dapat menggunakan perhitungan simbolis
sekarang.
\end{eulercomment}
\begin{eulerprompt}
>A &= [1,0]; B &= [0,1]; C &= [2,2]; // menentukan tiga titik A, B, C
\end{eulerprompt}
\begin{eulercomment}
Fungsi untuk garis dan lingkaran bekerja seperti fungsi Euler, tetapi
memberikan perhitungan simbolis.
\end{eulercomment}
\begin{eulerprompt}
>c &= lineThrough(B,C) // c=BC
\end{eulerprompt}
\begin{euleroutput}
  
                               [- 1, 2, 2]
  
\end{euleroutput}
\begin{eulercomment}
Kita bisa mendapatkan persamaan garis dengan mudah.
\end{eulercomment}
\begin{eulerprompt}
>$getLineEquation(c,x,y), $solve(%,y) | expand // persamaan garis c
\end{eulerprompt}
\begin{euleroutput}
  Maxima said:
  solve: all variables must not be numbers.
   -- an error. To debug this try: debugmode(true);
  
  Error in:
   $getLineEquation(c,x,y), $solve(%,y) | expand // persamaan gar ...
                                                ^
\end{euleroutput}
\begin{eulerprompt}
>$getLineEquation(lineThrough(A,[x1,y1]),x,y) // persamaan garis melalui A dan (x1, y1)
>h &= perpendicular(A,lineThrough(B,C)) // h melalui A tegak lurus BC
\end{eulerprompt}
\begin{euleroutput}
  
                                [2, 1, 2]
  
\end{euleroutput}
\begin{eulerprompt}
>Q &= lineIntersection(c,h) // Q titik potong garis c=BC dan h
\end{eulerprompt}
\begin{euleroutput}
  Maxima said:
  rat: replaced 9.983250083613754e-5 by 612914/6139423483 = 9.983250083613756e-5
  
  rat: replaced 3.986533601775671e-4 by 220554/553247563 = 3.986533601775666e-4
  
  rat: replaced 8.954327045205754e-4 by 584699/652979277 = 8.954327045205756e-4
  
  rat: replaced 0.001589120864678328 by 740868/466212493 = 0.00158912086467833
  
  rat: replaced 0.002478648480745763 by 878917/354595259 = 0.002478648480745762
  
  rat: replaced 0.003562926609036218 by 2735717/767828614 = 0.003562926609036219
  
  rat: replaced 0.004840846830973591 by 1164348/240525685 = 0.004840846830973582
  
  rat: replaced 0.006311281363933816 by 16515210/2616776063 = 0.006311281363933816
  
  rat: replaced 0.007973083174022497 by 2414321/302808957 = 0.007973083174022491
  
  rat: replaced 0.009825086090776508 by 1144049/116441626 = 0.009825086090776506
  
  rat: replaced 0.01186610492378118 by 1659683/139867548 = 0.01186610492378118
  
  rat: replaced 0.01409493558118687 by 986877/70016425 = 0.01409493558118684
  
  rat: replaced 0.01651035519011868 by 1738361/105289134 = 0.01651035519011867
  
  rat: replaced 0.01911112221896202 by 1475047/77182647 = 0.01911112221896199
  
  rat: replaced 0.02189597660151474 by 7711274/352177669 = 0.02189597660151473
  
  rat: replaced 0.02486363986299212 by 3887839/156366446 = 0.02486363986299209
  
  rat: replaced 0.0280128152478745 by 2263313/80795628 = 0.02801281524787455
  
  rat: replaced 0.03134218784958129 by 1116362/35618509 = 0.03134218784958124
  
  rat: replaced 0.03485042474195996 by 3920507/112495243 = 0.03485042474195998
  
  rat: replaced 0.03853617511257795 by 5379408/139593719 = 0.03853617511257795
  
  rat: replaced 0.04239807039780302 by 3385918/79860191 = 0.04239807039780308
  
  rat: replaced 0.04643472441965829 by 10918553/235137672 = 0.04643472441965828
  
  rat: replaced 0.05064473352443885 by 5036501/99447675 = 0.05064473352443886
  
  rat: replaced 0.05502667672307548 by 2932521/53292715 = 0.05502667672307557
  
  rat: replaced 0.05957911583323347 by 6320819/106091185 = 0.05957911583323346
  
  rat: replaced 0.06430059562312868 by 9893260/153859539 = 0.0643005956231287
  
  rat: replaced 0.06918964395705007 by 6012189/86894348 = 0.06918964395705
  
  rat: replaced 0.07424477194257195 by 6096479/82113243 = 0.07424477194257204
  
  rat: replaced 0.07946447407944118 by 5389689/67825139 = 0.07946447407944125
  
  rat: replaced 0.0848472284101276 by 9595393/113090235 = 0.08484722841012754
  
  rat: replaced 0.09039149667201674 by 3773144/41742245 = 0.09039149667201657
  
  rat: replaced 0.0960957244512361 by 5162056/53717853 = 0.09609572445123597
  
  rat: replaced 0.1019583413380946 by 1082663/10618680 = 0.1019583413380948
  
  rat: replaced 0.107977761084122 by 1922059/17800508 = 0.1079777610841219
  
  rat: replaced 0.1141523817606936 by 5923297/51889386 = 0.1141523817606938
  
  rat: replaced 0.1204805859192203 by 17634703/146369665 = 0.1204805859192204
  
  rat: replaced 0.1269607407528933 by 11368220/89541223 = 0.1269607407528932
  
  rat: replaced 0.1335911982599624 by 4657902/34866833 = 0.1335911982599624
  
  rat: replaced 0.1403702954085355 by 8528456/60756843 = 0.1403702954085353
  
  rat: replaced 0.1472963543028805 by 11128453/75551449 = 0.1472963543028804
  
  rat: replaced 0.1543676823512128 by 8170760/52930509 = 0.1543676823512126
  
  rat: replaced 0.1615825724349539 by 188109817/1164171446 = 0.1615825724349539
  
  rat: replaced 0.1689393030794406 by 5046974/29874481 = 0.1689393030794409
  
  rat: replaced 0.1764361386260728 by 6530305/37012287 = 0.176436138626073
  
  rat: replaced 0.1840713294058766 by 25189859/136848357 = 0.1840713294058766
  
  rat: replaced 0.1918431119144694 by 24326967/126806570 = 0.1918431119144694
  
  rat: replaced 0.1997497089884105 by 14902039/74603558 = 0.1997497089884104
  
  rat: replaced 0.2077893299829148 by 7281351/35041987 = 0.2077893299829145
  
  rat: replaced 0.2159601709509153 by 11348921/52550991 = 0.2159601709509151
  
  rat: replaced 0.2242604148234577 by 22385730/99820247 = 0.2242604148234576
  
  rat: replaced 0.2326882315914051 by 25615030/110083049 = 0.2326882315914051
  
  rat: replaced 0.2412417784884371 by 14523232/60201977 = 0.2412417784884373
  
  rat: replaced 0.2499192001753251 by 11309023/45250717 = 0.2499192001753254
  
  rat: replaced 0.2587186289254649 by 7582961/29309683 = 0.2587186289254647
  
  rat: replaced 0.267638184811648 by 17912865/66929407 = 0.2676381848116479
  
  rat: replaced 0.2766759758940514 by 27538925/99534934 = 0.2766759758940514
  
  rat: replaced 0.2858300984094321 by 29258587/102363562 = 0.2858300984094321
  
  rat: replaced 0.2950986369614998 by 7877677/26695064 = 0.2950986369614997
  
  rat: replaced 0.304479664712457 by 14469542/47522195 = 0.304479664712457
  
  rat: replaced 0.3139712435756791 by 8375733/26676752 = 0.3139712435756797
  
  rat: replaced 0.3235714244095225 by 178371467/551258404 = 0.3235714244095225
  
  rat: replaced 0.3332782472122374 by 5743591/17233621 = 0.333278247212237
  
  rat: replaced 0.3430897413179662 by 15588245/45434891 = 0.3430897413179664
  
  rat: replaced 0.3530039255938071 by 6523425/18479752 = 0.3530039255938067
  
  rat: replaced 0.3630188086379282 by 51253958/141188161 = 0.3630188086379282
  
  rat: replaced 0.373132388978704 by 9370061/25111894 = 0.3731323889787047
  
  rat: replaced 0.3833426552748616 by 11820697/30835851 = 0.3833426552748617
  
  rat: replaced 0.393647586516613 by 9153768/23253713 = 0.3936475865166135
  
  rat: replaced 0.4040451522277552 by 16634707/41170416 = 0.404045152227755
  
  rat: replaced 0.4145333126687146 by 2088920/5039209 = 0.4145333126687145
  
  rat: replaced 0.4251100190405208 by 24667763/58026774 = 0.4251100190405209
  
  rat: replaced 0.4357732136896836 by 10448574/23977091 = 0.435773213689684
  
  rat: replaced 0.4465208303139576 by 8346266/18691773 = 0.4465208303139568
  
  rat: replaced 0.4573507941689697 by 20158688/44077081 = 0.4573507941689696
  
  rat: replaced 0.4682610222756929 by 12818601/27374905 = 0.4682610222756937
  
  rat: replaced 0.4792494236287415 by 13652513/28487281 = 0.4792494236287416
  
  rat: replaced 0.4903138994054704 by 35114711/71616797 = 0.4903138994054705
  
  rat: replaced 0.5014523431758559 by 15102855/30118226 = 0.5014523431758564
  
  rat: replaced 0.5126626411131362 by 31697340/61828847 = 0.5126626411131361
  
  rat: replaced 0.5239426722051925 by 27432767/52358337 = 0.5239426722051924
  
  rat: replaced 0.5352903084666492 by 6124470/11441399 = 0.5352903084666482
  
  rat: replaced 0.5467034151516694 by 41717397/76307182 = 0.5467034151516694
  
  rat: replaced 0.5581798509674292 by 7494380/13426461 = 0.5581798509674292
  
  rat: replaced 0.5697174682882435 by 14609183/25642856 = 0.5697174682882438
  
  rat: replaced 0.581314113370329 by 14367580/24715691 = 0.5813141133703282
  
  rat: replaced 0.5929676265671738 by 9820294/16561265 = 0.5929676265671735
  
  rat: replaced 0.6046758425455033 by 23593213/39017952 = 0.6046758425455031
  
  rat: replaced 0.6164365905018095 by 15720181/25501700 = 0.6164365905018097
  
  rat: replaced 0.6282476943794307 by 53974636/85912987 = 0.6282476943794306
  
  rat: replaced 0.640106973086155 by 20459615/31962806 = 0.6401069730861552
  
  rat: replaced 0.652012240712328 by 51645100/79208789 = 0.652012240712328
  
  rat: replaced 0.6639613067494411 by 12215999/18398661 = 0.6639613067494422
  
  rat: replaced 0.6759519763091814 by 18558734/27455699 = 0.6759519763091808
  
  rat: replaced 0.6879820503429186 by 23500536/34158647 = 0.687982050342919
  
  rat: replaced 0.7000493258616074 by 29992669/42843651 = 0.7000493258616078
  
  rat: replaced 0.7121515961560857 by 10685401/15004391 = 0.7121515961560853
  
  rat: replaced 0.7242866510177421 by 11795807/16286103 = 0.7242866510177419
  
  rat: replaced 0.7364522769595366 by 14940657/20287339 = 0.7364522769595362
  
  rat: replaced 0.7486462574373463 by 42508133/56779998 = 0.7486462574373461
  
  rat: replaced 5.033291500140813e-5 by 263336/5231884543 = 5.033291500140813e-5
  
  rat: replaced 2.026599467560841e-4 by 407727/2011877564 = 2.02659946756084e-4
  
  rat: replaced 4.589658460211338e-4 by 352373/767754296 = 4.589658460211339e-4
  
  rat: replaced 8.21224965753764e-4 by 219501/267284860 = 8.212249657537654e-4
  
  rat: replaced 0.001291401063677061 by 174589/135193477 = 0.001291401063677059
  
  rat: replaced 0.001871447105906615 by 1078337/576204904 = 0.001871447105906617
  
  rat: replaced 0.002563305071654892 by 1323915/516487489 = 0.002563305071654891
  
  rat: replaced 0.003368905759035173 by 820537/243561874 = 0.003368905759035176
  
  rat: replaced 0.00429016859198364 by 7572857/1765165363 = 0.00429016859198364
  
  rat: replaced 0.005329001428317881 by 3020890/566877311 = 0.005329001428317882
  
  rat: replaced 0.006487300368953564 by 2580732/397812935 = 0.006487300368953564
  
  rat: replaced 0.007766949568295017 by 1049181/135082762 = 0.007766949568295028
  
  rat: replaced 0.009169821045822202 by 2408608/262666849 = 0.009169821045822193
  
  rat: replaced 0.01069777449888981 by 2325322/217365023 = 0.01069777449888982
  
  rat: replaced 0.01235265711675931 by 7449711/603085711 = 0.01235265711675931
  
  rat: replaced 0.01413630339588112 by 3774568/267012379 = 0.01413630339588113
  
  rat: replaced 0.0160505349564472 by 2619104/163178611 = 0.0160505349564472
  
  rat: replaced 0.01809716036023018 by 3107690/171722521 = 0.01809716036023021
  
  rat: replaced 0.02027797492972855 by 6791343/334912289 = 0.02027797492972854
  
  rat: replaced 0.02259476056863596 by 2685790/118867823 = 0.02259476056863597
  
  rat: replaced 0.0250492855836526 by 2956693/118035023 = 0.02504928558365258
  
  rat: replaced 0.02764330450765584 by 2138111/77346433 = 0.02764330450765583
  
  rat: replaced 0.03037855792424843 by 1678577/55255322 = 0.03037855792424846
  
  rat: replaced 0.03325677229370128 by 1488397/44754704 = 0.03325677229370124
  
  rat: replaced 0.03627965978030939 by 3229091/89005548 = 0.03627965978030943
  
  rat: replaced 0.03944891808117656 by 6094420/154488901 = 0.03944891808117659
  
  rat: replaced 0.04276623025644721 by 206826/4836199 = 0.04276623025644726
  
  rat: replaced 0.04623326456100163 by 7175941/155211644 = 0.04623326456100162
  
  rat: replaced 0.04985167427763171 by 2856261/57295187 = 0.04985167427763173
  
  rat: replaced 0.0536230975517149 by 8075629/150599823 = 0.05362309755171492
  
  rat: replaced 0.05754915722739962 by 12314906/213989337 = 0.05754915722739961
  
  rat: replaced 0.06163146068532366 by 10145753/164619707 = 0.06163146068532366
  
  rat: replaced 0.06587159968187639 by 5154956/78257641 = 0.06587159968187643
  
  rat: replaced 0.07027115019002506 by 3189686/45391117 = 0.07027115019002507
  
  rat: replaced 0.07483167224171838 by 4757796/63579977 = 0.0748316722417185
  
  rat: replaced 0.07955470977188528 by 7059961/88743470 = 0.07955470977188518
  
  rat: replaced 0.08444179046404166 by 17285418/204702173 = 0.08444179046404163
  
  rat: replaced 0.08949442559752452 by 6119169/68374862 = 0.08949442559752442
  
  rat: replaced 0.0947141098963642 by 2739857/28927654 = 0.09471410989636422
  
  rat: replaced 0.100102321379814 by 21380147/213582929 = 0.100102321379814
  
  rat: replaced 0.1056605212145493 by 8628153/81659194 = 0.1056605212145493
  
  rat: replaced 0.1113901535685515 by 4925969/44222661 = 0.1113901535685517
  
  rat: replaced 0.1172926454666934 by 7052303/60125705 = 0.1172926454666935
  
  rat: replaced 0.1233694066480375 by 17851649/144700777 = 0.1233694066480376
  
  rat: replaced 0.1296218294248629 by 13037238/100579031 = 0.1296218294248629
  
  rat: replaced 0.1360512885434353 by 20468361/150445918 = 0.1360512885434353
  
  rat: replaced 0.1426591410465347 by 8451499/59242604 = 0.1426591410465347
  
  rat: replaced 0.1494467261377502 by 40350618/270000013 = 0.1494467261377502
  
  rat: replaced 0.1564153650475627 by 30759845/196654881 = 0.1564153650475627
  
  rat: replaced 0.1635663609012215 by 11970848/73186491 = 0.1635663609012215
  
  rat: replaced 0.1709009985884339 by 3726835/21806982 = 0.1709009985884337
  
  rat: replaced 0.1784205446348769 by 7050541/39516419 = 0.178420544634877
  
  rat: replaced 0.1861262470755453 by 7913431/42516470 = 0.1861262470755451
  
  rat: replaced 0.1940193353299499 by 15356416/79148895 = 0.19401933532995
  
  rat: replaced 0.2021010200791761 by 21517868/106470853 = 0.202101020079176
  
  rat: replaced 0.2103724931448173 by 10133132/48167571 = 0.2103724931448173
  
  rat: replaced 0.2188349273697929 by 14393696/65774217 = 0.2188349273697929
  
  rat: replaced 0.2274894765010662 by 2362445/10384854 = 0.2274894765010659
  
  rat: replaced 0.2363372750742693 by 14238388/60246053 = 0.2363372750742692
  
  rat: replaced 0.2453794383002513 by 11843947/48267887 = 0.2453794383002513
  
  rat: replaced 0.2546170619535583 by 10437767/40993981 = 0.2546170619535585
  
  rat: replaced 0.2640512222628563 by 18572095/70335198 = 0.2640512222628562
  
  rat: replaced 0.2736829758033094 by 25733021/94024924 = 0.2736829758033094
  
  rat: replaced 0.2835133593909236 by 5354031/18884581 = 0.2835133593909232
  
  rat: replaced 0.2935433899788653 by 33562265/114334937 = 0.2935433899788654
  
  rat: replaced 0.3037740645557676 by 12785981/42090430 = 0.3037740645557672
  
  rat: replaced 0.3142063600460319 by 13879096/44171913 = 0.314206360046032
  
  rat: replaced 0.3248412332121354 by 13048490/40168823 = 0.3248412332121357
  
  rat: replaced 0.3356796205589581 by 12520681/37299497 = 0.3356796205589582
  
  rat: replaced 0.3467224382401299 by 27133151/78256115 = 0.3467224382401299
  
  rat: replaced 0.3579705819664191 by 32019579/89447515 = 0.3579705819664191
  
  rat: replaced 0.3694249269161592 by 12845283/34771024 = 0.3694249269161587
  
  rat: replaced 0.3810863276477343 by 12790304/33562747 = 0.381086327647734
  
  rat: replaced 0.3929556180141225 by 27557157/70127912 = 0.3929556180141225
  
  rat: replaced 0.4050336110795114 by 12582391/31065054 = 0.4050336110795107
  
  rat: replaced 0.4173210990379927 by 17616979/42214446 = 0.4173210990379928
  
  rat: replaced 0.4298188531343438 by 28764336/66921997 = 0.4298188531343439
  
  rat: replaced 0.4425276235869029 by 52612738/118891421 = 0.4425276235869029
  
  rat: replaced 0.4554481395125489 by 12438812/27311149 = 0.4554481395125485
  
  rat: replaced 0.4685811088537897 by 12910499/27552325 = 0.46858110885379
  
  rat: replaced 0.4819272183079686 by 11623658/24119115 = 0.4819272183079686
  
  rat: replaced 0.4954871332585954 by 40137729/81006602 = 0.4954871332585954
  
  rat: replaced 0.5092614977088081 by 27060617/53136978 = 0.5092614977088084
  
  rat: replaced 0.523250934216974 by 57357723/109618004 = 0.5232509342169741
  
  rat: replaced 0.5374560438344332 by 19984722/37183919 = 0.5374560438344328
  
  rat: replaced 0.551877406045395 by 10637804/19275665 = 0.5518774060453946
  
  rat: replaced 0.5665155787089895 by 22241852/39260795 = 0.5665155787089895
  
  rat: replaced 0.5813710980034821 by 10844268/18652919 = 0.5813710980034814
  
  rat: replaced 0.5964444783726564 by 13079224/21928653 = 0.596444478372657
  
  rat: replaced 0.6117362124743696 by 11199699/18308053 = 0.6117362124743685
  
  rat: replaced 0.6272467711312885 by 11338738/18076997 = 0.6272467711312891
  
  rat: replaced 0.6429766032838061 by 10161473/15803799 = 0.6429766032838053
  
  rat: replaced 0.6589261359451484 by 11120191/16876233 = 0.6589261359451484
  
  rat: replaced 0.6750957741586742 by 11234073/16640710 = 0.6750957741586747
  
  rat: replaced 0.6914859009573701 by 9571673/13842181 = 0.6914859009573708
  
  rat: replaced 0.7080968773255479 by 20218829/28553761 = 0.7080968773255474
  
  rat: replaced 0.7249290421627467 by 10945526/15098755 = 0.7249290421627479
  
  rat: replaced 0.7419827122498429 by 23520179/31699093 = 0.7419827122498426
  
  rat: replaced 0.7592581822173726 by 16709871/22008154 = 0.7592581822173727
  part: invalid index of list or matrix.
  #0: lineIntersection(g=[-1,2,2],h=[2,1,2])
   -- an error. To debug this try: debugmode(true);
  
  Error in:
  ... ersection(c,h) // Q titik potong garis c=BC dan h ...
                                                       ^
\end{euleroutput}
\begin{eulerprompt}
>$projectToLine(A,lineThrough(B,C)) // proyeksi A pada BC
\end{eulerprompt}
\begin{euleroutput}
  Maxima said:
  rat: replaced 5.033291500140813e-5 by 263336/5231884543 = 5.033291500140813e-5
  
  rat: replaced 2.026599467560841e-4 by 407727/2011877564 = 2.02659946756084e-4
  
  rat: replaced 4.589658460211338e-4 by 352373/767754296 = 4.589658460211339e-4
  
  rat: replaced 8.21224965753764e-4 by 219501/267284860 = 8.212249657537654e-4
  
  rat: replaced 0.001291401063677061 by 174589/135193477 = 0.001291401063677059
  
  rat: replaced 0.001871447105906615 by 1078337/576204904 = 0.001871447105906617
  
  rat: replaced 0.002563305071654892 by 1323915/516487489 = 0.002563305071654891
  
  rat: replaced 0.003368905759035173 by 820537/243561874 = 0.003368905759035176
  
  rat: replaced 0.00429016859198364 by 7572857/1765165363 = 0.00429016859198364
  
  rat: replaced 0.005329001428317881 by 3020890/566877311 = 0.005329001428317882
  
  rat: replaced 0.006487300368953564 by 2580732/397812935 = 0.006487300368953564
  
  rat: replaced 0.007766949568295017 by 1049181/135082762 = 0.007766949568295028
  
  rat: replaced 0.009169821045822202 by 2408608/262666849 = 0.009169821045822193
  
  rat: replaced 0.01069777449888981 by 2325322/217365023 = 0.01069777449888982
  
  rat: replaced 0.01235265711675931 by 7449711/603085711 = 0.01235265711675931
  
  rat: replaced 0.01413630339588112 by 3774568/267012379 = 0.01413630339588113
  
  rat: replaced 0.0160505349564472 by 2619104/163178611 = 0.0160505349564472
  
  rat: replaced 0.01809716036023018 by 3107690/171722521 = 0.01809716036023021
  
  rat: replaced 0.02027797492972855 by 6791343/334912289 = 0.02027797492972854
  
  rat: replaced 0.02259476056863596 by 2685790/118867823 = 0.02259476056863597
  
  rat: replaced 0.0250492855836526 by 2956693/118035023 = 0.02504928558365258
  
  rat: replaced 0.02764330450765584 by 2138111/77346433 = 0.02764330450765583
  
  rat: replaced 0.03037855792424843 by 1678577/55255322 = 0.03037855792424846
  
  rat: replaced 0.03325677229370128 by 1488397/44754704 = 0.03325677229370124
  
  rat: replaced 0.03627965978030939 by 3229091/89005548 = 0.03627965978030943
  
  rat: replaced 0.03944891808117656 by 6094420/154488901 = 0.03944891808117659
  
  rat: replaced 0.04276623025644721 by 206826/4836199 = 0.04276623025644726
  
  rat: replaced 0.04623326456100163 by 7175941/155211644 = 0.04623326456100162
  
  rat: replaced 0.04985167427763171 by 2856261/57295187 = 0.04985167427763173
  
  rat: replaced 0.0536230975517149 by 8075629/150599823 = 0.05362309755171492
  
  rat: replaced 0.05754915722739962 by 12314906/213989337 = 0.05754915722739961
  
  rat: replaced 0.06163146068532366 by 10145753/164619707 = 0.06163146068532366
  
  rat: replaced 0.06587159968187639 by 5154956/78257641 = 0.06587159968187643
  
  rat: replaced 0.07027115019002506 by 3189686/45391117 = 0.07027115019002507
  
  rat: replaced 0.07483167224171838 by 4757796/63579977 = 0.0748316722417185
  
  rat: replaced 0.07955470977188528 by 7059961/88743470 = 0.07955470977188518
  
  rat: replaced 0.08444179046404166 by 17285418/204702173 = 0.08444179046404163
  
  rat: replaced 0.08949442559752452 by 6119169/68374862 = 0.08949442559752442
  
  rat: replaced 0.0947141098963642 by 2739857/28927654 = 0.09471410989636422
  
  rat: replaced 0.100102321379814 by 21380147/213582929 = 0.100102321379814
  
  rat: replaced 0.1056605212145493 by 8628153/81659194 = 0.1056605212145493
  
  rat: replaced 0.1113901535685515 by 4925969/44222661 = 0.1113901535685517
  
  rat: replaced 0.1172926454666934 by 7052303/60125705 = 0.1172926454666935
  
  rat: replaced 0.1233694066480375 by 17851649/144700777 = 0.1233694066480376
  
  rat: replaced 0.1296218294248629 by 13037238/100579031 = 0.1296218294248629
  
  rat: replaced 0.1360512885434353 by 20468361/150445918 = 0.1360512885434353
  
  rat: replaced 0.1426591410465347 by 8451499/59242604 = 0.1426591410465347
  
  rat: replaced 0.1494467261377502 by 40350618/270000013 = 0.1494467261377502
  
  rat: replaced 0.1564153650475627 by 30759845/196654881 = 0.1564153650475627
  
  rat: replaced 0.1635663609012215 by 11970848/73186491 = 0.1635663609012215
  
  rat: replaced 0.1709009985884339 by 3726835/21806982 = 0.1709009985884337
  
  rat: replaced 0.1784205446348769 by 7050541/39516419 = 0.178420544634877
  
  rat: replaced 0.1861262470755453 by 7913431/42516470 = 0.1861262470755451
  
  rat: replaced 0.1940193353299499 by 15356416/79148895 = 0.19401933532995
  
  rat: replaced 0.2021010200791761 by 21517868/106470853 = 0.202101020079176
  
  rat: replaced 0.2103724931448173 by 10133132/48167571 = 0.2103724931448173
  
  rat: replaced 0.2188349273697929 by 14393696/65774217 = 0.2188349273697929
  
  rat: replaced 0.2274894765010662 by 2362445/10384854 = 0.2274894765010659
  
  rat: replaced 0.2363372750742693 by 14238388/60246053 = 0.2363372750742692
  
  rat: replaced 0.2453794383002513 by 11843947/48267887 = 0.2453794383002513
  
  rat: replaced 0.2546170619535583 by 10437767/40993981 = 0.2546170619535585
  
  rat: replaced 0.2640512222628563 by 18572095/70335198 = 0.2640512222628562
  
  rat: replaced 0.2736829758033094 by 25733021/94024924 = 0.2736829758033094
  
  rat: replaced 0.2835133593909236 by 5354031/18884581 = 0.2835133593909232
  
  rat: replaced 0.2935433899788653 by 33562265/114334937 = 0.2935433899788654
  
  rat: replaced 0.3037740645557676 by 12785981/42090430 = 0.3037740645557672
  
  rat: replaced 0.3142063600460319 by 13879096/44171913 = 0.314206360046032
  
  rat: replaced 0.3248412332121354 by 13048490/40168823 = 0.3248412332121357
  
  rat: replaced 0.3356796205589581 by 12520681/37299497 = 0.3356796205589582
  
  rat: replaced 0.3467224382401299 by 27133151/78256115 = 0.3467224382401299
  
  rat: replaced 0.3579705819664191 by 32019579/89447515 = 0.3579705819664191
  
  rat: replaced 0.3694249269161592 by 12845283/34771024 = 0.3694249269161587
  
  rat: replaced 0.3810863276477343 by 12790304/33562747 = 0.381086327647734
  
  rat: replaced 0.3929556180141225 by 27557157/70127912 = 0.3929556180141225
  
  rat: replaced 0.4050336110795114 by 12582391/31065054 = 0.4050336110795107
  
  rat: replaced 0.4173210990379927 by 17616979/42214446 = 0.4173210990379928
  
  rat: replaced 0.4298188531343438 by 28764336/66921997 = 0.4298188531343439
  
  rat: replaced 0.4425276235869029 by 52612738/118891421 = 0.4425276235869029
  
  rat: replaced 0.4554481395125489 by 12438812/27311149 = 0.4554481395125485
  
  rat: replaced 0.4685811088537897 by 12910499/27552325 = 0.46858110885379
  
  rat: replaced 0.4819272183079686 by 11623658/24119115 = 0.4819272183079686
  
  rat: replaced 0.4954871332585954 by 40137729/81006602 = 0.4954871332585954
  
  rat: replaced 0.5092614977088081 by 27060617/53136978 = 0.5092614977088084
  
  rat: replaced 0.523250934216974 by 57357723/109618004 = 0.5232509342169741
  
  rat: replaced 0.5374560438344332 by 19984722/37183919 = 0.5374560438344328
  
  rat: replaced 0.551877406045395 by 10637804/19275665 = 0.5518774060453946
  
  rat: replaced 0.5665155787089895 by 22241852/39260795 = 0.5665155787089895
  
  rat: replaced 0.5813710980034821 by 10844268/18652919 = 0.5813710980034814
  
  rat: replaced 0.5964444783726564 by 13079224/21928653 = 0.596444478372657
  
  rat: replaced 0.6117362124743696 by 11199699/18308053 = 0.6117362124743685
  
  rat: replaced 0.6272467711312885 by 11338738/18076997 = 0.6272467711312891
  
  rat: replaced 0.6429766032838061 by 10161473/15803799 = 0.6429766032838053
  
  rat: replaced 0.6589261359451484 by 11120191/16876233 = 0.6589261359451484
  
  rat: replaced 0.6750957741586742 by 11234073/16640710 = 0.6750957741586747
  
  rat: replaced 0.6914859009573701 by 9571673/13842181 = 0.6914859009573708
  
  rat: replaced 0.7080968773255479 by 20218829/28553761 = 0.7080968773255474
  
  rat: replaced 0.7249290421627467 by 10945526/15098755 = 0.7249290421627479
  
  rat: replaced 0.7419827122498429 by 23520179/31699093 = 0.7419827122498426
  
  rat: replaced 0.7592581822173726 by 16709871/22008154 = 0.7592581822173727
  
  rat: replaced 9.983250083613754e-5 by 612914/6139423483 = 9.983250083613756e-5
  
  rat: replaced 3.986533601775671e-4 by 220554/553247563 = 3.986533601775666e-4
  
  rat: replaced 8.954327045205754e-4 by 584699/652979277 = 8.954327045205756e-4
  
  rat: replaced 0.001589120864678328 by 740868/466212493 = 0.00158912086467833
  
  rat: replaced 0.002478648480745763 by 878917/354595259 = 0.002478648480745762
  
  rat: replaced 0.003562926609036218 by 2735717/767828614 = 0.003562926609036219
  
  rat: replaced 0.004840846830973591 by 1164348/240525685 = 0.004840846830973582
  
  rat: replaced 0.006311281363933816 by 16515210/2616776063 = 0.006311281363933816
  
  rat: replaced 0.007973083174022497 by 2414321/302808957 = 0.007973083174022491
  
  rat: replaced 0.009825086090776508 by 1144049/116441626 = 0.009825086090776506
  
  rat: replaced 0.01186610492378118 by 1659683/139867548 = 0.01186610492378118
  
  rat: replaced 0.01409493558118687 by 986877/70016425 = 0.01409493558118684
  
  rat: replaced 0.01651035519011868 by 1738361/105289134 = 0.01651035519011867
  
  rat: replaced 0.01911112221896202 by 1475047/77182647 = 0.01911112221896199
  
  rat: replaced 0.02189597660151474 by 7711274/352177669 = 0.02189597660151473
  
  rat: replaced 0.02486363986299212 by 3887839/156366446 = 0.02486363986299209
  
  rat: replaced 0.0280128152478745 by 2263313/80795628 = 0.02801281524787455
  
  rat: replaced 0.03134218784958129 by 1116362/35618509 = 0.03134218784958124
  
  rat: replaced 0.03485042474195996 by 3920507/112495243 = 0.03485042474195998
  
  rat: replaced 0.03853617511257795 by 5379408/139593719 = 0.03853617511257795
  
  rat: replaced 0.04239807039780302 by 3385918/79860191 = 0.04239807039780308
  
  rat: replaced 0.04643472441965829 by 10918553/235137672 = 0.04643472441965828
  
  rat: replaced 0.05064473352443885 by 5036501/99447675 = 0.05064473352443886
  
  rat: replaced 0.05502667672307548 by 2932521/53292715 = 0.05502667672307557
  
  rat: replaced 0.05957911583323347 by 6320819/106091185 = 0.05957911583323346
  
  rat: replaced 0.06430059562312868 by 9893260/153859539 = 0.0643005956231287
  
  rat: replaced 0.06918964395705007 by 6012189/86894348 = 0.06918964395705
  
  rat: replaced 0.07424477194257195 by 6096479/82113243 = 0.07424477194257204
  
  rat: replaced 0.07946447407944118 by 5389689/67825139 = 0.07946447407944125
  
  rat: replaced 0.0848472284101276 by 9595393/113090235 = 0.08484722841012754
  
  rat: replaced 0.09039149667201674 by 3773144/41742245 = 0.09039149667201657
  
  rat: replaced 0.0960957244512361 by 5162056/53717853 = 0.09609572445123597
  
  rat: replaced 0.1019583413380946 by 1082663/10618680 = 0.1019583413380948
  
  rat: replaced 0.107977761084122 by 1922059/17800508 = 0.1079777610841219
  
  rat: replaced 0.1141523817606936 by 5923297/51889386 = 0.1141523817606938
  
  rat: replaced 0.1204805859192203 by 17634703/146369665 = 0.1204805859192204
  
  rat: replaced 0.1269607407528933 by 11368220/89541223 = 0.1269607407528932
  
  rat: replaced 0.1335911982599624 by 4657902/34866833 = 0.1335911982599624
  
  rat: replaced 0.1403702954085355 by 8528456/60756843 = 0.1403702954085353
  
  rat: replaced 0.1472963543028805 by 11128453/75551449 = 0.1472963543028804
  
  rat: replaced 0.1543676823512128 by 8170760/52930509 = 0.1543676823512126
  
  rat: replaced 0.1615825724349539 by 188109817/1164171446 = 0.1615825724349539
  
  rat: replaced 0.1689393030794406 by 5046974/29874481 = 0.1689393030794409
  
  rat: replaced 0.1764361386260728 by 6530305/37012287 = 0.176436138626073
  
  rat: replaced 0.1840713294058766 by 25189859/136848357 = 0.1840713294058766
  
  rat: replaced 0.1918431119144694 by 24326967/126806570 = 0.1918431119144694
  
  rat: replaced 0.1997497089884105 by 14902039/74603558 = 0.1997497089884104
  
  rat: replaced 0.2077893299829148 by 7281351/35041987 = 0.2077893299829145
  
  rat: replaced 0.2159601709509153 by 11348921/52550991 = 0.2159601709509151
  
  rat: replaced 0.2242604148234577 by 22385730/99820247 = 0.2242604148234576
  
  rat: replaced 0.2326882315914051 by 25615030/110083049 = 0.2326882315914051
  
  rat: replaced 0.2412417784884371 by 14523232/60201977 = 0.2412417784884373
  
  rat: replaced 0.2499192001753251 by 11309023/45250717 = 0.2499192001753254
  
  rat: replaced 0.2587186289254649 by 7582961/29309683 = 0.2587186289254647
  
  rat: replaced 0.267638184811648 by 17912865/66929407 = 0.2676381848116479
  
  rat: replaced 0.2766759758940514 by 27538925/99534934 = 0.2766759758940514
  
  rat: replaced 0.2858300984094321 by 29258587/102363562 = 0.2858300984094321
  
  rat: replaced 0.2950986369614998 by 7877677/26695064 = 0.2950986369614997
  
  rat: replaced 0.304479664712457 by 14469542/47522195 = 0.304479664712457
  
  rat: replaced 0.3139712435756791 by 8375733/26676752 = 0.3139712435756797
  
  rat: replaced 0.3235714244095225 by 178371467/551258404 = 0.3235714244095225
  
  rat: replaced 0.3332782472122374 by 5743591/17233621 = 0.333278247212237
  
  rat: replaced 0.3430897413179662 by 15588245/45434891 = 0.3430897413179664
  
  rat: replaced 0.3530039255938071 by 6523425/18479752 = 0.3530039255938067
  
  rat: replaced 0.3630188086379282 by 51253958/141188161 = 0.3630188086379282
  
  rat: replaced 0.373132388978704 by 9370061/25111894 = 0.3731323889787047
  
  rat: replaced 0.3833426552748616 by 11820697/30835851 = 0.3833426552748617
  
  rat: replaced 0.393647586516613 by 9153768/23253713 = 0.3936475865166135
  
  rat: replaced 0.4040451522277552 by 16634707/41170416 = 0.404045152227755
  
  rat: replaced 0.4145333126687146 by 2088920/5039209 = 0.4145333126687145
  
  rat: replaced 0.4251100190405208 by 24667763/58026774 = 0.4251100190405209
  
  rat: replaced 0.4357732136896836 by 10448574/23977091 = 0.435773213689684
  
  rat: replaced 0.4465208303139576 by 8346266/18691773 = 0.4465208303139568
  
  rat: replaced 0.4573507941689697 by 20158688/44077081 = 0.4573507941689696
  
  rat: replaced 0.4682610222756929 by 12818601/27374905 = 0.4682610222756937
  
  rat: replaced 0.4792494236287415 by 13652513/28487281 = 0.4792494236287416
  
  rat: replaced 0.4903138994054704 by 35114711/71616797 = 0.4903138994054705
  
  rat: replaced 0.5014523431758559 by 15102855/30118226 = 0.5014523431758564
  
  rat: replaced 0.5126626411131362 by 31697340/61828847 = 0.5126626411131361
  
  rat: replaced 0.5239426722051925 by 27432767/52358337 = 0.5239426722051924
  
  rat: replaced 0.5352903084666492 by 6124470/11441399 = 0.5352903084666482
  
  rat: replaced 0.5467034151516694 by 41717397/76307182 = 0.5467034151516694
  
  rat: replaced 0.5581798509674292 by 7494380/13426461 = 0.5581798509674292
  
  rat: replaced 0.5697174682882435 by 14609183/25642856 = 0.5697174682882438
  
  rat: replaced 0.581314113370329 by 14367580/24715691 = 0.5813141133703282
  
  rat: replaced 0.5929676265671738 by 9820294/16561265 = 0.5929676265671735
  
  rat: replaced 0.6046758425455033 by 23593213/39017952 = 0.6046758425455031
  
  rat: replaced 0.6164365905018095 by 15720181/25501700 = 0.6164365905018097
  
  rat: replaced 0.6282476943794307 by 53974636/85912987 = 0.6282476943794306
  
  rat: replaced 0.640106973086155 by 20459615/31962806 = 0.6401069730861552
  
  rat: replaced 0.652012240712328 by 51645100/79208789 = 0.652012240712328
  
  rat: replaced 0.6639613067494411 by 12215999/18398661 = 0.6639613067494422
  
  rat: replaced 0.6759519763091814 by 18558734/27455699 = 0.6759519763091808
  
  rat: replaced 0.6879820503429186 by 23500536/34158647 = 0.687982050342919
  
  rat: replaced 0.7000493258616074 by 29992669/42843651 = 0.7000493258616078
  
  rat: replaced 0.7121515961560857 by 10685401/15004391 = 0.7121515961560853
  
  rat: replaced 0.7242866510177421 by 11795807/16286103 = 0.7242866510177419
  
  rat: replaced 0.7364522769595366 by 14940657/20287339 = 0.7364522769595362
  
  rat: replaced 0.7486462574373463 by 42508133/56779998 = 0.7486462574373461
  part: invalid index of list or matrix.
  #0: lineIntersection(g=[2,1,2],h=[-1,2,2])
  #1: projectToLine(a=[1,0],g=[-1,2,2])
   -- an error. To debug this try: debugmode(true);
  
  Error in:
   $projectToLine(A,lineThrough(B,C)) // proyeksi A pada BC ...
                                     ^
\end{euleroutput}
\begin{eulerprompt}
>$distance(A,Q) // jarak AQ
>cc &= circleThrough(A,B,C); $cc // (titik pusat dan jari-jari) lingkaran melalui A, B, C
\end{eulerprompt}
\begin{euleroutput}
  Maxima said:
  rat: replaced -4.98329175014009e-5 by -86001/1725786976 = -4.983291750140082e-5
  
  rat: replaced -1.986600267553235e-4 by -1133306/5704751069 = -1.986600267553234e-4
  
  rat: replaced -4.454664535081185e-4 by -474290/1064704191 = -4.454664535081181e-4
  
  rat: replaced -7.892275256562442e-4 by -1190199/1508055613 = -7.892275256562439e-4
  
  rat: replaced -0.001228908875712045 by -259907/211494119 = -0.001228908875712047
  
  rat: replaced -0.001763466544240408 by -5854594/3319934829 = -0.001763466544240408
  
  rat: replaced -0.002391847084253176 by -866601/362314550 = -0.002391847084253172
  
  rat: replaced -0.003112987666553255 by -5049204/1621980085 = -0.003112987666553255
  
  rat: replaced -0.00392581618601677 by -1241039/316122544 = -0.003925816186016774
  
  rat: replaced -0.004829251368802329 by -3015690/624463249 = -0.00482925136880233
  
  rat: replaced -0.005822202880477995 by -2532373/434951006 = -0.005822202880477991
  
  rat: replaced -0.006903571435053116 by -1331361/192851050 = -0.006903571435053115
  
  rat: replaced -0.008072248904906765 by -7953293/985263598 = -0.008072248904906766
  
  rat: replaced -0.009327118431599252 by -432515/46371771 = -0.009327118431599259
  
  rat: replaced -0.01066705453755698 by -2950074/276559381 = -0.01066705453755698
  
  rat: replaced -0.01209092323861904 by -1254816/103781653 = -0.01209092323861907
  
  rat: replaced -0.01359758215743526 by -1827823/134422648 = -0.01359758215743526
  
  rat: replaced -0.01518588063770274 by -9199276/605778237 = -0.01518588063770274
  
  rat: replaced -0.01685465985923026 by -2516580/149310637 = -0.01685465985923026
  
  rat: replaced -0.01860275295381958 by -2032371/109251088 = -0.01860275295381955
  
  rat: replaced -0.02042898512195129 by -1413911/69211025 = -0.02042898512195131
  
  rat: replaced -0.02233217375026381 by -3647892/163346929 = -0.02233217375026377
  
  rat: replaced -0.02431112852981362 by -1377268/56651751 = -0.02431112852981367
  
  rat: replaced -0.02636465157510504 by -2533336/96088355 = -0.02636465157510502
  
  rat: replaced -0.0284915375438782 by -9699307/340427644 = -0.02849153754387819
  
  rat: replaced -0.03069057375764189 by -7938451/258660886 = -0.03069057375764189
  
  rat: replaced -0.0329605403229406 by -2936449/89089832 = -0.03296054032294056
  
  rat: replaced -0.03530021025334285 by -5224432/148000025 = -0.03530021025334287
  
  rat: replaced -0.03770834959213837 by -2448749/64939172 = -0.03770834959213832
  
  rat: replaced -0.04018371753573358 by -2461511/61256428 = -0.04018371753573356
  
  rat: replaced -0.04272506655773012 by -13954421/326609696 = -0.04272506655773012
  
  rat: replaced -0.04533114253367693 by -2051558/45257143 = -0.04533114253367695
  
  rat: replaced -0.04800068486648146 by -16995415/354066094 = -0.04800068486648145
  
  rat: replaced -0.05073242661246818 by -3970295/78259513 = -0.05073242661246818
  
  rat: replaced -0.05352509460807248 by -3894269/72755948 = -0.05352509460807246
  
  rat: replaced -0.05637740959715515 by -11093364/196769665 = -0.05637740959715513
  
  rat: replaced -0.05928808635892763 by -3489209/58851773 = -0.05928808635892754
  
  rat: replaced -0.06225583383647254 by -3380435/54299088 = -0.06225583383647254
  
  rat: replaced -0.06527935526584844 by -7571267/115982564 = -0.06527935526584841
  
  rat: replaced -0.06835734830576551 by -8050241/117767017 = -0.06835734830576544
  
  rat: replaced -0.07148850516781785 by -5513427/77123266 = -0.07148850516781798
  
  rat: replaced -0.07467151274726203 by -2259975/30265558 = -0.07467151274726208
  
  rat: replaced -0.07790505275432569 by -657797/8443573 = -0.07790505275432569
  
  rat: replaced -0.08118780184603619 by -4832180/59518547 = -0.08118780184603633
  
  rat: replaced -0.08451843175855339 by -3076049/36395008 = -0.08451843175855327
  
  rat: replaced -0.08789560943999458 by -7150621/81353563 = -0.08789560943999465
  
  rat: replaced -0.0913179971837394 by -20067867/219758072 = -0.0913179971837394
  
  rat: replaced -0.09478425276219882 by -5487749/57897265 = -0.09478425276219869
  
  rat: replaced -0.09829302956103664 by -4406725/44832528 = -0.09829302956103658
  
  rat: replaced -0.1018429767138303 by -3912367/38415678 = -0.1018429767138302
  
  rat: replaced -0.1054327392371563 by -8451941/80164293 = -0.1054327392371564
  
  rat: replaced -0.1090609581660869 by -13126833/120362348 = -0.1090609581660869
  
  rat: replaced -0.112726270690086 by -2754747/24437489 = -0.112726270690086
  
  rat: replaced -0.116427310289289 by -22239618/191017193 = -0.116427310289289
  
  rat: replaced -0.1201627068711536 by -9494831/79016454 = -0.1201627068711537
  
  rat: replaced -0.1239310869074673 by -3190398/25743323 = -0.1239310869074672
  
  rat: replaced -0.1277310735717007 by -15999330/125257931 = -0.1277310735717006
  
  rat: replaced -0.1315612868766867 by -13929723/105880106 = -0.1315612868766867
  
  rat: replaced -0.1354203438126204 by -28035370/207024803 = -0.1354203438126204
  
  rat: replaced -0.1393068584853572 by -11590983/83204683 = -0.1393068584853571
  
  rat: replaced -0.1432194422550018 by -12738764/88945773 = -0.1432194422550018
  
  rat: replaced -0.1471567038747712 by -5246589/35653075 = -0.147156703874771
  
  rat: replaced -0.1511172496301179 by -4676629/30947023 = -0.1511172496301179
  
  rat: replaced -0.1550996834780995 by -15854305/102220099 = -0.1550996834780995
  
  rat: replaced -0.1591026071869839 by -9026555/56734174 = -0.159102607186984
  
  rat: replaced -0.1631246204760689 by -10435073/63969945 = -0.1631246204760689
  
  rat: replaced -0.1671643211557106 by -164873401/986295400 = -0.1671643211557106
  
  rat: replaced -0.1712203052675407 by -7017638/40986015 = -0.1712203052675406
  
  rat: replaced -0.1752911672248615 by -3184915/18169284 = -0.1752911672248615
  
  rat: replaced -0.1793754999532028 by -2646709/14755131 = -0.1793754999532027
  
  rat: replaced -0.1834718950310287 by -8392143/45740755 = -0.1834718950310287
  
  rat: replaced -0.1875789428305783 by -12888313/68708741 = -0.1875789428305781
  
  rat: replaced -0.1916952326588277 by -16014703/83542521 = -0.1916952326588277
  
  rat: replaced -0.1958193528985573 by -21279927/108671215 = -0.1958193528985574
  
  rat: replaced -0.1999498911495134 by -5994245/29978736 = -0.1999498911495134
  
  rat: replaced -0.2040854343696463 by -17847769/87452439 = -0.2040854343696464
  
  rat: replaced -0.2082245690164135 by -5203892/24991729 = -0.2082245690164134
  
  rat: replaced -0.2123658811881329 by -20393053/96027916 = -0.2123658811881328
  
  rat: replaced -0.2165079567653719 by -8489188/39209589 = -0.2165079567653719
  
  rat: replaced -0.2206493815523576 by -14881929/67446049 = -0.2206493815523575
  
  rat: replaced -0.2247887414183958 by -11437558/50881365 = -0.2247887414183955
  
  rat: replaced -0.2289246224392826 by -17547464/76651711 = -0.2289246224392825
  
  rat: replaced -0.2330556110386959 by -11148764/47837355 = -0.2330556110386956
  
  rat: replaced -0.2371802941295513 by -11052217/46598378 = -0.237180294129551
  
  rat: replaced -0.2412972592553108 by -36037383/149348497 = -0.2412972592553108
  
  rat: replaced -0.2454050947312253 by -4652365/18957899 = -0.2454050947312252
  
  rat: replaced -0.2495023897855041 by -6175634/24751803 = -0.2495023897855037
  
  rat: replaced -0.2535877347003893 by -11299519/44558618 = -0.2535877347003895
  
  rat: replaced -0.2576597209531272 by -6871877/26670358 = -0.2576597209531271
  
  rat: replaced -0.2617169413568191 by -2245730/8580759 = -0.2617169413568194
  
  rat: replaced -0.2657579902011391 by -10500993/39513367 = -0.2657579902011388
  
  rat: replaced -0.2697814633929034 by -21050552/78028163 = -0.2697814633929034
  
  rat: replaced -0.2737859585964791 by -1510231/5516101 = -0.2737859585964796
  
  rat: replaced -0.2777700753740163 by -9819093/35349715 = -0.2777700753740164
  
  rat: replaced -0.2817324153254904 by -10837378/38466919 = -0.2817324153254905
  
  rat: replaced -0.2856715822285418 by -17041418/59653879 = -0.2856715822285421
  
  rat: replaced -0.289586182178096 by -721506/2491507 = -0.2895861821780955
  
  rat: replaced -0.2934748237257534 by -11793110/40184401 = -0.2934748237257537
  
  rat: replaced -0.2973361180189332 by -15390047/51759763 = -0.2973361180189329
  
  rat: replaced -1.00165832502809e-4 by -535089/5342031176 = -1.00165832502809e-4
  
  rat: replaced -4.013199735114076e-4 by -779636/1942679287 = -4.013199735114075e-4
  
  rat: replaced -9.044322995292522e-4 by -524677/580117495 = -9.044322995292531e-4
  
  rat: replaced -0.001610452491410008 by -2370713/1472078818 = -0.001610452491410008
  
  rat: replaced -0.002520309939389107 by -92559/36725245 = -0.002520309939389104
  
  rat: replaced -0.003634913650147023 by -1950438/536584411 = -0.003634913650147022
  
  rat: replaced -0.004955152155908069 by -1126716/227382725 = -0.004955152155908062
  
  rat: replaced -0.006481893425588428 by -972955/150103517 = -0.006481893425588422
  
  rat: replaced -0.00821598477800041 by -318177/38726581 = -0.008215984778000413
  
  rat: replaced -0.01015825279712021 by -1696171/166974679 = -0.01015825279712021
  
  rat: replaced -0.01230950324943156 by -3071603/249531028 = -0.01230950324943157
  
  rat: replaced -0.01467052100334813 by -1108148/75535695 = -0.01467052100334815
  
  rat: replaced -0.01724206995072897 by -3656561/212072043 = -0.01724206995072896
  
  rat: replaced -0.02002489293048906 by -7918949/395455248 = -0.02002489293048907
  
  rat: replaced -0.02301971165431629 by -747397/32467696 = -0.02301971165431634
  
  rat: replaced -0.02622722663450017 by -4067487/155086432 = -0.02622722663450017
  
  rat: replaced -0.02964811711388246 by -2777477/93681396 = -0.02964811711388246
  
  rat: replaced -0.03328304099793292 by -563339/16925707 = -0.03328304099793291
  
  rat: replaced -0.03713263478895881 by -8128846/218913795 = -0.03713263478895882
  
  rat: replaced -0.04119751352245554 by -4789499/116256992 = -0.04119751352245549
  
  rat: replaced -0.0454782707056039 by -4689976/103125645 = -0.04547827070560383
  
  rat: replaced -0.04997547825791965 by -3780197/75641037 = -0.0499754782579197
  
  rat: replaced -0.05468968645406205 by -7762811/141942869 = -0.05468968645406202
  
  rat: replaced -0.05962142386880631 by -931586/15625021 = -0.05962142386880632
  
  rat: replaced -0.0647711973241876 by -3587937/55394020 = -0.06477119732418771
  
  rat: replaced -0.07013949183881846 by -8850563/126185160 = -0.07013949183881844
  
  rat: replaced -0.07572677057938781 by -6606579/87242318 = -0.07572677057938786
  
  rat: replaced -0.08153347481434448 by -9568762/117359919 = -0.0815334748143444
  
  rat: replaced -0.08756002386977008 by -10007103/114288491 = -0.08756002386977005
  
  rat: replaced -0.09380681508744848 by -4116683/43884690 = -0.0938068150874485
  
  rat: replaced -0.1002742237851297 by -7402097/73818542 = -0.1002742237851298
  
  rat: replaced -0.1069626032190006 by -4704154/43979427 = -0.1069626032190006
  
  rat: replaced -0.1138722845483578 by -6905284/60640603 = -0.1138722845483578
  
  rat: replaced -0.1210035768024932 by -4945013/40866668 = -0.1210035768024934
  
  rat: replaced -0.1283567668497909 by -174125156/1356571689 = -0.1283567668497909
  
  rat: replaced -0.1359321193690404 by -1669939/12285095 = -0.1359321193690403
  
  rat: replaced -0.1437298768229693 by -2883920/20064861 = -0.1437298768229693
  
  rat: replaced -0.1517502594339971 by -13646521/89927497 = -0.1517502594339971
  
  rat: replaced -0.1599934651622126 by -8139181/50871959 = -0.1599934651622124
  
  rat: replaced -0.1684596696855795 by -7231383/42926494 = -0.1684596696855793
  
  rat: replaced -0.1771490263823671 by -8210161/46346069 = -0.177149026382367
  
  rat: replaced -0.1860616663158136 by -37562009/201879354 = -0.1860616663158136
  
  rat: replaced -0.1951976982210191 by -5455884/27950555 = -0.1951976982210192
  
  rat: replaced -0.2045572084940737 by -22523003/110106132 = -0.2045572084940737
  
  rat: replaced -0.2141402611834163 by -32619650/152328431 = -0.2141402611834163
  
  rat: replaced -0.2239468979834299 by -13386607/59775809 = -0.2239468979834301
  
  rat: replaced -0.2339771382302741 by -10271620/43900101 = -0.2339771382302742
  
  rat: replaced -0.244230978899949 by -8014993/32817266 = -0.2442309788999486
  
  rat: replaced -0.2547083946085993 by -6543653/25690763 = -0.2547083946085992
  
  rat: replaced -0.2654093376150518 by -8928803/33641631 = -0.265409337615052
  
  rat: replaced -0.2763337378255902 by -18341761/66375395 = -0.2763337378255903
  
  rat: replaced -0.2874815028009638 by -9829665/34192339 = -0.2874815028009637
  
  rat: replaced -0.2988525177656313 by -54659344/182897385 = -0.2988525177656313
  
  rat: replaced -0.3104466456192388 by -11128869/35847928 = -0.3104466456192391
  
  rat: replaced -0.3222637269503297 by -3000119/9309515 = -0.3222637269503298
  
  rat: replaced -0.3343035800522847 by -17486063/52305940 = -0.3343035800522847
  
  rat: replaced -0.3465660009414936 by -56802607/163901268 = -0.3465660009414936
  
  rat: replaced -0.3590507633777529 by -11187457/31158427 = -0.3590507633777533
  
  rat: replaced -0.3717576188868896 by -26309122/70769557 = -0.3717576188868895
  
  rat: replaced -0.3846862967856085 by -11423435/29695456 = -0.3846862967856092
  
  rat: replaced -0.3978365042085601 by -16989224/42704035 = -0.3978365042085601
  
  rat: replaced -0.4112079261376275 by -11659135/28353381 = -0.4112079261376271
  
  rat: replaced -0.4248002254334272 by -67609726/159156521 = -0.4248002254334273
  
  rat: replaced -0.4386130428690231 by -22452660/51190133 = -0.4386130428690232
  
  rat: replaced -0.4526459971658492 by -5841603/12905456 = -0.4526459971658499
  
  rat: replaced -0.4668986850318365 by -9748690/20879669 = -0.4668986850318365
  
  rat: replaced -0.4813706812017424 by -28304029/58798822 = -0.4813706812017424
  
  rat: replaced -0.4960615384796762 by -18513661/37321299 = -0.4960615384796762
  
  rat: replaced -0.5109707877838195 by -12122645/23724732 = -0.5109707877838199
  
  rat: replaced -0.5260979381933327 by -2102905/3997174 = -0.5260979381933336
  
  rat: replaced -0.5414424769974477 by -13834851/25551839 = -0.5414424769974482
  
  rat: replaced -0.5570038697467375 by -46945257/84281743 = -0.5570038697467374
  
  rat: replaced -0.572781560306562 by -17582077/30695955 = -0.5727815603065616
  
  rat: replaced -0.5887749709126798 by -30776397/52271918 = -0.5887749709126802
  
  rat: replaced -0.6049835022290249 by -19188269/31717012 = -0.6049835022290246
  
  rat: replaced -0.6214065334076391 by -11602067/18670655 = -0.6214065334076389
  
  rat: replaced -0.6380434221507573 by -6649723/10422054 = -0.6380434221507584
  
  rat: replaced -0.6548935047750358 by -43869993/66987980 = -0.6548935047750357
  
  rat: replaced -0.6719560962779209 by -31668213/47128396 = -0.6719560962779213
  
  rat: replaced -0.6892304904061473 by -31252599/45344191 = -0.689230490406147
  
  rat: replaced -0.7067159597263644 by -7239052/10243227 = -0.7067159597263636
  
  rat: replaced -0.724411755697878 by -65966965/91062803 = -0.7244117556978781
  
  rat: replaced -0.742317108747504 by -29643877/39934250 = -0.7423171087475037
  
  rat: replaced -0.7604312283465253 by -19521554/25671689 = -0.760431228346526
  
  rat: replaced -0.778753303089744 by -17717453/22751047 = -0.7787533030897436
  
  rat: replaced -0.7972825007766203 by -6544613/8208650 = -0.7972825007766198
  
  rat: replaced -0.8160179684944936 by -15744063/19293770 = -0.8160179684944933
  
  rat: replaced -0.8349588327038714 by -31965589/38284030 = -0.8349588327038716
  
  rat: replaced -0.8541041993257835 by -22076179/25847173 = -0.8541041993257832
  
  rat: replaced -0.8734531538311887 by -59286729/67876255 = -0.8734531538311888
  
  rat: replaced -0.8930047613324276 by -6137127/6872446 = -0.8930047613324281
  
  rat: replaced -0.9127580666767096 by -13137137/14392792 = -0.9127580666767088
  
  rat: replaced -0.9327120945416275 by -15972523/17124816 = -0.932712094541629
  
  rat: replaced -0.9528658495326905 by -44894507/47115244 = -0.9528658495326905
  
  rat: replaced -0.9732183162828605 by -25482581/26183828 = -0.9732183162828598
  
  rat: replaced -0.9937684595540898 by -23211595/23357146 = -0.9937684595540911
  
  rat: replaced -1.014515224340843 by -33401253/32923363 = -1.014515224340843
  
  rat: replaced -1.035457535975596 by -9211102/8895683 = -1.035457535975596
  
  rat: replaced -1.056594300236306 by -24469996/23159311 = -1.056594300236307
  part: invalid index of list or matrix.
  #0: lineIntersection(g=[1,-1,0],h=[-1,-2,-7/2])
  #1: circleThrough(a=[1,0],b=[0,1],c=[2,2])
   -- an error. To debug this try: debugmode(true);
  
  Error in:
  cc &= circleThrough(A,B,C); $cc // (titik pusat dan jari-jari) ...
                            ^
\end{euleroutput}
\begin{eulerprompt}
>r&=getCircleRadius(cc); $r , $float(r) // tampilkan nilai jari-jari
>$computeAngle(A,C,B) // nilai <ACB
>$solve(getLineEquation(angleBisector(A,C,B),x,y),y)[1] // persamaan garis bagi <ACB
\end{eulerprompt}
\begin{euleroutput}
  Maxima said:
  solve: all variables must not be numbers.
   -- an error. To debug this try: debugmode(true);
  
  Error in:
  ... (getLineEquation(angleBisector(A,C,B),x,y),y)[1] // persamaan  ...
                                                       ^
\end{euleroutput}
\begin{eulerprompt}
>P &= lineIntersection(angleBisector(A,C,B),angleBisector(C,B,A)); $P // titik potong 2 garis bagi sudut
\end{eulerprompt}
\begin{euleroutput}
  Maxima said:
  rat: replaced -4.98329175014009e-5 by -86001/1725786976 = -4.983291750140082e-5
  
  rat: replaced -1.986600267553235e-4 by -1133306/5704751069 = -1.986600267553234e-4
  
  rat: replaced -4.454664535081185e-4 by -474290/1064704191 = -4.454664535081181e-4
  
  rat: replaced -7.892275256562442e-4 by -1190199/1508055613 = -7.892275256562439e-4
  
  rat: replaced -0.001228908875712045 by -259907/211494119 = -0.001228908875712047
  
  rat: replaced -0.001763466544240408 by -5854594/3319934829 = -0.001763466544240408
  
  rat: replaced -0.002391847084253176 by -866601/362314550 = -0.002391847084253172
  
  rat: replaced -0.003112987666553255 by -5049204/1621980085 = -0.003112987666553255
  
  rat: replaced -0.00392581618601677 by -1241039/316122544 = -0.003925816186016774
  
  rat: replaced -0.004829251368802329 by -3015690/624463249 = -0.00482925136880233
  
  rat: replaced -0.005822202880477995 by -2532373/434951006 = -0.005822202880477991
  
  rat: replaced -0.006903571435053116 by -1331361/192851050 = -0.006903571435053115
  
  rat: replaced -0.008072248904906765 by -7953293/985263598 = -0.008072248904906766
  
  rat: replaced -0.009327118431599252 by -432515/46371771 = -0.009327118431599259
  
  rat: replaced -0.01066705453755698 by -2950074/276559381 = -0.01066705453755698
  
  rat: replaced -0.01209092323861904 by -1254816/103781653 = -0.01209092323861907
  
  rat: replaced -0.01359758215743526 by -1827823/134422648 = -0.01359758215743526
  
  rat: replaced -0.01518588063770274 by -9199276/605778237 = -0.01518588063770274
  
  rat: replaced -0.01685465985923026 by -2516580/149310637 = -0.01685465985923026
  
  rat: replaced -0.01860275295381958 by -2032371/109251088 = -0.01860275295381955
  
  rat: replaced -0.02042898512195129 by -1413911/69211025 = -0.02042898512195131
  
  rat: replaced -0.02233217375026381 by -3647892/163346929 = -0.02233217375026377
  
  rat: replaced -0.02431112852981362 by -1377268/56651751 = -0.02431112852981367
  
  rat: replaced -0.02636465157510504 by -2533336/96088355 = -0.02636465157510502
  
  rat: replaced -0.0284915375438782 by -9699307/340427644 = -0.02849153754387819
  
  rat: replaced -0.03069057375764189 by -7938451/258660886 = -0.03069057375764189
  
  rat: replaced -0.0329605403229406 by -2936449/89089832 = -0.03296054032294056
  
  rat: replaced -0.03530021025334285 by -5224432/148000025 = -0.03530021025334287
  
  rat: replaced -0.03770834959213837 by -2448749/64939172 = -0.03770834959213832
  
  rat: replaced -0.04018371753573358 by -2461511/61256428 = -0.04018371753573356
  
  rat: replaced -0.04272506655773012 by -13954421/326609696 = -0.04272506655773012
  
  rat: replaced -0.04533114253367693 by -2051558/45257143 = -0.04533114253367695
  
  rat: replaced -0.04800068486648146 by -16995415/354066094 = -0.04800068486648145
  
  rat: replaced -0.05073242661246818 by -3970295/78259513 = -0.05073242661246818
  
  rat: replaced -0.05352509460807248 by -3894269/72755948 = -0.05352509460807246
  
  rat: replaced -0.05637740959715515 by -11093364/196769665 = -0.05637740959715513
  
  rat: replaced -0.05928808635892763 by -3489209/58851773 = -0.05928808635892754
  
  rat: replaced -0.06225583383647254 by -3380435/54299088 = -0.06225583383647254
  
  rat: replaced -0.06527935526584844 by -7571267/115982564 = -0.06527935526584841
  
  rat: replaced -0.06835734830576551 by -8050241/117767017 = -0.06835734830576544
  
  rat: replaced -0.07148850516781785 by -5513427/77123266 = -0.07148850516781798
  
  rat: replaced -0.07467151274726203 by -2259975/30265558 = -0.07467151274726208
  
  rat: replaced -0.07790505275432569 by -657797/8443573 = -0.07790505275432569
  
  rat: replaced -0.08118780184603619 by -4832180/59518547 = -0.08118780184603633
  
  rat: replaced -0.08451843175855339 by -3076049/36395008 = -0.08451843175855327
  
  rat: replaced -0.08789560943999458 by -7150621/81353563 = -0.08789560943999465
  
  rat: replaced -0.0913179971837394 by -20067867/219758072 = -0.0913179971837394
  
  rat: replaced -0.09478425276219882 by -5487749/57897265 = -0.09478425276219869
  
  rat: replaced -0.09829302956103664 by -4406725/44832528 = -0.09829302956103658
  
  rat: replaced -0.1018429767138303 by -3912367/38415678 = -0.1018429767138302
  
  rat: replaced -0.1054327392371563 by -8451941/80164293 = -0.1054327392371564
  
  rat: replaced -0.1090609581660869 by -13126833/120362348 = -0.1090609581660869
  
  rat: replaced -0.112726270690086 by -2754747/24437489 = -0.112726270690086
  
  rat: replaced -0.116427310289289 by -22239618/191017193 = -0.116427310289289
  
  rat: replaced -0.1201627068711536 by -9494831/79016454 = -0.1201627068711537
  
  rat: replaced -0.1239310869074673 by -3190398/25743323 = -0.1239310869074672
  
  rat: replaced -0.1277310735717007 by -15999330/125257931 = -0.1277310735717006
  
  rat: replaced -0.1315612868766867 by -13929723/105880106 = -0.1315612868766867
  
  rat: replaced -0.1354203438126204 by -28035370/207024803 = -0.1354203438126204
  
  rat: replaced -0.1393068584853572 by -11590983/83204683 = -0.1393068584853571
  
  rat: replaced -0.1432194422550018 by -12738764/88945773 = -0.1432194422550018
  
  rat: replaced -0.1471567038747712 by -5246589/35653075 = -0.147156703874771
  
  rat: replaced -0.1511172496301179 by -4676629/30947023 = -0.1511172496301179
  
  rat: replaced -0.1550996834780995 by -15854305/102220099 = -0.1550996834780995
  
  rat: replaced -0.1591026071869839 by -9026555/56734174 = -0.159102607186984
  
  rat: replaced -0.1631246204760689 by -10435073/63969945 = -0.1631246204760689
  
  rat: replaced -0.1671643211557106 by -164873401/986295400 = -0.1671643211557106
  
  rat: replaced -0.1712203052675407 by -7017638/40986015 = -0.1712203052675406
  
  rat: replaced -0.1752911672248615 by -3184915/18169284 = -0.1752911672248615
  
  rat: replaced -0.1793754999532028 by -2646709/14755131 = -0.1793754999532027
  
  rat: replaced -0.1834718950310287 by -8392143/45740755 = -0.1834718950310287
  
  rat: replaced -0.1875789428305783 by -12888313/68708741 = -0.1875789428305781
  
  rat: replaced -0.1916952326588277 by -16014703/83542521 = -0.1916952326588277
  
  rat: replaced -0.1958193528985573 by -21279927/108671215 = -0.1958193528985574
  
  rat: replaced -0.1999498911495134 by -5994245/29978736 = -0.1999498911495134
  
  rat: replaced -0.2040854343696463 by -17847769/87452439 = -0.2040854343696464
  
  rat: replaced -0.2082245690164135 by -5203892/24991729 = -0.2082245690164134
  
  rat: replaced -0.2123658811881329 by -20393053/96027916 = -0.2123658811881328
  
  rat: replaced -0.2165079567653719 by -8489188/39209589 = -0.2165079567653719
  
  rat: replaced -0.2206493815523576 by -14881929/67446049 = -0.2206493815523575
  
  rat: replaced -0.2247887414183958 by -11437558/50881365 = -0.2247887414183955
  
  rat: replaced -0.2289246224392826 by -17547464/76651711 = -0.2289246224392825
  
  rat: replaced -0.2330556110386959 by -11148764/47837355 = -0.2330556110386956
  
  rat: replaced -0.2371802941295513 by -11052217/46598378 = -0.237180294129551
  
  rat: replaced -0.2412972592553108 by -36037383/149348497 = -0.2412972592553108
  
  rat: replaced -0.2454050947312253 by -4652365/18957899 = -0.2454050947312252
  
  rat: replaced -0.2495023897855041 by -6175634/24751803 = -0.2495023897855037
  
  rat: replaced -0.2535877347003893 by -11299519/44558618 = -0.2535877347003895
  
  rat: replaced -0.2576597209531272 by -6871877/26670358 = -0.2576597209531271
  
  rat: replaced -0.2617169413568191 by -2245730/8580759 = -0.2617169413568194
  
  rat: replaced -0.2657579902011391 by -10500993/39513367 = -0.2657579902011388
  
  rat: replaced -0.2697814633929034 by -21050552/78028163 = -0.2697814633929034
  
  rat: replaced -0.2737859585964791 by -1510231/5516101 = -0.2737859585964796
  
  rat: replaced -0.2777700753740163 by -9819093/35349715 = -0.2777700753740164
  
  rat: replaced -0.2817324153254904 by -10837378/38466919 = -0.2817324153254905
  
  rat: replaced -0.2856715822285418 by -17041418/59653879 = -0.2856715822285421
  
  rat: replaced -0.289586182178096 by -721506/2491507 = -0.2895861821780955
  
  rat: replaced -0.2934748237257534 by -11793110/40184401 = -0.2934748237257537
  
  rat: replaced -0.2973361180189332 by -15390047/51759763 = -0.2973361180189329
  
  rat: replaced 1.66665833335744e-7 by 15819/94914474571 = 1.66665833335744e-7
  
  rat: replaced 4.999958333473664e-5 by 201389/4027813565 = 4.99995833347366e-5
  
  rat: replaced 1.33330666692022e-6 by 31771/23828726570 = 1.333306666920221e-6
  
  rat: replaced 1.999933334222437e-4 by 200030/1000183339 = 1.999933334222437e-4
  
  rat: replaced 4.499797504338432e-6 by 24036/5341573699 = 4.499797504338431e-6
  
  rat: replaced 4.499662510124569e-4 by 1162901/2584418270 = 4.499662510124571e-4
  
  rat: replaced 1.066581336583994e-5 by 58861/5518660226 = 1.066581336583993e-5
  
  rat: replaced 7.998933390220841e-4 by 1137431/1421978337 = 7.998933390220838e-4
  
  rat: replaced 2.083072932167196e-5 by 35635/1710693824 = 2.0830729321672e-5
  
  rat: replaced 0.001249739605033717 by 567943/454449069 = 0.001249739605033716
  
  rat: replaced 3.599352055540239e-5 by 98277/2730408098 = 3.599352055540234e-5
  
  rat: replaced 0.00179946006479581 by 479561/266502719 = 0.001799460064795812
  
  rat: replaced 5.71526624672386e-5 by 51154/895041417 = 5.715266246723866e-5
  
  rat: replaced 0.002448999746720415 by 1946227/794702818 = 0.002448999746720415
  
  rat: replaced 8.530603082730626e-5 by 121691/1426522824 = 8.530603082730627e-5
  
  rat: replaced 0.003198293697380561 by 2986741/933854512 = 0.003198293697380562
  
  rat: replaced 1.214508019889565e-4 by 158455/1304684674 = 1.214508019889563e-4
  
  rat: replaced 0.004047266988005727 by 2125334/525128193 = 0.004047266988005727
  
  rat: replaced 1.665833531718508e-4 by 142521/855553675 = 1.66583353171851e-4
  
  rat: replaced 0.004995834721974179 by 1957223/391770967 = 0.004995834721974179
  
  rat: replaced 2.216991628251896e-4 by 179571/809975995 = 2.216991628251896e-4
  
  rat: replaced 0.006043902043303184 by 1800665/297930871 = 0.006043902043303193
  
  rat: replaced 2.877927110806339e-4 by 1167733/4057548906 = 2.877927110806339e-4
  
  rat: replaced 0.00719136414613375 by 2476362/344352191 = 0.007191364146133747
  
  rat: replaced 3.658573803051457e-4 by 386279/1055818526 = 3.658573803051454e-4
  
  rat: replaced 0.00843810628521191 by 2079855/246483622 = 0.008438106285211924
  
  rat: replaced 4.5688535576352e-4 by 262978/575588595 = 4.568853557635206e-4
  
  rat: replaced 0.009784003787362772 by 1752551/179124113 = 0.009784003787362787
  
  rat: replaced 5.618675264007778e-4 by 150595/268025812 = 5.618675264007782e-4
  
  rat: replaced 0.01122892206395776 by 5450241/485375263 = 0.01122892206395776
  
  rat: replaced 6.817933857540259e-4 by 192316/282073725 = 6.817933857540258e-4
  
  rat: replaced 0.01277271662437307 by 3258991/255152533 = 0.01277271662437308
  
  rat: replaced 8.176509330039827e-4 by 105841/129445214 = 8.176509330039812e-4
  
  rat: replaced 0.01441523309043924 by 2330472/161667313 = 0.01441523309043925
  
  rat: replaced 9.704265741758145e-4 by 651321/671169790 = 9.704265741758132e-4
  
  rat: replaced 0.01615630721187855 by 19391318/1200232067 = 0.01615630721187855
  
  rat: replaced 0.001141105023499428 by 1259907/1104111343 = 0.001141105023499428
  
  rat: replaced 0.01799576488272969 by 4765614/264818641 = 0.01799576488272969
  
  rat: replaced 0.001330669204938795 by 1231154/925214167 = 0.001330669204938796
  
  rat: replaced 0.01993342215875837 by 2504519/125644206 = 0.01993342215875836
  
  rat: replaced 0.001540100153900437 by 276884/179783113 = 0.001540100153900439
  
  rat: replaced 0.02196908527585173 by 1298306/59096953 = 0.0219690852758517
  
  rat: replaced 0.001770376919130678 by 644389/363984072 = 0.001770376919130681
  
  rat: replaced 0.02410255066939448 by 2001286/83032125 = 0.02410255066939453
  
  rat: replaced 0.002022476464811601 by 1271955/628909667 = 0.002022476464811599
  
  rat: replaced 0.02633360499462523 by 2978115/113091808 = 0.02633360499462525
  
  rat: replaced 0.002297373572865413 by 1020913/444382669 = 0.002297373572865417
  
  rat: replaced 0.02866202514797045 by 1770713/61779061 = 0.02866202514797044
  
  rat: replaced 0.002596040745477063 by 1097643/422814242 = 0.002596040745477065
  
  rat: replaced 0.03108757828935527 by 5034207/161936287 = 0.03108757828935525
  
  rat: replaced 0.002919448107844891 by 906221/310408326 = 0.002919448107844891
  
  rat: replaced 0.03361002186548678 by 4553215/135471944 = 0.03361002186548678
  
  rat: replaced 0.003268563311168871 by 1379071/421919623 = 0.003268563311168867
  
  rat: replaced 0.03622910363410947 by 3082649/85087642 = 0.0362291036341094
  
  rat: replaced 0.003644351435886262 by 5966577/1637212301 = 0.003644351435886261
  
  rat: replaced 0.03894456168922911 by 4913415/126164342 = 0.03894456168922911
  
  rat: replaced 0.004047774895164447 by 572425/141417202 = 0.004047774895164451
  
  rat: replaced 0.04175612448730281 by 1734727/41544253 = 0.04175612448730273
  
  rat: replaced 0.004479793338660443 by 2952779/659132861 = 0.004479793338660444
  
  rat: replaced 0.04466351087439402 by 4691119/105032473 = 0.04466351087439405
  
  rat: replaced 0.0049413635565565 by 2524919/510976165 = 0.004941363556556498
  
  rat: replaced 0.04766643011428662 by 3536207/74186529 = 0.04766643011428665
  
  rat: replaced 0.005433439383882244 by 1361584/250593391 = 0.005433439383882235
  
  rat: replaced 0.05076458191755917 by 7710025/151878036 = 0.05076458191755916
  
  rat: replaced 0.005956971605131645 by 1447422/242979503 = 0.005956971605131648
  
  rat: replaced 0.0539576564716131 by 3377975/62604183 = 0.05395765647161309
  
  rat: replaced 0.006512907859185624 by 3695063/567344584 = 0.006512907859185626
  
  rat: replaced 0.05724533447165381 by 2560865/44734912 = 0.05724533447165382
  
  rat: replaced 0.007102192544548636 by 1363981/192050693 = 0.007102192544548642
  
  rat: replaced 0.06062728715262111 by 8274761/136485754 = 0.06062728715262107
  
  rat: replaced 0.007725766724910044 by 1464384/189545459 = 0.007725766724910038
  
  rat: replaced 0.06410317632206519 by 5287663/82486755 = 0.06410317632206528
  
  rat: replaced 0.00838456803503801 by 1113589/132814117 = 0.008384568035038023
  
  rat: replaced 0.06767265439396564 by 2921400/43169579 = 0.06767265439396572
  
  rat: replaced 0.009079530587017326 by 433906/47789475 = 0.00907953058701733
  
  rat: replaced 0.07133536442348987 by 7236103/101437808 = 0.07133536442348991
  
  rat: replaced 0.009811584876838586 by 1363090/138926587 = 0.009811584876838586
  
  rat: replaced 0.07509094014268702 by 9209133/122639735 = 0.07509094014268704
  
  rat: replaced 0.0105816576913495 by 1163729/109976058 = 0.01058165769134951
  
  rat: replaced 0.07893900599711501 by 5197067/65836489 = 0.07893900599711506
  
  rat: replaced 0.01139067201557714 by 13426050/1178688139 = 0.01139067201557714
  
  rat: replaced 0.08287917718339499 by 11217158/135343501 = 0.082879177183395
  
  rat: replaced 0.01223954694042984 by 2283101/186534764 = 0.01223954694042983
  
  rat: replaced 0.08691105968769186 by 5213115/59982182 = 0.08691105968769192
  
  rat: replaced 0.01312919757078923 by 3499615/266552086 = 0.01312919757078922
  
  rat: replaced 0.09103425032511492 by 5893225/64736349 = 0.09103425032511488
  
  rat: replaced 0.01406053493400045 by 2280713/162206702 = 0.01406053493400045
  
  rat: replaced 0.09524833678003664 by 9601787/100807923 = 0.09524833678003662
  
  rat: replaced 0.01503446588876983 by 200490/13335359 = 0.01503446588876985
  
  rat: replaced 0.09955289764732322 by 5687088/57126293 = 0.09955289764732328
  
  rat: replaced 0.01605189303448024 by 951971/59305840 = 0.01605189303448025
  
  rat: replaced 0.1039475024744748 by 10260011/98703776 = 0.1039475024744747
  
  rat: replaced 0.01711371462093175 by 9432386/551159477 = 0.01711371462093176
  
  rat: replaced 0.1084317118046711 by 14939691/137779721 = 0.1084317118046712
  
  rat: replaced 0.01822082445851714 by 2559788/140486947 = 0.01822082445851713
  
  rat: replaced 0.113005077220716 by 8478529/75027859 = 0.1130050772207161
  
  rat: replaced 0.01937411182884202 by 2983799/154009589 = 0.01937411182884203
  
  rat: replaced 0.1176671413898787 by 7123715/60541243 = 0.1176671413898786
  
  rat: replaced 0.02057446139579705 by 7167743/348380590 = 0.02057446139579705
  
  rat: replaced 0.1224174381096274 by 12172179/99431741 = 0.1224174381096274
  
  rat: replaced 0.02182275311709253 by 7415562/339808729 = 0.02182275311709253
  
  rat: replaced 0.1272554923542488 by 7277933/57191504 = 0.127255492354249
  
  rat: replaced 0.02311986215626333 by 2988661/129268115 = 0.02311986215626336
  
  rat: replaced 0.1321808203223502 by 3633064/27485561 = 0.1321808203223503
  
  rat: replaced 0.02446665879515308 by 1991976/81415939 = 0.02446665879515312
  
  rat: replaced 0.1371929294852391 by 56235017/409897341 = 0.1371929294852391
  
  rat: replaced 0.02586400834688696 by 5000736/193347293 = 0.02586400834688697
  
  rat: replaced 0.1422913186361759 by 9349741/65708443 = 0.1422913186361759
  
  rat: replaced 0.02731277106934082 by 858413/31428997 = 0.02731277106934084
  
  rat: replaced 0.1474754779404944 by 1549881/10509415 = 0.1474754779404943
  
  rat: replaced 0.02881380207911666 by 3754753/130310918 = 0.02881380207911666
  
  rat: replaced 0.152744888986584 by 5264425/34465474 = 0.1527448889865841
  
  rat: replaced 0.03036795126603076 by 4118329/135614318 = 0.03036795126603077
  
  rat: replaced 0.1580990248377314 by 5442776/34426373 = 0.1580990248377312
  
  rat: replaced 0.03197606320812652 by 3497683/109384416 = 0.03197606320812647
  
  rat: replaced 0.1635373500848132 by 12328488/75386375 = 0.1635373500848131
  
  rat: replaced 0.0336389770872163 by 3971799/118071337 = 0.03363897708721635
  
  rat: replaced 0.1690593208998367 by 20896917/123607009 = 0.1690593208998367
  
  rat: replaced 0.03535752660496472 by 1815732/51353479 = 0.03535752660496478
  
  rat: replaced 0.1746643850903219 by 2841592/16268869 = 0.1746643850903219
  
  rat: replaced 0.03713253989951881 by 3333721/89778965 = 0.03713253989951878
  
  rat: replaced 0.1803519821545206 by 4461007/24735004 = 0.1803519821545208
  
  rat: replaced 0.03896483946269502 by 8785771/225479461 = 0.03896483946269501
  
  rat: replaced 0.1861215433374662 by 4381209/23539505 = 0.1861215433374661
  
  rat: replaced 0.0408552420577305 by 3189084/78058135 = 0.04085524205773043
  
  rat: replaced 0.1919724916878484 by 72809759/379271834 = 0.1919724916878484
  
  rat: replaced 0.04280455863760801 by 7646593/178639688 = 0.04280455863760801
  
  rat: replaced 0.1979042421157076 by 26318167/132984350 = 0.1979042421157076
  
  rat: replaced 0.04481359426396048 by 20610430/459914683 = 0.04481359426396048
  
  rat: replaced 0.2039162014509444 by 8519416/41779005 = 0.2039162014509441
  
  rat: replaced 0.04688314802656623 by 3439140/73355569 = 0.04688314802656633
  
  rat: replaced 0.2100077685026351 by 50962787/242670961 = 0.2100077685026351
  
  rat: replaced 0.04901401296344043 by 4006732/81746663 = 0.04901401296344048
  
  rat: replaced 0.216178334119151 by 1347531/6233423 = 0.2161783341191509
  
  rat: replaced 0.05120697598153157 by 4148974/81023609 = 0.0512069759815315
  
  rat: replaced 0.2224272812490723 by 23234851/104460437 = 0.2224272812490723
  
  rat: replaced 0.05346281777803219 by 11998448/224426031 = 0.05346281777803218
  
  rat: replaced 0.2287539850028937 by 8185268/35781969 = 0.2287539850028935
  
  rat: replaced 0.05578231276230905 by 1398019/25062048 = 0.05578231276230897
  
  rat: replaced 0.2351578127155118 by 12642104/53760085 = 0.2351578127155119
  
  rat: replaced 0.05816622897846346 by 4451048/76522891 = 0.05816622897846345
  
  rat: replaced 0.2416381240094921 by 8002142/33116223 = 0.2416381240094923
  
  rat: replaced 0.06061532802852698 by 2146337/35409146 = 0.06061532802852686
  
  rat: replaced 0.2481942708591053 by 8882901/35790113 = 0.2481942708591057
  
  rat: replaced 0.0631303649963022 by 14651447/232082406 = 0.06313036499630222
  
  rat: replaced 0.2548255976551299 by 868346/3407609 = 0.25482559765513
  
  rat: replaced 0.06571208837185505 by 4240309/64528599 = 0.06571208837185509
  
  rat: replaced 0.2615314412704124 by 8212450/31401387 = 0.2615314412704127
  
  rat: replaced 0.06836123997666599 by 2716643/39739522 = 0.06836123997666604
  
  rat: replaced 0.2683111311261794 by 34459769/128432126 = 0.2683111311261794
  
  rat: replaced 0.07107855488944881 by 3146673/44270357 = 0.07107855488944893
  
  rat: replaced 0.2751639892590951 by 12552159/45617012 = 0.2751639892590949
  
  rat: replaced 0.07386476137264342 by 12898997/174629915 = 0.0738647613726434
  
  rat: replaced 0.2820893303890569 by 11134456/39471383 = 0.2820893303890568
  
  rat: replaced 0.07672058079958999 by 5073506/66129661 = 0.07672058079959007
  
  rat: replaced 0.2890864619877229 by 9583357/33150487 = 0.2890864619877228
  
  rat: replaced 0.07964672758239233 by 5672399/71219486 = 0.07964672758239227
  
  rat: replaced 0.2961546843477643 by 11052271/37319251 = 0.2961546843477647
  
  rat: replaced 0.08264390910047736 by 4686067/56701904 = 0.08264390910047748
  
  rat: replaced 0.3032932906528349 by 9918077/32701274 = 0.3032932906528351
  
  rat: replaced 0.0857128256298576 by 3585977/41837111 = 0.08571282562985766
  
  rat: replaced 0.3105015670482534 by 9320011/30015987 = 0.3105015670482533
  
  rat: replaced 0.08885417027310427 by 5751353/64728003 = 0.0888541702731042
  
  rat: replaced 0.3177787927123868 by 248395525/781661743 = 0.3177787927123868
  
  rat: replaced 0.09206862889003742 by 7305460/79347983 = 0.09206862889003745
  
  rat: replaced 0.3251242399287333 by 13842845/42577093 = 0.3251242399287335
  
  rat: replaced 0.09535688002914089 by 5971998/62627867 = 0.09535688002914103
  
  rat: replaced 0.3325371741586922 by 9318229/28021616 = 0.3325371741586923
  
  rat: replaced 0.0987195948597075 by 9821211/99485933 = 0.09871959485970745
  
  rat: replaced 0.3400168541150183 by 13391981/39386227 = 0.3400168541150184
  
  rat: replaced 0.1021574371047232 by 8336413/81603584 = 0.1021574371047232
  
  rat: replaced 0.3475625318359485 by 10097818/29053241 = 0.347562531835949
  
  rat: replaced 0.1056710629744951 by 5741011/54329074 = 0.105671062974495
  
  rat: replaced 0.3551734527599992 by 15867851/44676343 = 0.3551734527599987
  
  rat: replaced 0.1092611211010309 by 5551873/50812887 = 0.1092611211010309
  
  rat: replaced 0.3628488558014202 by 6897641/19009681 = 0.3628488558014203
  
  rat: replaced 0.1129282524731764 by 11548693/102265755 = 0.1129282524731764
  
  rat: replaced 0.3705879734263036 by 23358661/63031352 = 0.3705879734263038
  
  rat: replaced 0.1166730903725168 by 5656228/48479285 = 0.1166730903725168
  
  rat: replaced 0.3783900317293359 by 14241382/37636779 = 0.3783900317293358
  
  rat: replaced 0.1204962603100498 by 4057613/33674182 = 0.12049626031005
  
  rat: replaced 0.3862542505111889 by 3461217/8960981 = 0.3862542505111884
  
  rat: replaced 0.1243983799636342 by 7966447/64039797 = 0.1243983799636342
  
  rat: replaced 0.3941798433565377 by 5314214/13481699 = 0.3941798433565384
  
  rat: replaced 0.1283800591162231 by 796346/6203035 = 0.1283800591162229
  
  rat: replaced 0.4021660177127022 by 11567173/28762184 = 0.4021660177127022
  
  rat: replaced 0.1324418995948859 by 4716124/35609003 = 0.1324418995948862
  
  rat: replaced 0.4102119749689023 by 11320633/27597032 = 0.4102119749689024
  
  rat: replaced 0.1365844952106265 by 612971/4487852 = 0.1365844952106264
  
  rat: replaced 0.418316910536117 by 12225195/29224721 = 0.4183169105361177
  
  rat: replaced 0.140808431699002 by 10431632/74083859 = 0.1408084316990021
  
  rat: replaced 0.4264800139275439 by 7978696/18708253 = 0.4264800139275431
  
  rat: replaced 0.1451142866615502 by 3554077/24491572 = 0.1451142866615504
  
  rat: replaced 0.4347004688396462 by 20489554/47134879 = 0.4347004688396463
  
  rat: replaced 0.1495026295080298 by 26759297/178988805 = 0.1495026295080298
  
  rat: replaced 0.4429774532337832 by 23449796/52936771 = 0.4429774532337834
  
  rat: replaced 0.1539740213994798 by 16145763/104860306 = 0.1539740213994798
  
  rat: replaced 0.451310139418413 by 8841241/19590167 = 0.4513101394184133
  part: invalid index of list or matrix.
  #0: lineIntersection(g=[1,-1,0],h=[2-sqrt(5)/sqrt(2),1+sqrt(5)/sqrt(2),(3-sqrt(5)/sqrt(2))*(1+sqrt(5)/sqrt(2))/2+(2-sqrt(5)/sqrt(2))*(...)
   -- an error. To debug this try: debugmode(true);
  
  Error in:
  ... ection(angleBisector(A,C,B),angleBisector(C,B,A)); $P // titik ...
                                                       ^
\end{euleroutput}
\begin{eulerprompt}
>P() // hasilnya sama dengan perhitungan sebelumnya
\end{eulerprompt}
\begin{euleroutput}
  Function P needs at least 2 arguments!
  Use: P (x, n) 
  Error in:
  P() // hasilnya sama dengan perhitungan sebelumnya ...
     ^
\end{euleroutput}
\eulersubheading{Garis dan Lingkaran yang Berpotongan}
\begin{eulercomment}
Tentu saja, kita juga dapat memotong garis dengan lingkaran, dan
lingkaran dengan lingkaran.
\end{eulercomment}
\begin{eulerprompt}
>A &:= [1,0]; c=circleWithCenter(A,4);
>B &:= [1,2]; C &:= [2,1]; l=lineThrough(B,C);
>setPlotRange(5); plotCircle(c); plotLine(l);
\end{eulerprompt}
\begin{eulercomment}
Perpotongan garis dengan lingkaran menghasilkan dua titik dan jumlah
titik potong.
\end{eulercomment}
\begin{eulerprompt}
>\{P1,P2,f\}=lineCircleIntersections(l,c);
>P1, P2,
\end{eulerprompt}
\begin{euleroutput}
  [4.64575,  -1.64575]
  [-0.645751,  3.64575]
\end{euleroutput}
\begin{eulerprompt}
>plotPoint(P1); plotPoint(P2):
\end{eulerprompt}
\begin{eulercomment}
Begitu pula di Maxima.
\end{eulercomment}
\begin{eulerprompt}
>c &= circleWithCenter(A,4) // lingkaran dengan pusat A jari-jari 4
\end{eulerprompt}
\begin{euleroutput}
  
                                [1, 0, 4]
  
\end{euleroutput}
\begin{eulerprompt}
>l &= lineThrough(B,C) // garis l melalui B dan C
\end{eulerprompt}
\begin{euleroutput}
  
                                [1, 1, 3]
  
\end{euleroutput}
\begin{eulerprompt}
>$lineCircleIntersections(l,c) | radcan, // titik potong lingkaran c dan garis l
\end{eulerprompt}
\begin{euleroutput}
  Maxima said:
  rat: replaced -4.98329175014009e-5 by -86001/1725786976 = -4.983291750140082e-5
  
  rat: replaced -1.986600267553235e-4 by -1133306/5704751069 = -1.986600267553234e-4
  
  rat: replaced -4.454664535081185e-4 by -474290/1064704191 = -4.454664535081181e-4
  
  rat: replaced -7.892275256562442e-4 by -1190199/1508055613 = -7.892275256562439e-4
  
  rat: replaced -0.001228908875712045 by -259907/211494119 = -0.001228908875712047
  
  rat: replaced -0.001763466544240408 by -5854594/3319934829 = -0.001763466544240408
  
  rat: replaced -0.002391847084253176 by -866601/362314550 = -0.002391847084253172
  
  rat: replaced -0.003112987666553255 by -5049204/1621980085 = -0.003112987666553255
  
  rat: replaced -0.00392581618601677 by -1241039/316122544 = -0.003925816186016774
  
  rat: replaced -0.004829251368802329 by -3015690/624463249 = -0.00482925136880233
  
  rat: replaced -0.005822202880477995 by -2532373/434951006 = -0.005822202880477991
  
  rat: replaced -0.006903571435053116 by -1331361/192851050 = -0.006903571435053115
  
  rat: replaced -0.008072248904906765 by -7953293/985263598 = -0.008072248904906766
  
  rat: replaced -0.009327118431599252 by -432515/46371771 = -0.009327118431599259
  
  rat: replaced -0.01066705453755698 by -2950074/276559381 = -0.01066705453755698
  
  rat: replaced -0.01209092323861904 by -1254816/103781653 = -0.01209092323861907
  
  rat: replaced -0.01359758215743526 by -1827823/134422648 = -0.01359758215743526
  
  rat: replaced -0.01518588063770274 by -9199276/605778237 = -0.01518588063770274
  
  rat: replaced -0.01685465985923026 by -2516580/149310637 = -0.01685465985923026
  
  rat: replaced -0.01860275295381958 by -2032371/109251088 = -0.01860275295381955
  
  rat: replaced -0.02042898512195129 by -1413911/69211025 = -0.02042898512195131
  
  rat: replaced -0.02233217375026381 by -3647892/163346929 = -0.02233217375026377
  
  rat: replaced -0.02431112852981362 by -1377268/56651751 = -0.02431112852981367
  
  rat: replaced -0.02636465157510504 by -2533336/96088355 = -0.02636465157510502
  
  rat: replaced -0.0284915375438782 by -9699307/340427644 = -0.02849153754387819
  
  rat: replaced -0.03069057375764189 by -7938451/258660886 = -0.03069057375764189
  
  rat: replaced -0.0329605403229406 by -2936449/89089832 = -0.03296054032294056
  
  rat: replaced -0.03530021025334285 by -5224432/148000025 = -0.03530021025334287
  
  rat: replaced -0.03770834959213837 by -2448749/64939172 = -0.03770834959213832
  
  rat: replaced -0.04018371753573358 by -2461511/61256428 = -0.04018371753573356
  
  rat: replaced -0.04272506655773012 by -13954421/326609696 = -0.04272506655773012
  
  rat: replaced -0.04533114253367693 by -2051558/45257143 = -0.04533114253367695
  
  rat: replaced -0.04800068486648146 by -16995415/354066094 = -0.04800068486648145
  
  rat: replaced -0.05073242661246818 by -3970295/78259513 = -0.05073242661246818
  
  rat: replaced -0.05352509460807248 by -3894269/72755948 = -0.05352509460807246
  
  rat: replaced -0.05637740959715515 by -11093364/196769665 = -0.05637740959715513
  
  rat: replaced -0.05928808635892763 by -3489209/58851773 = -0.05928808635892754
  
  rat: replaced -0.06225583383647254 by -3380435/54299088 = -0.06225583383647254
  
  rat: replaced -0.06527935526584844 by -7571267/115982564 = -0.06527935526584841
  
  rat: replaced -0.06835734830576551 by -8050241/117767017 = -0.06835734830576544
  
  rat: replaced -0.07148850516781785 by -5513427/77123266 = -0.07148850516781798
  
  rat: replaced -0.07467151274726203 by -2259975/30265558 = -0.07467151274726208
  
  rat: replaced -0.07790505275432569 by -657797/8443573 = -0.07790505275432569
  
  rat: replaced -0.08118780184603619 by -4832180/59518547 = -0.08118780184603633
  
  rat: replaced -0.08451843175855339 by -3076049/36395008 = -0.08451843175855327
  
  rat: replaced -0.08789560943999458 by -7150621/81353563 = -0.08789560943999465
  
  rat: replaced -0.0913179971837394 by -20067867/219758072 = -0.0913179971837394
  
  rat: replaced -0.09478425276219882 by -5487749/57897265 = -0.09478425276219869
  
  rat: replaced -0.09829302956103664 by -4406725/44832528 = -0.09829302956103658
  
  rat: replaced -0.1018429767138303 by -3912367/38415678 = -0.1018429767138302
  
  rat: replaced -0.1054327392371563 by -8451941/80164293 = -0.1054327392371564
  
  rat: replaced -0.1090609581660869 by -13126833/120362348 = -0.1090609581660869
  
  rat: replaced -0.112726270690086 by -2754747/24437489 = -0.112726270690086
  
  rat: replaced -0.116427310289289 by -22239618/191017193 = -0.116427310289289
  
  rat: replaced -0.1201627068711536 by -9494831/79016454 = -0.1201627068711537
  
  rat: replaced -0.1239310869074673 by -3190398/25743323 = -0.1239310869074672
  
  rat: replaced -0.1277310735717007 by -15999330/125257931 = -0.1277310735717006
  
  rat: replaced -0.1315612868766867 by -13929723/105880106 = -0.1315612868766867
  
  rat: replaced -0.1354203438126204 by -28035370/207024803 = -0.1354203438126204
  
  rat: replaced -0.1393068584853572 by -11590983/83204683 = -0.1393068584853571
  
  rat: replaced -0.1432194422550018 by -12738764/88945773 = -0.1432194422550018
  
  rat: replaced -0.1471567038747712 by -5246589/35653075 = -0.147156703874771
  
  rat: replaced -0.1511172496301179 by -4676629/30947023 = -0.1511172496301179
  
  rat: replaced -0.1550996834780995 by -15854305/102220099 = -0.1550996834780995
  
  rat: replaced -0.1591026071869839 by -9026555/56734174 = -0.159102607186984
  
  rat: replaced -0.1631246204760689 by -10435073/63969945 = -0.1631246204760689
  
  rat: replaced -0.1671643211557106 by -164873401/986295400 = -0.1671643211557106
  
  rat: replaced -0.1712203052675407 by -7017638/40986015 = -0.1712203052675406
  
  rat: replaced -0.1752911672248615 by -3184915/18169284 = -0.1752911672248615
  
  rat: replaced -0.1793754999532028 by -2646709/14755131 = -0.1793754999532027
  
  rat: replaced -0.1834718950310287 by -8392143/45740755 = -0.1834718950310287
  
  rat: replaced -0.1875789428305783 by -12888313/68708741 = -0.1875789428305781
  
  rat: replaced -0.1916952326588277 by -16014703/83542521 = -0.1916952326588277
  
  rat: replaced -0.1958193528985573 by -21279927/108671215 = -0.1958193528985574
  
  rat: replaced -0.1999498911495134 by -5994245/29978736 = -0.1999498911495134
  
  rat: replaced -0.2040854343696463 by -17847769/87452439 = -0.2040854343696464
  
  rat: replaced -0.2082245690164135 by -5203892/24991729 = -0.2082245690164134
  
  rat: replaced -0.2123658811881329 by -20393053/96027916 = -0.2123658811881328
  
  rat: replaced -0.2165079567653719 by -8489188/39209589 = -0.2165079567653719
  
  rat: replaced -0.2206493815523576 by -14881929/67446049 = -0.2206493815523575
  
  rat: replaced -0.2247887414183958 by -11437558/50881365 = -0.2247887414183955
  
  rat: replaced -0.2289246224392826 by -17547464/76651711 = -0.2289246224392825
  
  rat: replaced -0.2330556110386959 by -11148764/47837355 = -0.2330556110386956
  
  rat: replaced -0.2371802941295513 by -11052217/46598378 = -0.237180294129551
  
  rat: replaced -0.2412972592553108 by -36037383/149348497 = -0.2412972592553108
  
  rat: replaced -0.2454050947312253 by -4652365/18957899 = -0.2454050947312252
  
  rat: replaced -0.2495023897855041 by -6175634/24751803 = -0.2495023897855037
  
  rat: replaced -0.2535877347003893 by -11299519/44558618 = -0.2535877347003895
  
  rat: replaced -0.2576597209531272 by -6871877/26670358 = -0.2576597209531271
  
  rat: replaced -0.2617169413568191 by -2245730/8580759 = -0.2617169413568194
  
  rat: replaced -0.2657579902011391 by -10500993/39513367 = -0.2657579902011388
  
  rat: replaced -0.2697814633929034 by -21050552/78028163 = -0.2697814633929034
  
  rat: replaced -0.2737859585964791 by -1510231/5516101 = -0.2737859585964796
  
  rat: replaced -0.2777700753740163 by -9819093/35349715 = -0.2777700753740164
  
  rat: replaced -0.2817324153254904 by -10837378/38466919 = -0.2817324153254905
  
  rat: replaced -0.2856715822285418 by -17041418/59653879 = -0.2856715822285421
  
  rat: replaced -0.289586182178096 by -721506/2491507 = -0.2895861821780955
  
  rat: replaced -0.2934748237257534 by -11793110/40184401 = -0.2934748237257537
  
  rat: replaced -0.2973361180189332 by -15390047/51759763 = -0.2973361180189329
  
  rat: replaced 5.016624916807239e-5 by 153117/3052191514 = 5.016624916807235e-5
  
  rat: replaced 2.013266400891639e-4 by 232411/1154397649 = 2.013266400891639e-4
  
  rat: replaced 4.544660485167953e-4 by 444871/978887205 = 4.544660485167952e-4
  
  rat: replaced 8.105591523879241e-4 by 1425236/1758336817 = 8.105591523879239e-4
  
  rat: replaced 0.001270570334355389 by 696221/547959433 = 0.00127057033435539
  
  rat: replaced 0.001835453585351213 by 1018402/554850315 = 0.001835453585351213
  
  rat: replaced 0.002506152409187654 by 484773/193433168 = 0.002506152409187653
  
  rat: replaced 0.003283599728207867 by 1007483/306822720 = 0.003283599728207872
  
  rat: replaced 0.004168717789994683 by 897113/215201183 = 0.004168717789994677
  
  rat: replaced 0.00516241807514603 by 757433/146720585 = 0.005162418075146034
  
  rat: replaced 0.006265601206128374 by 1194190/190594639 = 0.006265601206128363
  
  rat: replaced 0.007479156857214384 by 1971251/263565939 = 0.007479156857214391
  
  rat: replaced 0.008803963665517056 by 365844/41554465 = 0.008803963665517051
  
  rat: replaced 0.01024088914312629 by 1345773/131411734 = 0.01024088914312629
  
  rat: replaced 0.01179078959035854 by 1519715/128890011 = 0.01179078959035856
  
  rat: replaced 0.0134545100101271 by 2242921/166704027 = 0.01345451001012711
  
  rat: replaced 0.01523288402344322 by 1950407/128039247 = 0.01523288402344322
  
  rat: replaced 0.01712673378605437 by 1362867/79575418 = 0.01712673378605438
  
  rat: replaced 0.01913686990622912 by 1694449/88543686 = 0.01913686990622911
  
  rat: replaced 0.02126409136369717 by 9814128/461535263 = 0.02126409136369716
  
  rat: replaced 0.02350918542975217 by 2315819/98506986 = 0.02350918542975216
  
  rat: replaced 0.02587292758852516 by 3386321/130882792 = 0.02587292758852516
  
  rat: replaced 0.02835608145943683 by 10230271/360778728 = 0.02835608145943682
  
  rat: replaced 0.03095939872083586 by 14307719/462144602 = 0.03095939872083587
  
  rat: replaced 0.03368361903483233 by 4712088/139892569 = 0.03368361903483236
  
  rat: replaced 0.03652946997333167 by 4111522/112553563 = 0.03652946997333172
  
  rat: replaced 0.03949766694527834 by 8626745/218411508 = 0.03949766694527836
  
  rat: replaced 0.04258891312511537 by 3115258/73147159 = 0.04258891312511536
  
  rat: replaced 0.04580389938246726 by 2358579/51492974 = 0.04580389938246721
  
  rat: replaced 0.04914330421305446 by 2180747/44375262 = 0.04914330421305456
  
  rat: replaced 0.05260779367084312 by 4975224/94571995 = 0.05260779367084304
  
  rat: replaced 0.05619802130144141 by 1396735/24853811 = 0.05619802130144146
  
  rat: replaced 0.05991462807674475 by 6603037/110207427 = 0.05991462807674477
  
  rat: replaced 0.06375824233083943 by 6198842/97224167 = 0.0637582423308394
  
  rat: replaced 0.06772947969716975 by 4012504/59243095 = 0.06772947969716978
  
  rat: replaced 0.07182894304697524 by 5813372/80933559 = 0.07182894304697511
  
  rat: replaced 0.07605722242900365 by 14672328/192911699 = 0.07605722242900365
  
  rat: replaced 0.08041489501050719 by 3507279/43614793 = 0.0804148950105071
  
  rat: replaced 0.08490252501952561 by 2460362/28978667 = 0.08490252501952557
  
  rat: replaced 0.08952066368846451 by 4304415/48082921 = 0.08952066368846436
  
  rat: replaced 0.09426984919897213 by 3898288/41352437 = 0.09426984919897224
  
  rat: replaced 0.0991506066281217 by 11428253/115261554 = 0.09915060662812164
  
  rat: replaced 0.1041634478959041 by 7209817/69216382 = 0.1041634478959042
  
  rat: replaced 0.1093088717140371 by 3826731/35008421 = 0.109308871714037
  
  rat: replaced 0.1145873635360931 by 5173172/45146095 = 0.1145873635360932
  
  rat: replaced 0.1199993955089551 by 23218093/193485083 = 0.1199993955089551
  
  rat: replaced 0.1255454264256029 by 2445819/19481546 = 0.125545426425603
  
  rat: replaced 0.1312259016792331 by 9111136/69430927 = 0.131225901679233
  
  rat: replaced 0.1370412532187207 by 16597683/121114501 = 0.1370412532187207
  
  rat: replaced 0.1429918995054244 by 34253454/239548213 = 0.1429918995054244
  
  rat: replaced 0.1490782454713414 by 11997679/80479073 = 0.1490782454713414
  
  rat: replaced 0.1553006824786136 by 13065213/84128497 = 0.1553006824786136
  
  rat: replaced 0.1616595882803922 by 12686167/78474572 = 0.1616595882803923
  
  rat: replaced 0.1681553269830629 by 4527449/26924208 = 0.168155326983063
  
  rat: replaced 0.1747882490098353 by 23565700/134824281 = 0.1747882490098353
  
  rat: replaced 0.1815586910657007 by 4563713/25136296 = 0.1815586910657004
  
  rat: replaced 0.1884669761037622 by 8213146/43578701 = 0.1884669761037623
  
  rat: replaced 0.1955134132929397 by 7172626/36686107 = 0.1955134132929395
  
  rat: replaced 0.202698297987053 by 17668607/87167022 = 0.2026982979870529
  
  rat: replaced 0.2100219116952866 by 8269584/39374863 = 0.2100219116952864
  
  rat: replaced 0.2174845220540395 by 56596301/260231397 = 0.2174845220540395
  
  rat: replaced 0.2250863828001612 by 8187128/36373271 = 0.2250863828001611
  
  rat: replaced 0.2328277337455789 by 10320856/44328293 = 0.2328277337455787
  
  rat: replaced 0.2407088007533156 by 16964872/70478819 = 0.2407088007533157
  
  rat: replaced 0.2487297957149048 by 11063220/44478869 = 0.2487297957149045
  
  rat: replaced 0.2568909165292014 by 17200949/66958183 = 0.2568909165292015
  
  rat: replaced 0.2651923470825914 by 8866093/33432688 = 0.2651923470825918
  
  rat: replaced 0.2736342572306039 by 12664159/46281336 = 0.2736342572306037
  
  rat: replaced 0.2822168027809259 by 8116045/28758192 = 0.2822168027809259
  
  rat: replaced 0.2909401254778209 by 24764749/85119744 = 0.290940125477821
  
  rat: replaced 0.2998043529879556 by 28498628/95057419 = 0.2998043529879556
  
  rat: replaced 0.3088095988876323 by 13390352/43361191 = 0.308809598887632
  
  rat: replaced 0.3179559626514321 by 26241235/82531036 = 0.3179559626514321
  
  rat: replaced 0.3272435296422674 by 8247573/25203166 = 0.3272435296422679
  
  rat: replaced 0.3366723711028454 by 10805861/32096073 = 0.3366723711028449
  
  rat: replaced 0.3462425441485439 by 20967050/60555961 = 0.3462425441485438
  
  rat: replaced 0.3559540917617003 by 19053013/53526602 = 0.3559540917617001
  
  rat: replaced 0.3658070427873129 by 10401097/28433288 = 0.3658070427873132
  
  rat: replaced 0.3758014119301566 by 5923743/15762961 = 0.375801411930157
  
  rat: replaced 0.3859371997533123 by 2934328/7603123 = 0.3859371997533119
  
  rat: replaced 0.396214392678111 by 30414315/76762267 = 0.396214392678111
  
  rat: replaced 0.4066329629854911 by 13711485/33719561 = 0.4066329629854908
  
  rat: replaced 0.4171928688187707 by 20838614/49949593 = 0.4171928688187709
  
  rat: replaced 0.4278940541878331 by 16106690/37641771 = 0.427894054187833
  
  rat: replaced 0.4387364489747257 by 4869080/11097961 = 0.4387364489747261
  
  rat: replaced 0.4497199689406718 by 4550581/10118699 = 0.4497199689406711
  
  rat: replaced 0.4608445157344944 by 7970699/17295853 = 0.4608445157344943
  
  rat: replaced 0.4721099769024512 by 25424083/53852035 = 0.4721099769024513
  
  rat: replaced 0.48351622589948 by 17675673/36556525 = 0.4835162258994803
  
  rat: replaced 0.4950631221018528 by 7053395/14247466 = 0.495063122101853
  
  rat: replaced 0.5067505108212387 by 13754758/27143057 = 0.5067505108212388
  
  rat: replaced 0.5185782233201719 by 21662467/41772805 = 0.518578223320172
  
  rat: replaced 0.5305460768289253 by 10488897/19770002 = 0.530546076828925
  
  rat: replaced 0.5426538745637882 by 22388393/41257225 = 0.5426538745637886
  
  rat: replaced 0.5549014057467435 by 9960301/17949677 = 0.5549014057467441
  
  rat: replaced 0.5672884456265459 by 28078535/49496046 = 0.5672884456265456
  
  rat: replaced 0.5798147555011964 by 18086313/31193261 = 0.5798147555011962
  
  rat: replaced 0.5924800827418131 by 20592707/34756792 = 0.5924800827418134
  
  rat: replaced 0.6052841608178928 by 26813845/44299598 = 0.6052841608178927
  part: invalid index of list or matrix.
  #0: lineIntersection(g=[1,-1,1],h=[1,1,3])
  #1: projectToLine(a=[1,0],g=[1,1,3])
  #2: lineCircleIntersections(g=[1,1,3],c=[1,0,4])
   -- an error. To debug this try: debugmode(true);
  
  Error in:
   $lineCircleIntersections(l,c) | radcan, // titik potong lingka ...
                                        ^
\end{euleroutput}
\begin{eulercomment}
Akan ditunjukkan bahwa sudut-sudut yang menghadap bsuusr yang sama adalah sama besar.
\end{eulercomment}
\begin{eulerprompt}
>C=A+normalize([-2,-3])*4; plotPoint(C); plotSegment(P1,C); plotSegment(P2,C);
>degprint(computeAngle(P1,C,P2))
\end{eulerprompt}
\begin{euleroutput}
  69°17'42.68''
\end{euleroutput}
\begin{eulerprompt}
>C=A+normalize([-4,-3])*4; plotPoint(C); plotSegment(P1,C); plotSegment(P2,C);
>degprint(computeAngle(P1,C,P2))
\end{eulerprompt}
\begin{euleroutput}
  69°17'42.68''
\end{euleroutput}
\begin{eulerprompt}
>insimg;
\end{eulerprompt}
\eulersubheading{Garis Sumbu}
\begin{eulercomment}
Berikut adalah langkah-langkah menggambar garis sumbu ruas garis AB:

1. Gambar lingkaran dengan pusat A melalui B.\\
2. Gambar lingkaran dengan pusat B melalui A.\\
3. Tarik garis melallui kedua titik potong kedua lingkaran tersebut. Garis ini merupakan
garis sumbu (melalui titik tengah dan tegak lurus) AB.
\end{eulercomment}
\begin{eulerprompt}
>A=[2,2]; B=[-1,-2];
>c1=circleWithCenter(A,distance(A,B));
>c2=circleWithCenter(B,distance(A,B));
>\{P1,P2,f\}=circleCircleIntersections(c1,c2);
>l=lineThrough(P1,P2);
>setPlotRange(5); plotCircle(c1); plotCircle(c2);
>plotPoint(A); plotPoint(B); plotSegment(A,B); plotLine(l):
\end{eulerprompt}
\begin{eulercomment}
Selanjutnya, kami melakukan hal yang sama di Maxima dengan koordinat
umum.
\end{eulercomment}
\begin{eulerprompt}
>A &= [a1,a2]; B &= [b1,b2];
>c1 &= circleWithCenter(A,distance(A,B));
>c2 &= circleWithCenter(B,distance(A,B));
>P &= circleCircleIntersections(c1,c2); P1 &= P[1]; P2 &= P[2];
\end{eulerprompt}
\begin{eulercomment}
Persamaan untuk persimpangan cukup terlibat. Tetapi kita dapat
menyederhanakannya, jika kita memecahkan y.
\end{eulercomment}
\begin{eulerprompt}
>g &= getLineEquation(lineThrough(P1,P2),x,y);
>$solve(g,y)
\end{eulerprompt}
\begin{euleroutput}
  Maxima said:
  solve: all variables must not be numbers.
   -- an error. To debug this try: debugmode(true);
  
  Error in:
   $solve(g,y) ...
             ^
\end{euleroutput}
\begin{eulercomment}
Ini memang sama dengan tegak lurus tengah, yang dihitung dengan cara
yang sama sekali berbeda.
\end{eulercomment}
\begin{eulerprompt}
>$solve(getLineEquation(middlePerpendicular(A,B),x,y),y)
\end{eulerprompt}
\begin{euleroutput}
  Maxima said:
  solve: all variables must not be numbers.
   -- an error. To debug this try: debugmode(true);
  
  Error in:
  ... (getLineEquation(middlePerpendicular(A,B),x,y),y) ...
                                                       ^
\end{euleroutput}
\begin{eulerprompt}
>h &=getLineEquation(lineThrough(A,B),x,y);
>$solve(h,y)
\end{eulerprompt}
\begin{euleroutput}
  Maxima said:
  solve: all variables must not be numbers.
   -- an error. To debug this try: debugmode(true);
  
  Error in:
   $solve(h,y) ...
             ^
\end{euleroutput}
\begin{eulercomment}
Perhatikan hasil kali gradien garis g dan h adalah:

\end{eulercomment}
\begin{eulerformula}
\[
\frac{-(b_1-a_1)}{(b_2-a_2)}\times \frac{(b_2-a_2)}{(b_1-a_1)} = -1.
\]
\end{eulerformula}
\begin{eulercomment}
Artinya kedua garis tegak lurus.

\end{eulercomment}
\eulersubheading{Contoh 3: Rumus Heron}
\begin{eulercomment}
Rumus Heron menyatakan bahwa luas segitiga dengan panjang sisi-sisi a,
b dan c adalah:

\end{eulercomment}
\begin{eulerformula}
\[
L = \sqrt{s(s-a)(s-b)(s-c)}\quad \text{ dengan } s=(a+b+c)/2,
\]
\end{eulerformula}
\begin{eulercomment}
Untuk membuktikan hal ini kita misalkan C(0,0), B(a,0) dan A(x,y),
b=AC, c=AB. Luas segitiga ABC adalah

\end{eulercomment}
\begin{eulerformula}
\[
L_{\triangle ABC}=\frac{1}{2}a\times y.
\]
\end{eulerformula}
\begin{eulercomment}
Nilai y didapat dengan menyelesaikan sistem persamaan:

\end{eulercomment}
\begin{eulerformula}
\[
x^2+y^2=b^2, \quad (x-a)^2+y^2=c^2.
\]
\end{eulerformula}
\begin{eulerprompt}
>setPlotRange(-1,10,-1,8); plotPoint([0,0], "C(0,0)"); plotPoint([5.5,0], "B(a,0)");  ...
>plotPoint([7.5,6], "A(x,y)");
>plotSegment([0,0],[5.5,0], "a",25); plotSegment([5.5,0],[7.5,6],"c",15);  ...
>plotSegment([0,0],[7.5,6],"b",25); 
>plotSegment([7.5,6],[7.5,0],"t=y",25):
>&assume(a>0); sol &= solve([x^2+y^2=b^2,(x-a)^2+y^2=c^2],[x,y])
\end{eulerprompt}
\begin{euleroutput}
  Maxima said:
  fullmap: arguments must have same formal structure.
   -- an error. To debug this try: debugmode(true);
  
  Error in:
  ... sol &= solve([x^2+y^2=b^2,(x-a)^2+y^2=c^2],[x,y]) ...
                                                       ^
\end{euleroutput}
\begin{eulercomment}
Ekstrak solusi y.
\end{eulercomment}
\begin{eulerprompt}
>ysol &= y with sol[2][2]; $'y=sqrt(factor(ysol^2))
\end{eulerprompt}
\begin{euleroutput}
  Maxima said:
  at: improper argument: (3*x*sqrt(4+9*x^2))[2][2]
  #0: with(expr=[0,4.999958333473664e-5*r,1.999933334222437e-4*r,4.499662510124569e-4*r,7.998933390220841e-4*r,0.001...,eq=(3*x*sqrt(4+9*x^2))[2][2])
   -- an error. To debug this try: debugmode(true);
  
  Error in:
  ysol &= y with sol[2][2]; $'y=sqrt(factor(ysol^2)) ...
                          ^
\end{euleroutput}
\begin{eulercomment}
Kami mendapatkan rumus Heron.
\end{eulercomment}
\begin{eulerprompt}
>function H(a,b,c) &= sqrt(factor((ysol*a/2)^2)); $'H(a,b,c)=H(a,b,c)
>$'Luas=H(2,5,6) // luas segitiga dengan panjang sisi-sisi 2, 5, 6
\end{eulerprompt}
\begin{eulercomment}
Tentu saja, setiap segitiga persegi panjang adalah kasus yang
terkenal.
\end{eulercomment}
\begin{eulerprompt}
>H(3,4,5) //luas segitiga siku-siku dengan panjang sisi 3, 4, 5
\end{eulerprompt}
\begin{euleroutput}
  Variable or function ysol not found.
  Try "trace errors" to inspect local variables after errors.
  H:
      useglobal; return a*abs(ysol)/2 
  Error in:
  H(3,4,5) //luas segitiga siku-siku dengan panjang sisi 3, 4, 5 ...
          ^
\end{euleroutput}
\begin{eulercomment}
Dan juga jelas, bahwa ini adalah segitiga dengan luas maksimal dan dua
sisi 3 dan 4.
\end{eulercomment}
\begin{eulerprompt}
>aspect (1.5); plot2d(&H(3,4,x),1,7): // Kurva luas segitiga sengan panjang sisi 3, 4, x (1<= x <=7)
\end{eulerprompt}
\begin{euleroutput}
  Variable or function ysol not found.
  Error in expression: 3*abs(ysol)/2
   %ploteval:
      y0=f$(x[1],args());
  adaptiveevalone:
      s=%ploteval(g$,t;args());
  Try "trace errors" to inspect local variables after errors.
  plot2d:
      dw/n,dw/n^2,dw/n,auto;args());
\end{euleroutput}
\begin{eulercomment}
Kasus umum juga berfungsi.
\end{eulercomment}
\begin{eulerprompt}
>$solve(diff(H(a,b,c)^2,c)=0,c)
\end{eulerprompt}
\begin{euleroutput}
  Maxima said:
  diff: second argument must be a variable; found [1,0,4]
   -- an error. To debug this try: debugmode(true);
  
  Error in:
   $solve(diff(H(a,b,c)^2,c)=0,c) ...
                                ^
\end{euleroutput}
\begin{eulercomment}
Sekarang mari kita cari himpunan semua titik di mana b+c=d untuk
beberapa konstanta d. Diketahui bahwa ini adalah elips.
\end{eulercomment}
\begin{eulerprompt}
>s1 &= subst(d-c,b,sol[2]); $s1
\end{eulerprompt}
\begin{eulercomment}
Dan buat fungsi ini.
\end{eulercomment}
\begin{eulerprompt}
>function fx(a,c,d) &= rhs(s1[1]); $fx(a,c,d), function fy(a,c,d) &= rhs(s1[2]); $fy(a,c,d)
\end{eulerprompt}
\begin{eulercomment}
Sekarang kita bisa menggambar setnya. Sisi b bervariasi dari 1 hingga
4. Diketahui bahwa kita mendapatkan elips.
\end{eulercomment}
\begin{eulerprompt}
>aspect(1); plot2d(&fx(3,x,5),&fy(3,x,5),xmin=1,xmax=4,square=1):
\end{eulerprompt}
\begin{eulercomment}
Kita dapat memeriksa persamaan umum untuk elips ini, yaitu.

\end{eulercomment}
\begin{eulerformula}
\[
\frac{(x-x_m)^2}{u^2}+\frac{(y-y_m)}{v^2}=1,
\]
\end{eulerformula}
\begin{eulercomment}
di mana (xm,ym) adalah pusat, dan u dan v adalah setengah sumbu.
\end{eulercomment}
\begin{eulerprompt}
>$ratsimp((fx(a,c,d)-a/2)^2/u^2+fy(a,c,d)^2/v^2 with [u=d/2,v=sqrt(d^2-a^2)/2])
\end{eulerprompt}
\begin{eulercomment}
Kita lihat bahwa tinggi dan luas segitiga adalah maksimal untuk x=0.
Jadi luas segitiga dengan a+b+c=d maksimal jika segitiga sama sisi.
Kami ingin menurunkan ini secara analitis.
\end{eulercomment}
\begin{eulerprompt}
>eqns &= [diff(H(a,b,d-(a+b))^2,a)=0,diff(H(a,b,d-(a+b))^2,b)=0]; $eqns
\end{eulerprompt}
\begin{eulercomment}
Kami mendapatkan beberapa minima, yang termasuk dalam segitiga dengan
satu sisi 0, dan solusinya a=b=c=d/3.
\end{eulercomment}
\begin{eulerprompt}
>$solve(eqns,[a,b])
\end{eulerprompt}
\begin{eulercomment}
Ada juga metode Lagrange, memaksimalkan H(a,b,c)\textasciicircum{}2 terhadap a+b+d=d.
\end{eulercomment}
\begin{eulerprompt}
>&solve([diff(H(a,b,c)^2,a)=la,diff(H(a,b,c)^2,b)=la, ...
>   diff(H(a,b,c)^2,c)=la,a+b+c=d],[a,b,c,la])
\end{eulerprompt}
\begin{euleroutput}
  
                       d      d
          [[a = 0, b = -, c = -, la = 0], 
                       2      2
       d             d                d      d
  [a = -, b = 0, c = -, la = 0], [a = -, b = -, c = 0, la = 0], 
       2             2                2      2
                              3
       d      d      d       d
  [a = -, b = -, c = -, la = ---]]
       3      3      3       108
  
\end{euleroutput}
\begin{eulercomment}
Kita bisa membuat plot situasinya
\end{eulercomment}
\begin{eulercomment}
Pertama-tama atur poin di Maxima.
\end{eulercomment}
\begin{eulerprompt}
>A &= at([x,y],sol[2]); $A
\end{eulerprompt}
\begin{euleroutput}
  Maxima said:
  at: improper argument: (3*x*sqrt(4+9*x^2))[2]
   -- an error. To debug this try: debugmode(true);
  
  Error in:
  A &= at([x,y],sol[2]); $A ...
                       ^
\end{euleroutput}
\begin{eulerprompt}
>B &= [0,0]; $B, C &= [a,0]; $C
\end{eulerprompt}
\begin{eulercomment}
Kemudian atur rentang plot, dan plot titik-titiknya.
\end{eulercomment}
\begin{eulerprompt}
>setPlotRange(0,5,-2,3); ...
>a=4; b=3; c=2; ...
>plotPoint(mxmeval("B"),"B"); plotPoint(mxmeval("C"),"C"); ...
>plotPoint(mxmeval("A"),"A"):
\end{eulerprompt}
\begin{euleroutput}
  Variable a1 not found!
  Use global variables or parameters for string evaluation.
  Error in Evaluate, superfluous characters found.
  Try "trace errors" to inspect local variables after errors.
  mxmeval:
      return evaluate(mxm(s));
  Error in:
  ... otPoint(mxmeval("C"),"C"); plotPoint(mxmeval("A"),"A"): ...
                                                       ^
\end{euleroutput}
\begin{eulercomment}
Plot segmen.
\end{eulercomment}
\begin{eulerprompt}
>plotSegment(mxmeval("A"),mxmeval("C")); ...
>plotSegment(mxmeval("B"),mxmeval("C")); ...
>plotSegment(mxmeval("B"),mxmeval("A")):
\end{eulerprompt}
\begin{euleroutput}
  Variable a1 not found!
  Use global variables or parameters for string evaluation.
  Error in Evaluate, superfluous characters found.
  Try "trace errors" to inspect local variables after errors.
  mxmeval:
      return evaluate(mxm(s));
  Error in:
  plotSegment(mxmeval("A"),mxmeval("C")); plotSegment(mxmeval("B ...
                          ^
\end{euleroutput}
\begin{eulercomment}
Hitung tegak lurus tengah di Maxima.
\end{eulercomment}
\begin{eulerprompt}
>h &= middlePerpendicular(A,B); g &= middlePerpendicular(B,C);
\end{eulerprompt}
\begin{eulercomment}
Dan pusat lingkaran.
\end{eulercomment}
\begin{eulerprompt}
>U &= lineIntersection(h,g);
\end{eulerprompt}
\begin{euleroutput}
  Maxima said:
  rat: replaced 1.66665833335744e-7 by 15819/94914474571 = 1.66665833335744e-7
  
  rat: replaced 4.999958333473664e-5 by 201389/4027813565 = 4.99995833347366e-5
  
  rat: replaced 1.33330666692022e-6 by 31771/23828726570 = 1.333306666920221e-6
  
  rat: replaced 1.999933334222437e-4 by 200030/1000183339 = 1.999933334222437e-4
  
  rat: replaced 4.499797504338432e-6 by 24036/5341573699 = 4.499797504338431e-6
  
  rat: replaced 4.499662510124569e-4 by 1162901/2584418270 = 4.499662510124571e-4
  
  rat: replaced 1.066581336583994e-5 by 58861/5518660226 = 1.066581336583993e-5
  
  rat: replaced 7.998933390220841e-4 by 1137431/1421978337 = 7.998933390220838e-4
  
  rat: replaced 2.083072932167196e-5 by 35635/1710693824 = 2.0830729321672e-5
  
  rat: replaced 0.001249739605033717 by 567943/454449069 = 0.001249739605033716
  
  rat: replaced 3.599352055540239e-5 by 98277/2730408098 = 3.599352055540234e-5
  
  rat: replaced 0.00179946006479581 by 479561/266502719 = 0.001799460064795812
  
  rat: replaced 5.71526624672386e-5 by 51154/895041417 = 5.715266246723866e-5
  
  rat: replaced 0.002448999746720415 by 1946227/794702818 = 0.002448999746720415
  
  rat: replaced 8.530603082730626e-5 by 121691/1426522824 = 8.530603082730627e-5
  
  rat: replaced 0.003198293697380561 by 2986741/933854512 = 0.003198293697380562
  
  rat: replaced 1.214508019889565e-4 by 158455/1304684674 = 1.214508019889563e-4
  
  rat: replaced 0.004047266988005727 by 2125334/525128193 = 0.004047266988005727
  
  rat: replaced 1.665833531718508e-4 by 142521/855553675 = 1.66583353171851e-4
  
  rat: replaced 0.004995834721974179 by 1957223/391770967 = 0.004995834721974179
  
  rat: replaced 2.216991628251896e-4 by 179571/809975995 = 2.216991628251896e-4
  
  rat: replaced 0.006043902043303184 by 1800665/297930871 = 0.006043902043303193
  
  rat: replaced 2.877927110806339e-4 by 1167733/4057548906 = 2.877927110806339e-4
  
  rat: replaced 0.00719136414613375 by 2476362/344352191 = 0.007191364146133747
  
  rat: replaced 3.658573803051457e-4 by 386279/1055818526 = 3.658573803051454e-4
  
  rat: replaced 0.00843810628521191 by 2079855/246483622 = 0.008438106285211924
  
  rat: replaced 4.5688535576352e-4 by 262978/575588595 = 4.568853557635206e-4
  
  rat: replaced 0.009784003787362772 by 1752551/179124113 = 0.009784003787362787
  
  rat: replaced 5.618675264007778e-4 by 150595/268025812 = 5.618675264007782e-4
  
  rat: replaced 0.01122892206395776 by 5450241/485375263 = 0.01122892206395776
  
  rat: replaced 6.817933857540259e-4 by 192316/282073725 = 6.817933857540258e-4
  
  rat: replaced 0.01277271662437307 by 3258991/255152533 = 0.01277271662437308
  
  rat: replaced 8.176509330039827e-4 by 105841/129445214 = 8.176509330039812e-4
  
  rat: replaced 0.01441523309043924 by 2330472/161667313 = 0.01441523309043925
  
  rat: replaced 9.704265741758145e-4 by 651321/671169790 = 9.704265741758132e-4
  
  rat: replaced 0.01615630721187855 by 19391318/1200232067 = 0.01615630721187855
  
  rat: replaced 0.001141105023499428 by 1259907/1104111343 = 0.001141105023499428
  
  rat: replaced 0.01799576488272969 by 4765614/264818641 = 0.01799576488272969
  
  rat: replaced 0.001330669204938795 by 1231154/925214167 = 0.001330669204938796
  
  rat: replaced 0.01993342215875837 by 2504519/125644206 = 0.01993342215875836
  
  rat: replaced 0.001540100153900437 by 276884/179783113 = 0.001540100153900439
  
  rat: replaced 0.02196908527585173 by 1298306/59096953 = 0.0219690852758517
  
  rat: replaced 0.001770376919130678 by 644389/363984072 = 0.001770376919130681
  
  rat: replaced 0.02410255066939448 by 2001286/83032125 = 0.02410255066939453
  
  rat: replaced 0.002022476464811601 by 1271955/628909667 = 0.002022476464811599
  
  rat: replaced 0.02633360499462523 by 2978115/113091808 = 0.02633360499462525
  
  rat: replaced 0.002297373572865413 by 1020913/444382669 = 0.002297373572865417
  
  rat: replaced 0.02866202514797045 by 1770713/61779061 = 0.02866202514797044
  
  rat: replaced 0.002596040745477063 by 1097643/422814242 = 0.002596040745477065
  
  rat: replaced 0.03108757828935527 by 5034207/161936287 = 0.03108757828935525
  
  rat: replaced 0.002919448107844891 by 906221/310408326 = 0.002919448107844891
  
  rat: replaced 0.03361002186548678 by 4553215/135471944 = 0.03361002186548678
  
  rat: replaced 0.003268563311168871 by 1379071/421919623 = 0.003268563311168867
  
  rat: replaced 0.03622910363410947 by 3082649/85087642 = 0.0362291036341094
  
  rat: replaced 0.003644351435886262 by 5966577/1637212301 = 0.003644351435886261
  
  rat: replaced 0.03894456168922911 by 4913415/126164342 = 0.03894456168922911
  
  rat: replaced 0.004047774895164447 by 572425/141417202 = 0.004047774895164451
  
  rat: replaced 0.04175612448730281 by 1734727/41544253 = 0.04175612448730273
  
  rat: replaced 0.004479793338660443 by 2952779/659132861 = 0.004479793338660444
  
  rat: replaced 0.04466351087439402 by 4691119/105032473 = 0.04466351087439405
  
  rat: replaced 0.0049413635565565 by 2524919/510976165 = 0.004941363556556498
  
  rat: replaced 0.04766643011428662 by 3536207/74186529 = 0.04766643011428665
  
  rat: replaced 0.005433439383882244 by 1361584/250593391 = 0.005433439383882235
  
  rat: replaced 0.05076458191755917 by 7710025/151878036 = 0.05076458191755916
  
  rat: replaced 0.005956971605131645 by 1447422/242979503 = 0.005956971605131648
  
  rat: replaced 0.0539576564716131 by 3377975/62604183 = 0.05395765647161309
  
  rat: replaced 0.006512907859185624 by 3695063/567344584 = 0.006512907859185626
  
  rat: replaced 0.05724533447165381 by 2560865/44734912 = 0.05724533447165382
  
  rat: replaced 0.007102192544548636 by 1363981/192050693 = 0.007102192544548642
  
  rat: replaced 0.06062728715262111 by 8274761/136485754 = 0.06062728715262107
  
  rat: replaced 0.007725766724910044 by 1464384/189545459 = 0.007725766724910038
  
  rat: replaced 0.06410317632206519 by 5287663/82486755 = 0.06410317632206528
  
  rat: replaced 0.00838456803503801 by 1113589/132814117 = 0.008384568035038023
  
  rat: replaced 0.06767265439396564 by 2921400/43169579 = 0.06767265439396572
  
  rat: replaced 0.009079530587017326 by 433906/47789475 = 0.00907953058701733
  
  rat: replaced 0.07133536442348987 by 7236103/101437808 = 0.07133536442348991
  
  rat: replaced 0.009811584876838586 by 1363090/138926587 = 0.009811584876838586
  
  rat: replaced 0.07509094014268702 by 9209133/122639735 = 0.07509094014268704
  
  rat: replaced 0.0105816576913495 by 1163729/109976058 = 0.01058165769134951
  
  rat: replaced 0.07893900599711501 by 5197067/65836489 = 0.07893900599711506
  
  rat: replaced 0.01139067201557714 by 13426050/1178688139 = 0.01139067201557714
  
  rat: replaced 0.08287917718339499 by 11217158/135343501 = 0.082879177183395
  
  rat: replaced 0.01223954694042984 by 2283101/186534764 = 0.01223954694042983
  
  rat: replaced 0.08691105968769186 by 5213115/59982182 = 0.08691105968769192
  
  rat: replaced 0.01312919757078923 by 3499615/266552086 = 0.01312919757078922
  
  rat: replaced 0.09103425032511492 by 5893225/64736349 = 0.09103425032511488
  
  rat: replaced 0.01406053493400045 by 2280713/162206702 = 0.01406053493400045
  
  rat: replaced 0.09524833678003664 by 9601787/100807923 = 0.09524833678003662
  
  rat: replaced 0.01503446588876983 by 200490/13335359 = 0.01503446588876985
  
  rat: replaced 0.09955289764732322 by 5687088/57126293 = 0.09955289764732328
  
  rat: replaced 0.01605189303448024 by 951971/59305840 = 0.01605189303448025
  
  rat: replaced 0.1039475024744748 by 10260011/98703776 = 0.1039475024744747
  
  rat: replaced 0.01711371462093175 by 9432386/551159477 = 0.01711371462093176
  
  rat: replaced 0.1084317118046711 by 14939691/137779721 = 0.1084317118046712
  
  rat: replaced 0.01822082445851714 by 2559788/140486947 = 0.01822082445851713
  
  rat: replaced 0.113005077220716 by 8478529/75027859 = 0.1130050772207161
  
  rat: replaced 0.01937411182884202 by 2983799/154009589 = 0.01937411182884203
  
  rat: replaced 0.1176671413898787 by 7123715/60541243 = 0.1176671413898786
  
  rat: replaced 0.02057446139579705 by 7167743/348380590 = 0.02057446139579705
  
  rat: replaced 0.1224174381096274 by 12172179/99431741 = 0.1224174381096274
  
  rat: replaced 0.02182275311709253 by 7415562/339808729 = 0.02182275311709253
  
  rat: replaced 0.1272554923542488 by 7277933/57191504 = 0.127255492354249
  
  rat: replaced 0.02311986215626333 by 2988661/129268115 = 0.02311986215626336
  
  rat: replaced 0.1321808203223502 by 3633064/27485561 = 0.1321808203223503
  
  rat: replaced 0.02446665879515308 by 1991976/81415939 = 0.02446665879515312
  
  rat: replaced 0.1371929294852391 by 56235017/409897341 = 0.1371929294852391
  
  rat: replaced 0.02586400834688696 by 5000736/193347293 = 0.02586400834688697
  
  rat: replaced 0.1422913186361759 by 9349741/65708443 = 0.1422913186361759
  
  rat: replaced 0.02731277106934082 by 858413/31428997 = 0.02731277106934084
  
  rat: replaced 0.1474754779404944 by 1549881/10509415 = 0.1474754779404943
  
  rat: replaced 0.02881380207911666 by 3754753/130310918 = 0.02881380207911666
  
  rat: replaced 0.152744888986584 by 5264425/34465474 = 0.1527448889865841
  
  rat: replaced 0.03036795126603076 by 4118329/135614318 = 0.03036795126603077
  
  rat: replaced 0.1580990248377314 by 5442776/34426373 = 0.1580990248377312
  
  rat: replaced 0.03197606320812652 by 3497683/109384416 = 0.03197606320812647
  
  rat: replaced 0.1635373500848132 by 12328488/75386375 = 0.1635373500848131
  
  rat: replaced 0.0336389770872163 by 3971799/118071337 = 0.03363897708721635
  
  rat: replaced 0.1690593208998367 by 20896917/123607009 = 0.1690593208998367
  
  rat: replaced 0.03535752660496472 by 1815732/51353479 = 0.03535752660496478
  
  rat: replaced 0.1746643850903219 by 2841592/16268869 = 0.1746643850903219
  
  rat: replaced 0.03713253989951881 by 3333721/89778965 = 0.03713253989951878
  
  rat: replaced 0.1803519821545206 by 4461007/24735004 = 0.1803519821545208
  
  rat: replaced 0.03896483946269502 by 8785771/225479461 = 0.03896483946269501
  
  rat: replaced 0.1861215433374662 by 4381209/23539505 = 0.1861215433374661
  
  rat: replaced 0.0408552420577305 by 3189084/78058135 = 0.04085524205773043
  
  rat: replaced 0.1919724916878484 by 72809759/379271834 = 0.1919724916878484
  
  rat: replaced 0.04280455863760801 by 7646593/178639688 = 0.04280455863760801
  
  rat: replaced 0.1979042421157076 by 26318167/132984350 = 0.1979042421157076
  
  rat: replaced 0.04481359426396048 by 20610430/459914683 = 0.04481359426396048
  
  rat: replaced 0.2039162014509444 by 8519416/41779005 = 0.2039162014509441
  
  rat: replaced 0.04688314802656623 by 3439140/73355569 = 0.04688314802656633
  
  rat: replaced 0.2100077685026351 by 50962787/242670961 = 0.2100077685026351
  
  rat: replaced 0.04901401296344043 by 4006732/81746663 = 0.04901401296344048
  
  rat: replaced 0.216178334119151 by 1347531/6233423 = 0.2161783341191509
  
  rat: replaced 0.05120697598153157 by 4148974/81023609 = 0.0512069759815315
  
  rat: replaced 0.2224272812490723 by 23234851/104460437 = 0.2224272812490723
  
  rat: replaced 0.05346281777803219 by 11998448/224426031 = 0.05346281777803218
  
  rat: replaced 0.2287539850028937 by 8185268/35781969 = 0.2287539850028935
  
  rat: replaced 0.05578231276230905 by 1398019/25062048 = 0.05578231276230897
  
  rat: replaced 0.2351578127155118 by 12642104/53760085 = 0.2351578127155119
  
  rat: replaced 0.05816622897846346 by 4451048/76522891 = 0.05816622897846345
  
  rat: replaced 0.2416381240094921 by 8002142/33116223 = 0.2416381240094923
  
  rat: replaced 0.06061532802852698 by 2146337/35409146 = 0.06061532802852686
  
  rat: replaced 0.2481942708591053 by 8882901/35790113 = 0.2481942708591057
  
  rat: replaced 0.0631303649963022 by 14651447/232082406 = 0.06313036499630222
  
  rat: replaced 0.2548255976551299 by 868346/3407609 = 0.25482559765513
  
  rat: replaced 0.06571208837185505 by 4240309/64528599 = 0.06571208837185509
  
  rat: replaced 0.2615314412704124 by 8212450/31401387 = 0.2615314412704127
  
  rat: replaced 0.06836123997666599 by 2716643/39739522 = 0.06836123997666604
  
  rat: replaced 0.2683111311261794 by 34459769/128432126 = 0.2683111311261794
  
  rat: replaced 0.07107855488944881 by 3146673/44270357 = 0.07107855488944893
  
  rat: replaced 0.2751639892590951 by 12552159/45617012 = 0.2751639892590949
  
  rat: replaced 0.07386476137264342 by 12898997/174629915 = 0.0738647613726434
  
  rat: replaced 0.2820893303890569 by 11134456/39471383 = 0.2820893303890568
  
  rat: replaced 0.07672058079958999 by 5073506/66129661 = 0.07672058079959007
  
  rat: replaced 0.2890864619877229 by 9583357/33150487 = 0.2890864619877228
  
  rat: replaced 0.07964672758239233 by 5672399/71219486 = 0.07964672758239227
  
  rat: replaced 0.2961546843477643 by 11052271/37319251 = 0.2961546843477647
  
  rat: replaced 0.08264390910047736 by 4686067/56701904 = 0.08264390910047748
  
  rat: replaced 0.3032932906528349 by 9918077/32701274 = 0.3032932906528351
  
  rat: replaced 0.0857128256298576 by 3585977/41837111 = 0.08571282562985766
  
  rat: replaced 0.3105015670482534 by 9320011/30015987 = 0.3105015670482533
  
  rat: replaced 0.08885417027310427 by 5751353/64728003 = 0.0888541702731042
  
  rat: replaced 0.3177787927123868 by 248395525/781661743 = 0.3177787927123868
  
  rat: replaced 0.09206862889003742 by 7305460/79347983 = 0.09206862889003745
  
  rat: replaced 0.3251242399287333 by 13842845/42577093 = 0.3251242399287335
  
  rat: replaced 0.09535688002914089 by 5971998/62627867 = 0.09535688002914103
  
  rat: replaced 0.3325371741586922 by 9318229/28021616 = 0.3325371741586923
  
  rat: replaced 0.0987195948597075 by 9821211/99485933 = 0.09871959485970745
  
  rat: replaced 0.3400168541150183 by 13391981/39386227 = 0.3400168541150184
  
  rat: replaced 0.1021574371047232 by 8336413/81603584 = 0.1021574371047232
  
  rat: replaced 0.3475625318359485 by 10097818/29053241 = 0.347562531835949
  
  rat: replaced 0.1056710629744951 by 5741011/54329074 = 0.105671062974495
  
  rat: replaced 0.3551734527599992 by 15867851/44676343 = 0.3551734527599987
  
  rat: replaced 0.1092611211010309 by 5551873/50812887 = 0.1092611211010309
  
  rat: replaced 0.3628488558014202 by 6897641/19009681 = 0.3628488558014203
  
  rat: replaced 0.1129282524731764 by 11548693/102265755 = 0.1129282524731764
  
  rat: replaced 0.3705879734263036 by 23358661/63031352 = 0.3705879734263038
  
  rat: replaced 0.1166730903725168 by 5656228/48479285 = 0.1166730903725168
  
  rat: replaced 0.3783900317293359 by 14241382/37636779 = 0.3783900317293358
  
  rat: replaced 0.1204962603100498 by 4057613/33674182 = 0.12049626031005
  
  rat: replaced 0.3862542505111889 by 3461217/8960981 = 0.3862542505111884
  
  rat: replaced 0.1243983799636342 by 7966447/64039797 = 0.1243983799636342
  
  rat: replaced 0.3941798433565377 by 5314214/13481699 = 0.3941798433565384
  
  rat: replaced 0.1283800591162231 by 796346/6203035 = 0.1283800591162229
  
  rat: replaced 0.4021660177127022 by 11567173/28762184 = 0.4021660177127022
  
  rat: replaced 0.1324418995948859 by 4716124/35609003 = 0.1324418995948862
  
  rat: replaced 0.4102119749689023 by 11320633/27597032 = 0.4102119749689024
  
  rat: replaced 0.1365844952106265 by 612971/4487852 = 0.1365844952106264
  
  rat: replaced 0.418316910536117 by 12225195/29224721 = 0.4183169105361177
  
  rat: replaced 0.140808431699002 by 10431632/74083859 = 0.1408084316990021
  
  rat: replaced 0.4264800139275439 by 7978696/18708253 = 0.4264800139275431
  
  rat: replaced 0.1451142866615502 by 3554077/24491572 = 0.1451142866615504
  
  rat: replaced 0.4347004688396462 by 20489554/47134879 = 0.4347004688396463
  
  rat: replaced 0.1495026295080298 by 26759297/178988805 = 0.1495026295080298
  
  rat: replaced 0.4429774532337832 by 23449796/52936771 = 0.4429774532337834
  
  rat: replaced 0.1539740213994798 by 16145763/104860306 = 0.1539740213994798
  
  rat: replaced 0.451310139418413 by 8841241/19590167 = 0.4513101394184133
  
  rat: replaced -1.66665833335744e-7 by -15819/94914474571 = -1.66665833335744e-7
  
  rat: replaced -1.33330666692022e-6 by -31771/23828726570 = -1.333306666920221e-6
  
  rat: replaced -4.499797504338432e-6 by -24036/5341573699 = -4.499797504338431e-6
  
  rat: replaced -1.066581336583994e-5 by -58861/5518660226 = -1.066581336583993e-5
  
  rat: replaced -2.083072932167196e-5 by -35635/1710693824 = -2.0830729321672e-5
  
  rat: replaced -3.599352055540239e-5 by -98277/2730408098 = -3.599352055540234e-5
  
  rat: replaced -5.71526624672386e-5 by -51154/895041417 = -5.715266246723866e-5
  
  rat: replaced -8.530603082730626e-5 by -121691/1426522824 = -8.530603082730627e-5
  
  rat: replaced -1.214508019889565e-4 by -158455/1304684674 = -1.214508019889563e-4
  
  rat: replaced -1.665833531718508e-4 by -142521/855553675 = -1.66583353171851e-4
  
  rat: replaced -2.216991628251896e-4 by -179571/809975995 = -2.216991628251896e-4
  
  rat: replaced -2.877927110806339e-4 by -1167733/4057548906 = -2.877927110806339e-4
  
  rat: replaced -3.658573803051457e-4 by -386279/1055818526 = -3.658573803051454e-4
  
  rat: replaced -4.5688535576352e-4 by -262978/575588595 = -4.568853557635206e-4
  
  rat: replaced -5.618675264007778e-4 by -150595/268025812 = -5.618675264007782e-4
  
  rat: replaced -6.817933857540259e-4 by -192316/282073725 = -6.817933857540258e-4
  
  rat: replaced -8.176509330039827e-4 by -105841/129445214 = -8.176509330039812e-4
  
  rat: replaced -9.704265741758145e-4 by -651321/671169790 = -9.704265741758132e-4
  
  rat: replaced -0.001141105023499428 by -1259907/1104111343 = -0.001141105023499428
  
  rat: replaced -0.001330669204938795 by -1231154/925214167 = -0.001330669204938796
  
  rat: replaced -0.001540100153900437 by -276884/179783113 = -0.001540100153900439
  
  rat: replaced -0.001770376919130678 by -644389/363984072 = -0.001770376919130681
  
  rat: replaced -0.002022476464811601 by -1271955/628909667 = -0.002022476464811599
  
  rat: replaced -0.002297373572865413 by -1020913/444382669 = -0.002297373572865417
  
  rat: replaced -0.002596040745477063 by -1097643/422814242 = -0.002596040745477065
  
  rat: replaced -0.002919448107844891 by -906221/310408326 = -0.002919448107844891
  
  rat: replaced -0.003268563311168871 by -1379071/421919623 = -0.003268563311168867
  
  rat: replaced -0.003644351435886262 by -5966577/1637212301 = -0.003644351435886261
  
  rat: replaced -0.004047774895164447 by -572425/141417202 = -0.004047774895164451
  
  rat: replaced -0.004479793338660443 by -2952779/659132861 = -0.004479793338660444
  
  rat: replaced -0.0049413635565565 by -2524919/510976165 = -0.004941363556556498
  
  rat: replaced -0.005433439383882244 by -1361584/250593391 = -0.005433439383882235
  
  rat: replaced -0.005956971605131645 by -1447422/242979503 = -0.005956971605131648
  
  rat: replaced -0.006512907859185624 by -3695063/567344584 = -0.006512907859185626
  
  rat: replaced -0.007102192544548636 by -1363981/192050693 = -0.007102192544548642
  
  rat: replaced -0.007725766724910044 by -1464384/189545459 = -0.007725766724910038
  
  rat: replaced -0.00838456803503801 by -1113589/132814117 = -0.008384568035038023
  
  rat: replaced -0.009079530587017326 by -433906/47789475 = -0.00907953058701733
  
  rat: replaced -0.009811584876838586 by -1363090/138926587 = -0.009811584876838586
  
  rat: replaced -0.0105816576913495 by -1163729/109976058 = -0.01058165769134951
  
  rat: replaced -0.01139067201557714 by -13426050/1178688139 = -0.01139067201557714
  
  rat: replaced -0.01223954694042984 by -2283101/186534764 = -0.01223954694042983
  
  rat: replaced -0.01312919757078923 by -3499615/266552086 = -0.01312919757078922
  
  rat: replaced -0.01406053493400045 by -2280713/162206702 = -0.01406053493400045
  
  rat: replaced -0.01503446588876983 by -200490/13335359 = -0.01503446588876985
  
  rat: replaced -0.01605189303448024 by -951971/59305840 = -0.01605189303448025
  
  rat: replaced -0.01711371462093175 by -9432386/551159477 = -0.01711371462093176
  
  rat: replaced -0.01822082445851714 by -2559788/140486947 = -0.01822082445851713
  
  rat: replaced -0.01937411182884202 by -2983799/154009589 = -0.01937411182884203
  
  rat: replaced -0.02057446139579705 by -7167743/348380590 = -0.02057446139579705
  
  rat: replaced -0.02182275311709253 by -7415562/339808729 = -0.02182275311709253
  
  rat: replaced -0.02311986215626333 by -2988661/129268115 = -0.02311986215626336
  
  rat: replaced -0.02446665879515308 by -1991976/81415939 = -0.02446665879515312
  
  rat: replaced -0.02586400834688696 by -5000736/193347293 = -0.02586400834688697
  
  rat: replaced -0.02731277106934082 by -858413/31428997 = -0.02731277106934084
  
  rat: replaced -0.02881380207911666 by -3754753/130310918 = -0.02881380207911666
  
  rat: replaced -0.03036795126603076 by -4118329/135614318 = -0.03036795126603077
  
  rat: replaced -0.03197606320812652 by -3497683/109384416 = -0.03197606320812647
  
  rat: replaced -0.0336389770872163 by -3971799/118071337 = -0.03363897708721635
  
  rat: replaced -0.03535752660496472 by -1815732/51353479 = -0.03535752660496478
  
  rat: replaced -0.03713253989951881 by -3333721/89778965 = -0.03713253989951878
  
  rat: replaced -0.03896483946269502 by -8785771/225479461 = -0.03896483946269501
  
  rat: replaced -0.0408552420577305 by -3189084/78058135 = -0.04085524205773043
  
  rat: replaced -0.04280455863760801 by -7646593/178639688 = -0.04280455863760801
  
  rat: replaced -0.04481359426396048 by -20610430/459914683 = -0.04481359426396048
  
  rat: replaced -0.04688314802656623 by -3439140/73355569 = -0.04688314802656633
  
  rat: replaced -0.04901401296344043 by -4006732/81746663 = -0.04901401296344048
  
  rat: replaced -0.05120697598153157 by -4148974/81023609 = -0.0512069759815315
  
  rat: replaced -0.05346281777803219 by -11998448/224426031 = -0.05346281777803218
  
  rat: replaced -0.05578231276230905 by -1398019/25062048 = -0.05578231276230897
  
  rat: replaced -0.05816622897846346 by -4451048/76522891 = -0.05816622897846345
  
  rat: replaced -0.06061532802852698 by -2146337/35409146 = -0.06061532802852686
  
  rat: replaced -0.0631303649963022 by -14651447/232082406 = -0.06313036499630222
  
  rat: replaced -0.06571208837185505 by -4240309/64528599 = -0.06571208837185509
  
  rat: replaced -0.06836123997666599 by -2716643/39739522 = -0.06836123997666604
  
  rat: replaced -0.07107855488944881 by -3146673/44270357 = -0.07107855488944893
  
  rat: replaced -0.07386476137264342 by -12898997/174629915 = -0.0738647613726434
  
  rat: replaced -0.07672058079958999 by -5073506/66129661 = -0.07672058079959007
  
  rat: replaced -0.07964672758239233 by -5672399/71219486 = -0.07964672758239227
  
  rat: replaced -0.08264390910047736 by -4686067/56701904 = -0.08264390910047748
  
  rat: replaced -0.0857128256298576 by -3585977/41837111 = -0.08571282562985766
  
  rat: replaced -0.08885417027310427 by -5751353/64728003 = -0.0888541702731042
  
  rat: replaced -0.09206862889003742 by -7305460/79347983 = -0.09206862889003745
  
  rat: replaced -0.09535688002914089 by -5971998/62627867 = -0.09535688002914103
  
  rat: replaced -0.0987195948597075 by -9821211/99485933 = -0.09871959485970745
  
  rat: replaced -0.1021574371047232 by -8336413/81603584 = -0.1021574371047232
  
  rat: replaced -0.1056710629744951 by -5741011/54329074 = -0.105671062974495
  
  rat: replaced -0.1092611211010309 by -5551873/50812887 = -0.1092611211010309
  
  rat: replaced -0.1129282524731764 by -11548693/102265755 = -0.1129282524731764
  
  rat: replaced -0.1166730903725168 by -5656228/48479285 = -0.1166730903725168
  
  rat: replaced -0.1204962603100498 by -4057613/33674182 = -0.12049626031005
  
  rat: replaced -0.1243983799636342 by -7966447/64039797 = -0.1243983799636342
  
  rat: replaced -0.1283800591162231 by -796346/6203035 = -0.1283800591162229
  
  rat: replaced -0.1324418995948859 by -4716124/35609003 = -0.1324418995948862
  
  rat: replaced -0.1365844952106265 by -612971/4487852 = -0.1365844952106264
  
  rat: replaced -0.140808431699002 by -10431632/74083859 = -0.1408084316990021
  
  rat: replaced -0.1451142866615502 by -3554077/24491572 = -0.1451142866615504
  
  rat: replaced -0.1495026295080298 by -26759297/178988805 = -0.1495026295080298
  
  rat: replaced -0.1539740213994798 by -16145763/104860306 = -0.1539740213994798
  part: invalid index of list or matrix.
  #0: lineIntersection(g=[a1,a2,a1^2/2+a2^2/2],h=[-a,0,-a^2/2])
   -- an error. To debug this try: debugmode(true);
  
  Error in:
  U &= lineIntersection(h,g); ...
                            ^
\end{euleroutput}
\begin{eulercomment}
Kami mendapatkan rumus untuk jari-jari lingkaran.
\end{eulercomment}
\begin{eulerprompt}
>&assume(a>0,b>0,c>0); $distance(U,B) | radcan
\end{eulerprompt}
\begin{eulercomment}
Mari kita tambahkan ini ke plot.
\end{eulercomment}
\begin{eulerprompt}
>plotPoint(U()); ...
>plotCircle(circleWithCenter(mxmeval("U"),mxmeval("distance(U,C)"))):
\end{eulerprompt}
\begin{euleroutput}
  Function U not found.
  Try list ... to find functions!
  Error in:
  plotPoint(U()); plotCircle(circleWithCenter(mxmeval("U"),mxmev ...
               ^
\end{euleroutput}
\begin{eulercomment}
Menggunakan geometri, kami memperoleh rumus sederhana

\end{eulercomment}
\begin{eulerformula}
\[
\frac{a}{\sin(\alpha)}=2r
\]
\end{eulerformula}
\begin{eulercomment}
untuk radiusnya. Kami dapat memeriksa, apakah ini benar dengan Maxima.
Maxima akan memfaktorkan ini hanya jika kita kuadratkan.
\end{eulercomment}
\begin{eulerprompt}
>$c^2/sin(computeAngle(A,B,C))^2  | factor
\end{eulerprompt}
\eulersubheading{Contoh 4: Garis Euler dan Parabola}
\begin{eulercomment}
Garis Euler adalah garis yang ditentukan dari sembarang segitiga yang
tidak sama sisi. Ini adalah garis tengah segitiga, dan melewati
beberapa titik penting yang ditentukan dari segitiga, termasuk
orthocenter, circumcenter, centroid, titik Exeter dan pusat lingkaran
sembilan titik segitiga.

Untuk demonstrasi, kami menghitung dan memplot garis Euler dalam
sebuah segitiga.

Pertama, kita mendefinisikan sudut-sudut segitiga di Euler. Kami
menggunakan definisi, yang terlihat dalam ekspresi simbolis.
\end{eulercomment}
\begin{eulerprompt}
>A::=[-1,-1]; B::=[2,0]; C::=[1,2];
\end{eulerprompt}
\begin{eulercomment}
Untuk memplot objek geometris, kami menyiapkan area plot, dan
menambahkan titik ke sana. Semua plot objek geometris ditambahkan ke
plot saat ini.
\end{eulercomment}
\begin{eulerprompt}
>setPlotRange(3); plotPoint(A,"A"); plotPoint(B,"B"); plotPoint(C,"C");
\end{eulerprompt}
\begin{eulercomment}
Kita juga bisa menambahkan sisi segitiga.
\end{eulercomment}
\begin{eulerprompt}
>plotSegment(A,B,""); plotSegment(B,C,""); plotSegment(C,A,""):
\end{eulerprompt}
\begin{eulercomment}
Berikut adalah luas segitiga, menggunakan rumus determinan. Tentu
saja, kita harus mengambil nilai absolut dari hasil ini.
\end{eulercomment}
\begin{eulerprompt}
>$areaTriangle(A,B,C)
\end{eulerprompt}
\begin{eulercomment}
Kita dapat menghitung koefisien sisi c.
\end{eulercomment}
\begin{eulerprompt}
>c &= lineThrough(A,B)
\end{eulerprompt}
\begin{euleroutput}
  
                              [- 1, 3, - 2]
  
\end{euleroutput}
\begin{eulercomment}
Dan juga dapatkan rumus untuk baris ini.
\end{eulercomment}
\begin{eulerprompt}
>$getLineEquation(c,x,y)
\end{eulerprompt}
\begin{eulercomment}
Untuk bentuk Hesse, kita perlu menentukan sebuah titik, sehingga titik
tersebut berada di sisi positif dari bentuk Hesse. Memasukkan titik
menghasilkan jarak positif ke garis.
\end{eulercomment}
\begin{eulerprompt}
>$getHesseForm(c,x,y,C), $at(%,[x=C[1],y=C[2]])
\end{eulerprompt}
\begin{eulercomment}
Sekarang kita hitung lingkaran luar ABC.
\end{eulercomment}
\begin{eulerprompt}
>LL &= circleThrough(A,B,C); $getCircleEquation(LL,x,y)
\end{eulerprompt}
\begin{euleroutput}
  Maxima said:
  rat: replaced -5.049958083474387e-5 by -102157/2022927682 = -5.049958083474385e-5
  
  rat: replaced -2.039932534230044e-4 by -284619/1395237319 = -2.039932534230043e-4
  
  rat: replaced -4.634656435254722e-4 by -573493/1237401322 = -4.634656435254721e-4
  
  rat: replaced -8.31890779119604e-4 by -332331/399488741 = -8.318907791196046e-4
  
  rat: replaced -0.001312231792998733 by -448125/341498356 = -0.001312231792998734
  
  rat: replaced -0.001907440626462018 by -276030/144712237 = -0.001907440626462018
  
  rat: replaced -0.002620457734122131 by -2586613/987084419 = -0.002620457734122131
  
  rat: replaced -0.00345421178986248 by -3402379/984994322 = -0.00345421178986248
  
  rat: replaced -0.004411619393972596 by -966955/219183686 = -0.004411619393972597
  
  rat: replaced -0.005495584781489732 by -2798484/509224061 = -0.005495584781489734
  
  rat: replaced -0.006708999531778753 by -6054060/902378957 = -0.006708999531778753
  
  rat: replaced -0.008054742279375651 by -806546/100133061 = -0.00805474227937564
  
  rat: replaced -0.009535678426127348 by -4115324/431571181 = -0.009535678426127346
  
  rat: replaced -0.01115465985465333 by -2266398/203179481 = -0.01115465985465334
  
  rat: replaced -0.01291452464316009 by -2106925/163143829 = -0.01291452464316012
  
  rat: replaced -0.01481809678163515 by -2779203/187554653 = -0.01481809678163516
  
  rat: replaced -0.01686818588945119 by -7427428/440321683 = -0.01686818588945119
  
  rat: replaced -0.01906758693440599 by -2278085/119474216 = -0.019067586934406
  
  rat: replaced -0.02141907995322798 by -2316386/108145915 = -0.02141907995322801
  
  rat: replaced -0.02392542977357476 by -1665518/69612877 = -0.02392542977357479
  
  rat: replaced -0.02658938573755304 by -3678645/138350131 = -0.02658938573755308
  
  rat: replaced -0.02941368142678652 by -4053557/137811957 = -0.0294136814267865
  
  rat: replaced -0.03240103438906003 by -2629160/81144323 = -0.03240103438906009
  
  rat: replaced -0.03555414586656669 by -1834427/51595305 = -0.03555414586656674
  
  rat: replaced -0.03887570052578646 by -1643964/42287701 = -0.03887570052578645
  
  rat: replaced -0.04236836618902146 by -3055464/72116635 = -0.04236836618902143
  
  rat: replaced -0.04603479356761608 by -3139251/68193007 = -0.0460347935676161
  
  rat: replaced -0.04987761599688789 by -5203437/104324092 = -0.04987761599688785
  
  rat: replaced -0.05389944917279615 by -4533622/84112585 = -0.0538994491727962
  
  rat: replaced -0.05810289089037535 by -11687290/201148167 = -0.05810289089037535
  
  rat: replaced -0.06249052078395612 by -3949243/63197473 = -0.06249052078395603
  
  rat: replaced -0.0670649000692059 by -3281728/48933615 = -0.067064900069206
  
  rat: replaced -0.07182857128700804 by -4146139/57722699 = -0.07182857128700791
  
  rat: replaced -0.07678405804921068 by -1198255/15605518 = -0.07678405804921054
  
  rat: replaced -0.08193386478626702 by -5956639/72700574 = -0.08193386478626702
  
  rat: replaced -0.08728047649679532 by -2799808/32078285 = -0.08728047649679527
  
  rat: replaced -0.09282635849907966 by -10292829/110882611 = -0.09282635849907972
  
  rat: replaced -0.09857395618454184 by -4198057/42587892 = -0.09857395618454184
  
  rat: replaced -0.1045256947732028 by -30563827/292404916 = -0.1045256947732028
  
  rat: replaced -0.1106839790711635 by -8949559/80856860 = -0.1106839790711635
  
  rat: replaced -0.1170511932301264 by -9911603/84677505 = -0.1170511932301265
  
  rat: replaced -0.1236297005089814 by -19561703/158228184 = -0.1236297005089814
  
  rat: replaced -0.1304218430374826 by -4975231/38147222 = -0.1304218430374825
  
  rat: replaced -0.137429941582038 by -3502939/25488907 = -0.137429941582038
  
  rat: replaced -0.1446562953136327 by -15521432/107298697 = -0.1446562953136327
  
  rat: replaced -0.1521031815779155 by -18080502/118869979 = -0.1521031815779155
  
  rat: replaced -0.1597728556674664 by -37419026/234201397 = -0.1597728556674664
  
  rat: replaced -0.1676675505962674 by -22585897/134706429 = -0.1676675505962674
  
  rat: replaced -0.1757894768764047 by -15940893/90681725 = -0.1757894768764048
  
  rat: replaced -0.1841408222970185 by -7944795/43145213 = -0.1841408222970182
  
  rat: replaced -0.1927237517055264 by -1392861/7227241 = -0.1927237517055264
  
  rat: replaced -0.2015404067911402 by -1735485/8611102 = -0.2015404067911401
  
  rat: replaced -0.2105929058706983 by -10627754/50465869 = -0.2105929058706985
  
  rat: replaced -0.2198833436768368 by -9372347/42624179 = -0.2198833436768366
  
  rat: replaced -0.2294137911485169 by -7405273/32279110 = -0.2294137911485168
  
  rat: replaced -0.239186295223934 by -27692337/115777273 = -0.239186295223934
  
  rat: replaced -0.2492028786358237 by -8925310/35815437 = -0.249202878635824
  
  rat: replaced -0.2594655397091927 by -11150701/42975653 = -0.259465539709193
  
  rat: replaced -0.2699762521614856 by -11249087/41666950 = -0.2699762521614853
  
  rat: replaced -0.280736964905216 by -12097010/43090193 = -0.2807369649052164
  
  rat: replaced -0.2917496018530771 by -14831788/50837389 = -0.2917496018530771
  
  rat: replaced -0.3030160617255513 by -18597622/61375037 = -0.3030160617255514
  
  rat: replaced -0.3145382178610399 by -11102944/35299189 = -0.3145382178610392
  
  rat: replaced -0.3263179180285316 by -13053510/40002431 = -0.3263179180285318
  
  rat: replaced -0.3383569842428258 by -13796661/40775458 = -0.3383569842428257
  
  rat: replaced -0.3506572125823338 by -33496033/95523582 = -0.3506572125823338
  
  rat: replaced -0.3632203730094723 by -23086207/63559780 = -0.3632203730094724
  
  rat: replaced -0.376048209193667 by -22674222/60296051 = -0.3760482091936668
  
  rat: replaced -0.3891424383369902 by -33246815/85436107 = -0.3891424383369902
  
  rat: replaced -0.402504751002439 by -7793813/19363282 = -0.4025047510024385
  
  rat: replaced -0.4161368109448825 by -10481453/25187517 = -0.4161368109448819
  
  rat: replaced -0.4300402549446862 by -19565443/45496771 = -0.4300402549446861
  
  rat: replaced -0.4442166926440365 by -16102633/36249500 = -0.4442166926440365
  
  rat: replaced -0.4586677063859775 by -19404529/42306290 = -0.4586677063859771
  
  rat: replaced -0.4733948510561774 by -10262860/21679281 = -0.4733948510561766
  
  rat: replaced -0.4883996539274416 by -4159841/8517289 = -0.488399653927441
  
  rat: replaced -0.5036836145069872 by -13202363/26211619 = -0.5036836145069864
  
  rat: replaced -0.5192482043864929 by -12221370/23536663 = -0.5192482043864927
  
  rat: replaced -0.5350948670949413 by -52965833/98984005 = -0.5350948670949413
  
  rat: replaced -0.551225017954267 by -14288533/25921416 = -0.551225017954266
  
  rat: replaced -0.5676400439378262 by -25565995/45039097 = -0.5676400439378259
  
  rat: replaced -0.5843413035316997 by -18888222/32323955 = -0.5843413035316997
  
  rat: replaced -0.6013301265988455 by -5789399/9627655 = -0.6013301265988447
  
  rat: replaced -0.6186078142461149 by -4803773/7765458 = -0.618607814246114
  
  rat: replaced -0.6361756386941407 by -13914515/21872128 = -0.6361756386941407
  
  rat: replaced -0.6540348431501183 by -48160581/73636109 = -0.6540348431501181
  
  rat: replaced -0.6721866416834846 by -6617334/9844489 = -0.6721866416834841
  
  rat: replaced -0.690632219104513 by -16840135/24383651 = -0.6906322191045139
  
  rat: replaced -0.7093727308458327 by -29189494/41148317 = -0.7093727308458326
  
  rat: replaced -0.7284093028468864 by -13153959/18058472 = -0.7284093028468854
  
  rat: replaced -0.7477430314413382 by -12236470/16364539 = -0.7477430314413379
  
  rat: replaced -0.7673749832474404 by -39576757/51574208 = -0.7673749832474402
  
  rat: replaced -0.7873061950613714 by -6818881/8661028 = -0.7873061950613714
  
  rat: replaced -0.8075376737535601 by -20498953/25384516 = -0.807537673753559
  
  rat: replaced -0.8280703961679966 by -6989671/8440914 = -0.8280703961679979
  
  rat: replaced -0.84890530902455 by -25231431/29722315 = -0.8489053090245494
  
  rat: replaced -0.8700433288242969 by -9721738/11173855 = -0.8700433288242957
  
  rat: replaced -0.8914853417578728 by -33469619/37543656 = -0.8914853417578725
  
  rat: replaced -0.9132322036168524 by -21961040/24047597 = -0.913232203616852
  
  rat: replaced -1.503320816708814e-4 by -144269/959668744 = -1.503320816708812e-4
  
  rat: replaced -6.026466136005715e-4 by -554629/920322105 = -6.026466136005719e-4
  
  rat: replaced -0.001358898348046048 by -845667/622318072 = -0.001358898348046045
  
  rat: replaced -0.002421011643797932 by -1882206/777446075 = -0.002421011643797931
  
  rat: replaced -0.003790880273744496 by -1271426/335390703 = -0.003790880273744499
  
  rat: replaced -0.005470367235498236 by -1827828/334132595 = -0.005470367235498231
  
  rat: replaced -0.007461304565095722 by -763704/102355291 = -0.007461304565095712
  
  rat: replaced -0.009765493153796295 by -1498127/153410276 = -0.009765493153796295
  
  rat: replaced -0.01238470256799509 by -4127047/333237474 = -0.0123847025679951
  
  rat: replaced -0.01532067087226624 by -2377723/155197055 = -0.01532067087226623
  
  rat: replaced -0.01857510445555993 by -4650073/250338996 = -0.01857510445555993
  
  rat: replaced -0.02214967786056252 by -1400096/63210671 = -0.0221496778605625
  
  rat: replaced -0.02604603361624602 by -2361294/90658487 = -0.02604603361624599
  
  rat: replaced -0.03026578207361535 by -1660222/54854753 = -0.03026578207361539
  
  rat: replaced -0.03481050124467483 by -4348324/124914145 = -0.03481050124467489
  
  rat: replaced -0.03968173664462726 by -9988401/251712799 = -0.03968173664462728
  
  rat: replaced -0.04488100113732568 by -9262688/206383275 = -0.04488100113732569
  
  rat: replaced -0.05040977478398728 by -8488548/168390913 = -0.0504097747839873
  
  rat: replaced -0.05626950469518793 by -3521587/62584290 = -0.05626950469518788
  
  rat: replaced -0.06246160488615271 by -1724725/27612563 = -0.06246160488615272
  
  rat: replaced -0.06898745613535606 by -699061/10133161 = -0.06898745613535599
  
  rat: replaced -0.07584840584644481 by -7595322/100138189 = -0.07584840584644485
  
  rat: replaced -0.08304576791349888 by -2278089/27431729 = -0.083045767913499
  
  rat: replaced -0.09058082258964217 by -5411518/59742425 = -0.09058082258964212
  
  rat: replaced -0.09845481635901993 by -9650917/98023818 = -0.09845481635902001
  
  rat: replaced -0.1066689618121501 by -6098878/57175751 = -0.10666896181215
  
  rat: replaced -0.1152244375246662 by -22751561/197454303 = -0.1152244375246662
  
  rat: replaced -0.1241223879394598 by -36591823/294804375 = -0.1241223879394599
  
  rat: replaced -0.1333639232522373 by -2621363/19655713 = -0.1333639232522371
  
  rat: replaced -0.1429501193005029 by -14344978/100349535 = -0.142950119300503
  
  rat: replaced -0.1528820174559729 by -7599779/49710091 = -0.1528820174559729
  
  rat: replaced -0.163160624520442 by -10213526/62597983 = -0.1631606245204418
  
  rat: replaced -0.1737869126251026 by -16586501/95441600 = -0.1737869126251027
  
  rat: replaced -0.1847618191333327 by -5716910/30942053 = -0.1847618191333329
  
  rat: replaced -0.1960862465469606 by -4485287/22874052 = -0.1960862465469607
  
  rat: replaced -0.2077610624160157 by -6172898/29711525 = -0.2077610624160153
  
  rat: replaced -0.2197870992519729 by -9779676/44496133 = -0.2197870992519732
  
  rat: replaced -0.2321651544445043 by -12225287/52657717 = -0.2321651544445043
  
  rat: replaced -0.2448959901817382 by -6355852/25953271 = -0.2448959901817385
  
  rat: replaced -0.257980333374044 by -7376119/28591788 = -0.2579803333740443
  
  rat: replaced -0.2714188755813393 by -7521403/27711422 = -0.2714188755813397
  
  rat: replaced -0.2852122729439353 by -33856936/118707851 = -0.2852122729439353
  
  rat: replaced -0.2993611461169232 by -11605920/38768959 = -0.2993611461169231
  
  rat: replaced -0.3138660802081108 by -12935015/41211892 = -0.3138660802081108
  
  rat: replaced -0.3287276247195093 by -4721149/14361887 = -0.3287276247195093
  
  rat: replaced -0.3439462934923849 by -29044045/84443547 = -0.3439462934923849
  
  rat: replaced -0.359522564655877 by -32233940/89657627 = -0.359522564655877
  
  rat: replaced -0.3754568805791821 by -37000079/98546813 = -0.3754568805791822
  
  rat: replaced -0.39174964782732 by -19372366/49450883 = -0.3917496478273199
  
  rat: replaced -0.4084012371204762 by -18093351/44302880 = -0.4084012371204762
  
  rat: replaced -0.4254119832969315 by -17508065/41155552 = -0.4254119832969316
  
  rat: replaced -0.4427821852795774 by -7324893/16542881 = -0.4427821852795774
  
  rat: replaced -0.4605121060460234 by -10003471/21722493 = -0.4605121060460233
  
  rat: replaced -0.4786019726023018 by -60236280/125858821 = -0.4786019726023018
  
  rat: replaced -0.4970519759601649 by -15490512/31164773 = -0.497051975960165
  
  rat: replaced -0.5158622711179853 by -5589426/10835113 = -0.5158622711179847
  
  rat: replaced -0.5350329770452558 by -7868837/14707200 = -0.5350329770452568
  
  rat: replaced -0.5545641766706926 by -12815301/23108779 = -0.554564176670693
  
  rat: replaced -0.5744559168739426 by -13141311/22876100 = -0.5744559168739427
  
  rat: replaced -0.5947082084808951 by -14675891/24677465 = -0.5947082084808955
  
  rat: replaced -0.6153210262625995 by -19715525/32041039 = -0.6153210262626003
  
  rat: replaced -0.6362943089377887 by -24594815/38653206 = -0.636294308937789
  
  rat: replaced -0.6576279591790061 by -15639887/23782272 = -0.6576279591790053
  
  rat: replaced -0.6793218436223387 by -17047367/25094684 = -0.6793218436223385
  
  rat: replaced -0.701375792880754 by -24937969/35555788 = -0.701375792880754
  
  rat: replaced -0.7237896015610379 by -12792911/17674903 = -0.7237896015610382
  
  rat: replaced -0.7465630282843339 by -24894563/33345561 = -0.7465630282843344
  
  rat: replaced -0.76969579571028 by -11030405/14330863 = -0.7696957957102792
  
  rat: replaced -0.7931875905647454 by -17983947/22673006 = -0.7931875905647447
  
  rat: replaced -0.8170380636711536 by -35512807/43465303 = -0.817038063671154
  
  rat: replaced -0.8412468299854033 by -9419201/11196715 = -0.841246829985402
  
  rat: replaced -0.8658134686343698 by -33824443/39066663 = -0.8658134686343699
  
  rat: replaced -0.8907375229579941 by -1798799/2019449 = -0.890737522957995
  
  rat: replaced -0.9160185005549473 by -21232969/23179629 = -0.9160185005549485
  
  rat: replaced -0.9416558733318703 by -11919739/12658275 = -0.9416558733318718
  
  rat: replaced -0.967649077556183 by -17561986/18149127 = -0.9676490775561821
  
  rat: replaced -0.9939975139124576 by -37819354/38047735 = -0.9939975139124576
  
  rat: replaced -1.020700547562349 by -18875311/18492506 = -1.020700547562348
  
  rat: replaced -1.047757508208077 by -16616467/15859077 = -1.047757508208075
  
  rat: replaced -1.07516769015946 by -21797467/20273551 = -1.07516769015946
  
  rat: replaced -1.102930352404475 by -30072842/27266311 = -1.102930352404474
  
  rat: replaced -1.131044718683369 by -7906291/6990255 = -1.131044718683367
  
  rat: replaced -1.159509977566275 by -26421893/22787120 = -1.159509977566274
  
  rat: replaced -1.188325282534358 by -19245269/16195287 = -1.188325282534357
  
  rat: replaced -1.21748975206447 by -18221771/14966673 = -1.21748975206447
  
  rat: replaced -1.247002469717292 by -21540268/17273637 = -1.247002469717292
  
  rat: replaced -1.276862484228988 by -17179259/13454275 = -1.27686248422899
  
  rat: replaced -1.307068809606323 by -22169845/16961498 = -1.307068809606321
  
  rat: replaced -1.337620425225263 by -48765573/36456959 = -1.337620425225263
  
  rat: replaced -1.368516275933041 by -117856634/86120009 = -1.368516275933041
  
  rat: replaced -1.399755272153666 by -29694085/21213769 = -1.399755272153666
  
  rat: replaced -1.431336289996881 by -23110861/16146353 = -1.43133628999688
  
  rat: replaced -1.463258171370553 by -19245288/13152353 = -1.463258171370553
  
  rat: replaced -1.495519724096479 by -43164951/28862843 = -1.495519724096479
  
  rat: replaced -1.528119722029604 by -31224680/20433399 = -1.528119722029605
  
  rat: replaced -1.561056905180636 by -23576644/15103001 = -1.561056905180633
  
  rat: replaced -1.594329979842039 by -26705354/16750205 = -1.594329979842038
  
  rat: replaced -1.627937618717409 by -52734804/32393627 = -1.62793761871741
  
  rat: replaced -1.661878461054199 by -42611978/25640851 = -1.661878461054198
  part: invalid index of list or matrix.
  #0: lineIntersection(g=[-3,-1,-1],h=[-2,-3,-3/2])
  #1: circleThrough(a=[-1,-1],b=[2,0],c=[1,2])
   -- an error. To debug this try: debugmode(true);
  
  Error in:
  LL &= circleThrough(A,B,C); $getCircleEquation(LL,x,y) ...
                            ^
\end{euleroutput}
\begin{eulerprompt}
>O &= getCircleCenter(LL); $O
\end{eulerprompt}
\begin{eulercomment}
Gambarkan lingkaran dan pusatnya. Cu dan U adalah simbolis. Kami
mengevaluasi ekspresi ini untuk Euler.
\end{eulercomment}
\begin{eulerprompt}
>plotCircle(LL()); plotPoint(O(),"O"):
\end{eulerprompt}
\begin{euleroutput}
  Function LL not found.
  Try list ... to find functions!
  Error in:
  plotCircle(LL()); plotPoint(O(),"O"): ...
                 ^
\end{euleroutput}
\begin{eulercomment}
Kita dapat menghitung perpotongan ketinggian di ABC (orthocenter)
secara numerik dengan perintah berikut.
\end{eulercomment}
\begin{eulerprompt}
>H &= lineIntersection(perpendicular(A,lineThrough(C,B)),...
>  perpendicular(B,lineThrough(A,C))); $H
\end{eulerprompt}
\begin{euleroutput}
  Maxima said:
  rat: replaced -9.983250083613754e-5 by -612914/6139423483 = -9.983250083613756e-5
  
  rat: replaced -3.986533601775671e-4 by -220554/553247563 = -3.986533601775666e-4
  
  rat: replaced -8.954327045205754e-4 by -584699/652979277 = -8.954327045205756e-4
  
  rat: replaced -0.001589120864678328 by -740868/466212493 = -0.00158912086467833
  
  rat: replaced -0.002478648480745763 by -878917/354595259 = -0.002478648480745762
  
  rat: replaced -0.003562926609036218 by -2735717/767828614 = -0.003562926609036219
  
  rat: replaced -0.004840846830973591 by -1164348/240525685 = -0.004840846830973582
  
  rat: replaced -0.006311281363933816 by -16515210/2616776063 = -0.006311281363933816
  
  rat: replaced -0.007973083174022497 by -2414321/302808957 = -0.007973083174022491
  
  rat: replaced -0.009825086090776508 by -1144049/116441626 = -0.009825086090776506
  
  rat: replaced -0.01186610492378118 by -1659683/139867548 = -0.01186610492378118
  
  rat: replaced -0.01409493558118687 by -986877/70016425 = -0.01409493558118684
  
  rat: replaced -0.01651035519011868 by -1738361/105289134 = -0.01651035519011867
  
  rat: replaced -0.01911112221896202 by -1475047/77182647 = -0.01911112221896199
  
  rat: replaced -0.02189597660151474 by -7711274/352177669 = -0.02189597660151473
  
  rat: replaced -0.02486363986299212 by -3887839/156366446 = -0.02486363986299209
  
  rat: replaced -0.0280128152478745 by -2263313/80795628 = -0.02801281524787455
  
  rat: replaced -0.03134218784958129 by -1116362/35618509 = -0.03134218784958124
  
  rat: replaced -0.03485042474195996 by -3920507/112495243 = -0.03485042474195998
  
  rat: replaced -0.03853617511257795 by -5379408/139593719 = -0.03853617511257795
  
  rat: replaced -0.04239807039780302 by -3385918/79860191 = -0.04239807039780308
  
  rat: replaced -0.04643472441965829 by -10918553/235137672 = -0.04643472441965828
  
  rat: replaced -0.05064473352443885 by -5036501/99447675 = -0.05064473352443886
  
  rat: replaced -0.05502667672307548 by -2932521/53292715 = -0.05502667672307557
  
  rat: replaced -0.05957911583323347 by -6320819/106091185 = -0.05957911583323346
  
  rat: replaced -0.06430059562312868 by -9893260/153859539 = -0.0643005956231287
  
  rat: replaced -0.06918964395705007 by -6012189/86894348 = -0.06918964395705
  
  rat: replaced -0.07424477194257195 by -6096479/82113243 = -0.07424477194257204
  
  rat: replaced -0.07946447407944118 by -5389689/67825139 = -0.07946447407944125
  
  rat: replaced -0.0848472284101276 by -9595393/113090235 = -0.08484722841012754
  
  rat: replaced -0.09039149667201674 by -3773144/41742245 = -0.09039149667201657
  
  rat: replaced -0.0960957244512361 by -5162056/53717853 = -0.09609572445123597
  
  rat: replaced -0.1019583413380946 by -1082663/10618680 = -0.1019583413380948
  
  rat: replaced -0.107977761084122 by -1922059/17800508 = -0.1079777610841219
  
  rat: replaced -0.1141523817606936 by -5923297/51889386 = -0.1141523817606938
  
  rat: replaced -0.1204805859192203 by -17634703/146369665 = -0.1204805859192204
  
  rat: replaced -0.1269607407528933 by -11368220/89541223 = -0.1269607407528932
  
  rat: replaced -0.1335911982599624 by -4657902/34866833 = -0.1335911982599624
  
  rat: replaced -0.1403702954085355 by -8528456/60756843 = -0.1403702954085353
  
  rat: replaced -0.1472963543028805 by -11128453/75551449 = -0.1472963543028804
  
  rat: replaced -0.1543676823512128 by -8170760/52930509 = -0.1543676823512126
  
  rat: replaced -0.1615825724349539 by -188109817/1164171446 = -0.1615825724349539
  
  rat: replaced -0.1689393030794406 by -5046974/29874481 = -0.1689393030794409
  
  rat: replaced -0.1764361386260728 by -6530305/37012287 = -0.176436138626073
  
  rat: replaced -0.1840713294058766 by -25189859/136848357 = -0.1840713294058766
  
  rat: replaced -0.1918431119144694 by -24326967/126806570 = -0.1918431119144694
  
  rat: replaced -0.1997497089884105 by -14902039/74603558 = -0.1997497089884104
  
  rat: replaced -0.2077893299829148 by -7281351/35041987 = -0.2077893299829145
  
  rat: replaced -0.2159601709509153 by -11348921/52550991 = -0.2159601709509151
  
  rat: replaced -0.2242604148234577 by -22385730/99820247 = -0.2242604148234576
  
  rat: replaced -0.2326882315914051 by -25615030/110083049 = -0.2326882315914051
  
  rat: replaced -0.2412417784884371 by -14523232/60201977 = -0.2412417784884373
  
  rat: replaced -0.2499192001753251 by -11309023/45250717 = -0.2499192001753254
  
  rat: replaced -0.2587186289254649 by -7582961/29309683 = -0.2587186289254647
  
  rat: replaced -0.267638184811648 by -17912865/66929407 = -0.2676381848116479
  
  rat: replaced -0.2766759758940514 by -27538925/99534934 = -0.2766759758940514
  
  rat: replaced -0.2858300984094321 by -29258587/102363562 = -0.2858300984094321
  
  rat: replaced -0.2950986369614998 by -7877677/26695064 = -0.2950986369614997
  
  rat: replaced -0.304479664712457 by -14469542/47522195 = -0.304479664712457
  
  rat: replaced -0.3139712435756791 by -8375733/26676752 = -0.3139712435756797
  
  rat: replaced -0.3235714244095225 by -178371467/551258404 = -0.3235714244095225
  
  rat: replaced -0.3332782472122374 by -5743591/17233621 = -0.333278247212237
  
  rat: replaced -0.3430897413179662 by -15588245/45434891 = -0.3430897413179664
  
  rat: replaced -0.3530039255938071 by -6523425/18479752 = -0.3530039255938067
  
  rat: replaced -0.3630188086379282 by -51253958/141188161 = -0.3630188086379282
  
  rat: replaced -0.373132388978704 by -9370061/25111894 = -0.3731323889787047
  
  rat: replaced -0.3833426552748616 by -11820697/30835851 = -0.3833426552748617
  
  rat: replaced -0.393647586516613 by -9153768/23253713 = -0.3936475865166135
  
  rat: replaced -0.4040451522277552 by -16634707/41170416 = -0.404045152227755
  
  rat: replaced -0.4145333126687146 by -2088920/5039209 = -0.4145333126687145
  
  rat: replaced -0.4251100190405208 by -24667763/58026774 = -0.4251100190405209
  
  rat: replaced -0.4357732136896836 by -10448574/23977091 = -0.435773213689684
  
  rat: replaced -0.4465208303139576 by -8346266/18691773 = -0.4465208303139568
  
  rat: replaced -0.4573507941689697 by -20158688/44077081 = -0.4573507941689696
  
  rat: replaced -0.4682610222756929 by -12818601/27374905 = -0.4682610222756937
  
  rat: replaced -0.4792494236287415 by -13652513/28487281 = -0.4792494236287416
  
  rat: replaced -0.4903138994054704 by -35114711/71616797 = -0.4903138994054705
  
  rat: replaced -0.5014523431758559 by -15102855/30118226 = -0.5014523431758564
  
  rat: replaced -0.5126626411131362 by -31697340/61828847 = -0.5126626411131361
  
  rat: replaced -0.5239426722051925 by -27432767/52358337 = -0.5239426722051924
  
  rat: replaced -0.5352903084666492 by -6124470/11441399 = -0.5352903084666482
  
  rat: replaced -0.5467034151516694 by -41717397/76307182 = -0.5467034151516694
  
  rat: replaced -0.5581798509674292 by -7494380/13426461 = -0.5581798509674292
  
  rat: replaced -0.5697174682882435 by -14609183/25642856 = -0.5697174682882438
  
  rat: replaced -0.581314113370329 by -14367580/24715691 = -0.5813141133703282
  
  rat: replaced -0.5929676265671738 by -9820294/16561265 = -0.5929676265671735
  
  rat: replaced -0.6046758425455033 by -23593213/39017952 = -0.6046758425455031
  
  rat: replaced -0.6164365905018095 by -15720181/25501700 = -0.6164365905018097
  
  rat: replaced -0.6282476943794307 by -53974636/85912987 = -0.6282476943794306
  
  rat: replaced -0.640106973086155 by -20459615/31962806 = -0.6401069730861552
  
  rat: replaced -0.652012240712328 by -51645100/79208789 = -0.652012240712328
  
  rat: replaced -0.6639613067494411 by -12215999/18398661 = -0.6639613067494422
  
  rat: replaced -0.6759519763091814 by -18558734/27455699 = -0.6759519763091808
  
  rat: replaced -0.6879820503429186 by -23500536/34158647 = -0.687982050342919
  
  rat: replaced -0.7000493258616074 by -29992669/42843651 = -0.7000493258616078
  
  rat: replaced -0.7121515961560857 by -10685401/15004391 = -0.7121515961560853
  
  rat: replaced -0.7242866510177421 by -11795807/16286103 = -0.7242866510177419
  
  rat: replaced -0.7364522769595366 by -14940657/20287339 = -0.7364522769595362
  
  rat: replaced -0.7486462574373463 by -42508133/56779998 = -0.7486462574373461
  
  rat: replaced 1.503320816708814e-4 by 144269/959668744 = 1.503320816708812e-4
  
  rat: replaced 6.026466136005715e-4 by 554629/920322105 = 6.026466136005719e-4
  
  rat: replaced 0.001358898348046048 by 845667/622318072 = 0.001358898348046045
  
  rat: replaced 0.002421011643797932 by 1882206/777446075 = 0.002421011643797931
  
  rat: replaced 0.003790880273744496 by 1271426/335390703 = 0.003790880273744499
  
  rat: replaced 0.005470367235498236 by 1827828/334132595 = 0.005470367235498231
  
  rat: replaced 0.007461304565095722 by 763704/102355291 = 0.007461304565095712
  
  rat: replaced 0.009765493153796295 by 1498127/153410276 = 0.009765493153796295
  
  rat: replaced 0.01238470256799509 by 4127047/333237474 = 0.0123847025679951
  
  rat: replaced 0.01532067087226624 by 2377723/155197055 = 0.01532067087226623
  
  rat: replaced 0.01857510445555993 by 4650073/250338996 = 0.01857510445555993
  
  rat: replaced 0.02214967786056252 by 1400096/63210671 = 0.0221496778605625
  
  rat: replaced 0.02604603361624602 by 2361294/90658487 = 0.02604603361624599
  
  rat: replaced 0.03026578207361535 by 1660222/54854753 = 0.03026578207361539
  
  rat: replaced 0.03481050124467483 by 4348324/124914145 = 0.03481050124467489
  
  rat: replaced 0.03968173664462726 by 9988401/251712799 = 0.03968173664462728
  
  rat: replaced 0.04488100113732568 by 9262688/206383275 = 0.04488100113732569
  
  rat: replaced 0.05040977478398728 by 8488548/168390913 = 0.0504097747839873
  
  rat: replaced 0.05626950469518793 by 3521587/62584290 = 0.05626950469518788
  
  rat: replaced 0.06246160488615271 by 1724725/27612563 = 0.06246160488615272
  
  rat: replaced 0.06898745613535606 by 699061/10133161 = 0.06898745613535599
  
  rat: replaced 0.07584840584644481 by 7595322/100138189 = 0.07584840584644485
  
  rat: replaced 0.08304576791349888 by 2278089/27431729 = 0.083045767913499
  
  rat: replaced 0.09058082258964217 by 5411518/59742425 = 0.09058082258964212
  
  rat: replaced 0.09845481635901993 by 9650917/98023818 = 0.09845481635902001
  
  rat: replaced 0.1066689618121501 by 6098878/57175751 = 0.10666896181215
  
  rat: replaced 0.1152244375246662 by 22751561/197454303 = 0.1152244375246662
  
  rat: replaced 0.1241223879394598 by 36591823/294804375 = 0.1241223879394599
  
  rat: replaced 0.1333639232522373 by 2621363/19655713 = 0.1333639232522371
  
  rat: replaced 0.1429501193005029 by 14344978/100349535 = 0.142950119300503
  
  rat: replaced 0.1528820174559729 by 7599779/49710091 = 0.1528820174559729
  
  rat: replaced 0.163160624520442 by 10213526/62597983 = 0.1631606245204418
  
  rat: replaced 0.1737869126251026 by 16586501/95441600 = 0.1737869126251027
  
  rat: replaced 0.1847618191333327 by 5716910/30942053 = 0.1847618191333329
  
  rat: replaced 0.1960862465469606 by 4485287/22874052 = 0.1960862465469607
  
  rat: replaced 0.2077610624160157 by 6172898/29711525 = 0.2077610624160153
  
  rat: replaced 0.2197870992519729 by 9779676/44496133 = 0.2197870992519732
  
  rat: replaced 0.2321651544445043 by 12225287/52657717 = 0.2321651544445043
  
  rat: replaced 0.2448959901817382 by 6355852/25953271 = 0.2448959901817385
  
  rat: replaced 0.257980333374044 by 7376119/28591788 = 0.2579803333740443
  
  rat: replaced 0.2714188755813393 by 7521403/27711422 = 0.2714188755813397
  
  rat: replaced 0.2852122729439353 by 33856936/118707851 = 0.2852122729439353
  
  rat: replaced 0.2993611461169232 by 11605920/38768959 = 0.2993611461169231
  
  rat: replaced 0.3138660802081108 by 12935015/41211892 = 0.3138660802081108
  
  rat: replaced 0.3287276247195093 by 4721149/14361887 = 0.3287276247195093
  
  rat: replaced 0.3439462934923849 by 29044045/84443547 = 0.3439462934923849
  
  rat: replaced 0.359522564655877 by 32233940/89657627 = 0.359522564655877
  
  rat: replaced 0.3754568805791821 by 37000079/98546813 = 0.3754568805791822
  
  rat: replaced 0.39174964782732 by 19372366/49450883 = 0.3917496478273199
  
  rat: replaced 0.4084012371204762 by 18093351/44302880 = 0.4084012371204762
  
  rat: replaced 0.4254119832969315 by 17508065/41155552 = 0.4254119832969316
  
  rat: replaced 0.4427821852795774 by 7324893/16542881 = 0.4427821852795774
  
  rat: replaced 0.4605121060460234 by 10003471/21722493 = 0.4605121060460233
  
  rat: replaced 0.4786019726023018 by 60236280/125858821 = 0.4786019726023018
  
  rat: replaced 0.4970519759601649 by 15490512/31164773 = 0.497051975960165
  
  rat: replaced 0.5158622711179853 by 5589426/10835113 = 0.5158622711179847
  
  rat: replaced 0.5350329770452558 by 7868837/14707200 = 0.5350329770452568
  
  rat: replaced 0.5545641766706926 by 12815301/23108779 = 0.554564176670693
  
  rat: replaced 0.5744559168739426 by 13141311/22876100 = 0.5744559168739427
  
  rat: replaced 0.5947082084808951 by 14675891/24677465 = 0.5947082084808955
  
  rat: replaced 0.6153210262625995 by 19715525/32041039 = 0.6153210262626003
  
  rat: replaced 0.6362943089377887 by 24594815/38653206 = 0.636294308937789
  
  rat: replaced 0.6576279591790061 by 15639887/23782272 = 0.6576279591790053
  
  rat: replaced 0.6793218436223387 by 17047367/25094684 = 0.6793218436223385
  
  rat: replaced 0.701375792880754 by 24937969/35555788 = 0.701375792880754
  
  rat: replaced 0.7237896015610379 by 12792911/17674903 = 0.7237896015610382
  
  rat: replaced 0.7465630282843339 by 24894563/33345561 = 0.7465630282843344
  
  rat: replaced 0.76969579571028 by 11030405/14330863 = 0.7696957957102792
  
  rat: replaced 0.7931875905647454 by 17983947/22673006 = 0.7931875905647447
  
  rat: replaced 0.8170380636711536 by 35512807/43465303 = 0.817038063671154
  
  rat: replaced 0.8412468299854033 by 9419201/11196715 = 0.841246829985402
  
  rat: replaced 0.8658134686343698 by 33824443/39066663 = 0.8658134686343699
  
  rat: replaced 0.8907375229579941 by 1798799/2019449 = 0.890737522957995
  
  rat: replaced 0.9160185005549473 by 21232969/23179629 = 0.9160185005549485
  
  rat: replaced 0.9416558733318703 by 11919739/12658275 = 0.9416558733318718
  
  rat: replaced 0.967649077556183 by 17561986/18149127 = 0.9676490775561821
  
  rat: replaced 0.9939975139124576 by 37819354/38047735 = 0.9939975139124576
  
  rat: replaced 1.020700547562349 by 18875311/18492506 = 1.020700547562348
  
  rat: replaced 1.047757508208077 by 16616467/15859077 = 1.047757508208075
  
  rat: replaced 1.07516769015946 by 21797467/20273551 = 1.07516769015946
  
  rat: replaced 1.102930352404475 by 30072842/27266311 = 1.102930352404474
  
  rat: replaced 1.131044718683369 by 7906291/6990255 = 1.131044718683367
  
  rat: replaced 1.159509977566275 by 26421893/22787120 = 1.159509977566274
  
  rat: replaced 1.188325282534358 by 19245269/16195287 = 1.188325282534357
  
  rat: replaced 1.21748975206447 by 18221771/14966673 = 1.21748975206447
  
  rat: replaced 1.247002469717292 by 21540268/17273637 = 1.247002469717292
  
  rat: replaced 1.276862484228988 by 17179259/13454275 = 1.27686248422899
  
  rat: replaced 1.307068809606323 by 22169845/16961498 = 1.307068809606321
  
  rat: replaced 1.337620425225263 by 48765573/36456959 = 1.337620425225263
  
  rat: replaced 1.368516275933041 by 117856634/86120009 = 1.368516275933041
  
  rat: replaced 1.399755272153666 by 29694085/21213769 = 1.399755272153666
  
  rat: replaced 1.431336289996881 by 23110861/16146353 = 1.43133628999688
  
  rat: replaced 1.463258171370553 by 19245288/13152353 = 1.463258171370553
  
  rat: replaced 1.495519724096479 by 43164951/28862843 = 1.495519724096479
  
  rat: replaced 1.528119722029604 by 31224680/20433399 = 1.528119722029605
  
  rat: replaced 1.561056905180636 by 23576644/15103001 = 1.561056905180633
  
  rat: replaced 1.594329979842039 by 26705354/16750205 = 1.594329979842038
  
  rat: replaced 1.627937618717409 by 52734804/32393627 = 1.62793761871741
  
  rat: replaced 1.661878461054199 by 42611978/25640851 = 1.661878461054198
  part: invalid index of list or matrix.
  #0: lineIntersection(g=[1,-2,1],h=[2,3,4])
   -- an error. To debug this try: debugmode(true);
  
  Error in:
    perpendicular(B,lineThrough(A,C))); $H ...
                                      ^
\end{euleroutput}
\begin{eulercomment}
Sekarang kita dapat menghitung garis Euler dari segitiga.
\end{eulercomment}
\begin{eulerprompt}
>el &= lineThrough(H,O); $getLineEquation(el,x,y)
\end{eulerprompt}
\begin{eulercomment}
Tambahkan ke plot kami.
\end{eulercomment}
\begin{eulerprompt}
>plotPoint(H(),"H"); plotLine(el(),"Garis Euler"):
\end{eulerprompt}
\begin{euleroutput}
  Function H needs at least 3 arguments!
  Use: H (a, b, c) 
  Error in:
  plotPoint(H(),"H"); plotLine(el(),"Garis Euler"): ...
               ^
\end{euleroutput}
\begin{eulercomment}
Pusat gravitasi harus berada di garis ini.
\end{eulercomment}
\begin{eulerprompt}
>M &= (A+B+C)/3; $getLineEquation(el,x,y) with [x=M[1],y=M[2]]
>plotPoint(M(),"M"): // titik berat
\end{eulerprompt}
\begin{eulercomment}
Teorinya memberitahu kita MH=2*MO. Kita perlu menyederhanakan dengan
radcan untuk mencapai ini.
\end{eulercomment}
\begin{eulerprompt}
>$distance(M,H)/distance(M,O)|radcan
\end{eulerprompt}
\begin{eulercomment}
Fungsi termasuk fungsi untuk sudut juga.
\end{eulercomment}
\begin{eulerprompt}
>$computeAngle(A,C,B), degprint(%())
\end{eulerprompt}
\begin{euleroutput}
  60°15'18.43''
\end{euleroutput}
\begin{eulercomment}
Persamaan untuk pusat incircle tidak terlalu bagus.
\end{eulercomment}
\begin{eulerprompt}
>Q &= lineIntersection(angleBisector(A,C,B),angleBisector(C,B,A))|radcan; $Q
\end{eulerprompt}
\begin{euleroutput}
  Maxima said:
  rat: replaced 1.66665833335744e-7 by 15819/94914474571 = 1.66665833335744e-7
  
  rat: replaced 4.999958333473664e-5 by 201389/4027813565 = 4.99995833347366e-5
  
  rat: replaced 1.33330666692022e-6 by 31771/23828726570 = 1.333306666920221e-6
  
  rat: replaced 1.999933334222437e-4 by 200030/1000183339 = 1.999933334222437e-4
  
  rat: replaced 4.499797504338432e-6 by 24036/5341573699 = 4.499797504338431e-6
  
  rat: replaced 4.499662510124569e-4 by 1162901/2584418270 = 4.499662510124571e-4
  
  rat: replaced 1.066581336583994e-5 by 58861/5518660226 = 1.066581336583993e-5
  
  rat: replaced 7.998933390220841e-4 by 1137431/1421978337 = 7.998933390220838e-4
  
  rat: replaced 2.083072932167196e-5 by 35635/1710693824 = 2.0830729321672e-5
  
  rat: replaced 0.001249739605033717 by 567943/454449069 = 0.001249739605033716
  
  rat: replaced 3.599352055540239e-5 by 98277/2730408098 = 3.599352055540234e-5
  
  rat: replaced 0.00179946006479581 by 479561/266502719 = 0.001799460064795812
  
  rat: replaced 5.71526624672386e-5 by 51154/895041417 = 5.715266246723866e-5
  
  rat: replaced 0.002448999746720415 by 1946227/794702818 = 0.002448999746720415
  
  rat: replaced 8.530603082730626e-5 by 121691/1426522824 = 8.530603082730627e-5
  
  rat: replaced 0.003198293697380561 by 2986741/933854512 = 0.003198293697380562
  
  rat: replaced 1.214508019889565e-4 by 158455/1304684674 = 1.214508019889563e-4
  
  rat: replaced 0.004047266988005727 by 2125334/525128193 = 0.004047266988005727
  
  rat: replaced 1.665833531718508e-4 by 142521/855553675 = 1.66583353171851e-4
  
  rat: replaced 0.004995834721974179 by 1957223/391770967 = 0.004995834721974179
  
  rat: replaced 2.216991628251896e-4 by 179571/809975995 = 2.216991628251896e-4
  
  rat: replaced 0.006043902043303184 by 1800665/297930871 = 0.006043902043303193
  
  rat: replaced 2.877927110806339e-4 by 1167733/4057548906 = 2.877927110806339e-4
  
  rat: replaced 0.00719136414613375 by 2476362/344352191 = 0.007191364146133747
  
  rat: replaced 3.658573803051457e-4 by 386279/1055818526 = 3.658573803051454e-4
  
  rat: replaced 0.00843810628521191 by 2079855/246483622 = 0.008438106285211924
  
  rat: replaced 4.5688535576352e-4 by 262978/575588595 = 4.568853557635206e-4
  
  rat: replaced 0.009784003787362772 by 1752551/179124113 = 0.009784003787362787
  
  rat: replaced 5.618675264007778e-4 by 150595/268025812 = 5.618675264007782e-4
  
  rat: replaced 0.01122892206395776 by 5450241/485375263 = 0.01122892206395776
  
  rat: replaced 6.817933857540259e-4 by 192316/282073725 = 6.817933857540258e-4
  
  rat: replaced 0.01277271662437307 by 3258991/255152533 = 0.01277271662437308
  
  rat: replaced 8.176509330039827e-4 by 105841/129445214 = 8.176509330039812e-4
  
  rat: replaced 0.01441523309043924 by 2330472/161667313 = 0.01441523309043925
  
  rat: replaced 9.704265741758145e-4 by 651321/671169790 = 9.704265741758132e-4
  
  rat: replaced 0.01615630721187855 by 19391318/1200232067 = 0.01615630721187855
  
  rat: replaced 0.001141105023499428 by 1259907/1104111343 = 0.001141105023499428
  
  rat: replaced 0.01799576488272969 by 4765614/264818641 = 0.01799576488272969
  
  rat: replaced 0.001330669204938795 by 1231154/925214167 = 0.001330669204938796
  
  rat: replaced 0.01993342215875837 by 2504519/125644206 = 0.01993342215875836
  
  rat: replaced 0.001540100153900437 by 276884/179783113 = 0.001540100153900439
  
  rat: replaced 0.02196908527585173 by 1298306/59096953 = 0.0219690852758517
  
  rat: replaced 0.001770376919130678 by 644389/363984072 = 0.001770376919130681
  
  rat: replaced 0.02410255066939448 by 2001286/83032125 = 0.02410255066939453
  
  rat: replaced 0.002022476464811601 by 1271955/628909667 = 0.002022476464811599
  
  rat: replaced 0.02633360499462523 by 2978115/113091808 = 0.02633360499462525
  
  rat: replaced 0.002297373572865413 by 1020913/444382669 = 0.002297373572865417
  
  rat: replaced 0.02866202514797045 by 1770713/61779061 = 0.02866202514797044
  
  rat: replaced 0.002596040745477063 by 1097643/422814242 = 0.002596040745477065
  
  rat: replaced 0.03108757828935527 by 5034207/161936287 = 0.03108757828935525
  
  rat: replaced 0.002919448107844891 by 906221/310408326 = 0.002919448107844891
  
  rat: replaced 0.03361002186548678 by 4553215/135471944 = 0.03361002186548678
  
  rat: replaced 0.003268563311168871 by 1379071/421919623 = 0.003268563311168867
  
  rat: replaced 0.03622910363410947 by 3082649/85087642 = 0.0362291036341094
  
  rat: replaced 0.003644351435886262 by 5966577/1637212301 = 0.003644351435886261
  
  rat: replaced 0.03894456168922911 by 4913415/126164342 = 0.03894456168922911
  
  rat: replaced 0.004047774895164447 by 572425/141417202 = 0.004047774895164451
  
  rat: replaced 0.04175612448730281 by 1734727/41544253 = 0.04175612448730273
  
  rat: replaced 0.004479793338660443 by 2952779/659132861 = 0.004479793338660444
  
  rat: replaced 0.04466351087439402 by 4691119/105032473 = 0.04466351087439405
  
  rat: replaced 0.0049413635565565 by 2524919/510976165 = 0.004941363556556498
  
  rat: replaced 0.04766643011428662 by 3536207/74186529 = 0.04766643011428665
  
  rat: replaced 0.005433439383882244 by 1361584/250593391 = 0.005433439383882235
  
  rat: replaced 0.05076458191755917 by 7710025/151878036 = 0.05076458191755916
  
  rat: replaced 0.005956971605131645 by 1447422/242979503 = 0.005956971605131648
  
  rat: replaced 0.0539576564716131 by 3377975/62604183 = 0.05395765647161309
  
  rat: replaced 0.006512907859185624 by 3695063/567344584 = 0.006512907859185626
  
  rat: replaced 0.05724533447165381 by 2560865/44734912 = 0.05724533447165382
  
  rat: replaced 0.007102192544548636 by 1363981/192050693 = 0.007102192544548642
  
  rat: replaced 0.06062728715262111 by 8274761/136485754 = 0.06062728715262107
  
  rat: replaced 0.007725766724910044 by 1464384/189545459 = 0.007725766724910038
  
  rat: replaced 0.06410317632206519 by 5287663/82486755 = 0.06410317632206528
  
  rat: replaced 0.00838456803503801 by 1113589/132814117 = 0.008384568035038023
  
  rat: replaced 0.06767265439396564 by 2921400/43169579 = 0.06767265439396572
  
  rat: replaced 0.009079530587017326 by 433906/47789475 = 0.00907953058701733
  
  rat: replaced 0.07133536442348987 by 7236103/101437808 = 0.07133536442348991
  
  rat: replaced 0.009811584876838586 by 1363090/138926587 = 0.009811584876838586
  
  rat: replaced 0.07509094014268702 by 9209133/122639735 = 0.07509094014268704
  
  rat: replaced 0.0105816576913495 by 1163729/109976058 = 0.01058165769134951
  
  rat: replaced 0.07893900599711501 by 5197067/65836489 = 0.07893900599711506
  
  rat: replaced 0.01139067201557714 by 13426050/1178688139 = 0.01139067201557714
  
  rat: replaced 0.08287917718339499 by 11217158/135343501 = 0.082879177183395
  
  rat: replaced 0.01223954694042984 by 2283101/186534764 = 0.01223954694042983
  
  rat: replaced 0.08691105968769186 by 5213115/59982182 = 0.08691105968769192
  
  rat: replaced 0.01312919757078923 by 3499615/266552086 = 0.01312919757078922
  
  rat: replaced 0.09103425032511492 by 5893225/64736349 = 0.09103425032511488
  
  rat: replaced 0.01406053493400045 by 2280713/162206702 = 0.01406053493400045
  
  rat: replaced 0.09524833678003664 by 9601787/100807923 = 0.09524833678003662
  
  rat: replaced 0.01503446588876983 by 200490/13335359 = 0.01503446588876985
  
  rat: replaced 0.09955289764732322 by 5687088/57126293 = 0.09955289764732328
  
  rat: replaced 0.01605189303448024 by 951971/59305840 = 0.01605189303448025
  
  rat: replaced 0.1039475024744748 by 10260011/98703776 = 0.1039475024744747
  
  rat: replaced 0.01711371462093175 by 9432386/551159477 = 0.01711371462093176
  
  rat: replaced 0.1084317118046711 by 14939691/137779721 = 0.1084317118046712
  
  rat: replaced 0.01822082445851714 by 2559788/140486947 = 0.01822082445851713
  
  rat: replaced 0.113005077220716 by 8478529/75027859 = 0.1130050772207161
  
  rat: replaced 0.01937411182884202 by 2983799/154009589 = 0.01937411182884203
  
  rat: replaced 0.1176671413898787 by 7123715/60541243 = 0.1176671413898786
  
  rat: replaced 0.02057446139579705 by 7167743/348380590 = 0.02057446139579705
  
  rat: replaced 0.1224174381096274 by 12172179/99431741 = 0.1224174381096274
  
  rat: replaced 0.02182275311709253 by 7415562/339808729 = 0.02182275311709253
  
  rat: replaced 0.1272554923542488 by 7277933/57191504 = 0.127255492354249
  
  rat: replaced 0.02311986215626333 by 2988661/129268115 = 0.02311986215626336
  
  rat: replaced 0.1321808203223502 by 3633064/27485561 = 0.1321808203223503
  
  rat: replaced 0.02446665879515308 by 1991976/81415939 = 0.02446665879515312
  
  rat: replaced 0.1371929294852391 by 56235017/409897341 = 0.1371929294852391
  
  rat: replaced 0.02586400834688696 by 5000736/193347293 = 0.02586400834688697
  
  rat: replaced 0.1422913186361759 by 9349741/65708443 = 0.1422913186361759
  
  rat: replaced 0.02731277106934082 by 858413/31428997 = 0.02731277106934084
  
  rat: replaced 0.1474754779404944 by 1549881/10509415 = 0.1474754779404943
  
  rat: replaced 0.02881380207911666 by 3754753/130310918 = 0.02881380207911666
  
  rat: replaced 0.152744888986584 by 5264425/34465474 = 0.1527448889865841
  
  rat: replaced 0.03036795126603076 by 4118329/135614318 = 0.03036795126603077
  
  rat: replaced 0.1580990248377314 by 5442776/34426373 = 0.1580990248377312
  
  rat: replaced 0.03197606320812652 by 3497683/109384416 = 0.03197606320812647
  
  rat: replaced 0.1635373500848132 by 12328488/75386375 = 0.1635373500848131
  
  rat: replaced 0.0336389770872163 by 3971799/118071337 = 0.03363897708721635
  
  rat: replaced 0.1690593208998367 by 20896917/123607009 = 0.1690593208998367
  
  rat: replaced 0.03535752660496472 by 1815732/51353479 = 0.03535752660496478
  
  rat: replaced 0.1746643850903219 by 2841592/16268869 = 0.1746643850903219
  
  rat: replaced 0.03713253989951881 by 3333721/89778965 = 0.03713253989951878
  
  rat: replaced 0.1803519821545206 by 4461007/24735004 = 0.1803519821545208
  
  rat: replaced 0.03896483946269502 by 8785771/225479461 = 0.03896483946269501
  
  rat: replaced 0.1861215433374662 by 4381209/23539505 = 0.1861215433374661
  
  rat: replaced 0.0408552420577305 by 3189084/78058135 = 0.04085524205773043
  
  rat: replaced 0.1919724916878484 by 72809759/379271834 = 0.1919724916878484
  
  rat: replaced 0.04280455863760801 by 7646593/178639688 = 0.04280455863760801
  
  rat: replaced 0.1979042421157076 by 26318167/132984350 = 0.1979042421157076
  
  rat: replaced 0.04481359426396048 by 20610430/459914683 = 0.04481359426396048
  
  rat: replaced 0.2039162014509444 by 8519416/41779005 = 0.2039162014509441
  
  rat: replaced 0.04688314802656623 by 3439140/73355569 = 0.04688314802656633
  
  rat: replaced 0.2100077685026351 by 50962787/242670961 = 0.2100077685026351
  
  rat: replaced 0.04901401296344043 by 4006732/81746663 = 0.04901401296344048
  
  rat: replaced 0.216178334119151 by 1347531/6233423 = 0.2161783341191509
  
  rat: replaced 0.05120697598153157 by 4148974/81023609 = 0.0512069759815315
  
  rat: replaced 0.2224272812490723 by 23234851/104460437 = 0.2224272812490723
  
  rat: replaced 0.05346281777803219 by 11998448/224426031 = 0.05346281777803218
  
  rat: replaced 0.2287539850028937 by 8185268/35781969 = 0.2287539850028935
  
  rat: replaced 0.05578231276230905 by 1398019/25062048 = 0.05578231276230897
  
  rat: replaced 0.2351578127155118 by 12642104/53760085 = 0.2351578127155119
  
  rat: replaced 0.05816622897846346 by 4451048/76522891 = 0.05816622897846345
  
  rat: replaced 0.2416381240094921 by 8002142/33116223 = 0.2416381240094923
  
  rat: replaced 0.06061532802852698 by 2146337/35409146 = 0.06061532802852686
  
  rat: replaced 0.2481942708591053 by 8882901/35790113 = 0.2481942708591057
  
  rat: replaced 0.0631303649963022 by 14651447/232082406 = 0.06313036499630222
  
  rat: replaced 0.2548255976551299 by 868346/3407609 = 0.25482559765513
  
  rat: replaced 0.06571208837185505 by 4240309/64528599 = 0.06571208837185509
  
  rat: replaced 0.2615314412704124 by 8212450/31401387 = 0.2615314412704127
  
  rat: replaced 0.06836123997666599 by 2716643/39739522 = 0.06836123997666604
  
  rat: replaced 0.2683111311261794 by 34459769/128432126 = 0.2683111311261794
  
  rat: replaced 0.07107855488944881 by 3146673/44270357 = 0.07107855488944893
  
  rat: replaced 0.2751639892590951 by 12552159/45617012 = 0.2751639892590949
  
  rat: replaced 0.07386476137264342 by 12898997/174629915 = 0.0738647613726434
  
  rat: replaced 0.2820893303890569 by 11134456/39471383 = 0.2820893303890568
  
  rat: replaced 0.07672058079958999 by 5073506/66129661 = 0.07672058079959007
  
  rat: replaced 0.2890864619877229 by 9583357/33150487 = 0.2890864619877228
  
  rat: replaced 0.07964672758239233 by 5672399/71219486 = 0.07964672758239227
  
  rat: replaced 0.2961546843477643 by 11052271/37319251 = 0.2961546843477647
  
  rat: replaced 0.08264390910047736 by 4686067/56701904 = 0.08264390910047748
  
  rat: replaced 0.3032932906528349 by 9918077/32701274 = 0.3032932906528351
  
  rat: replaced 0.0857128256298576 by 3585977/41837111 = 0.08571282562985766
  
  rat: replaced 0.3105015670482534 by 9320011/30015987 = 0.3105015670482533
  
  rat: replaced 0.08885417027310427 by 5751353/64728003 = 0.0888541702731042
  
  rat: replaced 0.3177787927123868 by 248395525/781661743 = 0.3177787927123868
  
  rat: replaced 0.09206862889003742 by 7305460/79347983 = 0.09206862889003745
  
  rat: replaced 0.3251242399287333 by 13842845/42577093 = 0.3251242399287335
  
  rat: replaced 0.09535688002914089 by 5971998/62627867 = 0.09535688002914103
  
  rat: replaced 0.3325371741586922 by 9318229/28021616 = 0.3325371741586923
  
  rat: replaced 0.0987195948597075 by 9821211/99485933 = 0.09871959485970745
  
  rat: replaced 0.3400168541150183 by 13391981/39386227 = 0.3400168541150184
  
  rat: replaced 0.1021574371047232 by 8336413/81603584 = 0.1021574371047232
  
  rat: replaced 0.3475625318359485 by 10097818/29053241 = 0.347562531835949
  
  rat: replaced 0.1056710629744951 by 5741011/54329074 = 0.105671062974495
  
  rat: replaced 0.3551734527599992 by 15867851/44676343 = 0.3551734527599987
  
  rat: replaced 0.1092611211010309 by 5551873/50812887 = 0.1092611211010309
  
  rat: replaced 0.3628488558014202 by 6897641/19009681 = 0.3628488558014203
  
  rat: replaced 0.1129282524731764 by 11548693/102265755 = 0.1129282524731764
  
  rat: replaced 0.3705879734263036 by 23358661/63031352 = 0.3705879734263038
  
  rat: replaced 0.1166730903725168 by 5656228/48479285 = 0.1166730903725168
  
  rat: replaced 0.3783900317293359 by 14241382/37636779 = 0.3783900317293358
  
  rat: replaced 0.1204962603100498 by 4057613/33674182 = 0.12049626031005
  
  rat: replaced 0.3862542505111889 by 3461217/8960981 = 0.3862542505111884
  
  rat: replaced 0.1243983799636342 by 7966447/64039797 = 0.1243983799636342
  
  rat: replaced 0.3941798433565377 by 5314214/13481699 = 0.3941798433565384
  
  rat: replaced 0.1283800591162231 by 796346/6203035 = 0.1283800591162229
  
  rat: replaced 0.4021660177127022 by 11567173/28762184 = 0.4021660177127022
  
  rat: replaced 0.1324418995948859 by 4716124/35609003 = 0.1324418995948862
  
  rat: replaced 0.4102119749689023 by 11320633/27597032 = 0.4102119749689024
  
  rat: replaced 0.1365844952106265 by 612971/4487852 = 0.1365844952106264
  
  rat: replaced 0.418316910536117 by 12225195/29224721 = 0.4183169105361177
  
  rat: replaced 0.140808431699002 by 10431632/74083859 = 0.1408084316990021
  
  rat: replaced 0.4264800139275439 by 7978696/18708253 = 0.4264800139275431
  
  rat: replaced 0.1451142866615502 by 3554077/24491572 = 0.1451142866615504
  
  rat: replaced 0.4347004688396462 by 20489554/47134879 = 0.4347004688396463
  
  rat: replaced 0.1495026295080298 by 26759297/178988805 = 0.1495026295080298
  
  rat: replaced 0.4429774532337832 by 23449796/52936771 = 0.4429774532337834
  
  rat: replaced 0.1539740213994798 by 16145763/104860306 = 0.1539740213994798
  
  rat: replaced 0.451310139418413 by 8841241/19590167 = 0.4513101394184133
  
  rat: replaced 4.999958333473664e-5 by 201389/4027813565 = 4.99995833347366e-5
  
  rat: replaced 1.66665833335744e-7 by 15819/94914474571 = 1.66665833335744e-7
  
  rat: replaced 1.999933334222437e-4 by 200030/1000183339 = 1.999933334222437e-4
  
  rat: replaced 1.33330666692022e-6 by 31771/23828726570 = 1.333306666920221e-6
  
  rat: replaced 4.499662510124569e-4 by 1162901/2584418270 = 4.499662510124571e-4
  
  rat: replaced 4.499797504338432e-6 by 24036/5341573699 = 4.499797504338431e-6
  
  rat: replaced 7.998933390220841e-4 by 1137431/1421978337 = 7.998933390220838e-4
  
  rat: replaced 1.066581336583994e-5 by 58861/5518660226 = 1.066581336583993e-5
  
  rat: replaced 0.001249739605033717 by 567943/454449069 = 0.001249739605033716
  
  rat: replaced 2.083072932167196e-5 by 35635/1710693824 = 2.0830729321672e-5
  
  rat: replaced 0.00179946006479581 by 479561/266502719 = 0.001799460064795812
  
  rat: replaced 3.599352055540239e-5 by 98277/2730408098 = 3.599352055540234e-5
  
  rat: replaced 0.002448999746720415 by 1946227/794702818 = 0.002448999746720415
  
  rat: replaced 5.71526624672386e-5 by 51154/895041417 = 5.715266246723866e-5
  
  rat: replaced 0.003198293697380561 by 2986741/933854512 = 0.003198293697380562
  
  rat: replaced 8.530603082730626e-5 by 121691/1426522824 = 8.530603082730627e-5
  
  rat: replaced 0.004047266988005727 by 2125334/525128193 = 0.004047266988005727
  
  rat: replaced 1.214508019889565e-4 by 158455/1304684674 = 1.214508019889563e-4
  
  rat: replaced 0.004995834721974179 by 1957223/391770967 = 0.004995834721974179
  
  rat: replaced 1.665833531718508e-4 by 142521/855553675 = 1.66583353171851e-4
  
  rat: replaced 0.006043902043303184 by 1800665/297930871 = 0.006043902043303193
  
  rat: replaced 2.216991628251896e-4 by 179571/809975995 = 2.216991628251896e-4
  
  rat: replaced 0.00719136414613375 by 2476362/344352191 = 0.007191364146133747
  
  rat: replaced 2.877927110806339e-4 by 1167733/4057548906 = 2.877927110806339e-4
  
  rat: replaced 0.00843810628521191 by 2079855/246483622 = 0.008438106285211924
  
  rat: replaced 3.658573803051457e-4 by 386279/1055818526 = 3.658573803051454e-4
  
  rat: replaced 0.009784003787362772 by 1752551/179124113 = 0.009784003787362787
  
  rat: replaced 4.5688535576352e-4 by 262978/575588595 = 4.568853557635206e-4
  
  rat: replaced 0.01122892206395776 by 5450241/485375263 = 0.01122892206395776
  
  rat: replaced 5.618675264007778e-4 by 150595/268025812 = 5.618675264007782e-4
  
  rat: replaced 0.01277271662437307 by 3258991/255152533 = 0.01277271662437308
  
  rat: replaced 6.817933857540259e-4 by 192316/282073725 = 6.817933857540258e-4
  
  rat: replaced 0.01441523309043924 by 2330472/161667313 = 0.01441523309043925
  
  rat: replaced 8.176509330039827e-4 by 105841/129445214 = 8.176509330039812e-4
  
  rat: replaced 0.01615630721187855 by 19391318/1200232067 = 0.01615630721187855
  
  rat: replaced 9.704265741758145e-4 by 651321/671169790 = 9.704265741758132e-4
  
  rat: replaced 0.01799576488272969 by 4765614/264818641 = 0.01799576488272969
  
  rat: replaced 0.001141105023499428 by 1259907/1104111343 = 0.001141105023499428
  
  rat: replaced 0.01993342215875837 by 2504519/125644206 = 0.01993342215875836
  
  rat: replaced 0.001330669204938795 by 1231154/925214167 = 0.001330669204938796
  
  rat: replaced 0.02196908527585173 by 1298306/59096953 = 0.0219690852758517
  
  rat: replaced 0.001540100153900437 by 276884/179783113 = 0.001540100153900439
  
  rat: replaced 0.02410255066939448 by 2001286/83032125 = 0.02410255066939453
  
  rat: replaced 0.001770376919130678 by 644389/363984072 = 0.001770376919130681
  
  rat: replaced 0.02633360499462523 by 2978115/113091808 = 0.02633360499462525
  
  rat: replaced 0.002022476464811601 by 1271955/628909667 = 0.002022476464811599
  
  rat: replaced 0.02866202514797045 by 1770713/61779061 = 0.02866202514797044
  
  rat: replaced 0.002297373572865413 by 1020913/444382669 = 0.002297373572865417
  
  rat: replaced 0.03108757828935527 by 5034207/161936287 = 0.03108757828935525
  
  rat: replaced 0.002596040745477063 by 1097643/422814242 = 0.002596040745477065
  
  rat: replaced 0.03361002186548678 by 4553215/135471944 = 0.03361002186548678
  
  rat: replaced 0.002919448107844891 by 906221/310408326 = 0.002919448107844891
  
  rat: replaced 0.03622910363410947 by 3082649/85087642 = 0.0362291036341094
  
  rat: replaced 0.003268563311168871 by 1379071/421919623 = 0.003268563311168867
  
  rat: replaced 0.03894456168922911 by 4913415/126164342 = 0.03894456168922911
  
  rat: replaced 0.003644351435886262 by 5966577/1637212301 = 0.003644351435886261
  
  rat: replaced 0.04175612448730281 by 1734727/41544253 = 0.04175612448730273
  
  rat: replaced 0.004047774895164447 by 572425/141417202 = 0.004047774895164451
  
  rat: replaced 0.04466351087439402 by 4691119/105032473 = 0.04466351087439405
  
  rat: replaced 0.004479793338660443 by 2952779/659132861 = 0.004479793338660444
  
  rat: replaced 0.04766643011428662 by 3536207/74186529 = 0.04766643011428665
  
  rat: replaced 0.0049413635565565 by 2524919/510976165 = 0.004941363556556498
  
  rat: replaced 0.05076458191755917 by 7710025/151878036 = 0.05076458191755916
  
  rat: replaced 0.005433439383882244 by 1361584/250593391 = 0.005433439383882235
  
  rat: replaced 0.0539576564716131 by 3377975/62604183 = 0.05395765647161309
  
  rat: replaced 0.005956971605131645 by 1447422/242979503 = 0.005956971605131648
  
  rat: replaced 0.05724533447165381 by 2560865/44734912 = 0.05724533447165382
  
  rat: replaced 0.006512907859185624 by 3695063/567344584 = 0.006512907859185626
  
  rat: replaced 0.06062728715262111 by 8274761/136485754 = 0.06062728715262107
  
  rat: replaced 0.007102192544548636 by 1363981/192050693 = 0.007102192544548642
  
  rat: replaced 0.06410317632206519 by 5287663/82486755 = 0.06410317632206528
  
  rat: replaced 0.007725766724910044 by 1464384/189545459 = 0.007725766724910038
  
  rat: replaced 0.06767265439396564 by 2921400/43169579 = 0.06767265439396572
  
  rat: replaced 0.00838456803503801 by 1113589/132814117 = 0.008384568035038023
  
  rat: replaced 0.07133536442348987 by 7236103/101437808 = 0.07133536442348991
  
  rat: replaced 0.009079530587017326 by 433906/47789475 = 0.00907953058701733
  
  rat: replaced 0.07509094014268702 by 9209133/122639735 = 0.07509094014268704
  
  rat: replaced 0.009811584876838586 by 1363090/138926587 = 0.009811584876838586
  
  rat: replaced 0.07893900599711501 by 5197067/65836489 = 0.07893900599711506
  
  rat: replaced 0.0105816576913495 by 1163729/109976058 = 0.01058165769134951
  
  rat: replaced 0.08287917718339499 by 11217158/135343501 = 0.082879177183395
  
  rat: replaced 0.01139067201557714 by 13426050/1178688139 = 0.01139067201557714
  
  rat: replaced 0.08691105968769186 by 5213115/59982182 = 0.08691105968769192
  
  rat: replaced 0.01223954694042984 by 2283101/186534764 = 0.01223954694042983
  
  rat: replaced 0.09103425032511492 by 5893225/64736349 = 0.09103425032511488
  
  rat: replaced 0.01312919757078923 by 3499615/266552086 = 0.01312919757078922
  
  rat: replaced 0.09524833678003664 by 9601787/100807923 = 0.09524833678003662
  
  rat: replaced 0.01406053493400045 by 2280713/162206702 = 0.01406053493400045
  
  rat: replaced 0.09955289764732322 by 5687088/57126293 = 0.09955289764732328
  
  rat: replaced 0.01503446588876983 by 200490/13335359 = 0.01503446588876985
  
  rat: replaced 0.1039475024744748 by 10260011/98703776 = 0.1039475024744747
  
  rat: replaced 0.01605189303448024 by 951971/59305840 = 0.01605189303448025
  
  rat: replaced 0.1084317118046711 by 14939691/137779721 = 0.1084317118046712
  
  rat: replaced 0.01711371462093175 by 9432386/551159477 = 0.01711371462093176
  
  rat: replaced 0.113005077220716 by 8478529/75027859 = 0.1130050772207161
  
  rat: replaced 0.01822082445851714 by 2559788/140486947 = 0.01822082445851713
  
  rat: replaced 0.1176671413898787 by 7123715/60541243 = 0.1176671413898786
  
  rat: replaced 0.01937411182884202 by 2983799/154009589 = 0.01937411182884203
  
  rat: replaced 0.1224174381096274 by 12172179/99431741 = 0.1224174381096274
  
  rat: replaced 0.02057446139579705 by 7167743/348380590 = 0.02057446139579705
  
  rat: replaced 0.1272554923542488 by 7277933/57191504 = 0.127255492354249
  
  rat: replaced 0.02182275311709253 by 7415562/339808729 = 0.02182275311709253
  
  rat: replaced 0.1321808203223502 by 3633064/27485561 = 0.1321808203223503
  
  rat: replaced 0.02311986215626333 by 2988661/129268115 = 0.02311986215626336
  
  rat: replaced 0.1371929294852391 by 56235017/409897341 = 0.1371929294852391
  
  rat: replaced 0.02446665879515308 by 1991976/81415939 = 0.02446665879515312
  
  rat: replaced 0.1422913186361759 by 9349741/65708443 = 0.1422913186361759
  
  rat: replaced 0.02586400834688696 by 5000736/193347293 = 0.02586400834688697
  
  rat: replaced 0.1474754779404944 by 1549881/10509415 = 0.1474754779404943
  
  rat: replaced 0.02731277106934082 by 858413/31428997 = 0.02731277106934084
  
  rat: replaced 0.152744888986584 by 5264425/34465474 = 0.1527448889865841
  
  rat: replaced 0.02881380207911666 by 3754753/130310918 = 0.02881380207911666
  
  rat: replaced 0.1580990248377314 by 5442776/34426373 = 0.1580990248377312
  
  rat: replaced 0.03036795126603076 by 4118329/135614318 = 0.03036795126603077
  
  rat: replaced 0.1635373500848132 by 12328488/75386375 = 0.1635373500848131
  
  rat: replaced 0.03197606320812652 by 3497683/109384416 = 0.03197606320812647
  
  rat: replaced 0.1690593208998367 by 20896917/123607009 = 0.1690593208998367
  
  rat: replaced 0.0336389770872163 by 3971799/118071337 = 0.03363897708721635
  
  rat: replaced 0.1746643850903219 by 2841592/16268869 = 0.1746643850903219
  
  rat: replaced 0.03535752660496472 by 1815732/51353479 = 0.03535752660496478
  
  rat: replaced 0.1803519821545206 by 4461007/24735004 = 0.1803519821545208
  
  rat: replaced 0.03713253989951881 by 3333721/89778965 = 0.03713253989951878
  
  rat: replaced 0.1861215433374662 by 4381209/23539505 = 0.1861215433374661
  
  rat: replaced 0.03896483946269502 by 8785771/225479461 = 0.03896483946269501
  
  rat: replaced 0.1919724916878484 by 72809759/379271834 = 0.1919724916878484
  
  rat: replaced 0.0408552420577305 by 3189084/78058135 = 0.04085524205773043
  
  rat: replaced 0.1979042421157076 by 26318167/132984350 = 0.1979042421157076
  
  rat: replaced 0.04280455863760801 by 7646593/178639688 = 0.04280455863760801
  
  rat: replaced 0.2039162014509444 by 8519416/41779005 = 0.2039162014509441
  
  rat: replaced 0.04481359426396048 by 20610430/459914683 = 0.04481359426396048
  
  rat: replaced 0.2100077685026351 by 50962787/242670961 = 0.2100077685026351
  
  rat: replaced 0.04688314802656623 by 3439140/73355569 = 0.04688314802656633
  
  rat: replaced 0.216178334119151 by 1347531/6233423 = 0.2161783341191509
  
  rat: replaced 0.04901401296344043 by 4006732/81746663 = 0.04901401296344048
  
  rat: replaced 0.2224272812490723 by 23234851/104460437 = 0.2224272812490723
  
  rat: replaced 0.05120697598153157 by 4148974/81023609 = 0.0512069759815315
  
  rat: replaced 0.2287539850028937 by 8185268/35781969 = 0.2287539850028935
  
  rat: replaced 0.05346281777803219 by 11998448/224426031 = 0.05346281777803218
  
  rat: replaced 0.2351578127155118 by 12642104/53760085 = 0.2351578127155119
  
  rat: replaced 0.05578231276230905 by 1398019/25062048 = 0.05578231276230897
  
  rat: replaced 0.2416381240094921 by 8002142/33116223 = 0.2416381240094923
  
  rat: replaced 0.05816622897846346 by 4451048/76522891 = 0.05816622897846345
  
  rat: replaced 0.2481942708591053 by 8882901/35790113 = 0.2481942708591057
  
  rat: replaced 0.06061532802852698 by 2146337/35409146 = 0.06061532802852686
  
  rat: replaced 0.2548255976551299 by 868346/3407609 = 0.25482559765513
  
  rat: replaced 0.0631303649963022 by 14651447/232082406 = 0.06313036499630222
  
  rat: replaced 0.2615314412704124 by 8212450/31401387 = 0.2615314412704127
  
  rat: replaced 0.06571208837185505 by 4240309/64528599 = 0.06571208837185509
  
  rat: replaced 0.2683111311261794 by 34459769/128432126 = 0.2683111311261794
  
  rat: replaced 0.06836123997666599 by 2716643/39739522 = 0.06836123997666604
  
  rat: replaced 0.2751639892590951 by 12552159/45617012 = 0.2751639892590949
  
  rat: replaced 0.07107855488944881 by 3146673/44270357 = 0.07107855488944893
  
  rat: replaced 0.2820893303890569 by 11134456/39471383 = 0.2820893303890568
  
  rat: replaced 0.07386476137264342 by 12898997/174629915 = 0.0738647613726434
  
  rat: replaced 0.2890864619877229 by 9583357/33150487 = 0.2890864619877228
  
  rat: replaced 0.07672058079958999 by 5073506/66129661 = 0.07672058079959007
  
  rat: replaced 0.2961546843477643 by 11052271/37319251 = 0.2961546843477647
  
  rat: replaced 0.07964672758239233 by 5672399/71219486 = 0.07964672758239227
  
  rat: replaced 0.3032932906528349 by 9918077/32701274 = 0.3032932906528351
  
  rat: replaced 0.08264390910047736 by 4686067/56701904 = 0.08264390910047748
  
  rat: replaced 0.3105015670482534 by 9320011/30015987 = 0.3105015670482533
  
  rat: replaced 0.0857128256298576 by 3585977/41837111 = 0.08571282562985766
  
  rat: replaced 0.3177787927123868 by 248395525/781661743 = 0.3177787927123868
  
  rat: replaced 0.08885417027310427 by 5751353/64728003 = 0.0888541702731042
  
  rat: replaced 0.3251242399287333 by 13842845/42577093 = 0.3251242399287335
  
  rat: replaced 0.09206862889003742 by 7305460/79347983 = 0.09206862889003745
  
  rat: replaced 0.3325371741586922 by 9318229/28021616 = 0.3325371741586923
  
  rat: replaced 0.09535688002914089 by 5971998/62627867 = 0.09535688002914103
  
  rat: replaced 0.3400168541150183 by 13391981/39386227 = 0.3400168541150184
  
  rat: replaced 0.0987195948597075 by 9821211/99485933 = 0.09871959485970745
  
  rat: replaced 0.3475625318359485 by 10097818/29053241 = 0.347562531835949
  
  rat: replaced 0.1021574371047232 by 8336413/81603584 = 0.1021574371047232
  
  rat: replaced 0.3551734527599992 by 15867851/44676343 = 0.3551734527599987
  
  rat: replaced 0.1056710629744951 by 5741011/54329074 = 0.105671062974495
  
  rat: replaced 0.3628488558014202 by 6897641/19009681 = 0.3628488558014203
  
  rat: replaced 0.1092611211010309 by 5551873/50812887 = 0.1092611211010309
  
  rat: replaced 0.3705879734263036 by 23358661/63031352 = 0.3705879734263038
  
  rat: replaced 0.1129282524731764 by 11548693/102265755 = 0.1129282524731764
  
  rat: replaced 0.3783900317293359 by 14241382/37636779 = 0.3783900317293358
  
  rat: replaced 0.1166730903725168 by 5656228/48479285 = 0.1166730903725168
  
  rat: replaced 0.3862542505111889 by 3461217/8960981 = 0.3862542505111884
  
  rat: replaced 0.1204962603100498 by 4057613/33674182 = 0.12049626031005
  
  rat: replaced 0.3941798433565377 by 5314214/13481699 = 0.3941798433565384
  
  rat: replaced 0.1243983799636342 by 7966447/64039797 = 0.1243983799636342
  
  rat: replaced 0.4021660177127022 by 11567173/28762184 = 0.4021660177127022
  
  rat: replaced 0.1283800591162231 by 796346/6203035 = 0.1283800591162229
  
  rat: replaced 0.4102119749689023 by 11320633/27597032 = 0.4102119749689024
  
  rat: replaced 0.1324418995948859 by 4716124/35609003 = 0.1324418995948862
  
  rat: replaced 0.418316910536117 by 12225195/29224721 = 0.4183169105361177
  
  rat: replaced 0.1365844952106265 by 612971/4487852 = 0.1365844952106264
  
  rat: replaced 0.4264800139275439 by 7978696/18708253 = 0.4264800139275431
  
  rat: replaced 0.140808431699002 by 10431632/74083859 = 0.1408084316990021
  
  rat: replaced 0.4347004688396462 by 20489554/47134879 = 0.4347004688396463
  
  rat: replaced 0.1451142866615502 by 3554077/24491572 = 0.1451142866615504
  
  rat: replaced 0.4429774532337832 by 23449796/52936771 = 0.4429774532337834
  
  rat: replaced 0.1495026295080298 by 26759297/178988805 = 0.1495026295080298
  
  rat: replaced 0.451310139418413 by 8841241/19590167 = 0.4513101394184133
  
  rat: replaced 0.1539740213994798 by 16145763/104860306 = 0.1539740213994798
  part: invalid index of list or matrix.
  #0: lineIntersection(g=[-2-sqrt(13)/sqrt(5),-3-sqrt(13)/(2*sqrt(5))+sqrt(5)*sqrt(13)/2,sqrt(13)*(-2-sqrt(13)/sqrt(5))/(2*sq...,h=[-1+3*sqrt(5)/sqrt(10),2+sqrt(5)/sqrt(10),(2-sqrt(5)/sqrt(10))*(2+sqrt(5)/sqrt(10))/2+(3-3*sqrt(5)/s...)
   -- an error. To debug this try: debugmode(true);
  
  Error in:
  ... angleBisector(A,C,B),angleBisector(C,B,A))|radcan; $Q ...
                                                       ^
\end{euleroutput}
\begin{eulercomment}
Mari kita hitung juga ekspresi untuk jari-jari lingkaran yang
tertulis.
\end{eulercomment}
\begin{eulerprompt}
>r &= distance(Q,projectToLine(Q,lineThrough(A,B)))|ratsimp; $r
\end{eulerprompt}
\begin{euleroutput}
  Maxima said:
  rat: replaced 1.498320841708742e-4 by 1329822/8875415485 = 1.498320841708742e-4
  
  rat: replaced 5.986466935998108e-4 by 398723/666040595 = 5.986466935998098e-4
  
  rat: replaced 0.001345398955533032 by 4525441/3363642421 = 0.001345398955533032
  
  rat: replaced 0.002389014203700413 by 1071627/448564516 = 0.00238901420370041
  
  rat: replaced 0.00372838808577948 by 661903/177530607 = 0.003728388085779485
  
  rat: replaced 0.005362386673832029 by 5230891/975478144 = 0.005362386673832028
  
  rat: replaced 0.007289846577694006 by 32241346/4422774287 = 0.007289846577694006
  
  rat: replaced 0.009509575061314376 by 2146493/225719129 = 0.009509575061314364
  
  rat: replaced 0.01202035016202822 by 1789188/148846579 = 0.01202035016202825
  
  rat: replaced 0.01482092081275069 by 2581665/174190594 = 0.01482092081275066
  
  rat: replaced 0.01791000696708436 by 5107285/285163764 = 0.01791000696708436
  
  rat: replaced 0.02128629972732062 by 3323295/156123659 = 0.02128629972732064
  
  rat: replaced 0.02494846147533059 by 4548287/182307314 = 0.02494846147533061
  
  rat: replaced 0.02889512600632479 by 3147802/108938857 = 0.02889512600632481
  
  rat: replaced 0.0331248986654725 by 5858625/176864692 = 0.03312489866547248
  
  rat: replaced 0.03763635648736519 by 10043830/266865099 = 0.03763635648736518
  
  rat: replaced 0.04242804833831373 by 4635713/109260576 = 0.04242804833831372
  
  rat: replaced 0.04749849506145984 by 5610259/118114458 = 0.04749849506145979
  
  rat: replaced 0.05284618962468965 by 4237503/80185592 = 0.05284618962468968
  
  rat: replaced 0.05846959727133633 by 3317197/56733707 = 0.05846959727133642
  
  rat: replaced 0.06436715567365475 by 13427433/208606903 = 0.06436715567365477
  
  rat: replaced 0.07053727508905278 by 8025659/113778977 = 0.07053727508905269
  
  rat: replaced 0.07697833851906408 by 6306881/81930594 = 0.07697833851906408
  
  rat: replaced 0.08368870187104593 by 4282086/51166835 = 0.08368870187104596
  
  rat: replaced 0.09066669412258874 by 2175091/23989967 = 0.09066669412258883
  
  rat: replaced 0.09791061748861546 by 8290049/84669561 = 0.09791061748861554
  
  rat: replaced 0.1054187475911595 by 8501563/80645646 = 0.1054187475911595
  
  rat: replaced 0.1131893336318011 by 6539019/57770629 = 0.113189333631801
  
  rat: replaced 0.121220598566744 by 5779101/47674249 = 0.1212205985667441
  
  rat: replaced 0.1295107392845216 by 7134865/55090914 = 0.1295107392845216
  
  rat: replaced 0.1380579267863034 by 6113057/44278928 = 0.1380579267863034
  
  rat: replaced 0.1468603063687953 by 6311140/42973763 = 0.1468603063687953
  
  rat: replaced 0.1559159978097077 by 4027079/25828517 = 0.1559159978097078
  
  rat: replaced 0.1652230955557758 by 10597125/64138279 = 0.1652230955557757
  
  rat: replaced 0.1747796689133147 by 9649007/55206690 = 0.1747796689133147
  
  rat: replaced 0.1845837622412855 by 6871913/37229239 = 0.1845837622412857
  
  rat: replaced 0.1946333951468589 by 39341769/202132676 = 0.1946333951468589
  
  rat: replaced 0.2049265626834523 by 10758647/52500012 = 0.2049265626834523
  
  rat: replaced 0.2154612355512225 by 33702610/156420759 = 0.2154612355512225
  
  rat: replaced 0.2262353602999955 by 2338161/10335082 = 0.2262353602999957
  
  rat: replaced 0.2372468595346078 by 7573078/31920667 = 0.2372468595346081
  
  rat: replaced 0.2484936321226457 by 3764353/15148690 = 0.2484936321226456
  
  rat: replaced 0.2599735534045555 by 26335713/101301508 = 0.2599735534045554
  
  rat: replaced 0.2716844754061095 by 29831699/109802737 = 0.2716844754061094
  
  rat: replaced 0.2836242270531998 by 15100773/53242183 = 0.2836242270531995
  
  rat: replaced 0.2957906143889442 by 2942977/9949528 = 0.2957906143889439
  
  rat: replaced 0.3081814207930817 by 12077608/39189929 = 0.3081814207930818
  
  rat: replaced 0.3207944072036307 by 9185023/28632117 = 0.3207944072036308
  
  rat: replaced 0.333627312340794 by 5228336/15671187 = 0.3336273123407946
  
  rat: replaced 0.346677852933085 by 15615111/45042136 = 0.3466778529330847
  
  rat: replaced 0.3599437239456539 by 7564465/21015688 = 0.3599437239456543
  
  rat: replaced 0.3734225988107874 by 7702871/20627758 = 0.3734225988107869
  
  rat: replaced 0.3871121296605642 by 97723109/252441351 = 0.3871121296605642
  
  rat: replaced 0.4010099475616409 by 3146543/7846546 = 0.4010099475616405
  
  rat: replaced 0.4151136627521425 by 6219049/14981557 = 0.4151136627521425
  
  rat: replaced 0.4294208648806354 by 26148647/60892819 = 0.4294208648806356
  
  rat: replaced 0.4439291232471635 by 19525684/43983787 = 0.4439291232471638
  
  rat: replaced 0.458635987046313 by 38604672/84172793 = 0.4586359870463132
  
  rat: replaced 0.4735389856122937 by 11146199/23538081 = 0.4735389856122935
  
  rat: replaced 0.488635628666001 by 13946471/28541658 = 0.4886356286660011
  
  rat: replaced 0.5039234065640431 by 5948069/11803518 = 0.503923406564043
  
  rat: replaced 0.5193997905497036 by 24027011/46259185 = 0.5193997905497038
  
  rat: replaced 0.5350622330058146 by 7363779/13762472 = 0.5350622330058147
  
  rat: replaced 0.5509081677095147 by 8130825/14758948 = 0.5509081677095142
  
  rat: replaced 0.5669350100888726 by 10250363/18080314 = 0.5669350100888735
  
  rat: replaced 0.5831401574813392 by 37655026/64572857 = 0.5831401574813393
  
  rat: replaced 0.5995209893940125 by 30778651/51338738 = 0.5995209893940128
  
  rat: replaced 0.6160748677656853 by 23698401/38466755 = 0.616074867765685
  
  rat: replaced 0.6327991372306488 by 5052598/7984521 = 0.6327991372306492
  
  rat: replaced 0.6496911253842265 by 60646047/93345968 = 0.6496911253842266
  
  rat: replaced 0.666748143050013 by 30125566/45182827 = 0.6667481430500132
  
  rat: replaced 0.6839674845487889 by 8953739/13090884 = 0.6839674845487899
  
  rat: replaced 0.7013464279690875 by 7888577/11247761 = 0.7013464279690864
  
  rat: replaced 0.7188822354393821 by 16662338/23178119 = 0.7188822354393815
  
  rat: replaced 0.7365721534018723 by 13899283/18870226 = 0.7365721534018723
  
  rat: replaced 0.7544134128878366 by 16270763/21567436 = 0.754413412887837
  
  rat: replaced 0.7724032297945274 by 8203205/10620366 = 0.7724032297945287
  
  rat: replaced 0.7905388051635788 by 10794522/13654639 = 0.7905388051635784
  
  rat: replaced 0.8088173254609005 by 16745047/20703126 = 0.808817325460899
  
  rat: replaced 0.8272359628580275 by 20291194/24528907 = 0.827235962858027
  
  rat: replaced 0.8457918755149025 by 10996366/13001267 = 0.8457918755149018
  
  rat: replaced 0.8644822078640563 by 9158500/10594203 = 0.8644822078640555
  
  rat: replaced 0.8833040908961625 by 13759446/15577247 = 0.8833040908961641
  
  rat: replaced 0.9022546424469358 by 19827819/21975857 = 0.9022546424469362
  
  rat: replaced 0.9213309674853474 by 60458149/65620446 = 0.9213309674853475
  
  rat: replaced 0.9405301584031224 by 11658841/12396031 = 0.9405301584031212
  
  rat: replaced 0.9598492953055026 by 26214088/27310629 = 0.9598492953055018
  
  rat: replaced 0.9792854463032298 by 35089005/35831233 = 0.9792854463032293
  
  rat: replaced 0.9988356678057343 by 15735752/15754095 = 0.9988356678057356
  
  rat: replaced 1.018497004815491 by 16202286/15908035 = 1.018497004815491
  
  rat: replaced 1.038266491223517 by 17763365/17108676 = 1.038266491223517
  
  rat: replaced 1.058141150105979 by 33730321/31876958 = 1.058141150105979
  
  rat: replaced 1.078117994021884 by 51996446/48228901 = 1.078117994021883
  
  rat: replaced 1.098194025311821 by 124719922/113568203 = 1.098194025311821
  
  rat: replaced 1.118366236397724 by 92837336/83011569 = 1.118366236397724
  
  rat: replaced 1.13863161008363 by 20601995/18093644 = 1.138631610083629
  
  rat: replaced 1.158987119857388 by 20626233/17796775 = 1.15898711985739
  
  rat: replaced 1.17942973019332 by 4098089/3474636 = 1.179429730193321
  
  rat: replaced 1.199956396855759 by 17442145/14535649 = 1.199956396855758
  part: invalid index of list or matrix.
  #0: lineIntersection(g=[3,1,[3,1] . Q],h=[-1,3,-2])
  #1: projectToLine(a=Q,g=[-1,3,-2])
   -- an error. To debug this try: debugmode(true);
  
  Error in:
  ... ance(Q,projectToLine(Q,lineThrough(A,B)))|ratsimp; $r ...
                                                       ^
\end{euleroutput}
\begin{eulerprompt}
>LD &=  circleWithCenter(Q,r); // Lingkaran dalam
\end{eulerprompt}
\begin{eulercomment}
Mari kita tambahkan ini ke plot.
\end{eulercomment}
\begin{eulerprompt}
>color(5); plotCircle(LD()):
\end{eulerprompt}
\begin{euleroutput}
  Q is not a variable!
  Error in expression: [Q[1],Q[2],cc[3]]
  Error in:
  color(5); plotCircle(LD()): ...
                           ^
\end{euleroutput}
\eulersubheading{Parabola}
\begin{eulercomment}
Selanjutnya akan dicari persamaan tempat kedudukan titik-titik yang berjarak sama ke titik C
dan ke garis AB.
\end{eulercomment}
\begin{eulerprompt}
>p &= getHesseForm(lineThrough(A,B),x,y,C)-distance([x,y],C); $p='0
\end{eulerprompt}
\begin{eulercomment}
Persamaan tersebut dapat digambar menjadi satu dengan gambar sebelumnya.
\end{eulercomment}
\begin{eulerprompt}
>plot2d(p,level=0,add=1,contourcolor=6):
\end{eulerprompt}
\begin{euleroutput}
  Wrong argument!
  
  Cannot combine a symbolic expression here.
  Did you want to create a symbolic expression?
  Then start with &.
  
  Error in expression: [-sqrt(5),-sqrt((2-4.999958333473664e-5*r)^2+(1-1.66665833335744e-7*r)^2),-sqrt((2-1.999933334222437e-4*r)^2+(1-1.33330666692022e-6*r)^2),-sqrt((2-4.499662510124569e-4*r)^2+(1-4.499797504338432e-6*r)^2),-sqrt((2-7.998933390220841e-4*r)^2+(1-1.066581336583994e-5*r)^2),-sqrt((2-0.001249739605033717*r)^2+(1-2.083072932167196e-5*r)^2),-sqrt((2-0.00179946006479581*r)^2+(1-3.599352055540239e-5*r)^2),-sqrt((2-0.002448999746720415*r)^2+(1-5.71526624672386e-5*r)^2),-sqrt((2-0.003198293697380561*r)^2+(1-8.530603082730626e-5*r)^2),-sqrt((2-0.004047266988005727*r)^2+(1-1.214508019889565e-4*r)^2),-sqrt((2-0.004995834721974179*r)^2+(1-1.665833531718508e-4*r)^2),-sqrt((2-0.006043902043303184*r)^2+(1-2.216991628251896e-4*r)^2),-sqrt((2-0.00719136414613375*r)^2+(1-2.877927110806339e-4*r)^2),-sqrt((2-0.00843810628521191*r)^2+(1-3.658573803051457e-4*r)^2),-sqrt((2-0.009784003787362772*r)^2+(1-4.5688535576352e-4*r)^2),-sqrt((2-0.01122892206395776*r)^2+(1-5.618675264007778e-4*r)^2),-sqrt((2-0.01277271662437307*r)^2+(1-6.817933857540259e-4*r)^2),-sqrt((2-0.01441523309043924*r)^2+(1-8.176509330039827e-4*r)^2),-sqrt((2-0.01615630721187855*r)^2+(1-9.704265741758145e-4*r)^2),-sqrt((2-0.01799576488272969*r)^2+(1-0.001141105023499428*r)^2),-sqrt((2-0.01993342215875837*r)^2+(1-0.001330669204938795*r)^2),-sqrt((2-0.02196908527585173*r)^2+(1-0.001540100153900437*r)^2),-sqrt((2-0.02410255066939448*r)^2+(1-0.001770376919130678*r)^2),-sqrt((2-0.02633360499462523*r)^2+(1-0.002022476464811601*r)^2),-sqrt((2-0.02866202514797045*r)^2+(1-0.002297373572865413*r)^2),-sqrt((2-0.03108757828935527*r)^2+(1-0.002596040745477063*r)^2),-sqrt((2-0.03361002186548678*r)^2+(1-0.002919448107844891*r)^2),-sqrt((2-0.03622910363410947*r)^2+(1-0.003268563311168871*r)^2),-sqrt((2-0.03894456168922911*r)^2+(1-0.003644351435886262*r)^2),-sqrt((2-0.04175612448730281*r)^2+(1-0.004047774895164447*r)^2),-sqrt((2-0.04466351087439402*r)^2+(1-0.004479793338660443*r)^2),-sqrt((2-0.04766643011428662*r)^2+(1-0.0049413635565565*r)^2),-sqrt((2-0.05076458191755917*r)^2+(1-0.005433439383882244*r)^2),-sqrt((2-0.0539576564716131*r)^2+(1-0.005956971605131645*r)^2),-sqrt((2-0.05724533447165381*r)^2+(1-0.006512907859185624*r)^2),-sqrt((2-0.06062728715262111*r)^2+(1-0.007102192544548636*r)^2),-sqrt((2-0.06410317632206519*r)^2+(1-0.007725766724910044*r)^2),-sqrt((2-0.06767265439396564*r)^2+(1-0.00838456803503801*r)^2),-sqrt((2-0.07133536442348987*r)^2+(1-0.009079530587017326*r)^2),-sqrt((2-0.07509094014268702*r)^2+(1-0.009811584876838586*r)^2),-sqrt((2-0.07893900599711501*r)^2+(1-0.0105816576913495*r)^2),-sqrt((2-0.08287917718339499*r)^2+(1-0.01139067201557714*r)^2),-sqrt((2-0.08691105968769186*r)^2+(1-0.01223954694042984*r)^2),-sqrt((2-0.09103425032511492*r)^2+(1-0.01312919757078923*r)^2),-sqrt((2-0.09524833678003664*r)^2+(1-0.01406053493400045*r)^2),-sqrt((2-0.09955289764732322*r)^2+(1-0.01503446588876983*r)^2),-sqrt((2-0.1039475024744748*r)^2+(1-0.01605189303448024*r)^2),-sqrt((2-0.1084317118046711*r)^2+(1-0.01711371462093175*r)^2),-sqrt((2-0.113005077220716*r)^2+(1-0.01822082445851714*r)^2),-sqrt((2-0.1176671413898787*r)^2+(1-0.01937411182884202*r)^2),-sqrt((2-0.1224174381096274*r)^2+(1-0.02057446139579705*r)^2),-sqrt((2-0.1272554923542488*r)^2+(1-0.02182275311709253*r)^2),-sqrt((2-0.1321808203223502*r)^2+(1-0.02311986215626333*r)^2),-sqrt((2-0.1371929294852391*r)^2+(1-0.02446665879515308*r)^2),-sqrt((2-0.1422913186361759*r)^2+(1-0.02586400834688696*r)^2),-sqrt((2-0.1474754779404944*r)^2+(1-0.02731277106934082*r)^2),-sqrt((2-0.152744888986584*r)^2+(1-0.02881380207911666*r)^2),-sqrt((2-0.1580990248377314*r)^2+(1-0.03036795126603076*r)^2),-sqrt((2-0.1635373500848132*r)^2+(1-0.03197606320812652*r)^2),-sqrt((2-0.1690593208998367*r)^2+(1-0.0336389770872163*r)^2),-sqrt((2-0.1746643850903219*r)^2+(1-0.03535752660496472*r)^2),-sqrt((2-0.1803519821545206*r)^2+(1-0.03713253989951881*r)^2),-sqrt((2-0.1861215433374662*r)^2+(1-0.03896483946269502*r)^2),-sqrt((2-0.1919724916878484*r)^2+(1-0.0408552420577305*r)^2),-sqrt((2-0.1979042421157076*r)^2+(1-0.04280455863760801*r)^2),-sqrt((2-0.2039162014509444*r)^2+(1-0.04481359426396048*r)^2),-sqrt((2-0.2100077685026351*r)^2+(1-0.04688314802656623*r)^2),-sqrt((2-0.216178334119151*r)^2+(1-0.04901401296344043*r)^2),-sqrt((2-0.2224272812490723*r)^2+(1-0.05120697598153157*r)^2),-sqrt((2-0.2287539850028937*r)^2+(1-0.05346281777803219*r)^2),-sqrt((2-0.2351578127155118*r)^2+(1-0.05578231276230905*r)^2),-sqrt((2-0.2416381240094921*r)^2+(1-0.05816622897846346*r)^2),-sqrt((2-0.2481942708591053*r)^2+(1-0.06061532802852698*r)^2),-sqrt((2-0.2548255976551299*r)^2+(1-0.0631303649963022*r)^2),-sqrt((2-0.2615314412704124*r)^2+(1-0.06571208837185505*r)^2),-sqrt((2-0.2683111311261794*r)^2+(1-0.06836123997666599*r)^2),-sqrt((2-0.2751639892590951*r)^2+(1-0.07107855488944881*r)^2),-sqrt((2-0.2820893303890569*r)^2+(1-0.07386476137264342*r)^2),-sqrt((2-0.2890864619877229*r)^2+(1-0.07672058079958999*r)^2),-sqrt((2-0.2961546843477643*r)^2+(1-0.07964672758239233*r)^2),-sqrt((2-0.3032932906528349*r)^2+(1-0.08264390910047736*r)^2),-sqrt((2-0.3105015670482534*r)^2+(1-0.0857128256298576*r)^2),-sqrt((2-0.3177787927123868*r)^2+(1-0.08885417027310427*r)^2),-sqrt((2-0.3251242399287333*r)^2+(1-0.09206862889003742*r)^2),-sqrt((2-0.3325371741586922*r)^2+(1-0.09535688002914089*r)^2),-sqrt((2-0.3400168541150183*r)^2+(1-0.0987195948597075*r)^2),-sqrt((2-0.3475625318359485*r)^2+(1-0.1021574371047232*r)^2),-sqrt((2-0.3551734527599992*r)^2+(1-0.1056710629744951*r)^2),-sqrt((2-0.3628488558014202*r)^2+(1-0.1092611211010309*r)^2),-sqrt((2-0.3705879734263036*r)^2+(1-0.1129282524731764*r)^2),-sqrt((2-0.3783900317293359*r)^2+(1-0.1166730903725168*r)^2),-sqrt((2-0.3862542505111889*r)^2+(1-0.1204962603100498*r)^2),-sqrt((2-0.3941798433565377*r)^2+(1-0.1243983799636342*r)^2),-sqrt((2-0.4021660177127022*r)^2+(1-0.1283800591162231*r)^2),-sqrt((2-0.4102119749689023*r)^2+(1-0.1324418995948859*r)^2),-sqrt((2-0.418316910536117*r)^2+(1-0.1365844952106265*r)^2),-sqrt((2-0.4264800139275439*r)^2+(1-0.140808431699002*r)^2),-sqrt((2-0.4347004688396462*r)^2+(1-0.1451142866615502*r)^2),-sqrt((2-0.4429774532337832*r)^2+(1-0.1495026295080298*r)^2),-sqrt((2-0.451310139418413*r)^2+(1-0.1539740213994798*r)^2)]+(if [2/sqrt(10),(2+1.498320841708742e-4*r)/sqrt(10),(2+5.986466935998108e-4*r)/sqrt(10),(2+0.001345398955533032*r)/sqrt(10),(2+0.002389014203700413*r)/sqrt(10),(2+0.00372838808577948*r)/sqrt(10),(2+0.005362386673832029*r)/sqrt(10),(2+0.007289846577694006*r)/sqrt(10),(2+0.009509575061314376*r)/sqrt(10),(2+0.01202035016202822*r)/sqrt(10),(2+0.01482092081275069*r)/sqrt(10),(2+0.01791000696708436*r)/sqrt(10),(2+0.02128629972732062*r)/sqrt(10),(2+0.02494846147533059*r)/sqrt(10),(2+0.02889512600632479*r)/sqrt(10),(2+0.0331248986654725*r)/sqrt(10),(2+0.03763635648736519*r)/sqrt(10),(2+0.04242804833831373*r)/sqrt(10),(2+0.04749849506145984*r)/sqrt(10),(2+0.05284618962468965*r)/sqrt(10),(2+0.05846959727133633*r)/sqrt(10),(2+0.06436715567365475*r)/sqrt(10),(2+0.07053727508905278*r)/sqrt(10),(2+0.07697833851906408*r)/sqrt(10),(2+0.08368870187104593*r)/sqrt(10),(2+0.09066669412258874*r)/sqrt(10),(2+0.09791061748861546*r)/sqrt(10),(2+0.1054187475911595*r)/sqrt(10),(2+0.1131893336318011*r)/sqrt(10),(2+0.121220598566744*r)/sqrt(10),(2+0.1295107392845216*r)/sqrt(10),(2+0.1380579267863034*r)/sqrt(10),(2+0.1468603063687953*r)/sqrt(10),(2+0.1559159978097077*r)/sqrt(10),(2+0.1652230955557758*r)/sqrt(10),(2+0.1747796689133147*r)/sqrt(10),(2+0.1845837622412855*r)/sqrt(10),(2+0.1946333951468589*r)/sqrt(10),(2+0.2049265626834523*r)/sqrt(10),(2+0.2154612355512225*r)/sqrt(10),(2+0.2262353602999955*r)/sqrt(10),(2+0.2372468595346078*r)/sqrt(10),(2+0.2484936321226457*r)/sqrt(10),(2+0.2599735534045555*r)/sqrt(10),(2+0.2716844754061095*r)/sqrt(10),(2+0.2836242270531998*r)/sqrt(10),(2+0.2957906143889442*r)/sqrt(10),(2+0.3081814207930817*r)/sqrt(10),(2+0.3207944072036307*r)/sqrt(10),(2+0.333627312340794*r)/sqrt(10),(2+0.346677852933085*r)/sqrt(10),(2+0.3599437239456539*r)/sqrt(10),(2+0.3734225988107874*r)/sqrt(10),(2+0.3871121296605642*r)/sqrt(10),(2+0.4010099475616409*r)/sqrt(10),(2+0.4151136627521425*r)/sqrt(10),(2+0.4294208648806354*r)/sqrt(10),(2+0.4439291232471635*r)/sqrt(10),(2+0.458635987046313*r)/sqrt(10),(2+0.4735389856122937*r)/sqrt(10),(2+0.488635628666001*r)/sqrt(10),(2+0.5039234065640431*r)/sqrt(10),(2+0.5193997905497036*r)/sqrt(10),(2+0.5350622330058146*r)/sqrt(10),(2+0.5509081677095147*r)/sqrt(10),(2+0.5669350100888726*r)/sqrt(10),(2+0.5831401574813392*r)/sqrt(10),(2+0.5995209893940125*r)/sqrt(10),(2+0.6160748677656853*r)/sqrt(10),(2+0.6327991372306488*r)/sqrt(10),(2+0.6496911253842265*r)/sqrt(10),(2+0.666748143050013*r)/sqrt(10),(2+0.6839674845487889*r)/sqrt(10),(2+0.7013464279690875*r)/sqrt(10),(2+0.7188822354393821*r)/sqrt(10),(2+0.7365721534018723*r)/sqrt(10),(2+0.7544134128878366*r)/sqrt(10),(2+0.7724032297945274*r)/sqrt(10),(2+0.7905388051635788*r)/sqrt(10),(2+0.8088173254609005*r)/sqrt(10),(2+0.8272359628580275*r)/sqrt(10),(2+0.8457918755149025*r)/sqrt(10),(2+0.8644822078640563*r)/sqrt(10),(2+0.8833040908961625*r)/sqrt(10),(2+0.9022546424469358*r)/sqrt(10),(2+0.9213309674853474*r)/sqrt(10),(2+0.9405301584031224*r)/sqrt(10),(2+0.9598492953055026*r)/sqrt(10),(2+0.9792854463032298*r)/sqrt(10),(2+0.9988356678057343*r)/sqrt(10),(2+1.018497004815491*r)/sqrt(10),(2+1.038266491223517*r)/sqrt(10),(2+1.058141150105979*r)/sqrt(10),(2+1.078117994021884*r)/sqrt(10),(2+1.098194025311821*r)/sqrt(10),(2+1.118366236397724*r)/sqrt(10),(2+1.13863161008363*r)/sqrt(10),(2+1.158987119857388*r)/sqrt(10),(2+1.17942973019332*r)/sqrt(10),(2+1.199956396855759*r)/sqrt(10)]< 0then [-2/sqrt(10),-(2+1.498320841708742e-4*r)/sqrt(10),-(2+5.986466935998108e-4*r)/sqrt(10),-(2+0.001345398955533032*r)/sqrt(10),-(2+0.002389014203700413*r)/sqrt(10),-(2+0.00372838808577948*r)/sqrt(10),-(2+0.005362386673832029*r)/sqrt(10),-(2+0.007289846577694006*r)/sqrt(10),-(2+0.009509575061314376*r)/sqrt(10),-(2+0.01202035016202822*r)/sqrt(10),-(2+0.01482092081275069*r)/sqrt(10),-(2+0.01791000696708436*r)/sqrt(10),-(2+0.02128629972732062*r)/sqrt(10),-(2+0.02494846147533059*r)/sqrt(10),-(2+0.02889512600632479*r)/sqrt(10),-(2+0.0331248986654725*r)/sqrt(10),-(2+0.03763635648736519*r)/sqrt(10),-(2+0.04242804833831373*r)/sqrt(10),-(2+0.04749849506145984*r)/sqrt(10),-(2+0.05284618962468965*r)/sqrt(10),-(2+0.05846959727133633*r)/sqrt(10),-(2+0.06436715567365475*r)/sqrt(10),-(2+0.07053727508905278*r)/sqrt(10),-(2+0.07697833851906408*r)/sqrt(10),-(2+0.08368870187104593*r)/sqrt(10),-(2+0.09066669412258874*r)/sqrt(10),-(2+0.09791061748861546*r)/sqrt(10),-(2+0.1054187475911595*r)/sqrt(10),-(2+0.1131893336318011*r)/sqrt(10),-(2+0.121220598566744*r)/sqrt(10),-(2+0.1295107392845216*r)/sqrt(10),-(2+0.1380579267863034*r)/sqrt(10),-(2+0.1468603063687953*r)/sqrt(10),-(2+0.1559159978097077*r)/sqrt(10),-(2+0.1652230955557758*r)/sqrt(10),-(2+0.1747796689133147*r)/sqrt(10),-(2+0.1845837622412855*r)/sqrt(10),-(2+0.1946333951468589*r)/sqrt(10),-(2+0.2049265626834523*r)/sqrt(10),-(2+0.2154612355512225*r)/sqrt(10),-(2+0.2262353602999955*r)/sqrt(10),-(2+0.2372468595346078*r)/sqrt(10),-(2+0.2484936321226457*r)/sqrt(10),-(2+0.2599735534045555*r)/sqrt(10),-(2+0.2716844754061095*r)/sqrt(10),-(2+0.2836242270531998*r)/sqrt(10),-(2+0.2957906143889442*r)/sqrt(10),-(2+0.3081814207930817*r)/sqrt(10),-(2+0.3207944072036307*r)/sqrt(10),-(2+0.333627312340794*r)/sqrt(10),-(2+0.346677852933085*r)/sqrt(10),-(2+0.3599437239456539*r)/sqrt(10),-(2+0.3734225988107874*r)/sqrt(10),-(2+0.3871121296605642*r)/sqrt(10),-(2+0.4010099475616409*r)/sqrt(10),-(2+0.4151136627521425*r)/sqrt(10),-(2+0.4294208648806354*r)/sqrt(10),-(2+0.4439291232471635*r)/sqrt(10),-(2+0.458635987046313*r)/sqrt(10),-(2+0.4735389856122937*r)/sqrt(10),-(2+0.488635628666001*r)/sqrt(10),-(2+0.5039234065640431*r)/sqrt(10),-(2+0.5193997905497036*r)/sqrt(10),-(2+0.5350622330058146*r)/sqrt(10),-(2+0.5509081677095147*r)/sqrt(10),-(2+0.5669350100888726*r)/sqrt(10),-(2+0.5831401574813392*r)/sqrt(10),-(2+0.5995209893940125*r)/sqrt(10),-(2+0.6160748677656853*r)/sqrt(10),-(2+0.6327991372306488*r)/sqrt(10),-(2+0.6496911253842265*r)/sqrt(10),-(2+0.666748143050013*r)/sqrt(10),-(2+0.6839674845487889*r)/sqrt(10),-(2+0.7013464279690875*r)/sqrt(10),-(2+0.7188822354393821*r)/sqrt(10),-(2+0.7365721534018723*r)/sqrt(10),-(2+0.7544134128878366*r)/sqrt(10),-(2+0.7724032297945274*r)/sqrt(10),-(2+0.7905388051635788*r)/sqrt(10),-(2+0.8088173254609005*r)/sqrt(10),-(2+0.8272359628580275*r)/sqrt(10),-(2+0.8457918755149025*r)/sqrt(10),-(2+0.8644822078640563*r)/sqrt(10),-(2+0.8833040908961625*r)/sqrt(10),-(2+0.9022546424469358*r)/sqrt(10),-(2+0.9213309674853474*r)/sqrt(10),-(2+0.9405301584031224*r)/sqrt(10),-(2+0.9598492953055026*r)/sqrt(10),-(2+0.9792854463032298*r)/sqrt(10),-(2+0.9988356678057343*r)/sqrt(10),-(2+1.018497004815491*r)/sqrt(10),-(2+1.038266491223517*r)/sqrt(10),-(2+1.058141150105979*r)/sqrt(10),-(2+1.078117994021884*r)/sqrt(10),-(2+1.098194025311821*r)/sqrt(10),-(2+1.118366236397724*r)/sqrt(10),-(2+1.13863161008363*r)/sqrt(10),-(2+1.158987119857388*r)/sqrt(10),-(2+1.17942973019332*r)/sqrt(10),-(2+1.199956396855759*r)/sqrt(10)]else [2/sqrt(10),(2+1.498320841708742e-4*r)/sqrt(10),(2+5.986466935998108e-4*r)/sqrt(10),(2+0.001345398955533032*r)/sqrt(10),(2+0.002389014203700413*r)/sqrt(10),(2+0.00372838808577948*r)/sqrt(10),(2+0.005362386673832029*r)/sqrt(10),(2+0.007289846577694006*r)/sqrt(10),(2+0.009509575061314376*r)/sqrt(10),(2+0.01202035016202822*r)/sqrt(10),(2+0.01482092081275069*r)/sqrt(10),(2+0.01791000696708436*r)/sqrt(10),(2+0.02128629972732062*r)/sqrt(10),(2+0.02494846147533059*r)/sqrt(10),(2+0.02889512600632479*r)/sqrt(10),(2+0.0331248986654725*r)/sqrt(10),(2+0.03763635648736519*r)/sqrt(10),(2+0.04242804833831373*r)/sqrt(10),(2+0.04749849506145984*r)/sqrt(10),(2+0.05284618962468965*r)/sqrt(10),(2+0.05846959727133633*r)/sqrt(10),(2+0.06436715567365475*r)/sqrt(10),(2+0.07053727508905278*r)/sqrt(10),(2+0.07697833851906408*r)/sqrt(10),(2+0.08368870187104593*r)/sqrt(10),(2+0.09066669412258874*r)/sqrt(10),(2+0.09791061748861546*r)/sqrt(10),(2+0.1054187475911595*r)/sqrt(10),(2+0.1131893336318011*r)/sqrt(10),(2+0.121220598566744*r)/sqrt(10),(2+0.1295107392845216*r)/sqrt(10),(2+0.1380579267863034*r)/sqrt(10),(2+0.1468603063687953*r)/sqrt(10),(2+0.1559159978097077*r)/sqrt(10),(2+0.1652230955557758*r)/sqrt(10),(2+0.1747796689133147*r)/sqrt(10),(2+0.1845837622412855*r)/sqrt(10),(2+0.1946333951468589*r)/sqrt(10),(2+0.2049265626834523*r)/sqrt(10),(2+0.2154612355512225*r)/sqrt(10),(2+0.2262353602999955*r)/sqrt(10),(2+0.2372468595346078*r)/sqrt(10),(2+0.2484936321226457*r)/sqrt(10),(2+0.2599735534045555*r)/sqrt(10),(2+0.2716844754061095*r)/sqrt(10),(2+0.2836242270531998*r)/sqrt(10),(2+0.2957906143889442*r)/sqrt(10),(2+0.3081814207930817*r)/sqrt(10),(2+0.3207944072036307*r)/sqrt(10),(2+0.333627312340794*r)/sqrt(10),(2+0.346677852933085*r)/sqrt(10),(2+0.3599437239456539*r)/sqrt(10),(2+0.3734225988107874*r)/sqrt(10),(2+0.3871121296605642*r)/sqrt(10),(2+0.4010099475616409*r)/sqrt(10),(2+0.4151136627521425*r)/sqrt(10),(2+0.4294208648806354*r)/sqrt(10),(2+0.4439291232471635*r)/sqrt(10),(2+0.458635987046313*r)/sqrt(10),(2+0.4735389856122937*r)/sqrt(10),(2+0.488635628666001*r)/sqrt(10),(2+0.5039234065640431*r)/sqrt(10),(2+0.5193997905497036*r)/sqrt(10),(2+0.5350622330058146*r)/sqrt(10),(2+0.5509081677095147*r)/sqrt(10),(2+0.5669350100888726*r)/sqrt(10),(2+0.5831401574813392*r)/sqrt(10),(2+0.5995209893940125*r)/sqrt(10),(2+0.6160748677656853*r)/sqrt(10),(2+0.6327991372306488*r)/sqrt(10),(2+0.6496911253842265*r)/sqrt(10),(2+0.666748143050013*r)/sqrt(10),(2+0.6839674845487889*r)/sqrt(10),(2+0.7013464279690875*r)/sqrt(10),(2+0.7188822354393821*r)/sqrt(10),(2+0.7365721534018723*r)/sqrt(10),(2+0.7544134128878366*r)/sqrt(10),(2+0.7724032297945274*r)/sqrt(10),(2+0.7905388051635788*r)/sqrt(10),(2+0.8088173254609005*r)/sqrt(10),(2+0.8272359628580275*r)/sqrt(10),(2+0.8457918755149025*r)/sqrt(10),(2+0.8644822078640563*r)/sqrt(10),(2+0.8833040908961625*r)/sqrt(10),(2+0.9022546424469358*r)/sqrt(10),(2+0.9213309674853474*r)/sqrt(10),(2+0.9405301584031224*r)/sqrt(10),(2+0.9598492953055026*r)/sqrt(10),(2+0.9792854463032298*r)/sqrt(10),(2+0.9988356678057343*r)/sqrt(10),(2+1.018497004815491*r)/sqrt(10),(2+1.038266491223517*r)/sqrt(10),(2+1.058141150105979*r)/sqrt(10),(2+1.078117994021884*r)/sqrt(10),(2+1.098194025311821*r)/sqrt(10),(2+1.118366236397724*r)/sqrt(10),(2+1.13863161008363*r)/sqrt(10),(2+1.158987119857388*r)/sqrt(10),(2+1.17942973019332*r)/sqrt(10),(2+1.199956396855759*r)/sqrt(10)])
   %ploteval2:
      if maps then return %mapexpression2(x,y,f$;args());
  fcontour:
      Z=%ploteval2(f$,X,Y,maps;args());
  Try "trace errors" to inspect local variables after errors.
  plot2d:
      =style,=outline,=frame);
\end{euleroutput}
\begin{eulercomment}
Ini seharusnya menjadi beberapa fungsi, tetapi pemecah default Maxima
hanya dapat menemukan solusinya, jika kita kuadratkan persamaannya.
Akibatnya, kami mendapatkan solusi palsu.
\end{eulercomment}
\begin{eulerprompt}
>akar &= solve(getHesseForm(lineThrough(A,B),x,y,C)^2-distance([x,y],C)^2,y)
\end{eulerprompt}
\begin{euleroutput}
  Maxima said:
  solve: all variables must not be numbers.
   -- an error. To debug this try: debugmode(true);
  
  Error in:
  ... (lineThrough(A,B),x,y,C)^2-distance([x,y],C)^2,y) ...
                                                       ^
\end{euleroutput}
\begin{eulercomment}
Solusi pertama adalah

maxima: akar[1]

Menambahkan solusi pertama ke plot menunjukkan, bahwa itu memang jalan
yang kita cari. Teorinya memberi tahu kita bahwa itu adalah parabola
yang diputar.
\end{eulercomment}
\begin{eulerprompt}
>plot2d(&rhs(akar[1]),add=1):
>function g(x) &= rhs(akar[1]); $'g(x)= g(x)// fungsi yang mendefinisikan kurva di atas
>T &=[-1, g(-1)]; // ambil sebarang titik pada kurva tersebut
>dTC &= distance(T,C); $fullratsimp(dTC), $float(%) // jarak T ke C
>U &= projectToLine(T,lineThrough(A,B)); $U // proyeksi T pada garis AB 
\end{eulerprompt}
\begin{euleroutput}
  Maxima said:
  rat: replaced 5.049958083474387e-5 by 102157/2022927682 = 5.049958083474385e-5
  
  rat: replaced 2.039932534230044e-4 by 284619/1395237319 = 2.039932534230043e-4
  
  rat: replaced 4.634656435254722e-4 by 573493/1237401322 = 4.634656435254721e-4
  
  rat: replaced 8.31890779119604e-4 by 332331/399488741 = 8.318907791196046e-4
  
  rat: replaced 0.001312231792998733 by 448125/341498356 = 0.001312231792998734
  
  rat: replaced 0.001907440626462018 by 276030/144712237 = 0.001907440626462018
  
  rat: replaced 0.002620457734122131 by 2586613/987084419 = 0.002620457734122131
  
  rat: replaced 0.00345421178986248 by 3402379/984994322 = 0.00345421178986248
  
  rat: replaced 0.004411619393972596 by 966955/219183686 = 0.004411619393972597
  
  rat: replaced 0.005495584781489732 by 2798484/509224061 = 0.005495584781489734
  
  rat: replaced 0.006708999531778753 by 6054060/902378957 = 0.006708999531778753
  
  rat: replaced 0.008054742279375651 by 806546/100133061 = 0.00805474227937564
  
  rat: replaced 0.009535678426127348 by 4115324/431571181 = 0.009535678426127346
  
  rat: replaced 0.01115465985465333 by 2266398/203179481 = 0.01115465985465334
  
  rat: replaced 0.01291452464316009 by 2106925/163143829 = 0.01291452464316012
  
  rat: replaced 0.01481809678163515 by 2779203/187554653 = 0.01481809678163516
  
  rat: replaced 0.01686818588945119 by 7427428/440321683 = 0.01686818588945119
  
  rat: replaced 0.01906758693440599 by 2278085/119474216 = 0.019067586934406
  
  rat: replaced 0.02141907995322798 by 2316386/108145915 = 0.02141907995322801
  
  rat: replaced 0.02392542977357476 by 1665518/69612877 = 0.02392542977357479
  
  rat: replaced 0.02658938573755304 by 3678645/138350131 = 0.02658938573755308
  
  rat: replaced 0.02941368142678652 by 4053557/137811957 = 0.0294136814267865
  
  rat: replaced 0.03240103438906003 by 2629160/81144323 = 0.03240103438906009
  
  rat: replaced 0.03555414586656669 by 1834427/51595305 = 0.03555414586656674
  
  rat: replaced 0.03887570052578646 by 1643964/42287701 = 0.03887570052578645
  
  rat: replaced 0.04236836618902146 by 3055464/72116635 = 0.04236836618902143
  
  rat: replaced 0.04603479356761608 by 3139251/68193007 = 0.0460347935676161
  
  rat: replaced 0.04987761599688789 by 5203437/104324092 = 0.04987761599688785
  
  rat: replaced 0.05389944917279615 by 4533622/84112585 = 0.0538994491727962
  
  rat: replaced 0.05810289089037535 by 11687290/201148167 = 0.05810289089037535
  
  rat: replaced 0.06249052078395612 by 3949243/63197473 = 0.06249052078395603
  
  rat: replaced 0.0670649000692059 by 3281728/48933615 = 0.067064900069206
  
  rat: replaced 0.07182857128700804 by 4146139/57722699 = 0.07182857128700791
  
  rat: replaced 0.07678405804921068 by 1198255/15605518 = 0.07678405804921054
  
  rat: replaced 0.08193386478626702 by 5956639/72700574 = 0.08193386478626702
  
  rat: replaced 0.08728047649679532 by 2799808/32078285 = 0.08728047649679527
  
  rat: replaced 0.09282635849907966 by 10292829/110882611 = 0.09282635849907972
  
  rat: replaced 0.09857395618454184 by 4198057/42587892 = 0.09857395618454184
  
  rat: replaced 0.1045256947732028 by 30563827/292404916 = 0.1045256947732028
  
  rat: replaced 0.1106839790711635 by 8949559/80856860 = 0.1106839790711635
  
  rat: replaced 0.1170511932301264 by 9911603/84677505 = 0.1170511932301265
  
  rat: replaced 0.1236297005089814 by 19561703/158228184 = 0.1236297005089814
  
  rat: replaced 0.1304218430374826 by 4975231/38147222 = 0.1304218430374825
  
  rat: replaced 0.137429941582038 by 3502939/25488907 = 0.137429941582038
  
  rat: replaced 0.1446562953136327 by 15521432/107298697 = 0.1446562953136327
  
  rat: replaced 0.1521031815779155 by 18080502/118869979 = 0.1521031815779155
  
  rat: replaced 0.1597728556674664 by 37419026/234201397 = 0.1597728556674664
  
  rat: replaced 0.1676675505962674 by 22585897/134706429 = 0.1676675505962674
  
  rat: replaced 0.1757894768764047 by 15940893/90681725 = 0.1757894768764048
  
  rat: replaced 0.1841408222970185 by 7944795/43145213 = 0.1841408222970182
  
  rat: replaced 0.1927237517055264 by 1392861/7227241 = 0.1927237517055264
  
  rat: replaced 0.2015404067911402 by 1735485/8611102 = 0.2015404067911401
  
  rat: replaced 0.2105929058706983 by 10627754/50465869 = 0.2105929058706985
  
  rat: replaced 0.2198833436768368 by 9372347/42624179 = 0.2198833436768366
  
  rat: replaced 0.2294137911485169 by 7405273/32279110 = 0.2294137911485168
  
  rat: replaced 0.239186295223934 by 27692337/115777273 = 0.239186295223934
  
  rat: replaced 0.2492028786358237 by 8925310/35815437 = 0.249202878635824
  
  rat: replaced 0.2594655397091927 by 11150701/42975653 = 0.259465539709193
  
  rat: replaced 0.2699762521614856 by 11249087/41666950 = 0.2699762521614853
  
  rat: replaced 0.280736964905216 by 12097010/43090193 = 0.2807369649052164
  
  rat: replaced 0.2917496018530771 by 14831788/50837389 = 0.2917496018530771
  
  rat: replaced 0.3030160617255513 by 18597622/61375037 = 0.3030160617255514
  
  rat: replaced 0.3145382178610399 by 11102944/35299189 = 0.3145382178610392
  
  rat: replaced 0.3263179180285316 by 13053510/40002431 = 0.3263179180285318
  
  rat: replaced 0.3383569842428258 by 13796661/40775458 = 0.3383569842428257
  
  rat: replaced 0.3506572125823338 by 33496033/95523582 = 0.3506572125823338
  
  rat: replaced 0.3632203730094723 by 23086207/63559780 = 0.3632203730094724
  
  rat: replaced 0.376048209193667 by 22674222/60296051 = 0.3760482091936668
  
  rat: replaced 0.3891424383369902 by 33246815/85436107 = 0.3891424383369902
  
  rat: replaced 0.402504751002439 by 7793813/19363282 = 0.4025047510024385
  
  rat: replaced 0.4161368109448825 by 10481453/25187517 = 0.4161368109448819
  
  rat: replaced 0.4300402549446862 by 19565443/45496771 = 0.4300402549446861
  
  rat: replaced 0.4442166926440365 by 16102633/36249500 = 0.4442166926440365
  
  rat: replaced 0.4586677063859775 by 19404529/42306290 = 0.4586677063859771
  
  rat: replaced 0.4733948510561774 by 10262860/21679281 = 0.4733948510561766
  
  rat: replaced 0.4883996539274416 by 4159841/8517289 = 0.488399653927441
  
  rat: replaced 0.5036836145069872 by 13202363/26211619 = 0.5036836145069864
  
  rat: replaced 0.5192482043864929 by 12221370/23536663 = 0.5192482043864927
  
  rat: replaced 0.5350948670949413 by 52965833/98984005 = 0.5350948670949413
  
  rat: replaced 0.551225017954267 by 14288533/25921416 = 0.551225017954266
  
  rat: replaced 0.5676400439378262 by 25565995/45039097 = 0.5676400439378259
  
  rat: replaced 0.5843413035316997 by 18888222/32323955 = 0.5843413035316997
  
  rat: replaced 0.6013301265988455 by 5789399/9627655 = 0.6013301265988447
  
  rat: replaced 0.6186078142461149 by 4803773/7765458 = 0.618607814246114
  
  rat: replaced 0.6361756386941407 by 13914515/21872128 = 0.6361756386941407
  
  rat: replaced 0.6540348431501183 by 48160581/73636109 = 0.6540348431501181
  
  rat: replaced 0.6721866416834846 by 6617334/9844489 = 0.6721866416834841
  
  rat: replaced 0.690632219104513 by 16840135/24383651 = 0.6906322191045139
  
  rat: replaced 0.7093727308458327 by 29189494/41148317 = 0.7093727308458326
  
  rat: replaced 0.7284093028468864 by 13153959/18058472 = 0.7284093028468854
  
  rat: replaced 0.7477430314413382 by 12236470/16364539 = 0.7477430314413379
  
  rat: replaced 0.7673749832474404 by 39576757/51574208 = 0.7673749832474402
  
  rat: replaced 0.7873061950613714 by 6818881/8661028 = 0.7873061950613714
  
  rat: replaced 0.8075376737535601 by 20498953/25384516 = 0.807537673753559
  
  rat: replaced 0.8280703961679966 by 6989671/8440914 = 0.8280703961679979
  
  rat: replaced 0.84890530902455 by 25231431/29722315 = 0.8489053090245494
  
  rat: replaced 0.8700433288242969 by 9721738/11173855 = 0.8700433288242957
  
  rat: replaced 0.8914853417578728 by 33469619/37543656 = 0.8914853417578725
  
  rat: replaced 0.9132322036168524 by 21961040/24047597 = 0.913232203616852
  
  rat: replaced 1.498320841708742e-4 by 1329822/8875415485 = 1.498320841708742e-4
  
  rat: replaced 5.986466935998108e-4 by 398723/666040595 = 5.986466935998098e-4
  
  rat: replaced 0.001345398955533032 by 4525441/3363642421 = 0.001345398955533032
  
  rat: replaced 0.002389014203700413 by 1071627/448564516 = 0.00238901420370041
  
  rat: replaced 0.00372838808577948 by 661903/177530607 = 0.003728388085779485
  
  rat: replaced 0.005362386673832029 by 5230891/975478144 = 0.005362386673832028
  
  rat: replaced 0.007289846577694006 by 32241346/4422774287 = 0.007289846577694006
  
  rat: replaced 0.009509575061314376 by 2146493/225719129 = 0.009509575061314364
  
  rat: replaced 0.01202035016202822 by 1789188/148846579 = 0.01202035016202825
  
  rat: replaced 0.01482092081275069 by 2581665/174190594 = 0.01482092081275066
  
  rat: replaced 0.01791000696708436 by 5107285/285163764 = 0.01791000696708436
  
  rat: replaced 0.02128629972732062 by 3323295/156123659 = 0.02128629972732064
  
  rat: replaced 0.02494846147533059 by 4548287/182307314 = 0.02494846147533061
  
  rat: replaced 0.02889512600632479 by 3147802/108938857 = 0.02889512600632481
  
  rat: replaced 0.0331248986654725 by 5858625/176864692 = 0.03312489866547248
  
  rat: replaced 0.03763635648736519 by 10043830/266865099 = 0.03763635648736518
  
  rat: replaced 0.04242804833831373 by 4635713/109260576 = 0.04242804833831372
  
  rat: replaced 0.04749849506145984 by 5610259/118114458 = 0.04749849506145979
  
  rat: replaced 0.05284618962468965 by 4237503/80185592 = 0.05284618962468968
  
  rat: replaced 0.05846959727133633 by 3317197/56733707 = 0.05846959727133642
  
  rat: replaced 0.06436715567365475 by 13427433/208606903 = 0.06436715567365477
  
  rat: replaced 0.07053727508905278 by 8025659/113778977 = 0.07053727508905269
  
  rat: replaced 0.07697833851906408 by 6306881/81930594 = 0.07697833851906408
  
  rat: replaced 0.08368870187104593 by 4282086/51166835 = 0.08368870187104596
  
  rat: replaced 0.09066669412258874 by 2175091/23989967 = 0.09066669412258883
  
  rat: replaced 0.09791061748861546 by 8290049/84669561 = 0.09791061748861554
  
  rat: replaced 0.1054187475911595 by 8501563/80645646 = 0.1054187475911595
  
  rat: replaced 0.1131893336318011 by 6539019/57770629 = 0.113189333631801
  
  rat: replaced 0.121220598566744 by 5779101/47674249 = 0.1212205985667441
  
  rat: replaced 0.1295107392845216 by 7134865/55090914 = 0.1295107392845216
  
  rat: replaced 0.1380579267863034 by 6113057/44278928 = 0.1380579267863034
  
  rat: replaced 0.1468603063687953 by 6311140/42973763 = 0.1468603063687953
  
  rat: replaced 0.1559159978097077 by 4027079/25828517 = 0.1559159978097078
  
  rat: replaced 0.1652230955557758 by 10597125/64138279 = 0.1652230955557757
  
  rat: replaced 0.1747796689133147 by 9649007/55206690 = 0.1747796689133147
  
  rat: replaced 0.1845837622412855 by 6871913/37229239 = 0.1845837622412857
  
  rat: replaced 0.1946333951468589 by 39341769/202132676 = 0.1946333951468589
  
  rat: replaced 0.2049265626834523 by 10758647/52500012 = 0.2049265626834523
  
  rat: replaced 0.2154612355512225 by 33702610/156420759 = 0.2154612355512225
  
  rat: replaced 0.2262353602999955 by 2338161/10335082 = 0.2262353602999957
  
  rat: replaced 0.2372468595346078 by 7573078/31920667 = 0.2372468595346081
  
  rat: replaced 0.2484936321226457 by 3764353/15148690 = 0.2484936321226456
  
  rat: replaced 0.2599735534045555 by 26335713/101301508 = 0.2599735534045554
  
  rat: replaced 0.2716844754061095 by 29831699/109802737 = 0.2716844754061094
  
  rat: replaced 0.2836242270531998 by 15100773/53242183 = 0.2836242270531995
  
  rat: replaced 0.2957906143889442 by 2942977/9949528 = 0.2957906143889439
  
  rat: replaced 0.3081814207930817 by 12077608/39189929 = 0.3081814207930818
  
  rat: replaced 0.3207944072036307 by 9185023/28632117 = 0.3207944072036308
  
  rat: replaced 0.333627312340794 by 5228336/15671187 = 0.3336273123407946
  
  rat: replaced 0.346677852933085 by 15615111/45042136 = 0.3466778529330847
  
  rat: replaced 0.3599437239456539 by 7564465/21015688 = 0.3599437239456543
  
  rat: replaced 0.3734225988107874 by 7702871/20627758 = 0.3734225988107869
  
  rat: replaced 0.3871121296605642 by 97723109/252441351 = 0.3871121296605642
  
  rat: replaced 0.4010099475616409 by 3146543/7846546 = 0.4010099475616405
  
  rat: replaced 0.4151136627521425 by 6219049/14981557 = 0.4151136627521425
  
  rat: replaced 0.4294208648806354 by 26148647/60892819 = 0.4294208648806356
  
  rat: replaced 0.4439291232471635 by 19525684/43983787 = 0.4439291232471638
  
  rat: replaced 0.458635987046313 by 38604672/84172793 = 0.4586359870463132
  
  rat: replaced 0.4735389856122937 by 11146199/23538081 = 0.4735389856122935
  
  rat: replaced 0.488635628666001 by 13946471/28541658 = 0.4886356286660011
  
  rat: replaced 0.5039234065640431 by 5948069/11803518 = 0.503923406564043
  
  rat: replaced 0.5193997905497036 by 24027011/46259185 = 0.5193997905497038
  
  rat: replaced 0.5350622330058146 by 7363779/13762472 = 0.5350622330058147
  
  rat: replaced 0.5509081677095147 by 8130825/14758948 = 0.5509081677095142
  
  rat: replaced 0.5669350100888726 by 10250363/18080314 = 0.5669350100888735
  
  rat: replaced 0.5831401574813392 by 37655026/64572857 = 0.5831401574813393
  
  rat: replaced 0.5995209893940125 by 30778651/51338738 = 0.5995209893940128
  
  rat: replaced 0.6160748677656853 by 23698401/38466755 = 0.616074867765685
  
  rat: replaced 0.6327991372306488 by 5052598/7984521 = 0.6327991372306492
  
  rat: replaced 0.6496911253842265 by 60646047/93345968 = 0.6496911253842266
  
  rat: replaced 0.666748143050013 by 30125566/45182827 = 0.6667481430500132
  
  rat: replaced 0.6839674845487889 by 8953739/13090884 = 0.6839674845487899
  
  rat: replaced 0.7013464279690875 by 7888577/11247761 = 0.7013464279690864
  
  rat: replaced 0.7188822354393821 by 16662338/23178119 = 0.7188822354393815
  
  rat: replaced 0.7365721534018723 by 13899283/18870226 = 0.7365721534018723
  
  rat: replaced 0.7544134128878366 by 16270763/21567436 = 0.754413412887837
  
  rat: replaced 0.7724032297945274 by 8203205/10620366 = 0.7724032297945287
  
  rat: replaced 0.7905388051635788 by 10794522/13654639 = 0.7905388051635784
  
  rat: replaced 0.8088173254609005 by 16745047/20703126 = 0.808817325460899
  
  rat: replaced 0.8272359628580275 by 20291194/24528907 = 0.827235962858027
  
  rat: replaced 0.8457918755149025 by 10996366/13001267 = 0.8457918755149018
  
  rat: replaced 0.8644822078640563 by 9158500/10594203 = 0.8644822078640555
  
  rat: replaced 0.8833040908961625 by 13759446/15577247 = 0.8833040908961641
  
  rat: replaced 0.9022546424469358 by 19827819/21975857 = 0.9022546424469362
  
  rat: replaced 0.9213309674853474 by 60458149/65620446 = 0.9213309674853475
  
  rat: replaced 0.9405301584031224 by 11658841/12396031 = 0.9405301584031212
  
  rat: replaced 0.9598492953055026 by 26214088/27310629 = 0.9598492953055018
  
  rat: replaced 0.9792854463032298 by 35089005/35831233 = 0.9792854463032293
  
  rat: replaced 0.9988356678057343 by 15735752/15754095 = 0.9988356678057356
  
  rat: replaced 1.018497004815491 by 16202286/15908035 = 1.018497004815491
  
  rat: replaced 1.038266491223517 by 17763365/17108676 = 1.038266491223517
  
  rat: replaced 1.058141150105979 by 33730321/31876958 = 1.058141150105979
  
  rat: replaced 1.078117994021884 by 51996446/48228901 = 1.078117994021883
  
  rat: replaced 1.098194025311821 by 124719922/113568203 = 1.098194025311821
  
  rat: replaced 1.118366236397724 by 92837336/83011569 = 1.118366236397724
  
  rat: replaced 1.13863161008363 by 20601995/18093644 = 1.138631610083629
  
  rat: replaced 1.158987119857388 by 20626233/17796775 = 1.15898711985739
  
  rat: replaced 1.17942973019332 by 4098089/3474636 = 1.179429730193321
  
  rat: replaced 1.199956396855759 by 17442145/14535649 = 1.199956396855758
  part: invalid index of list or matrix.
  #0: lineIntersection(g=[3,1,-3],h=[-1,3,-2])
  #1: projectToLine(a=[-1,0],g=[-1,3,-2])
   -- an error. To debug this try: debugmode(true);
  
  Error in:
  U &= projectToLine(T,lineThrough(A,B)); $U // proyeksi T pada  ...
                                        ^
\end{euleroutput}
\begin{eulerprompt}
>dU2AB &= distance(T,U); $fullratsimp(dU2AB), $float(%) // jatak T ke AB
\end{eulerprompt}
\begin{eulercomment}
Ternyata jarak T ke C sama dengan jarak T ke AB. Coba Anda pilih titik T yang lain dan
ulangi perhitungan-perhitungan di atas untuk menunjukkan bahwa hasilnya juga sama.
\end{eulercomment}
\eulersubheading{Contoh 5: Trigonometri Rasional}
\begin{eulercomment}
Ini terinspirasi dari ceramah N.J.Wildberger. Dalam bukunya "Divine
Proportions", Wildberger mengusulkan untuk mengganti pengertian klasik
tentang jarak dan sudut dengan kuadrat dan penyebaran. Dengan
menggunakan ini, memang mungkin untuk menghindari fungsi trigonometri
dalam banyak contoh, dan tetap "rasional".

Berikut ini, saya memperkenalkan konsep, dan memecahkan beberapa
masalah. Saya menggunakan perhitungan simbolik Maxima di sini, yang
menyembunyikan keuntungan utama dari trigonometri rasional bahwa
perhitungan hanya dapat dilakukan dengan kertas dan pensil. Anda
diundang untuk memeriksa hasil tanpa komputer.

Intinya adalah bahwa perhitungan rasional simbolis sering kali
menghasilkan hasil yang sederhana. Sebaliknya, trigonometri klasik
menghasilkan hasil trigonometri yang rumit, yang hanya mengevaluasi
perkiraan numerik.
\end{eulercomment}
\begin{eulerprompt}
>load geometry;
\end{eulerprompt}
\begin{eulercomment}
Untuk pengenalan pertama, kami menggunakan segitiga persegi panjang
dengan proporsi Mesir terkenal 3, 4 dan 5. Perintah berikut adalah
perintah Euler untuk merencanakan geometri bidang yang terdapat dalam
file Euler "geometry.e".
\end{eulercomment}
\begin{eulerprompt}
>C&:=[0,0]; A&:=[4,0]; B&:=[0,3]; ...
>setPlotRange(-1,5,-1,5); ...
>plotPoint(A,"A"); plotPoint(B,"B"); plotPoint(C,"C"); ...
>plotSegment(B,A,"c"); plotSegment(A,C,"b"); plotSegment(C,B,"a"); ...
>insimg(30);
\end{eulerprompt}
\begin{eulercomment}
Tentu saja,

\end{eulercomment}
\begin{eulerformula}
\[
\sin(w_a)=\frac{a}{c},
\]
\end{eulerformula}
\begin{eulercomment}
di mana wa adalah sudut di A. Cara yang biasa untuk menghitung sudut
ini, adalah dengan mengambil invers dari fungsi sinus. Hasilnya adalah
sudut yang tidak dapat dicerna, yang hanya dapat dicetak kira-kira.
\end{eulercomment}
\begin{eulerprompt}
>wa := arcsin(3/5); degprint(wa)
\end{eulerprompt}
\begin{euleroutput}
  36°52'11.63''
\end{euleroutput}
\begin{eulercomment}
Trigonometri rasional mencoba menghindari hal ini.

Gagasan pertama trigonometri rasional adalah kuadran, yang
menggantikan jarak. Sebenarnya, itu hanya jarak kuadrat. Berikut ini,
a, b, dan c menunjukkan kuadrat dari sisi-sisinya.

Teorema Pythogoras menjadi a+b=c.
\end{eulercomment}
\begin{eulerprompt}
>a &= 3^2; b &= 4^2; c &= 5^2; &a+b=c
\end{eulerprompt}
\begin{euleroutput}
  
                                 25 = 25
  
\end{euleroutput}
\begin{eulercomment}
Pengertian kedua dari trigonometri rasional adalah penyebaran. Spread
mengukur pembukaan antar baris. Ini adalah 0, jika garis-garisnya
sejajar, dan 1, jika garis-garisnya persegi panjang. Ini adalah
kuadrat sinus sudut antara dua garis.

Penyebaran garis AB dan AC pada gambar di atas didefinisikan sebagai:

\end{eulercomment}
\begin{eulerformula}
\[
s_a = \sin(\alpha)^2 = \frac{a}{c},
\]
\end{eulerformula}
\begin{eulercomment}
di mana a dan c adalah kuadrat dari sembarang segitiga siku-siku
dengan salah satu sudut di A.
\end{eulercomment}
\begin{eulerprompt}
>sa &= a/c; $sa
\end{eulerprompt}
\begin{eulercomment}
Ini lebih mudah dihitung daripada sudut, tentu saja. Tetapi Anda
kehilangan properti bahwa sudut dapat ditambahkan dengan mudah.

Tentu saja, kita dapat mengonversi nilai perkiraan untuk sudut wa
menjadi sprad, dan mencetaknya sebagai pecahan.
\end{eulercomment}
\begin{eulerprompt}
>fracprint(sin(wa)^2)
\end{eulerprompt}
\begin{euleroutput}
  9/25
\end{euleroutput}
\begin{eulercomment}
Hukum kosinus trgonometri klasik diterjemahkan menjadi "hukum silang"
berikut.

\end{eulercomment}
\begin{eulerformula}
\[
(c+b-a)^2 = 4 b c \, (1-s_a)
\]
\end{eulerformula}
\begin{eulercomment}
Di sini a, b, dan c adalah kuadrat dari sisi-sisi segitiga, dan sa
adalah penyebaran sudut A. Sisi a, seperti biasa, berhadapan dengan
sudut A.

Hukum ini diimplementasikan dalam file geometri.e yang kami muat ke
Euler.
\end{eulercomment}
\begin{eulerprompt}
>$crosslaw(aa,bb,cc,saa)
\end{eulerprompt}
\begin{eulercomment}
Dalam kasus kami, kami mendapatkan
\end{eulercomment}
\begin{eulerprompt}
>$crosslaw(a,b,c,sa)
\end{eulerprompt}
\begin{eulercomment}
Mari kita gunakan crosslaw ini untuk mencari spread di A. Untuk
melakukan ini, kita buat crosslaw untuk kuadran a, b, dan c, dan
selesaikan untuk spread yang tidak diketahui sa.

Anda dapat melakukannya dengan tangan dengan mudah, tetapi saya
menggunakan Maxima. Tentu saja, kami mendapatkan hasilnya, kami sudah
memilikinya.
\end{eulercomment}
\begin{eulerprompt}
>$crosslaw(a,b,c,x), $solve(%,x)
\end{eulerprompt}
\begin{euleroutput}
  Maxima said:
  solve: all variables must not be numbers.
   -- an error. To debug this try: debugmode(true);
  
  Error in:
   $crosslaw(a,b,c,x), $solve(%,x) ...
                                 ^
\end{euleroutput}
\begin{eulercomment}
Kita sudah tahu ini. Definisi spread adalah kasus khusus dari
crosslaw.

Kita juga dapat menyelesaikan ini untuk umum a,b,c. Hasilnya adalah
rumus yang menghitung penyebaran sudut segitiga yang diberikan kuadrat
dari ketiga sisinya.
\end{eulercomment}
\begin{eulerprompt}
>$solve(crosslaw(aa,bb,cc,x),x)
\end{eulerprompt}
\begin{euleroutput}
  Maxima said:
  solve: all variables must not be numbers.
   -- an error. To debug this try: debugmode(true);
  
  Error in:
   $solve(crosslaw(aa,bb,cc,x),x) ...
                                ^
\end{euleroutput}
\begin{eulercomment}
Kita bisa membuat fungsi dari hasilnya. Fungsi seperti itu sudah
didefinisikan dalam file geometri.e dari Euler.
\end{eulercomment}
\begin{eulerprompt}
>$spread(a,b,c)
\end{eulerprompt}
\begin{eulercomment}
Sebagai contoh, kita dapat menggunakannya untuk menghitung sudut
segitiga dengan sisi

\end{eulercomment}
\begin{eulerformula}
\[
a, \quad a, \quad \frac{4a}{7}
\]
\end{eulerformula}
\begin{eulercomment}
Hasilnya rasional, yang tidak begitu mudah didapat jika kita
menggunakan trigonometri klasik.
\end{eulercomment}
\begin{eulerprompt}
>$spread(a,a,4*a/7)
\end{eulerprompt}
\begin{eulercomment}
Ini adalah sudut dalam derajat.
\end{eulercomment}
\begin{eulerprompt}
>degprint(arcsin(sqrt(6/7)))
\end{eulerprompt}
\begin{euleroutput}
  67°47'32.44''
\end{euleroutput}
\eulersubheading{Contoh lain}
\begin{eulercomment}
Sekarang, mari kita coba contoh yang lebih maju.

Kami mengatur tiga sudut segitiga sebagai berikut.
\end{eulercomment}
\begin{eulerprompt}
>A&:=[1,2]; B&:=[4,3]; C&:=[0,4]; ...
>setPlotRange(-1,5,1,7); ...
>plotPoint(A,"A"); plotPoint(B,"B"); plotPoint(C,"C"); ...
>plotSegment(B,A,"c"); plotSegment(A,C,"b"); plotSegment(C,B,"a"); ...
>insimg;
\end{eulerprompt}
\begin{eulercomment}
Menggunakan Pythogoras, mudah untuk menghitung jarak antara dua titik.
Saya pertama kali menggunakan jarak fungsi file Euler untuk geometri.
Jarak fungsi menggunakan geometri klasik.
\end{eulercomment}
\begin{eulerprompt}
>$distance(A,B)
\end{eulerprompt}
\begin{eulercomment}
Euler juga mengandung fungsi untuk kuadran antara dua titik.

Dalam contoh berikut, karena c+b bukan a, maka segitiga itu bukan
persegi panjang.
\end{eulercomment}
\begin{eulerprompt}
>c &= quad(A,B); $c, b &= quad(A,C); $b, a &= quad(B,C); $a,
\end{eulerprompt}
\begin{eulercomment}
Pertama, mari kita hitung sudut tradisional. Fungsi computeAngle
menggunakan metode biasa berdasarkan hasil kali titik dua vektor.
Hasilnya adalah beberapa pendekatan floating point.

\end{eulercomment}
\begin{eulerformula}
\[
A=<1,2>\quad B=<4,3>,\quad C=<0,4>
\]
\end{eulerformula}
\begin{eulerformula}
\[
\mathbf{a}=C-B=<-4,1>,\quad \mathbf{c}=A-B=<-3,-1>,\quad \beta=\angle ABC
\]
\end{eulerformula}
\begin{eulerformula}
\[
\mathbf{a}.\mathbf{c}=|\mathbf{a}|.|\mathbf{c}|\cos \beta
\]
\end{eulerformula}
\begin{eulerformula}
\[
\cos \angle ABC =\cos\beta=\frac{\mathbf{a}.\mathbf{c}}{|\mathbf{a}|.|\mathbf{c}|}=\frac{12-1}{\sqrt{17}\sqrt{10}}=\frac{11}{\sqrt{17}\sqrt{10}}
\]
\end{eulerformula}
\begin{eulerprompt}
>wb &= computeAngle(A,B,C); $wb, $(wb/pi*180)()
\end{eulerprompt}
\begin{euleroutput}
  32.4711922908
\end{euleroutput}
\begin{eulercomment}
Dengan menggunakan pensil dan kertas, kita dapat melakukan hal yang
sama dengan hukum silang. Kami memasukkan kuadran a, b, dan c ke dalam
hukum silang dan menyelesaikan x.
\end{eulercomment}
\begin{eulerprompt}
>$crosslaw(a,b,c,x), $solve(%,x), //(b+c-a)^=4b.c(1-x)
\end{eulerprompt}
\begin{euleroutput}
  Maxima said:
  solve: all variables must not be numbers.
   -- an error. To debug this try: debugmode(true);
  
  Error in:
   $crosslaw(a,b,c,x), $solve(%,x), //(b+c-a)^=4b.c(1-x) ...
                                 ^
\end{euleroutput}
\begin{eulercomment}
Yaitu, apa yang dilakukan oleh penyebaran fungsi yang didefinisikan
dalam "geometry.e".
\end{eulercomment}
\begin{eulerprompt}
>sb &= spread(b,a,c); $sb
\end{eulerprompt}
\begin{eulercomment}
Maxima mendapatkan hasil yang sama menggunakan trigonometri biasa,
jika kita memaksanya. Itu menyelesaikan istilah sin(arccos(...))
menjadi hasil pecahan. Sebagian besar siswa tidak dapat melakukan ini.
\end{eulercomment}
\begin{eulerprompt}
>$sin(computeAngle(A,B,C))^2
\end{eulerprompt}
\begin{eulercomment}
Setelah kita memiliki spread di B, kita dapat menghitung tinggi ha di
sisi a. Ingat bahwa

\end{eulercomment}
\begin{eulerformula}
\[
s_b=\frac{h_a}{c}
\]
\end{eulerformula}
\begin{eulercomment}
Menurut definisi.
\end{eulercomment}
\begin{eulerprompt}
>ha &= c*sb; $ha
\end{eulerprompt}
\begin{eulercomment}
Gambar berikut telah dihasilkan dengan program geometri C.a.R., yang
dapat menggambar kuadrat dan menyebar.

image: (20) Rational\_Geometry\_CaR.png

Menurut definisi, panjang ha adalah akar kuadrat dari kuadratnya.
\end{eulercomment}
\begin{eulerprompt}
>$sqrt(ha)
\end{eulerprompt}
\begin{eulercomment}
Sekarang kita dapat menghitung luas segitiga. Jangan lupa, bahwa kita
berhadapan dengan kuadrat!
\end{eulercomment}
\begin{eulerprompt}
>$sqrt(ha)*sqrt(a)/2
\end{eulerprompt}
\begin{eulercomment}
Rumus determinan biasa menghasilkan hasil yang sama.
\end{eulercomment}
\begin{eulerprompt}
>$areaTriangle(B,A,C)
\end{eulerprompt}
\eulersubheading{Rumus Heron}
\begin{eulercomment}
Sekarang, mari kita selesaikan masalah ini secara umum!
\end{eulercomment}
\begin{eulerprompt}
>&remvalue(a,b,c,sb,ha);
\end{eulerprompt}
\begin{eulercomment}
Pertama kita hitung spread di B untuk segitiga dengan sisi a, b, dan
c. Kemudian kita menghitung luas kuadrat ("quadrea"?), faktorkan
dengan Maxima, dan kita mendapatkan rumus Heron yang terkenal.

Memang, ini sulit dilakukan dengan pensil dan kertas.
\end{eulercomment}
\begin{eulerprompt}
>$spread(b^2,c^2,a^2), $factor(%*c^2*a^2/4)
\end{eulerprompt}
\eulersubheading{Aturan Triple Spread}
\begin{eulercomment}
Kerugian dari spread adalah mereka tidak lagi hanya menambahkan sudut
yang sama.

Namun, tiga spread dari sebuah segitiga memenuhi aturan "triple
spread" berikut.
\end{eulercomment}
\begin{eulerprompt}
>&remvalue(sa,sb,sc); $triplespread(sa,sb,sc)
\end{eulerprompt}
\begin{eulercomment}
Aturan ini berlaku untuk setiap tiga sudut yang menambah 180 °.

\end{eulercomment}
\begin{eulerformula}
\[
\alpha+\beta+\gamma=\pi
\]
\end{eulerformula}
\begin{eulercomment}
Sejak menyebar

\end{eulercomment}
\begin{eulerformula}
\[
\alpha, \pi-\alpha
\]
\end{eulerformula}
\begin{eulercomment}
sama, aturan triple spread juga benar, jika

\end{eulercomment}
\begin{eulerformula}
\[
\alpha+\beta=\gamma
\]
\end{eulerformula}
\begin{eulercomment}
Karena penyebaran sudut negatif adalah sama, aturan penyebaran rangkap
tiga juga berlaku, jika

\end{eulercomment}
\begin{eulerformula}
\[
\alpha+\beta+\gamma=0
\]
\end{eulerformula}
\begin{eulercomment}
Misalnya, kita dapat menghitung penyebaran sudut 60°. Ini 3/4.
Persamaan memiliki solusi kedua, bagaimanapun, di mana semua spread
adalah 0.
\end{eulercomment}
\begin{eulerprompt}
>$solve(triplespread(x,x,x),x)
\end{eulerprompt}
\begin{euleroutput}
  Maxima said:
  solve: all variables must not be numbers.
   -- an error. To debug this try: debugmode(true);
  
  Error in:
   $solve(triplespread(x,x,x),x) ...
                               ^
\end{euleroutput}
\begin{eulercomment}
Sebaran 90° jelas 1. Jika dua sudut dijumlahkan menjadi 90°,
sebarannya menyelesaikan persamaan sebaran rangkap tiga dengan a,b,1.
Dengan perhitungan berikut kita mendapatkan a+b=1.
\end{eulercomment}
\begin{eulerprompt}
>$triplespread(x,y,1), $solve(%,x)
\end{eulerprompt}
\begin{euleroutput}
  Maxima said:
  solve: all variables must not be numbers.
   -- an error. To debug this try: debugmode(true);
  
  Error in:
   $triplespread(x,y,1), $solve(%,x) ...
                                   ^
\end{euleroutput}
\begin{eulercomment}
Karena sebaran 180°-t sama dengan sebaran t, rumus sebaran rangkap
tiga juga berlaku, jika satu sudut adalah jumlah atau selisih dua
sudut lainnya.

Jadi kita dapat menemukan penyebaran sudut berlipat ganda. Perhatikan
bahwa ada dua solusi lagi. Kami membuat ini fungsi.
\end{eulercomment}
\begin{eulerprompt}
>$solve(triplespread(a,a,x),x), function doublespread(a) &= factor(rhs(%[1]))
\end{eulerprompt}
\begin{euleroutput}
  Maxima said:
  solve: all variables must not be numbers.
   -- an error. To debug this try: debugmode(true);
  
  Error in:
   $solve(triplespread(a,a,x),x), function doublespread(a) &= fac ...
                               ^
\end{euleroutput}
\eulersubheading{Pembagi Sudut}
\begin{eulercomment}
Ini situasinya, kita sudah tahu.
\end{eulercomment}
\begin{eulerprompt}
>C&:=[0,0]; A&:=[4,0]; B&:=[0,3]; ...
>setPlotRange(-1,5,-1,5); ...
>plotPoint(A,"A"); plotPoint(B,"B"); plotPoint(C,"C"); ...
>plotSegment(B,A,"c"); plotSegment(A,C,"b"); plotSegment(C,B,"a"); ...
>insimg;
\end{eulerprompt}
\begin{eulercomment}
Mari kita hitung panjang garis bagi sudut di A. Tetapi kita ingin
menyelesaikannya untuk umum a,b,c.
\end{eulercomment}
\begin{eulerprompt}
>&remvalue(a,b,c);
\end{eulerprompt}
\begin{eulercomment}
Jadi pertama-tama kita hitung penyebaran sudut yang dibagi dua di A,
dengan menggunakan rumus sebaran rangkap tiga.

Masalah dengan rumus ini muncul lagi. Ini memiliki dua solusi. Kita
harus memilih yang benar. Solusi lainnya mengacu pada sudut terbelah
180 °-wa.
\end{eulercomment}
\begin{eulerprompt}
>$triplespread(x,x,a/(a+b)), $solve(%,x), sa2 &= rhs(%[1]); $sa2
\end{eulerprompt}
\begin{euleroutput}
  Maxima said:
  solve: all variables must not be numbers.
   -- an error. To debug this try: debugmode(true);
  
  Error in:
   $triplespread(x,x,a/(a+b)), $solve(%,x), sa2 &= rhs(%[1]); $sa ...
                                         ^
\end{euleroutput}
\begin{eulercomment}
Mari kita periksa persegi panjang Mesir.
\end{eulercomment}
\begin{eulerprompt}
>$sa2 with [a=3^2,b=4^2]
\end{eulerprompt}
\begin{eulercomment}
Kami dapat mencetak sudut dalam Euler, setelah mentransfer penyebaran
ke radian.
\end{eulercomment}
\begin{eulerprompt}
>wa2 := arcsin(sqrt(1/10)); degprint(wa2)
\end{eulerprompt}
\begin{euleroutput}
  18°26'5.82''
\end{euleroutput}
\begin{eulercomment}
Titik P adalah perpotongan garis bagi sudut dengan sumbu y.
\end{eulercomment}
\begin{eulerprompt}
>P := [0,tan(wa2)*4]
\end{eulerprompt}
\begin{euleroutput}
  [0,  1.33333]
\end{euleroutput}
\begin{eulerprompt}
>plotPoint(P,"P"); plotSegment(A,P):
\end{eulerprompt}
\begin{eulercomment}
Mari kita periksa sudut dalam contoh spesifik kita.
\end{eulercomment}
\begin{eulerprompt}
>computeAngle(C,A,P), computeAngle(P,A,B)
\end{eulerprompt}
\begin{euleroutput}
  0.321750554397
  0.321750554397
\end{euleroutput}
\begin{eulercomment}
Sekarang kita hitung panjang garis bagi AP.

Kami menggunakan teorema sinus dalam segitiga APC. Teorema ini
menyatakan bahwa

\end{eulercomment}
\begin{eulerformula}
\[
\frac{BC}{\sin(w_a)} = \frac{AC}{\sin(w_b)} = \frac{AB}{\sin(w_c)}
\]
\end{eulerformula}
\begin{eulercomment}
berlaku dalam segitiga apa pun. Kuadratkan, itu diterjemahkan ke dalam
apa yang disebut "hukum penyebaran"

\end{eulercomment}
\begin{eulerformula}
\[
\frac{a}{s_a} = \frac{b}{s_b} = \frac{c}{s_b}
\]
\end{eulerformula}
\begin{eulercomment}
di mana a,b,c menunjukkan qudrances.

Karena spread CPA adalah 1-sa2, kita dapatkan darinya bisa/1=b/(1-sa2)
dan dapat menghitung bisa (kuadran dari garis-bagi sudut).
\end{eulercomment}
\begin{eulerprompt}
>&factor(ratsimp(b/(1-sa2))); bisa &= %; $bisa
\end{eulerprompt}
\begin{eulercomment}
Mari kita periksa rumus ini untuk nilai-nilai Mesir kita.
\end{eulercomment}
\begin{eulerprompt}
>sqrt(mxmeval("at(bisa,[a=3^2,b=4^2])")), distance(A,P)
\end{eulerprompt}
\begin{euleroutput}
  Variable sa2 not found!
  Use global variables or parameters for string evaluation.
  Error in Evaluate, superfluous characters found.
  Try "trace errors" to inspect local variables after errors.
  mxmeval:
      return evaluate(mxm(s));
  Error in:
  sqrt(mxmeval("at(bisa,[a=3^2,b=4^2])")), distance(A,P) ...
                                        ^
\end{euleroutput}
\begin{eulercomment}
Kita juga dapat menghitung P menggunakan rumus spread.
\end{eulercomment}
\begin{eulerprompt}
>py&=factor(ratsimp(sa2*bisa)); $py
\end{eulerprompt}
\begin{eulercomment}
Nilainya sama dengan yang kita dapatkan dengan rumus trigonometri.
\end{eulercomment}
\begin{eulerprompt}
>sqrt(mxmeval("at(py,[a=3^2,b=4^2])"))
\end{eulerprompt}
\begin{euleroutput}
  Variable sa2 not found!
  Use global variables or parameters for string evaluation.
  Error in Evaluate, superfluous characters found.
  Try "trace errors" to inspect local variables after errors.
  mxmeval:
      return evaluate(mxm(s));
  Error in:
  sqrt(mxmeval("at(py,[a=3^2,b=4^2])")) ...
                                      ^
\end{euleroutput}
\eulersubheading{Sudut Akord}
\begin{eulercomment}
Perhatikan situasi berikut.
\end{eulercomment}
\begin{eulerprompt}
>setPlotRange(1.2); ...
>color(1); plotCircle(circleWithCenter([0,0],1)); ...
>A:=[cos(1),sin(1)]; B:=[cos(2),sin(2)]; C:=[cos(6),sin(6)]; ...
>plotPoint(A,"A"); plotPoint(B,"B"); plotPoint(C,"C"); ...
>color(3); plotSegment(A,B,"c"); plotSegment(A,C,"b"); plotSegment(C,B,"a"); ...
>color(1); O:=[0,0];  plotPoint(O,"0"); ...
>plotSegment(A,O); plotSegment(B,O); plotSegment(C,O,"r"); ...
>insimg;
\end{eulerprompt}
\begin{eulercomment}
Kita dapat menggunakan Maxima untuk menyelesaikan rumus penyebaran
rangkap tiga untuk sudut-sudut di pusat O untuk r. Jadi kita
mendapatkan rumus untuk jari-jari kuadrat dari pericircle dalam hal
kuadrat dari sisi.

Kali ini, Maxima menghasilkan beberapa nol kompleks, yang kita
abaikan.
\end{eulercomment}
\begin{eulerprompt}
>&remvalue(a,b,c,r); // hapus nilai-nilai sebelumnya untuk perhitungan baru
>rabc &= rhs(solve(triplespread(spread(b,r,r),spread(a,r,r),spread(c,r,r)),r)[4]); $rabc
\end{eulerprompt}
\begin{eulercomment}
Kita dapat menjadikannya sebagai fungsi Euler.
\end{eulercomment}
\begin{eulerprompt}
>function periradius(a,b,c) &= rabc;
\end{eulerprompt}
\begin{eulercomment}
Mari kita periksa hasilnya untuk poin A,B,C.
\end{eulercomment}
\begin{eulerprompt}
>a:=quadrance(B,C); b:=quadrance(A,C); c:=quadrance(A,B);
\end{eulerprompt}
\begin{eulercomment}
Jari-jarinya memang 1.
\end{eulercomment}
\begin{eulerprompt}
>periradius(a,b,c)
\end{eulerprompt}
\begin{euleroutput}
  1
\end{euleroutput}
\begin{eulercomment}
Faktanya, spread CBA hanya bergantung pada b dan c. Ini adalah teorema
sudut chord.
\end{eulercomment}
\begin{eulerprompt}
>$spread(b,a,c)*rabc | ratsimp
\end{eulerprompt}
\begin{eulercomment}
Sebenarnya spreadnya adalah b/(4r), dan kita melihat bahwa sudut chord
dari chord b adalah setengah dari sudut pusat.
\end{eulercomment}
\begin{eulerprompt}
>$doublespread(b/(4*r))-spread(b,r,r) | ratsimp
\end{eulerprompt}
\eulersubheading{Contoh 6: Jarak Minimal pada Bidang}
\begin{eulercomment}
\end{eulercomment}
\eulersubheading{Catatan awal}
\begin{eulercomment}
Fungsi yang, ke titik M di bidang, menetapkan jarak AM antara titik
tetap A dan M, memiliki garis level yang agak sederhana: lingkaran
berpusat di A.
\end{eulercomment}
\begin{eulerprompt}
>&remvalue();
>A=[-1,-1];
>function d1(x,y):=sqrt((x-A[1])^2+(y-A[2])^2)
>fcontour("d1",xmin=-2,xmax=0,ymin=-2,ymax=0,hue=1, ...
>title="If you see ellipses, please set your window square"):
\end{eulerprompt}
\begin{eulercomment}
dan grafiknya juga agak sederhana: bagian atas kerucut:
\end{eulercomment}
\begin{eulerprompt}
>plot3d("d1",xmin=-2,xmax=0,ymin=-2,ymax=0):
\end{eulerprompt}
\begin{eulercomment}
Tentu saja minimal 0 dicapai di A.

\end{eulercomment}
\eulersubheading{Dua poin}
\begin{eulercomment}
Sekarang kita lihat fungsi MA+MB dimana A dan B adalah dua titik
(tetap). Ini adalah "fakta yang diketahui" bahwa kurva level adalah
elips, titik fokusnya adalah A dan B; kecuali untuk AB minimum yang
konstan pada segmen [AB]:
\end{eulercomment}
\begin{eulerprompt}
>B=[1,-1];
>function d2(x,y):=d1(x,y)+sqrt((x-B[1])^2+(y-B[2])^2)
>fcontour("d2",xmin=-2,xmax=2,ymin=-3,ymax=1,hue=1):
\end{eulerprompt}
\begin{eulercomment}
Grafiknya lebih menarik:
\end{eulercomment}
\begin{eulerprompt}
>plot3d("d2",xmin=-2,xmax=2,ymin=-3,ymax=1):
\end{eulerprompt}
\begin{eulercomment}
Pembatasan garis (AB) lebih terkenal:
\end{eulercomment}
\begin{eulerprompt}
>plot2d("abs(x+1)+abs(x-1)",xmin=-3,xmax=3):
\end{eulerprompt}
\eulersubheading{Tiga poin}
\begin{eulercomment}
Sekarang hal-hal yang kurang sederhana: Ini sedikit kurang terkenal
bahwa MA+MB+MC mencapai minimum pada satu titik pesawat tetapi untuk
menentukan itu kurang sederhana:

1) Jika salah satu sudut segitiga ABC lebih dari 120° (katakanlah di
A), maka minimum dicapai pada titik ini (misalnya AB+AC).

Contoh:
\end{eulercomment}
\begin{eulerprompt}
>C=[-4,1];
>function d3(x,y):=d2(x,y)+sqrt((x-C[1])^2+(y-C[2])^2)
>plot3d("d3",xmin=-5,xmax=3,ymin=-4,ymax=4);
>insimg;
>fcontour("d3",xmin=-4,xmax=1,ymin=-2,ymax=2,hue=1,title="The minimum is on A");
>P=(A_B_C_A)'; plot2d(P[1],P[2],add=1,color=12);
>insimg;
\end{eulerprompt}
\begin{eulercomment}
2) Tetapi jika semua sudut segitiga ABC kurang dari 120 °, minimumnya
adalah pada titik F di bagian dalam segitiga, yang merupakan
satu-satunya titik yang melihat sisi-sisi ABC dengan sudut yang sama
(maka masing-masing 120 ° ):
\end{eulercomment}
\begin{eulerprompt}
>C=[-0.5,1];
>plot3d("d3",xmin=-2,xmax=2,ymin=-2,ymax=2):
>fcontour("d3",xmin=-2,xmax=2,ymin=-2,ymax=2,hue=1,title="The Fermat point");
>P=(A_B_C_A)'; plot2d(P[1],P[2],add=1,color=12);
>insimg;
\end{eulerprompt}
\begin{eulercomment}
Merupakan kegiatan yang menarik untuk mewujudkan gambar di atas dengan
perangkat lunak geometri; misalnya, saya tahu soft yang ditulis di
Jawa yang memiliki instruksi "garis kontur" ...

Semua ini di atas telah ditemukan oleh seorang hakim Perancis bernama
Pierre de Fermat; dia menulis surat kepada dilettants lain seperti
pendeta Marin Mersenne dan Blaise Pascal yang bekerja di pajak
penghasilan. Jadi titik unik F sedemikian rupa sehingga FA+FB+FC
minimal, disebut titik Fermat segitiga. Tetapi tampaknya beberapa
tahun sebelumnya, Torriccelli Italia telah menemukan titik ini sebelum
Fermat melakukannya! Bagaimanapun tradisinya adalah mencatat poin ini
F...

\end{eulercomment}
\eulersubheading{Empat poin}
\begin{eulercomment}
Langkah selanjutnya adalah menambahkan 4 titik D dan mencoba
meminimalkan MA+MB+MC+MD; katakan bahwa Anda adalah operator TV kabel
dan ingin mencari di bidang mana Anda harus meletakkan antena sehingga
Anda dapat memberi makan empat desa dan menggunakan panjang kabel
sesedikit mungkin!
\end{eulercomment}
\begin{eulerprompt}
>D=[1,1];
>function d4(x,y):=d3(x,y)+sqrt((x-D[1])^2+(y-D[2])^2)
>plot3d("d4",xmin=-1.5,xmax=1.5,ymin=-1.5,ymax=1.5):
>fcontour("d4",xmin=-1.5,xmax=1.5,ymin=-1.5,ymax=1.5,hue=1);
>P=(A_B_C_D)'; plot2d(P[1],P[2],points=1,add=1,color=12);
>insimg;
\end{eulerprompt}
\begin{eulercomment}
Masih ada minimum dan tidak tercapai di salah satu simpul A, B, C atau
D:
\end{eulercomment}
\begin{eulerprompt}
>function f(x):=d4(x[1],x[2])
>neldermin("f",[0.2,0.2])
\end{eulerprompt}
\begin{euleroutput}
  [0.142858,  0.142857]
\end{euleroutput}
\begin{eulercomment}
Tampaknya dalam kasus ini, koordinat titik optimal adalah rasional
atau mendekati rasional...

Sekarang ABCD adalah persegi, kami berharap bahwa titik optimal akan
menjadi pusat ABCD:
\end{eulercomment}
\begin{eulerprompt}
>C=[-1,1];
>plot3d("d4",xmin=-1,xmax=1,ymin=-1,ymax=1):
>fcontour("d4",xmin=-1.5,xmax=1.5,ymin=-1.5,ymax=1.5,hue=1);
>P=(A_B_C_D)'; plot2d(P[1],P[2],add=1,color=12,points=1);
>insimg;
\end{eulerprompt}
\eulersubheading{Contoh 7: Bola Dandelin dengan Povray}
\begin{eulercomment}
Anda dapat menjalankan demonstrasi ini, jika Anda telah menginstal
Povray, dan pvengine.exe di jalur program.

Pertama kita hitung jari-jari bola.

Jika Anda melihat gambar di bawah, Anda melihat bahwa kita membutuhkan
dua lingkaran yang menyentuh dua garis yang membentuk kerucut, dan
satu garis yang membentuk bidang yang memotong kerucut.

Kami menggunakan file geometri.e dari Euler untuk ini.
\end{eulercomment}
\begin{eulerprompt}
>load geometry;
\end{eulerprompt}
\begin{eulercomment}
Pertama dua garis yang membentuk kerucut.
\end{eulercomment}
\begin{eulerprompt}
>g1 &= lineThrough([0,0],[1,a])
\end{eulerprompt}
\begin{euleroutput}
  
                               [- a, 1, 0]
  
\end{euleroutput}
\begin{eulerprompt}
>g2 &= lineThrough([0,0],[-1,a])
\end{eulerprompt}
\begin{euleroutput}
  
                              [- a, - 1, 0]
  
\end{euleroutput}
\begin{eulercomment}
Kemudian saya baris ketiga.
\end{eulercomment}
\begin{eulerprompt}
>g &= lineThrough([-1,0],[1,1])
\end{eulerprompt}
\begin{euleroutput}
  
                               [- 1, 2, 1]
  
\end{euleroutput}
\begin{eulercomment}
Kami merencanakan semuanya sejauh ini.
\end{eulercomment}
\begin{eulerprompt}
>setPlotRange(-1,1,0,2);
>color(black); plotLine(g(),"")
>a:=2; color(blue); plotLine(g1(),""), plotLine(g2(),""):
\end{eulerprompt}
\begin{eulercomment}
Sekarang kita ambil titik umum pada sumbu y.
\end{eulercomment}
\begin{eulerprompt}
>P &= [0,u]
\end{eulerprompt}
\begin{euleroutput}
  
                                  [0, u]
  
\end{euleroutput}
\begin{eulercomment}
Hitung jarak ke g1.
\end{eulercomment}
\begin{eulerprompt}
>d1 &= distance(P,projectToLine(P,g1)); $d1
\end{eulerprompt}
\begin{euleroutput}
  Maxima said:
  rat: replaced 1.66665833335744e-7 by 15819/94914474571 = 1.66665833335744e-7
  
  rat: replaced 4.999958333473664e-5 by 201389/4027813565 = 4.99995833347366e-5
  
  rat: replaced 1.33330666692022e-6 by 31771/23828726570 = 1.333306666920221e-6
  
  rat: replaced 1.999933334222437e-4 by 200030/1000183339 = 1.999933334222437e-4
  
  rat: replaced 4.499797504338432e-6 by 24036/5341573699 = 4.499797504338431e-6
  
  rat: replaced 4.499662510124569e-4 by 1162901/2584418270 = 4.499662510124571e-4
  
  rat: replaced 1.066581336583994e-5 by 58861/5518660226 = 1.066581336583993e-5
  
  rat: replaced 7.998933390220841e-4 by 1137431/1421978337 = 7.998933390220838e-4
  
  rat: replaced 2.083072932167196e-5 by 35635/1710693824 = 2.0830729321672e-5
  
  rat: replaced 0.001249739605033717 by 567943/454449069 = 0.001249739605033716
  
  rat: replaced 3.599352055540239e-5 by 98277/2730408098 = 3.599352055540234e-5
  
  rat: replaced 0.00179946006479581 by 479561/266502719 = 0.001799460064795812
  
  rat: replaced 5.71526624672386e-5 by 51154/895041417 = 5.715266246723866e-5
  
  rat: replaced 0.002448999746720415 by 1946227/794702818 = 0.002448999746720415
  
  rat: replaced 8.530603082730626e-5 by 121691/1426522824 = 8.530603082730627e-5
  
  rat: replaced 0.003198293697380561 by 2986741/933854512 = 0.003198293697380562
  
  rat: replaced 1.214508019889565e-4 by 158455/1304684674 = 1.214508019889563e-4
  
  rat: replaced 0.004047266988005727 by 2125334/525128193 = 0.004047266988005727
  
  rat: replaced 1.665833531718508e-4 by 142521/855553675 = 1.66583353171851e-4
  
  rat: replaced 0.004995834721974179 by 1957223/391770967 = 0.004995834721974179
  
  rat: replaced 2.216991628251896e-4 by 179571/809975995 = 2.216991628251896e-4
  
  rat: replaced 0.006043902043303184 by 1800665/297930871 = 0.006043902043303193
  
  rat: replaced 2.877927110806339e-4 by 1167733/4057548906 = 2.877927110806339e-4
  
  rat: replaced 0.00719136414613375 by 2476362/344352191 = 0.007191364146133747
  
  rat: replaced 3.658573803051457e-4 by 386279/1055818526 = 3.658573803051454e-4
  
  rat: replaced 0.00843810628521191 by 2079855/246483622 = 0.008438106285211924
  
  rat: replaced 4.5688535576352e-4 by 262978/575588595 = 4.568853557635206e-4
  
  rat: replaced 0.009784003787362772 by 1752551/179124113 = 0.009784003787362787
  
  rat: replaced 5.618675264007778e-4 by 150595/268025812 = 5.618675264007782e-4
  
  rat: replaced 0.01122892206395776 by 5450241/485375263 = 0.01122892206395776
  
  rat: replaced 6.817933857540259e-4 by 192316/282073725 = 6.817933857540258e-4
  
  rat: replaced 0.01277271662437307 by 3258991/255152533 = 0.01277271662437308
  
  rat: replaced 8.176509330039827e-4 by 105841/129445214 = 8.176509330039812e-4
  
  rat: replaced 0.01441523309043924 by 2330472/161667313 = 0.01441523309043925
  
  rat: replaced 9.704265741758145e-4 by 651321/671169790 = 9.704265741758132e-4
  
  rat: replaced 0.01615630721187855 by 19391318/1200232067 = 0.01615630721187855
  
  rat: replaced 0.001141105023499428 by 1259907/1104111343 = 0.001141105023499428
  
  rat: replaced 0.01799576488272969 by 4765614/264818641 = 0.01799576488272969
  
  rat: replaced 0.001330669204938795 by 1231154/925214167 = 0.001330669204938796
  
  rat: replaced 0.01993342215875837 by 2504519/125644206 = 0.01993342215875836
  
  rat: replaced 0.001540100153900437 by 276884/179783113 = 0.001540100153900439
  
  rat: replaced 0.02196908527585173 by 1298306/59096953 = 0.0219690852758517
  
  rat: replaced 0.001770376919130678 by 644389/363984072 = 0.001770376919130681
  
  rat: replaced 0.02410255066939448 by 2001286/83032125 = 0.02410255066939453
  
  rat: replaced 0.002022476464811601 by 1271955/628909667 = 0.002022476464811599
  
  rat: replaced 0.02633360499462523 by 2978115/113091808 = 0.02633360499462525
  
  rat: replaced 0.002297373572865413 by 1020913/444382669 = 0.002297373572865417
  
  rat: replaced 0.02866202514797045 by 1770713/61779061 = 0.02866202514797044
  
  rat: replaced 0.002596040745477063 by 1097643/422814242 = 0.002596040745477065
  
  rat: replaced 0.03108757828935527 by 5034207/161936287 = 0.03108757828935525
  
  rat: replaced 0.002919448107844891 by 906221/310408326 = 0.002919448107844891
  
  rat: replaced 0.03361002186548678 by 4553215/135471944 = 0.03361002186548678
  
  rat: replaced 0.003268563311168871 by 1379071/421919623 = 0.003268563311168867
  
  rat: replaced 0.03622910363410947 by 3082649/85087642 = 0.0362291036341094
  
  rat: replaced 0.003644351435886262 by 5966577/1637212301 = 0.003644351435886261
  
  rat: replaced 0.03894456168922911 by 4913415/126164342 = 0.03894456168922911
  
  rat: replaced 0.004047774895164447 by 572425/141417202 = 0.004047774895164451
  
  rat: replaced 0.04175612448730281 by 1734727/41544253 = 0.04175612448730273
  
  rat: replaced 0.004479793338660443 by 2952779/659132861 = 0.004479793338660444
  
  rat: replaced 0.04466351087439402 by 4691119/105032473 = 0.04466351087439405
  
  rat: replaced 0.0049413635565565 by 2524919/510976165 = 0.004941363556556498
  
  rat: replaced 0.04766643011428662 by 3536207/74186529 = 0.04766643011428665
  
  rat: replaced 0.005433439383882244 by 1361584/250593391 = 0.005433439383882235
  
  rat: replaced 0.05076458191755917 by 7710025/151878036 = 0.05076458191755916
  
  rat: replaced 0.005956971605131645 by 1447422/242979503 = 0.005956971605131648
  
  rat: replaced 0.0539576564716131 by 3377975/62604183 = 0.05395765647161309
  
  rat: replaced 0.006512907859185624 by 3695063/567344584 = 0.006512907859185626
  
  rat: replaced 0.05724533447165381 by 2560865/44734912 = 0.05724533447165382
  
  rat: replaced 0.007102192544548636 by 1363981/192050693 = 0.007102192544548642
  
  rat: replaced 0.06062728715262111 by 8274761/136485754 = 0.06062728715262107
  
  rat: replaced 0.007725766724910044 by 1464384/189545459 = 0.007725766724910038
  
  rat: replaced 0.06410317632206519 by 5287663/82486755 = 0.06410317632206528
  
  rat: replaced 0.00838456803503801 by 1113589/132814117 = 0.008384568035038023
  
  rat: replaced 0.06767265439396564 by 2921400/43169579 = 0.06767265439396572
  
  rat: replaced 0.009079530587017326 by 433906/47789475 = 0.00907953058701733
  
  rat: replaced 0.07133536442348987 by 7236103/101437808 = 0.07133536442348991
  
  rat: replaced 0.009811584876838586 by 1363090/138926587 = 0.009811584876838586
  
  rat: replaced 0.07509094014268702 by 9209133/122639735 = 0.07509094014268704
  
  rat: replaced 0.0105816576913495 by 1163729/109976058 = 0.01058165769134951
  
  rat: replaced 0.07893900599711501 by 5197067/65836489 = 0.07893900599711506
  
  rat: replaced 0.01139067201557714 by 13426050/1178688139 = 0.01139067201557714
  
  rat: replaced 0.08287917718339499 by 11217158/135343501 = 0.082879177183395
  
  rat: replaced 0.01223954694042984 by 2283101/186534764 = 0.01223954694042983
  
  rat: replaced 0.08691105968769186 by 5213115/59982182 = 0.08691105968769192
  
  rat: replaced 0.01312919757078923 by 3499615/266552086 = 0.01312919757078922
  
  rat: replaced 0.09103425032511492 by 5893225/64736349 = 0.09103425032511488
  
  rat: replaced 0.01406053493400045 by 2280713/162206702 = 0.01406053493400045
  
  rat: replaced 0.09524833678003664 by 9601787/100807923 = 0.09524833678003662
  
  rat: replaced 0.01503446588876983 by 200490/13335359 = 0.01503446588876985
  
  rat: replaced 0.09955289764732322 by 5687088/57126293 = 0.09955289764732328
  
  rat: replaced 0.01605189303448024 by 951971/59305840 = 0.01605189303448025
  
  rat: replaced 0.1039475024744748 by 10260011/98703776 = 0.1039475024744747
  
  rat: replaced 0.01711371462093175 by 9432386/551159477 = 0.01711371462093176
  
  rat: replaced 0.1084317118046711 by 14939691/137779721 = 0.1084317118046712
  
  rat: replaced 0.01822082445851714 by 2559788/140486947 = 0.01822082445851713
  
  rat: replaced 0.113005077220716 by 8478529/75027859 = 0.1130050772207161
  
  rat: replaced 0.01937411182884202 by 2983799/154009589 = 0.01937411182884203
  
  rat: replaced 0.1176671413898787 by 7123715/60541243 = 0.1176671413898786
  
  rat: replaced 0.02057446139579705 by 7167743/348380590 = 0.02057446139579705
  
  rat: replaced 0.1224174381096274 by 12172179/99431741 = 0.1224174381096274
  
  rat: replaced 0.02182275311709253 by 7415562/339808729 = 0.02182275311709253
  
  rat: replaced 0.1272554923542488 by 7277933/57191504 = 0.127255492354249
  
  rat: replaced 0.02311986215626333 by 2988661/129268115 = 0.02311986215626336
  
  rat: replaced 0.1321808203223502 by 3633064/27485561 = 0.1321808203223503
  
  rat: replaced 0.02446665879515308 by 1991976/81415939 = 0.02446665879515312
  
  rat: replaced 0.1371929294852391 by 56235017/409897341 = 0.1371929294852391
  
  rat: replaced 0.02586400834688696 by 5000736/193347293 = 0.02586400834688697
  
  rat: replaced 0.1422913186361759 by 9349741/65708443 = 0.1422913186361759
  
  rat: replaced 0.02731277106934082 by 858413/31428997 = 0.02731277106934084
  
  rat: replaced 0.1474754779404944 by 1549881/10509415 = 0.1474754779404943
  
  rat: replaced 0.02881380207911666 by 3754753/130310918 = 0.02881380207911666
  
  rat: replaced 0.152744888986584 by 5264425/34465474 = 0.1527448889865841
  
  rat: replaced 0.03036795126603076 by 4118329/135614318 = 0.03036795126603077
  
  rat: replaced 0.1580990248377314 by 5442776/34426373 = 0.1580990248377312
  
  rat: replaced 0.03197606320812652 by 3497683/109384416 = 0.03197606320812647
  
  rat: replaced 0.1635373500848132 by 12328488/75386375 = 0.1635373500848131
  
  rat: replaced 0.0336389770872163 by 3971799/118071337 = 0.03363897708721635
  
  rat: replaced 0.1690593208998367 by 20896917/123607009 = 0.1690593208998367
  
  rat: replaced 0.03535752660496472 by 1815732/51353479 = 0.03535752660496478
  
  rat: replaced 0.1746643850903219 by 2841592/16268869 = 0.1746643850903219
  
  rat: replaced 0.03713253989951881 by 3333721/89778965 = 0.03713253989951878
  
  rat: replaced 0.1803519821545206 by 4461007/24735004 = 0.1803519821545208
  
  rat: replaced 0.03896483946269502 by 8785771/225479461 = 0.03896483946269501
  
  rat: replaced 0.1861215433374662 by 4381209/23539505 = 0.1861215433374661
  
  rat: replaced 0.0408552420577305 by 3189084/78058135 = 0.04085524205773043
  
  rat: replaced 0.1919724916878484 by 72809759/379271834 = 0.1919724916878484
  
  rat: replaced 0.04280455863760801 by 7646593/178639688 = 0.04280455863760801
  
  rat: replaced 0.1979042421157076 by 26318167/132984350 = 0.1979042421157076
  
  rat: replaced 0.04481359426396048 by 20610430/459914683 = 0.04481359426396048
  
  rat: replaced 0.2039162014509444 by 8519416/41779005 = 0.2039162014509441
  
  rat: replaced 0.04688314802656623 by 3439140/73355569 = 0.04688314802656633
  
  rat: replaced 0.2100077685026351 by 50962787/242670961 = 0.2100077685026351
  
  rat: replaced 0.04901401296344043 by 4006732/81746663 = 0.04901401296344048
  
  rat: replaced 0.216178334119151 by 1347531/6233423 = 0.2161783341191509
  
  rat: replaced 0.05120697598153157 by 4148974/81023609 = 0.0512069759815315
  
  rat: replaced 0.2224272812490723 by 23234851/104460437 = 0.2224272812490723
  
  rat: replaced 0.05346281777803219 by 11998448/224426031 = 0.05346281777803218
  
  rat: replaced 0.2287539850028937 by 8185268/35781969 = 0.2287539850028935
  
  rat: replaced 0.05578231276230905 by 1398019/25062048 = 0.05578231276230897
  
  rat: replaced 0.2351578127155118 by 12642104/53760085 = 0.2351578127155119
  
  rat: replaced 0.05816622897846346 by 4451048/76522891 = 0.05816622897846345
  
  rat: replaced 0.2416381240094921 by 8002142/33116223 = 0.2416381240094923
  
  rat: replaced 0.06061532802852698 by 2146337/35409146 = 0.06061532802852686
  
  rat: replaced 0.2481942708591053 by 8882901/35790113 = 0.2481942708591057
  
  rat: replaced 0.0631303649963022 by 14651447/232082406 = 0.06313036499630222
  
  rat: replaced 0.2548255976551299 by 868346/3407609 = 0.25482559765513
  
  rat: replaced 0.06571208837185505 by 4240309/64528599 = 0.06571208837185509
  
  rat: replaced 0.2615314412704124 by 8212450/31401387 = 0.2615314412704127
  
  rat: replaced 0.06836123997666599 by 2716643/39739522 = 0.06836123997666604
  
  rat: replaced 0.2683111311261794 by 34459769/128432126 = 0.2683111311261794
  
  rat: replaced 0.07107855488944881 by 3146673/44270357 = 0.07107855488944893
  
  rat: replaced 0.2751639892590951 by 12552159/45617012 = 0.2751639892590949
  
  rat: replaced 0.07386476137264342 by 12898997/174629915 = 0.0738647613726434
  
  rat: replaced 0.2820893303890569 by 11134456/39471383 = 0.2820893303890568
  
  rat: replaced 0.07672058079958999 by 5073506/66129661 = 0.07672058079959007
  
  rat: replaced 0.2890864619877229 by 9583357/33150487 = 0.2890864619877228
  
  rat: replaced 0.07964672758239233 by 5672399/71219486 = 0.07964672758239227
  
  rat: replaced 0.2961546843477643 by 11052271/37319251 = 0.2961546843477647
  
  rat: replaced 0.08264390910047736 by 4686067/56701904 = 0.08264390910047748
  
  rat: replaced 0.3032932906528349 by 9918077/32701274 = 0.3032932906528351
  
  rat: replaced 0.0857128256298576 by 3585977/41837111 = 0.08571282562985766
  
  rat: replaced 0.3105015670482534 by 9320011/30015987 = 0.3105015670482533
  
  rat: replaced 0.08885417027310427 by 5751353/64728003 = 0.0888541702731042
  
  rat: replaced 0.3177787927123868 by 248395525/781661743 = 0.3177787927123868
  
  rat: replaced 0.09206862889003742 by 7305460/79347983 = 0.09206862889003745
  
  rat: replaced 0.3251242399287333 by 13842845/42577093 = 0.3251242399287335
  
  rat: replaced 0.09535688002914089 by 5971998/62627867 = 0.09535688002914103
  
  rat: replaced 0.3325371741586922 by 9318229/28021616 = 0.3325371741586923
  
  rat: replaced 0.0987195948597075 by 9821211/99485933 = 0.09871959485970745
  
  rat: replaced 0.3400168541150183 by 13391981/39386227 = 0.3400168541150184
  
  rat: replaced 0.1021574371047232 by 8336413/81603584 = 0.1021574371047232
  
  rat: replaced 0.3475625318359485 by 10097818/29053241 = 0.347562531835949
  
  rat: replaced 0.1056710629744951 by 5741011/54329074 = 0.105671062974495
  
  rat: replaced 0.3551734527599992 by 15867851/44676343 = 0.3551734527599987
  
  rat: replaced 0.1092611211010309 by 5551873/50812887 = 0.1092611211010309
  
  rat: replaced 0.3628488558014202 by 6897641/19009681 = 0.3628488558014203
  
  rat: replaced 0.1129282524731764 by 11548693/102265755 = 0.1129282524731764
  
  rat: replaced 0.3705879734263036 by 23358661/63031352 = 0.3705879734263038
  
  rat: replaced 0.1166730903725168 by 5656228/48479285 = 0.1166730903725168
  
  rat: replaced 0.3783900317293359 by 14241382/37636779 = 0.3783900317293358
  
  rat: replaced 0.1204962603100498 by 4057613/33674182 = 0.12049626031005
  
  rat: replaced 0.3862542505111889 by 3461217/8960981 = 0.3862542505111884
  
  rat: replaced 0.1243983799636342 by 7966447/64039797 = 0.1243983799636342
  
  rat: replaced 0.3941798433565377 by 5314214/13481699 = 0.3941798433565384
  
  rat: replaced 0.1283800591162231 by 796346/6203035 = 0.1283800591162229
  
  rat: replaced 0.4021660177127022 by 11567173/28762184 = 0.4021660177127022
  
  rat: replaced 0.1324418995948859 by 4716124/35609003 = 0.1324418995948862
  
  rat: replaced 0.4102119749689023 by 11320633/27597032 = 0.4102119749689024
  
  rat: replaced 0.1365844952106265 by 612971/4487852 = 0.1365844952106264
  
  rat: replaced 0.418316910536117 by 12225195/29224721 = 0.4183169105361177
  
  rat: replaced 0.140808431699002 by 10431632/74083859 = 0.1408084316990021
  
  rat: replaced 0.4264800139275439 by 7978696/18708253 = 0.4264800139275431
  
  rat: replaced 0.1451142866615502 by 3554077/24491572 = 0.1451142866615504
  
  rat: replaced 0.4347004688396462 by 20489554/47134879 = 0.4347004688396463
  
  rat: replaced 0.1495026295080298 by 26759297/178988805 = 0.1495026295080298
  
  rat: replaced 0.4429774532337832 by 23449796/52936771 = 0.4429774532337834
  
  rat: replaced 0.1539740213994798 by 16145763/104860306 = 0.1539740213994798
  
  rat: replaced 0.451310139418413 by 8841241/19590167 = 0.4513101394184133
  
  rat: replaced 4.999958333473664e-5 by 201389/4027813565 = 4.99995833347366e-5
  
  rat: replaced -1.66665833335744e-7 by -15819/94914474571 = -1.66665833335744e-7
  
  rat: replaced 1.999933334222437e-4 by 200030/1000183339 = 1.999933334222437e-4
  
  rat: replaced -1.33330666692022e-6 by -31771/23828726570 = -1.333306666920221e-6
  
  rat: replaced 4.499662510124569e-4 by 1162901/2584418270 = 4.499662510124571e-4
  
  rat: replaced -4.499797504338432e-6 by -24036/5341573699 = -4.499797504338431e-6
  
  rat: replaced 7.998933390220841e-4 by 1137431/1421978337 = 7.998933390220838e-4
  
  rat: replaced -1.066581336583994e-5 by -58861/5518660226 = -1.066581336583993e-5
  
  rat: replaced 0.001249739605033717 by 567943/454449069 = 0.001249739605033716
  
  rat: replaced -2.083072932167196e-5 by -35635/1710693824 = -2.0830729321672e-5
  
  rat: replaced 0.00179946006479581 by 479561/266502719 = 0.001799460064795812
  
  rat: replaced -3.599352055540239e-5 by -98277/2730408098 = -3.599352055540234e-5
  
  rat: replaced 0.002448999746720415 by 1946227/794702818 = 0.002448999746720415
  
  rat: replaced -5.71526624672386e-5 by -51154/895041417 = -5.715266246723866e-5
  
  rat: replaced 0.003198293697380561 by 2986741/933854512 = 0.003198293697380562
  
  rat: replaced -8.530603082730626e-5 by -121691/1426522824 = -8.530603082730627e-5
  
  rat: replaced 0.004047266988005727 by 2125334/525128193 = 0.004047266988005727
  
  rat: replaced -1.214508019889565e-4 by -158455/1304684674 = -1.214508019889563e-4
  
  rat: replaced 0.004995834721974179 by 1957223/391770967 = 0.004995834721974179
  
  rat: replaced -1.665833531718508e-4 by -142521/855553675 = -1.66583353171851e-4
  
  rat: replaced 0.006043902043303184 by 1800665/297930871 = 0.006043902043303193
  
  rat: replaced -2.216991628251896e-4 by -179571/809975995 = -2.216991628251896e-4
  
  rat: replaced 0.00719136414613375 by 2476362/344352191 = 0.007191364146133747
  
  rat: replaced -2.877927110806339e-4 by -1167733/4057548906 = -2.877927110806339e-4
  
  rat: replaced 0.00843810628521191 by 2079855/246483622 = 0.008438106285211924
  
  rat: replaced -3.658573803051457e-4 by -386279/1055818526 = -3.658573803051454e-4
  
  rat: replaced 0.009784003787362772 by 1752551/179124113 = 0.009784003787362787
  
  rat: replaced -4.5688535576352e-4 by -262978/575588595 = -4.568853557635206e-4
  
  rat: replaced 0.01122892206395776 by 5450241/485375263 = 0.01122892206395776
  
  rat: replaced -5.618675264007778e-4 by -150595/268025812 = -5.618675264007782e-4
  
  rat: replaced 0.01277271662437307 by 3258991/255152533 = 0.01277271662437308
  
  rat: replaced -6.817933857540259e-4 by -192316/282073725 = -6.817933857540258e-4
  
  rat: replaced 0.01441523309043924 by 2330472/161667313 = 0.01441523309043925
  
  rat: replaced -8.176509330039827e-4 by -105841/129445214 = -8.176509330039812e-4
  
  rat: replaced 0.01615630721187855 by 19391318/1200232067 = 0.01615630721187855
  
  rat: replaced -9.704265741758145e-4 by -651321/671169790 = -9.704265741758132e-4
  
  rat: replaced 0.01799576488272969 by 4765614/264818641 = 0.01799576488272969
  
  rat: replaced -0.001141105023499428 by -1259907/1104111343 = -0.001141105023499428
  
  rat: replaced 0.01993342215875837 by 2504519/125644206 = 0.01993342215875836
  
  rat: replaced -0.001330669204938795 by -1231154/925214167 = -0.001330669204938796
  
  rat: replaced 0.02196908527585173 by 1298306/59096953 = 0.0219690852758517
  
  rat: replaced -0.001540100153900437 by -276884/179783113 = -0.001540100153900439
  
  rat: replaced 0.02410255066939448 by 2001286/83032125 = 0.02410255066939453
  
  rat: replaced -0.001770376919130678 by -644389/363984072 = -0.001770376919130681
  
  rat: replaced 0.02633360499462523 by 2978115/113091808 = 0.02633360499462525
  
  rat: replaced -0.002022476464811601 by -1271955/628909667 = -0.002022476464811599
  
  rat: replaced 0.02866202514797045 by 1770713/61779061 = 0.02866202514797044
  
  rat: replaced -0.002297373572865413 by -1020913/444382669 = -0.002297373572865417
  
  rat: replaced 0.03108757828935527 by 5034207/161936287 = 0.03108757828935525
  
  rat: replaced -0.002596040745477063 by -1097643/422814242 = -0.002596040745477065
  
  rat: replaced 0.03361002186548678 by 4553215/135471944 = 0.03361002186548678
  
  rat: replaced -0.002919448107844891 by -906221/310408326 = -0.002919448107844891
  
  rat: replaced 0.03622910363410947 by 3082649/85087642 = 0.0362291036341094
  
  rat: replaced -0.003268563311168871 by -1379071/421919623 = -0.003268563311168867
  
  rat: replaced 0.03894456168922911 by 4913415/126164342 = 0.03894456168922911
  
  rat: replaced -0.003644351435886262 by -5966577/1637212301 = -0.003644351435886261
  
  rat: replaced 0.04175612448730281 by 1734727/41544253 = 0.04175612448730273
  
  rat: replaced -0.004047774895164447 by -572425/141417202 = -0.004047774895164451
  
  rat: replaced 0.04466351087439402 by 4691119/105032473 = 0.04466351087439405
  
  rat: replaced -0.004479793338660443 by -2952779/659132861 = -0.004479793338660444
  
  rat: replaced 0.04766643011428662 by 3536207/74186529 = 0.04766643011428665
  
  rat: replaced -0.0049413635565565 by -2524919/510976165 = -0.004941363556556498
  
  rat: replaced 0.05076458191755917 by 7710025/151878036 = 0.05076458191755916
  
  rat: replaced -0.005433439383882244 by -1361584/250593391 = -0.005433439383882235
  
  rat: replaced 0.0539576564716131 by 3377975/62604183 = 0.05395765647161309
  
  rat: replaced -0.005956971605131645 by -1447422/242979503 = -0.005956971605131648
  
  rat: replaced 0.05724533447165381 by 2560865/44734912 = 0.05724533447165382
  
  rat: replaced -0.006512907859185624 by -3695063/567344584 = -0.006512907859185626
  
  rat: replaced 0.06062728715262111 by 8274761/136485754 = 0.06062728715262107
  
  rat: replaced -0.007102192544548636 by -1363981/192050693 = -0.007102192544548642
  
  rat: replaced 0.06410317632206519 by 5287663/82486755 = 0.06410317632206528
  
  rat: replaced -0.007725766724910044 by -1464384/189545459 = -0.007725766724910038
  
  rat: replaced 0.06767265439396564 by 2921400/43169579 = 0.06767265439396572
  
  rat: replaced -0.00838456803503801 by -1113589/132814117 = -0.008384568035038023
  
  rat: replaced 0.07133536442348987 by 7236103/101437808 = 0.07133536442348991
  
  rat: replaced -0.009079530587017326 by -433906/47789475 = -0.00907953058701733
  
  rat: replaced 0.07509094014268702 by 9209133/122639735 = 0.07509094014268704
  
  rat: replaced -0.009811584876838586 by -1363090/138926587 = -0.009811584876838586
  
  rat: replaced 0.07893900599711501 by 5197067/65836489 = 0.07893900599711506
  
  rat: replaced -0.0105816576913495 by -1163729/109976058 = -0.01058165769134951
  
  rat: replaced 0.08287917718339499 by 11217158/135343501 = 0.082879177183395
  
  rat: replaced -0.01139067201557714 by -13426050/1178688139 = -0.01139067201557714
  
  rat: replaced 0.08691105968769186 by 5213115/59982182 = 0.08691105968769192
  
  rat: replaced -0.01223954694042984 by -2283101/186534764 = -0.01223954694042983
  
  rat: replaced 0.09103425032511492 by 5893225/64736349 = 0.09103425032511488
  
  rat: replaced -0.01312919757078923 by -3499615/266552086 = -0.01312919757078922
  
  rat: replaced 0.09524833678003664 by 9601787/100807923 = 0.09524833678003662
  
  rat: replaced -0.01406053493400045 by -2280713/162206702 = -0.01406053493400045
  
  rat: replaced 0.09955289764732322 by 5687088/57126293 = 0.09955289764732328
  
  rat: replaced -0.01503446588876983 by -200490/13335359 = -0.01503446588876985
  
  rat: replaced 0.1039475024744748 by 10260011/98703776 = 0.1039475024744747
  
  rat: replaced -0.01605189303448024 by -951971/59305840 = -0.01605189303448025
  
  rat: replaced 0.1084317118046711 by 14939691/137779721 = 0.1084317118046712
  
  rat: replaced -0.01711371462093175 by -9432386/551159477 = -0.01711371462093176
  
  rat: replaced 0.113005077220716 by 8478529/75027859 = 0.1130050772207161
  
  rat: replaced -0.01822082445851714 by -2559788/140486947 = -0.01822082445851713
  
  rat: replaced 0.1176671413898787 by 7123715/60541243 = 0.1176671413898786
  
  rat: replaced -0.01937411182884202 by -2983799/154009589 = -0.01937411182884203
  
  rat: replaced 0.1224174381096274 by 12172179/99431741 = 0.1224174381096274
  
  rat: replaced -0.02057446139579705 by -7167743/348380590 = -0.02057446139579705
  
  rat: replaced 0.1272554923542488 by 7277933/57191504 = 0.127255492354249
  
  rat: replaced -0.02182275311709253 by -7415562/339808729 = -0.02182275311709253
  
  rat: replaced 0.1321808203223502 by 3633064/27485561 = 0.1321808203223503
  
  rat: replaced -0.02311986215626333 by -2988661/129268115 = -0.02311986215626336
  
  rat: replaced 0.1371929294852391 by 56235017/409897341 = 0.1371929294852391
  
  rat: replaced -0.02446665879515308 by -1991976/81415939 = -0.02446665879515312
  
  rat: replaced 0.1422913186361759 by 9349741/65708443 = 0.1422913186361759
  
  rat: replaced -0.02586400834688696 by -5000736/193347293 = -0.02586400834688697
  
  rat: replaced 0.1474754779404944 by 1549881/10509415 = 0.1474754779404943
  
  rat: replaced -0.02731277106934082 by -858413/31428997 = -0.02731277106934084
  
  rat: replaced 0.152744888986584 by 5264425/34465474 = 0.1527448889865841
  
  rat: replaced -0.02881380207911666 by -3754753/130310918 = -0.02881380207911666
  
  rat: replaced 0.1580990248377314 by 5442776/34426373 = 0.1580990248377312
  
  rat: replaced -0.03036795126603076 by -4118329/135614318 = -0.03036795126603077
  
  rat: replaced 0.1635373500848132 by 12328488/75386375 = 0.1635373500848131
  
  rat: replaced -0.03197606320812652 by -3497683/109384416 = -0.03197606320812647
  
  rat: replaced 0.1690593208998367 by 20896917/123607009 = 0.1690593208998367
  
  rat: replaced -0.0336389770872163 by -3971799/118071337 = -0.03363897708721635
  
  rat: replaced 0.1746643850903219 by 2841592/16268869 = 0.1746643850903219
  
  rat: replaced -0.03535752660496472 by -1815732/51353479 = -0.03535752660496478
  
  rat: replaced 0.1803519821545206 by 4461007/24735004 = 0.1803519821545208
  
  rat: replaced -0.03713253989951881 by -3333721/89778965 = -0.03713253989951878
  
  rat: replaced 0.1861215433374662 by 4381209/23539505 = 0.1861215433374661
  
  rat: replaced -0.03896483946269502 by -8785771/225479461 = -0.03896483946269501
  
  rat: replaced 0.1919724916878484 by 72809759/379271834 = 0.1919724916878484
  
  rat: replaced -0.0408552420577305 by -3189084/78058135 = -0.04085524205773043
  
  rat: replaced 0.1979042421157076 by 26318167/132984350 = 0.1979042421157076
  
  rat: replaced -0.04280455863760801 by -7646593/178639688 = -0.04280455863760801
  
  rat: replaced 0.2039162014509444 by 8519416/41779005 = 0.2039162014509441
  
  rat: replaced -0.04481359426396048 by -20610430/459914683 = -0.04481359426396048
  
  rat: replaced 0.2100077685026351 by 50962787/242670961 = 0.2100077685026351
  
  rat: replaced -0.04688314802656623 by -3439140/73355569 = -0.04688314802656633
  
  rat: replaced 0.216178334119151 by 1347531/6233423 = 0.2161783341191509
  
  rat: replaced -0.04901401296344043 by -4006732/81746663 = -0.04901401296344048
  
  rat: replaced 0.2224272812490723 by 23234851/104460437 = 0.2224272812490723
  
  rat: replaced -0.05120697598153157 by -4148974/81023609 = -0.0512069759815315
  
  rat: replaced 0.2287539850028937 by 8185268/35781969 = 0.2287539850028935
  
  rat: replaced -0.05346281777803219 by -11998448/224426031 = -0.05346281777803218
  
  rat: replaced 0.2351578127155118 by 12642104/53760085 = 0.2351578127155119
  
  rat: replaced -0.05578231276230905 by -1398019/25062048 = -0.05578231276230897
  
  rat: replaced 0.2416381240094921 by 8002142/33116223 = 0.2416381240094923
  
  rat: replaced -0.05816622897846346 by -4451048/76522891 = -0.05816622897846345
  
  rat: replaced 0.2481942708591053 by 8882901/35790113 = 0.2481942708591057
  
  rat: replaced -0.06061532802852698 by -2146337/35409146 = -0.06061532802852686
  
  rat: replaced 0.2548255976551299 by 868346/3407609 = 0.25482559765513
  
  rat: replaced -0.0631303649963022 by -14651447/232082406 = -0.06313036499630222
  
  rat: replaced 0.2615314412704124 by 8212450/31401387 = 0.2615314412704127
  
  rat: replaced -0.06571208837185505 by -4240309/64528599 = -0.06571208837185509
  
  rat: replaced 0.2683111311261794 by 34459769/128432126 = 0.2683111311261794
  
  rat: replaced -0.06836123997666599 by -2716643/39739522 = -0.06836123997666604
  
  rat: replaced 0.2751639892590951 by 12552159/45617012 = 0.2751639892590949
  
  rat: replaced -0.07107855488944881 by -3146673/44270357 = -0.07107855488944893
  
  rat: replaced 0.2820893303890569 by 11134456/39471383 = 0.2820893303890568
  
  rat: replaced -0.07386476137264342 by -12898997/174629915 = -0.0738647613726434
  
  rat: replaced 0.2890864619877229 by 9583357/33150487 = 0.2890864619877228
  
  rat: replaced -0.07672058079958999 by -5073506/66129661 = -0.07672058079959007
  
  rat: replaced 0.2961546843477643 by 11052271/37319251 = 0.2961546843477647
  
  rat: replaced -0.07964672758239233 by -5672399/71219486 = -0.07964672758239227
  
  rat: replaced 0.3032932906528349 by 9918077/32701274 = 0.3032932906528351
  
  rat: replaced -0.08264390910047736 by -4686067/56701904 = -0.08264390910047748
  
  rat: replaced 0.3105015670482534 by 9320011/30015987 = 0.3105015670482533
  
  rat: replaced -0.0857128256298576 by -3585977/41837111 = -0.08571282562985766
  
  rat: replaced 0.3177787927123868 by 248395525/781661743 = 0.3177787927123868
  
  rat: replaced -0.08885417027310427 by -5751353/64728003 = -0.0888541702731042
  
  rat: replaced 0.3251242399287333 by 13842845/42577093 = 0.3251242399287335
  
  rat: replaced -0.09206862889003742 by -7305460/79347983 = -0.09206862889003745
  
  rat: replaced 0.3325371741586922 by 9318229/28021616 = 0.3325371741586923
  
  rat: replaced -0.09535688002914089 by -5971998/62627867 = -0.09535688002914103
  
  rat: replaced 0.3400168541150183 by 13391981/39386227 = 0.3400168541150184
  
  rat: replaced -0.0987195948597075 by -9821211/99485933 = -0.09871959485970745
  
  rat: replaced 0.3475625318359485 by 10097818/29053241 = 0.347562531835949
  
  rat: replaced -0.1021574371047232 by -8336413/81603584 = -0.1021574371047232
  
  rat: replaced 0.3551734527599992 by 15867851/44676343 = 0.3551734527599987
  
  rat: replaced -0.1056710629744951 by -5741011/54329074 = -0.105671062974495
  
  rat: replaced 0.3628488558014202 by 6897641/19009681 = 0.3628488558014203
  
  rat: replaced -0.1092611211010309 by -5551873/50812887 = -0.1092611211010309
  
  rat: replaced 0.3705879734263036 by 23358661/63031352 = 0.3705879734263038
  
  rat: replaced -0.1129282524731764 by -11548693/102265755 = -0.1129282524731764
  
  rat: replaced 0.3783900317293359 by 14241382/37636779 = 0.3783900317293358
  
  rat: replaced -0.1166730903725168 by -5656228/48479285 = -0.1166730903725168
  
  rat: replaced 0.3862542505111889 by 3461217/8960981 = 0.3862542505111884
  
  rat: replaced -0.1204962603100498 by -4057613/33674182 = -0.12049626031005
  
  rat: replaced 0.3941798433565377 by 5314214/13481699 = 0.3941798433565384
  
  rat: replaced -0.1243983799636342 by -7966447/64039797 = -0.1243983799636342
  
  rat: replaced 0.4021660177127022 by 11567173/28762184 = 0.4021660177127022
  
  rat: replaced -0.1283800591162231 by -796346/6203035 = -0.1283800591162229
  
  rat: replaced 0.4102119749689023 by 11320633/27597032 = 0.4102119749689024
  
  rat: replaced -0.1324418995948859 by -4716124/35609003 = -0.1324418995948862
  
  rat: replaced 0.418316910536117 by 12225195/29224721 = 0.4183169105361177
  
  rat: replaced -0.1365844952106265 by -612971/4487852 = -0.1365844952106264
  
  rat: replaced 0.4264800139275439 by 7978696/18708253 = 0.4264800139275431
  
  rat: replaced -0.140808431699002 by -10431632/74083859 = -0.1408084316990021
  
  rat: replaced 0.4347004688396462 by 20489554/47134879 = 0.4347004688396463
  
  rat: replaced -0.1451142866615502 by -3554077/24491572 = -0.1451142866615504
  
  rat: replaced 0.4429774532337832 by 23449796/52936771 = 0.4429774532337834
  
  rat: replaced -0.1495026295080298 by -26759297/178988805 = -0.1495026295080298
  
  rat: replaced 0.451310139418413 by 8841241/19590167 = 0.4513101394184133
  
  rat: replaced -0.1539740213994798 by -16145763/104860306 = -0.1539740213994798
  part: invalid index of list or matrix.
  #0: lineIntersection(g=[1,a,a*u],h=[-a,1,0])
  #1: projectToLine(a=[0,u],g=[-a,1,0])
   -- an error. To debug this try: debugmode(true);
  
  Error in:
  d1 &= distance(P,projectToLine(P,g1)); $d1 ...
                                       ^
\end{euleroutput}
\begin{eulercomment}
Hitung jarak ke g.
\end{eulercomment}
\begin{eulerprompt}
>d &= distance(P,projectToLine(P,g)); $d
\end{eulerprompt}
\begin{euleroutput}
  Maxima said:
  rat: replaced 5.033291500140813e-5 by 263336/5231884543 = 5.033291500140813e-5
  
  rat: replaced 2.026599467560841e-4 by 407727/2011877564 = 2.02659946756084e-4
  
  rat: replaced 4.589658460211338e-4 by 352373/767754296 = 4.589658460211339e-4
  
  rat: replaced 8.21224965753764e-4 by 219501/267284860 = 8.212249657537654e-4
  
  rat: replaced 0.001291401063677061 by 174589/135193477 = 0.001291401063677059
  
  rat: replaced 0.001871447105906615 by 1078337/576204904 = 0.001871447105906617
  
  rat: replaced 0.002563305071654892 by 1323915/516487489 = 0.002563305071654891
  
  rat: replaced 0.003368905759035173 by 820537/243561874 = 0.003368905759035176
  
  rat: replaced 0.00429016859198364 by 7572857/1765165363 = 0.00429016859198364
  
  rat: replaced 0.005329001428317881 by 3020890/566877311 = 0.005329001428317882
  
  rat: replaced 0.006487300368953564 by 2580732/397812935 = 0.006487300368953564
  
  rat: replaced 0.007766949568295017 by 1049181/135082762 = 0.007766949568295028
  
  rat: replaced 0.009169821045822202 by 2408608/262666849 = 0.009169821045822193
  
  rat: replaced 0.01069777449888981 by 2325322/217365023 = 0.01069777449888982
  
  rat: replaced 0.01235265711675931 by 7449711/603085711 = 0.01235265711675931
  
  rat: replaced 0.01413630339588112 by 3774568/267012379 = 0.01413630339588113
  
  rat: replaced 0.0160505349564472 by 2619104/163178611 = 0.0160505349564472
  
  rat: replaced 0.01809716036023018 by 3107690/171722521 = 0.01809716036023021
  
  rat: replaced 0.02027797492972855 by 6791343/334912289 = 0.02027797492972854
  
  rat: replaced 0.02259476056863596 by 2685790/118867823 = 0.02259476056863597
  
  rat: replaced 0.0250492855836526 by 2956693/118035023 = 0.02504928558365258
  
  rat: replaced 0.02764330450765584 by 2138111/77346433 = 0.02764330450765583
  
  rat: replaced 0.03037855792424843 by 1678577/55255322 = 0.03037855792424846
  
  rat: replaced 0.03325677229370128 by 1488397/44754704 = 0.03325677229370124
  
  rat: replaced 0.03627965978030939 by 3229091/89005548 = 0.03627965978030943
  
  rat: replaced 0.03944891808117656 by 6094420/154488901 = 0.03944891808117659
  
  rat: replaced 0.04276623025644721 by 206826/4836199 = 0.04276623025644726
  
  rat: replaced 0.04623326456100163 by 7175941/155211644 = 0.04623326456100162
  
  rat: replaced 0.04985167427763171 by 2856261/57295187 = 0.04985167427763173
  
  rat: replaced 0.0536230975517149 by 8075629/150599823 = 0.05362309755171492
  
  rat: replaced 0.05754915722739962 by 12314906/213989337 = 0.05754915722739961
  
  rat: replaced 0.06163146068532366 by 10145753/164619707 = 0.06163146068532366
  
  rat: replaced 0.06587159968187639 by 5154956/78257641 = 0.06587159968187643
  
  rat: replaced 0.07027115019002506 by 3189686/45391117 = 0.07027115019002507
  
  rat: replaced 0.07483167224171838 by 4757796/63579977 = 0.0748316722417185
  
  rat: replaced 0.07955470977188528 by 7059961/88743470 = 0.07955470977188518
  
  rat: replaced 0.08444179046404166 by 17285418/204702173 = 0.08444179046404163
  
  rat: replaced 0.08949442559752452 by 6119169/68374862 = 0.08949442559752442
  
  rat: replaced 0.0947141098963642 by 2739857/28927654 = 0.09471410989636422
  
  rat: replaced 0.100102321379814 by 21380147/213582929 = 0.100102321379814
  
  rat: replaced 0.1056605212145493 by 8628153/81659194 = 0.1056605212145493
  
  rat: replaced 0.1113901535685515 by 4925969/44222661 = 0.1113901535685517
  
  rat: replaced 0.1172926454666934 by 7052303/60125705 = 0.1172926454666935
  
  rat: replaced 0.1233694066480375 by 17851649/144700777 = 0.1233694066480376
  
  rat: replaced 0.1296218294248629 by 13037238/100579031 = 0.1296218294248629
  
  rat: replaced 0.1360512885434353 by 20468361/150445918 = 0.1360512885434353
  
  rat: replaced 0.1426591410465347 by 8451499/59242604 = 0.1426591410465347
  
  rat: replaced 0.1494467261377502 by 40350618/270000013 = 0.1494467261377502
  
  rat: replaced 0.1564153650475627 by 30759845/196654881 = 0.1564153650475627
  
  rat: replaced 0.1635663609012215 by 11970848/73186491 = 0.1635663609012215
  
  rat: replaced 0.1709009985884339 by 3726835/21806982 = 0.1709009985884337
  
  rat: replaced 0.1784205446348769 by 7050541/39516419 = 0.178420544634877
  
  rat: replaced 0.1861262470755453 by 7913431/42516470 = 0.1861262470755451
  
  rat: replaced 0.1940193353299499 by 15356416/79148895 = 0.19401933532995
  
  rat: replaced 0.2021010200791761 by 21517868/106470853 = 0.202101020079176
  
  rat: replaced 0.2103724931448173 by 10133132/48167571 = 0.2103724931448173
  
  rat: replaced 0.2188349273697929 by 14393696/65774217 = 0.2188349273697929
  
  rat: replaced 0.2274894765010662 by 2362445/10384854 = 0.2274894765010659
  
  rat: replaced 0.2363372750742693 by 14238388/60246053 = 0.2363372750742692
  
  rat: replaced 0.2453794383002513 by 11843947/48267887 = 0.2453794383002513
  
  rat: replaced 0.2546170619535583 by 10437767/40993981 = 0.2546170619535585
  
  rat: replaced 0.2640512222628563 by 18572095/70335198 = 0.2640512222628562
  
  rat: replaced 0.2736829758033094 by 25733021/94024924 = 0.2736829758033094
  
  rat: replaced 0.2835133593909236 by 5354031/18884581 = 0.2835133593909232
  
  rat: replaced 0.2935433899788653 by 33562265/114334937 = 0.2935433899788654
  
  rat: replaced 0.3037740645557676 by 12785981/42090430 = 0.3037740645557672
  
  rat: replaced 0.3142063600460319 by 13879096/44171913 = 0.314206360046032
  
  rat: replaced 0.3248412332121354 by 13048490/40168823 = 0.3248412332121357
  
  rat: replaced 0.3356796205589581 by 12520681/37299497 = 0.3356796205589582
  
  rat: replaced 0.3467224382401299 by 27133151/78256115 = 0.3467224382401299
  
  rat: replaced 0.3579705819664191 by 32019579/89447515 = 0.3579705819664191
  
  rat: replaced 0.3694249269161592 by 12845283/34771024 = 0.3694249269161587
  
  rat: replaced 0.3810863276477343 by 12790304/33562747 = 0.381086327647734
  
  rat: replaced 0.3929556180141225 by 27557157/70127912 = 0.3929556180141225
  
  rat: replaced 0.4050336110795114 by 12582391/31065054 = 0.4050336110795107
  
  rat: replaced 0.4173210990379927 by 17616979/42214446 = 0.4173210990379928
  
  rat: replaced 0.4298188531343438 by 28764336/66921997 = 0.4298188531343439
  
  rat: replaced 0.4425276235869029 by 52612738/118891421 = 0.4425276235869029
  
  rat: replaced 0.4554481395125489 by 12438812/27311149 = 0.4554481395125485
  
  rat: replaced 0.4685811088537897 by 12910499/27552325 = 0.46858110885379
  
  rat: replaced 0.4819272183079686 by 11623658/24119115 = 0.4819272183079686
  
  rat: replaced 0.4954871332585954 by 40137729/81006602 = 0.4954871332585954
  
  rat: replaced 0.5092614977088081 by 27060617/53136978 = 0.5092614977088084
  
  rat: replaced 0.523250934216974 by 57357723/109618004 = 0.5232509342169741
  
  rat: replaced 0.5374560438344332 by 19984722/37183919 = 0.5374560438344328
  
  rat: replaced 0.551877406045395 by 10637804/19275665 = 0.5518774060453946
  
  rat: replaced 0.5665155787089895 by 22241852/39260795 = 0.5665155787089895
  
  rat: replaced 0.5813710980034821 by 10844268/18652919 = 0.5813710980034814
  
  rat: replaced 0.5964444783726564 by 13079224/21928653 = 0.596444478372657
  
  rat: replaced 0.6117362124743696 by 11199699/18308053 = 0.6117362124743685
  
  rat: replaced 0.6272467711312885 by 11338738/18076997 = 0.6272467711312891
  
  rat: replaced 0.6429766032838061 by 10161473/15803799 = 0.6429766032838053
  
  rat: replaced 0.6589261359451484 by 11120191/16876233 = 0.6589261359451484
  
  rat: replaced 0.6750957741586742 by 11234073/16640710 = 0.6750957741586747
  
  rat: replaced 0.6914859009573701 by 9571673/13842181 = 0.6914859009573708
  
  rat: replaced 0.7080968773255479 by 20218829/28553761 = 0.7080968773255474
  
  rat: replaced 0.7249290421627467 by 10945526/15098755 = 0.7249290421627479
  
  rat: replaced 0.7419827122498429 by 23520179/31699093 = 0.7419827122498426
  
  rat: replaced 0.7592581822173726 by 16709871/22008154 = 0.7592581822173727
  
  rat: replaced 9.983250083613754e-5 by 612914/6139423483 = 9.983250083613756e-5
  
  rat: replaced 3.986533601775671e-4 by 220554/553247563 = 3.986533601775666e-4
  
  rat: replaced 8.954327045205754e-4 by 584699/652979277 = 8.954327045205756e-4
  
  rat: replaced 0.001589120864678328 by 740868/466212493 = 0.00158912086467833
  
  rat: replaced 0.002478648480745763 by 878917/354595259 = 0.002478648480745762
  
  rat: replaced 0.003562926609036218 by 2735717/767828614 = 0.003562926609036219
  
  rat: replaced 0.004840846830973591 by 1164348/240525685 = 0.004840846830973582
  
  rat: replaced 0.006311281363933816 by 16515210/2616776063 = 0.006311281363933816
  
  rat: replaced 0.007973083174022497 by 2414321/302808957 = 0.007973083174022491
  
  rat: replaced 0.009825086090776508 by 1144049/116441626 = 0.009825086090776506
  
  rat: replaced 0.01186610492378118 by 1659683/139867548 = 0.01186610492378118
  
  rat: replaced 0.01409493558118687 by 986877/70016425 = 0.01409493558118684
  
  rat: replaced 0.01651035519011868 by 1738361/105289134 = 0.01651035519011867
  
  rat: replaced 0.01911112221896202 by 1475047/77182647 = 0.01911112221896199
  
  rat: replaced 0.02189597660151474 by 7711274/352177669 = 0.02189597660151473
  
  rat: replaced 0.02486363986299212 by 3887839/156366446 = 0.02486363986299209
  
  rat: replaced 0.0280128152478745 by 2263313/80795628 = 0.02801281524787455
  
  rat: replaced 0.03134218784958129 by 1116362/35618509 = 0.03134218784958124
  
  rat: replaced 0.03485042474195996 by 3920507/112495243 = 0.03485042474195998
  
  rat: replaced 0.03853617511257795 by 5379408/139593719 = 0.03853617511257795
  
  rat: replaced 0.04239807039780302 by 3385918/79860191 = 0.04239807039780308
  
  rat: replaced 0.04643472441965829 by 10918553/235137672 = 0.04643472441965828
  
  rat: replaced 0.05064473352443885 by 5036501/99447675 = 0.05064473352443886
  
  rat: replaced 0.05502667672307548 by 2932521/53292715 = 0.05502667672307557
  
  rat: replaced 0.05957911583323347 by 6320819/106091185 = 0.05957911583323346
  
  rat: replaced 0.06430059562312868 by 9893260/153859539 = 0.0643005956231287
  
  rat: replaced 0.06918964395705007 by 6012189/86894348 = 0.06918964395705
  
  rat: replaced 0.07424477194257195 by 6096479/82113243 = 0.07424477194257204
  
  rat: replaced 0.07946447407944118 by 5389689/67825139 = 0.07946447407944125
  
  rat: replaced 0.0848472284101276 by 9595393/113090235 = 0.08484722841012754
  
  rat: replaced 0.09039149667201674 by 3773144/41742245 = 0.09039149667201657
  
  rat: replaced 0.0960957244512361 by 5162056/53717853 = 0.09609572445123597
  
  rat: replaced 0.1019583413380946 by 1082663/10618680 = 0.1019583413380948
  
  rat: replaced 0.107977761084122 by 1922059/17800508 = 0.1079777610841219
  
  rat: replaced 0.1141523817606936 by 5923297/51889386 = 0.1141523817606938
  
  rat: replaced 0.1204805859192203 by 17634703/146369665 = 0.1204805859192204
  
  rat: replaced 0.1269607407528933 by 11368220/89541223 = 0.1269607407528932
  
  rat: replaced 0.1335911982599624 by 4657902/34866833 = 0.1335911982599624
  
  rat: replaced 0.1403702954085355 by 8528456/60756843 = 0.1403702954085353
  
  rat: replaced 0.1472963543028805 by 11128453/75551449 = 0.1472963543028804
  
  rat: replaced 0.1543676823512128 by 8170760/52930509 = 0.1543676823512126
  
  rat: replaced 0.1615825724349539 by 188109817/1164171446 = 0.1615825724349539
  
  rat: replaced 0.1689393030794406 by 5046974/29874481 = 0.1689393030794409
  
  rat: replaced 0.1764361386260728 by 6530305/37012287 = 0.176436138626073
  
  rat: replaced 0.1840713294058766 by 25189859/136848357 = 0.1840713294058766
  
  rat: replaced 0.1918431119144694 by 24326967/126806570 = 0.1918431119144694
  
  rat: replaced 0.1997497089884105 by 14902039/74603558 = 0.1997497089884104
  
  rat: replaced 0.2077893299829148 by 7281351/35041987 = 0.2077893299829145
  
  rat: replaced 0.2159601709509153 by 11348921/52550991 = 0.2159601709509151
  
  rat: replaced 0.2242604148234577 by 22385730/99820247 = 0.2242604148234576
  
  rat: replaced 0.2326882315914051 by 25615030/110083049 = 0.2326882315914051
  
  rat: replaced 0.2412417784884371 by 14523232/60201977 = 0.2412417784884373
  
  rat: replaced 0.2499192001753251 by 11309023/45250717 = 0.2499192001753254
  
  rat: replaced 0.2587186289254649 by 7582961/29309683 = 0.2587186289254647
  
  rat: replaced 0.267638184811648 by 17912865/66929407 = 0.2676381848116479
  
  rat: replaced 0.2766759758940514 by 27538925/99534934 = 0.2766759758940514
  
  rat: replaced 0.2858300984094321 by 29258587/102363562 = 0.2858300984094321
  
  rat: replaced 0.2950986369614998 by 7877677/26695064 = 0.2950986369614997
  
  rat: replaced 0.304479664712457 by 14469542/47522195 = 0.304479664712457
  
  rat: replaced 0.3139712435756791 by 8375733/26676752 = 0.3139712435756797
  
  rat: replaced 0.3235714244095225 by 178371467/551258404 = 0.3235714244095225
  
  rat: replaced 0.3332782472122374 by 5743591/17233621 = 0.333278247212237
  
  rat: replaced 0.3430897413179662 by 15588245/45434891 = 0.3430897413179664
  
  rat: replaced 0.3530039255938071 by 6523425/18479752 = 0.3530039255938067
  
  rat: replaced 0.3630188086379282 by 51253958/141188161 = 0.3630188086379282
  
  rat: replaced 0.373132388978704 by 9370061/25111894 = 0.3731323889787047
  
  rat: replaced 0.3833426552748616 by 11820697/30835851 = 0.3833426552748617
  
  rat: replaced 0.393647586516613 by 9153768/23253713 = 0.3936475865166135
  
  rat: replaced 0.4040451522277552 by 16634707/41170416 = 0.404045152227755
  
  rat: replaced 0.4145333126687146 by 2088920/5039209 = 0.4145333126687145
  
  rat: replaced 0.4251100190405208 by 24667763/58026774 = 0.4251100190405209
  
  rat: replaced 0.4357732136896836 by 10448574/23977091 = 0.435773213689684
  
  rat: replaced 0.4465208303139576 by 8346266/18691773 = 0.4465208303139568
  
  rat: replaced 0.4573507941689697 by 20158688/44077081 = 0.4573507941689696
  
  rat: replaced 0.4682610222756929 by 12818601/27374905 = 0.4682610222756937
  
  rat: replaced 0.4792494236287415 by 13652513/28487281 = 0.4792494236287416
  
  rat: replaced 0.4903138994054704 by 35114711/71616797 = 0.4903138994054705
  
  rat: replaced 0.5014523431758559 by 15102855/30118226 = 0.5014523431758564
  
  rat: replaced 0.5126626411131362 by 31697340/61828847 = 0.5126626411131361
  
  rat: replaced 0.5239426722051925 by 27432767/52358337 = 0.5239426722051924
  
  rat: replaced 0.5352903084666492 by 6124470/11441399 = 0.5352903084666482
  
  rat: replaced 0.5467034151516694 by 41717397/76307182 = 0.5467034151516694
  
  rat: replaced 0.5581798509674292 by 7494380/13426461 = 0.5581798509674292
  
  rat: replaced 0.5697174682882435 by 14609183/25642856 = 0.5697174682882438
  
  rat: replaced 0.581314113370329 by 14367580/24715691 = 0.5813141133703282
  
  rat: replaced 0.5929676265671738 by 9820294/16561265 = 0.5929676265671735
  
  rat: replaced 0.6046758425455033 by 23593213/39017952 = 0.6046758425455031
  
  rat: replaced 0.6164365905018095 by 15720181/25501700 = 0.6164365905018097
  
  rat: replaced 0.6282476943794307 by 53974636/85912987 = 0.6282476943794306
  
  rat: replaced 0.640106973086155 by 20459615/31962806 = 0.6401069730861552
  
  rat: replaced 0.652012240712328 by 51645100/79208789 = 0.652012240712328
  
  rat: replaced 0.6639613067494411 by 12215999/18398661 = 0.6639613067494422
  
  rat: replaced 0.6759519763091814 by 18558734/27455699 = 0.6759519763091808
  
  rat: replaced 0.6879820503429186 by 23500536/34158647 = 0.687982050342919
  
  rat: replaced 0.7000493258616074 by 29992669/42843651 = 0.7000493258616078
  
  rat: replaced 0.7121515961560857 by 10685401/15004391 = 0.7121515961560853
  
  rat: replaced 0.7242866510177421 by 11795807/16286103 = 0.7242866510177419
  
  rat: replaced 0.7364522769595366 by 14940657/20287339 = 0.7364522769595362
  
  rat: replaced 0.7486462574373463 by 42508133/56779998 = 0.7486462574373461
  part: invalid index of list or matrix.
  #0: lineIntersection(g=[2,1,u],h=[-1,2,1])
  #1: projectToLine(a=[0,u],g=[-1,2,1])
   -- an error. To debug this try: debugmode(true);
  
  Error in:
  d &= distance(P,projectToLine(P,g)); $d ...
                                     ^
\end{euleroutput}
\begin{eulercomment}
Dan temukan pusat kedua lingkaran yang jaraknya sama.
\end{eulercomment}
\begin{eulerprompt}
>sol &= solve(d1^2=d^2,u); $sol
\end{eulerprompt}
\begin{eulercomment}
Ada dua solusi.

Kami mengevaluasi solusi simbolis, dan menemukan kedua pusat, dan
kedua jarak.
\end{eulercomment}
\begin{eulerprompt}
>u := sol()
\end{eulerprompt}
\begin{euleroutput}
  []
\end{euleroutput}
\begin{eulerprompt}
>dd := d()
\end{eulerprompt}
\begin{euleroutput}
  Function d needs at least one argument!
  Use: d (n) 
  Error in:
  dd := d() ...
           ^
\end{euleroutput}
\begin{eulercomment}
Plot lingkaran ke dalam gambar.
\end{eulercomment}
\begin{eulerprompt}
>color(red);
>plotCircle(circleWithCenter([0,u[1]],dd[1]),"");
\end{eulerprompt}
\begin{euleroutput}
  Index 1 out of bounds!
  Error in:
  plotCircle(circleWithCenter([0,u[1]],dd[1]),""); ...
                                     ^
\end{euleroutput}
\begin{eulerprompt}
>plotCircle(circleWithCenter([0,u[2]],dd[2]),"");
\end{eulerprompt}
\begin{euleroutput}
  Index 2 out of bounds!
  Error in:
  plotCircle(circleWithCenter([0,u[2]],dd[2]),""); ...
                                     ^
\end{euleroutput}
\begin{eulerprompt}
>insimg;
\end{eulerprompt}
\eulersubheading{Plot dengan Povray}
\begin{eulercomment}
Selanjutnya kami merencanakan semuanya dengan Povray. Perhatikan bahwa
Anda mengubah perintah apa pun dalam urutan perintah Povray berikut,
dan menjalankan kembali semua perintah dengan Shift-Return.

Pertama kita memuat fungsi povray.
\end{eulercomment}
\begin{eulerprompt}
>load povray;
>defaultpovray="C:\(\backslash\)Program Files\(\backslash\)POV-Ray\(\backslash\)v3.7\(\backslash\)bin\(\backslash\)pvengine.exe"
\end{eulerprompt}
\begin{euleroutput}
  C:\(\backslash\)Program Files\(\backslash\)POV-Ray\(\backslash\)v3.7\(\backslash\)bin\(\backslash\)pvengine.exe
\end{euleroutput}
\begin{eulercomment}
Kami mengatur adegan dengan tepat.
\end{eulercomment}
\begin{eulerprompt}
>povstart(zoom=11,center=[0,0,0.5],height=10°,angle=140°);
\end{eulerprompt}
\begin{eulercomment}
Selanjutnya kita menulis dua bidang ke file Povray.
\end{eulercomment}
\begin{eulerprompt}
>writeln(povsphere([0,0,u[1]],dd[1],povlook(red)));
\end{eulerprompt}
\begin{euleroutput}
  Index 1 out of bounds!
  Error in:
  writeln(povsphere([0,0,u[1]],dd[1],povlook(red))); ...
                             ^
\end{euleroutput}
\begin{eulerprompt}
>writeln(povsphere([0,0,u[2]],dd[2],povlook(red)));
\end{eulerprompt}
\begin{euleroutput}
  Index 2 out of bounds!
  Error in:
  writeln(povsphere([0,0,u[2]],dd[2],povlook(red))); ...
                             ^
\end{euleroutput}
\begin{eulercomment}
Dan kerucutnya, transparan.
\end{eulercomment}
\begin{eulerprompt}
>writeln(povcone([0,0,0],0,[0,0,a],1,povlook(lightgray,1)));
\end{eulerprompt}
\begin{eulercomment}
Kami menghasilkan bidang terbatas pada kerucut.
\end{eulercomment}
\begin{eulerprompt}
>gp=g();
>pc=povcone([0,0,0],0,[0,0,a],1,"");
>vp=[gp[1],0,gp[2]]; dp=gp[3];
>writeln(povplane(vp,dp,povlook(blue,0.5),pc));
\end{eulerprompt}
\begin{eulercomment}
Sekarang kita menghasilkan dua titik pada lingkaran, di mana bola
menyentuh kerucut.
\end{eulercomment}
\begin{eulerprompt}
>function turnz(v) := return [-v[2],v[1],v[3]]
>P1=projectToLine([0,u[1]],g1()); P1=turnz([P1[1],0,P1[2]]);
\end{eulerprompt}
\begin{euleroutput}
  Index 1 out of bounds!
  Error in:
  P1=projectToLine([0,u[1]],g1()); P1=turnz([P1[1],0,P1[2]]); ...
                          ^
\end{euleroutput}
\begin{eulerprompt}
>writeln(povpoint(P1,povlook(yellow)));
\end{eulerprompt}
\begin{euleroutput}
  Function povpoint needs a vector for P
  Error in:
  writeln(povpoint(P1,povlook(yellow))); ...
                                      ^
\end{euleroutput}
\begin{eulerprompt}
>P2=projectToLine([0,u[2]],g1()); P2=turnz([P2[1],0,P2[2]]);
\end{eulerprompt}
\begin{euleroutput}
  Index 2 out of bounds!
  Error in:
  P2=projectToLine([0,u[2]],g1()); P2=turnz([P2[1],0,P2[2]]); ...
                          ^
\end{euleroutput}
\begin{eulerprompt}
>writeln(povpoint(P2,povlook(yellow)));
\end{eulerprompt}
\begin{euleroutput}
  Function povpoint needs a vector for P
  Error in:
  writeln(povpoint(P2,povlook(yellow))); ...
                                      ^
\end{euleroutput}
\begin{eulercomment}
Kemudian kami menghasilkan dua titik di mana bola menyentuh bidang.
Ini adalah fokus dari elips.
\end{eulercomment}
\begin{eulerprompt}
>P3=projectToLine([0,u[1]],g()); P3=[P3[1],0,P3[2]];
\end{eulerprompt}
\begin{euleroutput}
  Index 1 out of bounds!
  Error in:
  P3=projectToLine([0,u[1]],g()); P3=[P3[1],0,P3[2]]; ...
                          ^
\end{euleroutput}
\begin{eulerprompt}
>writeln(povpoint(P3,povlook(yellow)));
\end{eulerprompt}
\begin{euleroutput}
  Variable or function P3 not found.
  Error in:
  writeln(povpoint(P3,povlook(yellow))); ...
                     ^
\end{euleroutput}
\begin{eulerprompt}
>P4=projectToLine([0,u[2]],g()); P4=[P4[1],0,P4[2]];
\end{eulerprompt}
\begin{euleroutput}
  Index 2 out of bounds!
  Error in:
  P4=projectToLine([0,u[2]],g()); P4=[P4[1],0,P4[2]]; ...
                          ^
\end{euleroutput}
\begin{eulerprompt}
>writeln(povpoint(P4,povlook(yellow)));
\end{eulerprompt}
\begin{euleroutput}
  Variable or function P4 not found.
  Error in:
  writeln(povpoint(P4,povlook(yellow))); ...
                     ^
\end{euleroutput}
\begin{eulercomment}
Selanjutnya kita hitung perpotongan P1P2 dengan bidang.
\end{eulercomment}
\begin{eulerprompt}
>t1=scalp(vp,P1)-dp; t2=scalp(vp,P2)-dp; P5=P1+t1/(t1-t2)*(P2-P1);
\end{eulerprompt}
\begin{euleroutput}
  Matrix expected in scalp!
  Error in:
  t1=scalp(vp,P1)-dp; t2=scalp(vp,P2)-dp; P5=P1+t1/(t1-t2)*(P2-P ...
                 ^
\end{euleroutput}
\begin{eulerprompt}
>writeln(povpoint(P5,povlook(yellow)));
\end{eulerprompt}
\begin{euleroutput}
  Variable or function P5 not found.
  Error in:
  writeln(povpoint(P5,povlook(yellow))); ...
                     ^
\end{euleroutput}
\begin{eulercomment}
Kami menghubungkan titik-titik dengan segmen garis.
\end{eulercomment}
\begin{eulerprompt}
>writeln(povsegment(P1,P2,povlook(yellow)));
\end{eulerprompt}
\begin{euleroutput}
  Function povsegment needs a vector for P1
  Error in:
  writeln(povsegment(P1,P2,povlook(yellow))); ...
                                           ^
\end{euleroutput}
\begin{eulerprompt}
>writeln(povsegment(P5,P3,povlook(yellow)));
\end{eulerprompt}
\begin{euleroutput}
  Variable or function P5 not found.
  Error in:
  writeln(povsegment(P5,P3,povlook(yellow))); ...
                       ^
\end{euleroutput}
\begin{eulerprompt}
>writeln(povsegment(P5,P4,povlook(yellow)));
\end{eulerprompt}
\begin{euleroutput}
  Variable or function P5 not found.
  Error in:
  writeln(povsegment(P5,P4,povlook(yellow))); ...
                       ^
\end{euleroutput}
\begin{eulercomment}
Sekarang kita menghasilkan pita abu-abu, di mana bola menyentuh
kerucut.
\end{eulercomment}
\begin{eulerprompt}
>pcw=povcone([0,0,0],0,[0,0,a],1.01);
>pc1=povcylinder([0,0,P1[3]-defaultpointsize/2],[0,0,P1[3]+defaultpointsize/2],1);
\end{eulerprompt}
\begin{euleroutput}
  Index 3 out of range for string (need string vector).
  Error in:
  pc1=povcylinder([0,0,P1[3]-defaultpointsize/2],[0,0,P1[3]+defa ...
                            ^
\end{euleroutput}
\begin{eulerprompt}
>writeln(povintersection([pcw,pc1],povlook(gray)));
\end{eulerprompt}
\begin{euleroutput}
  Variable pc1 not found!
  Error in:
  writeln(povintersection([pcw,pc1],povlook(gray))); ...
                                  ^
\end{euleroutput}
\begin{eulerprompt}
>pc2=povcylinder([0,0,P2[3]-defaultpointsize/2],[0,0,P2[3]+defaultpointsize/2],1);
\end{eulerprompt}
\begin{euleroutput}
  Index 3 out of range for string (need string vector).
  Error in:
  pc2=povcylinder([0,0,P2[3]-defaultpointsize/2],[0,0,P2[3]+defa ...
                            ^
\end{euleroutput}
\begin{eulerprompt}
>writeln(povintersection([pcw,pc2],povlook(gray)));
\end{eulerprompt}
\begin{euleroutput}
  Variable pc2 not found!
  Error in:
  writeln(povintersection([pcw,pc2],povlook(gray))); ...
                                  ^
\end{euleroutput}
\begin{eulercomment}
Mulai program Povray.
\end{eulercomment}
\begin{eulerprompt}
>povend();
\end{eulerprompt}
\begin{eulercomment}
Untuk mendapatkan Anaglyph ini kita perlu memasukkan semuanya ke dalam
fungsi scene. Fungsi ini akan digunakan dua kali kemudian.
\end{eulercomment}
\begin{eulerprompt}
>function scene () ...
\end{eulerprompt}
\begin{eulerudf}
  global a,u,dd,g,g1,defaultpointsize;
  writeln(povsphere([0,0,u[1]],dd[1],povlook(red)));
  writeln(povsphere([0,0,u[2]],dd[2],povlook(red)));
  writeln(povcone([0,0,0],0,[0,0,a],1,povlook(lightgray,1)));
  gp=g();
  pc=povcone([0,0,0],0,[0,0,a],1,"");
  vp=[gp[1],0,gp[2]]; dp=gp[3];
  writeln(povplane(vp,dp,povlook(blue,0.5),pc));
  P1=projectToLine([0,u[1]],g1()); P1=turnz([P1[1],0,P1[2]]);
  writeln(povpoint(P1,povlook(yellow)));
  P2=projectToLine([0,u[2]],g1()); P2=turnz([P2[1],0,P2[2]]);
  writeln(povpoint(P2,povlook(yellow)));
  P3=projectToLine([0,u[1]],g()); P3=[P3[1],0,P3[2]];
  writeln(povpoint(P3,povlook(yellow)));
  P4=projectToLine([0,u[2]],g()); P4=[P4[1],0,P4[2]];
  writeln(povpoint(P4,povlook(yellow)));
  t1=scalp(vp,P1)-dp; t2=scalp(vp,P2)-dp; P5=P1+t1/(t1-t2)*(P2-P1);
  writeln(povpoint(P5,povlook(yellow)));
  writeln(povsegment(P1,P2,povlook(yellow)));
  writeln(povsegment(P5,P3,povlook(yellow)));
  writeln(povsegment(P5,P4,povlook(yellow)));
  pcw=povcone([0,0,0],0,[0,0,a],1.01);
  pc1=povcylinder([0,0,P1[3]-defaultpointsize/2],[0,0,P1[3]+defaultpointsize/2],1);
  writeln(povintersection([pcw,pc1],povlook(gray)));
  pc2=povcylinder([0,0,P2[3]-defaultpointsize/2],[0,0,P2[3]+defaultpointsize/2],1);
  writeln(povintersection([pcw,pc2],povlook(gray)));
  endfunction
\end{eulerudf}
\begin{eulercomment}
Anda membutuhkan kacamata merah/sian untuk menghargai efek berikut.
\end{eulercomment}
\begin{eulerprompt}
>povanaglyph("scene",zoom=11,center=[0,0,0.5],height=10°,angle=140°);
\end{eulerprompt}
\begin{euleroutput}
  Global variable dd not found in "global" command.
  scene:
      global a,u,dd,g,g1,defaultpointsize;
  Try "trace errors" to inspect local variables after errors.
  povanaglyph:
      scene$(args());
\end{euleroutput}
\eulersubheading{Contoh 8: Geometri Bumi}
\begin{eulercomment}
Dalam buku catatan ini, kami ingin melakukan beberapa perhitungan
sferis. Fungsi-fungsi tersebut terdapat dalam file "spherical.e" di
folder contoh. Kita perlu memuat file itu terlebih dahulu.
\end{eulercomment}
\begin{eulerprompt}
>load "spherical.e";
\end{eulerprompt}
\begin{eulercomment}
Untuk memasukkan posisi geografis, kami menggunakan vektor dengan dua
koordinat dalam radian (utara dan timur, nilai negatif untuk selatan
dan barat). Berikut koordinat Kampus FMIPA UNY.
\end{eulercomment}
\begin{eulerprompt}
>FMIPA=[rad(-7,-46.467),rad(110,23.05)]
\end{eulerprompt}
\begin{euleroutput}
  [-0.13569,  1.92657]
\end{euleroutput}
\begin{eulercomment}
Anda dapat mencetak posisi ini dengan sposprint (cetak posisi
spherical).
\end{eulercomment}
\begin{eulerprompt}
>sposprint(FMIPA) // posisi garis lintang dan garis bujur FMIPA UNY
\end{eulerprompt}
\begin{euleroutput}
  S 7°46.467' E 110°23.050'
\end{euleroutput}
\begin{eulercomment}
Mari kita tambahkan dua kota lagi, Solo dan Semarang.
\end{eulercomment}
\begin{eulerprompt}
>Solo=[rad(-7,-34.333),rad(110,49.683)]; Semarang=[rad(-6,-59.05),rad(110,24.533)];
>sposprint(Solo), sposprint(Semarang),
\end{eulerprompt}
\begin{euleroutput}
  S 7°34.333' E 110°49.683'
  S 6°59.050' E 110°24.533'
\end{euleroutput}
\begin{eulercomment}
Pertama kita menghitung vektor dari satu ke yang lain pada bola ideal.
Vektor ini [pos,jarak] dalam radian. Untuk menghitung jarak di bumi,
kita kalikan dengan jari-jari bumi pada garis lintang 7°.
\end{eulercomment}
\begin{eulerprompt}
>br=svector(FMIPA,Solo); degprint(br[1]), br[2]*rearth(7°)->km // perkiraan jarak FMIPA-Solo
\end{eulerprompt}
\begin{euleroutput}
  65°20'26.60''
  53.8945384608
\end{euleroutput}
\begin{eulercomment}
Ini adalah perkiraan yang baik. Rutinitas berikut menggunakan
perkiraan yang lebih baik. Pada jarak yang begitu pendek hasilnya
hampir sama.
\end{eulercomment}
\begin{eulerprompt}
>esdist(FMIPA,Semarang)->" km", // perkiraan jarak FMIPA-Semarang
\end{eulerprompt}
\begin{euleroutput}
  88.0114026318 km
\end{euleroutput}
\begin{eulercomment}
Ada fungsi untuk heading, dengan mempertimbangkan bentuk elips bumi.
Sekali lagi, kami mencetak dengan cara yang canggih.
\end{eulercomment}
\begin{eulerprompt}
>sdegprint(esdir(FMIPA,Solo))
\end{eulerprompt}
\begin{euleroutput}
       65.34°
\end{euleroutput}
\begin{eulercomment}
Sudut segitiga melebihi 180° pada bola.
\end{eulercomment}
\begin{eulerprompt}
>asum=sangle(Solo,FMIPA,Semarang)+sangle(FMIPA,Solo,Semarang)+sangle(FMIPA,Semarang,Solo); degprint(asum)
\end{eulerprompt}
\begin{euleroutput}
  180°0'10.77''
\end{euleroutput}
\begin{eulercomment}
Ini dapat digunakan untuk menghitung luas segitiga. Catatan: Untuk
segitiga kecil, ini tidak akurat karena kesalahan pengurangan dalam
asum-pi.
\end{eulercomment}
\begin{eulerprompt}
>(asum-pi)*rearth(48°)^2->" km^2", // perkiraan luas segitiga FMIPA-Solo-Semarang
\end{eulerprompt}
\begin{euleroutput}
  2116.02948749 km^2
\end{euleroutput}
\begin{eulercomment}
Ada fungsi untuk ini, yang menggunakan garis lintang rata-rata
segitiga untuk menghitung jari-jari bumi, dan menangani kesalahan
pembulatan untuk segitiga yang sangat kecil.
\end{eulercomment}
\begin{eulerprompt}
>esarea(Solo,FMIPA,Semarang)->" km^2", //perkiraan yang sama dengan fungsi esarea()
\end{eulerprompt}
\begin{euleroutput}
  2123.64310526 km^2
\end{euleroutput}
\begin{eulercomment}
Kita juga dapat menambahkan vektor ke posisi. Sebuah vektor berisi
heading dan jarak, keduanya dalam radian. Untuk mendapatkan vektor,
kami menggunakan vektor. Untuk menambahkan vektor ke posisi, kami
menggunakan vektor sadd.
\end{eulercomment}
\begin{eulerprompt}
>v=svector(FMIPA,Solo); sposprint(saddvector(FMIPA,v)), sposprint(Solo),
\end{eulerprompt}
\begin{euleroutput}
  S 7°34.333' E 110°49.683'
  S 7°34.333' E 110°49.683'
\end{euleroutput}
\begin{eulercomment}
Fungsi-fungsi ini mengasumsikan bola yang ideal. Hal yang sama di
bumi.
\end{eulercomment}
\begin{eulerprompt}
>sposprint(esadd(FMIPA,esdir(FMIPA,Solo),esdist(FMIPA,Solo))), sposprint(Solo),
\end{eulerprompt}
\begin{euleroutput}
  S 7°34.333' E 110°49.683'
  S 7°34.333' E 110°49.683'
\end{euleroutput}
\begin{eulercomment}
Mari kita beralih ke contoh yang lebih besar, Tugu Jogja dan Monas
Jakarta (menggunakan Google Earth untuk mencari koordinatnya).
\end{eulercomment}
\begin{eulerprompt}
>Tugu=[-7.7833°,110.3661°]; Monas=[-6.175°,106.811944°];
>sposprint(Tugu), sposprint(Monas)
\end{eulerprompt}
\begin{euleroutput}
  S 7°46.998' E 110°21.966'
  S 6°10.500' E 106°48.717'
\end{euleroutput}
\begin{eulercomment}
Menurut Google Earth, jaraknya adalah 429,66 km. Kami mendapatkan
pendekatan yang baik.
\end{eulercomment}
\begin{eulerprompt}
>esdist(Tugu,Monas)->" km", // perkiraan jarak Tugu Jogja - Monas Jakarta
\end{eulerprompt}
\begin{euleroutput}
  431.565659488 km
\end{euleroutput}
\begin{eulercomment}
Judulnya sama dengan judul yang dihitung di Google Earth.
\end{eulercomment}
\begin{eulerprompt}
>degprint(esdir(Tugu,Monas))
\end{eulerprompt}
\begin{euleroutput}
  294°17'2.85''
\end{euleroutput}
\begin{eulercomment}
Namun, kita tidak lagi mendapatkan posisi target yang tepat, jika kita
menambahkan heading dan jarak ke posisi semula. Hal ini terjadi,
karena kita tidak menghitung fungsi invers secara tepat, tetapi
mengambil perkiraan jari-jari bumi di sepanjang jalan.
\end{eulercomment}
\begin{eulerprompt}
>sposprint(esadd(Tugu,esdir(Tugu,Monas),esdist(Tugu,Monas)))
\end{eulerprompt}
\begin{euleroutput}
  S 6°10.500' E 106°48.717'
\end{euleroutput}
\begin{eulercomment}
Namun, kesalahannya tidak besar.
\end{eulercomment}
\begin{eulerprompt}
>sposprint(Monas),
\end{eulerprompt}
\begin{euleroutput}
  S 6°10.500' E 106°48.717'
\end{euleroutput}
\begin{eulercomment}
Tentu kita tidak bisa berlayar dengan tujuan yang sama dari satu
tujuan ke tujuan lainnya, jika kita ingin menempuh jalur terpendek.
Bayangkan, Anda terbang NE mulai dari titik mana pun di bumi. Kemudian
Anda akan berputar ke kutub utara. Lingkaran besar tidak mengikuti
heading yang konstan!

Perhitungan berikut menunjukkan bahwa kami jauh dari tujuan yang
benar, jika kami menggunakan pos yang sama selama perjalanan kami.
\end{eulercomment}
\begin{eulerprompt}
>dist=esdist(Tugu,Monas); hd=esdir(Tugu,Monas);
\end{eulerprompt}
\begin{eulercomment}
Sekarang kita tambahkan 10 kali sepersepuluh dari jarak, menggunakan
pos ke Monas, kita sampai di Tugu.
\end{eulercomment}
\begin{eulerprompt}
>p=Tugu; loop 1 to 10; p=esadd(p,hd,dist/10); end;
\end{eulerprompt}
\begin{eulercomment}
Hasilnya jauh.
\end{eulercomment}
\begin{eulerprompt}
>sposprint(p), skmprint(esdist(p,Monas))
\end{eulerprompt}
\begin{euleroutput}
  S 6°11.250' E 106°48.372'
       1.529km
\end{euleroutput}
\begin{eulercomment}
Sebagai contoh lain, mari kita ambil dua titik di bumi pada garis
lintang yang sama.
\end{eulercomment}
\begin{eulerprompt}
>P1=[30°,10°]; P2=[30°,50°];
\end{eulerprompt}
\begin{eulercomment}
Jalur terpendek dari P1 ke P2 bukanlah lingkaran garis lintang 30°,
melainkan jalur terpendek yang dimulai 10° lebih jauh ke utara di P1.
\end{eulercomment}
\begin{eulerprompt}
>sdegprint(esdir(P1,P2))
\end{eulerprompt}
\begin{euleroutput}
       79.69°
\end{euleroutput}
\begin{eulercomment}
Tapi, jika kita mengikuti pembacaan kompas ini, kita akan berputar ke
kutub utara! Jadi kita harus menyesuaikan arah kita di sepanjang
jalan. Untuk tujuan kasar, kami menyesuaikannya pada 1/10 dari total
jarak.
\end{eulercomment}
\begin{eulerprompt}
>p=P1;  dist=esdist(P1,P2); ...
>  loop 1 to 10; dir=esdir(p,P2); sdegprint(dir), p=esadd(p,dir,dist/10); end;
\end{eulerprompt}
\begin{euleroutput}
       79.69°
       81.67°
       83.71°
       85.78°
       87.89°
       90.00°
       92.12°
       94.22°
       96.29°
       98.33°
\end{euleroutput}
\begin{eulercomment}
Jaraknya tidak tepat, karena kita akan menambahkan sedikit kesalahan,
jika kita mengikuti heading yang sama terlalu lama.
\end{eulercomment}
\begin{eulerprompt}
>skmprint(esdist(p,P2))
\end{eulerprompt}
\begin{euleroutput}
       0.203km
\end{euleroutput}
\begin{eulercomment}
Kami mendapatkan perkiraan yang baik, jika kami menyesuaikan pos
setelah setiap 1/100 dari total jarak dari Tugu ke Monas.
\end{eulercomment}
\begin{eulerprompt}
>p=Tugu; dist=esdist(Tugu,Monas); ...
>  loop 1 to 100; p=esadd(p,esdir(p,Monas),dist/100); end;
>skmprint(esdist(p,Monas))
\end{eulerprompt}
\begin{euleroutput}
       0.000km
\end{euleroutput}
\begin{eulercomment}
Untuk keperluan navigasi, kita bisa mendapatkan urutan posisi GPS di
sepanjang lingkaran besar menuju Monas dengan fungsi navigasi.
\end{eulercomment}
\begin{eulerprompt}
>load spherical; v=navigate(Tugu,Monas,10); ...
>  loop 1 to rows(v); sposprint(v[#]), end;
\end{eulerprompt}
\begin{euleroutput}
  S 7°46.998' E 110°21.966'
  S 7°37.422' E 110°0.573'
  S 7°27.829' E 109°39.196'
  S 7°18.219' E 109°17.834'
  S 7°8.592' E 108°56.488'
  S 6°58.948' E 108°35.157'
  S 6°49.289' E 108°13.841'
  S 6°39.614' E 107°52.539'
  S 6°29.924' E 107°31.251'
  S 6°20.219' E 107°9.977'
  S 6°10.500' E 106°48.717'
\end{euleroutput}
\begin{eulercomment}
Kami menulis sebuah fungsi, yang memplot bumi, dua posisi, dan posisi
di antaranya.
\end{eulercomment}
\begin{eulerprompt}
>function testplot ...
\end{eulerprompt}
\begin{eulerudf}
  useglobal;
  plotearth;
  plotpos(Tugu,"Tugu Jogja"); plotpos(Monas,"Tugu Monas");
  plotposline(v);
  endfunction
\end{eulerudf}
\begin{eulercomment}
Sekarang rencanakan semuanya.
\end{eulercomment}
\begin{eulerprompt}
>plot3d("testplot",angle=25, height=6,>own,>user,zoom=4):
\end{eulerprompt}
\begin{eulercomment}
Atau gunakan plot3d untuk mendapatkan tampilan anaglyph. Ini terlihat
sangat bagus dengan kacamata merah/sian.
\end{eulercomment}
\begin{eulerprompt}
>plot3d("testplot",angle=25,height=6,distance=5,own=1,anaglyph=1,zoom=4):
\end{eulerprompt}
\eulersubheading{MENCOBA RUMUS-RUMUS PADA MATERI DI ATAS}
\eulersubheading{Geometri Simbolik}
\begin{eulerprompt}
>A &= [2,0]; B &= [0,2]; C &= [3,3]; // menentukan tiga titik A, B, C
>c &= lineThrough(B,C) // c=BC
\end{eulerprompt}
\begin{euleroutput}
  
                               [- 1, 3, 6]
  
\end{euleroutput}
\begin{eulerprompt}
>$getLineEquation(c,x,y), $solve(%,y) | expand // persamaan garis c
\end{eulerprompt}
\begin{euleroutput}
  Maxima said:
  solve: all variables must not be numbers.
   -- an error. To debug this try: debugmode(true);
  
  Error in:
   $getLineEquation(c,x,y), $solve(%,y) | expand // persamaan gar ...
                                                ^
\end{euleroutput}
\begin{eulerprompt}
>h &= perpendicular(A,lineThrough(B,C)) // h melalui A tegak lurus BC
\end{eulerprompt}
\begin{euleroutput}
  
                                [3, 1, 6]
  
\end{euleroutput}
\begin{eulerprompt}
>Q &= lineIntersection(c,h) // Q titik potong garis c=BC dan h
\end{eulerprompt}
\begin{euleroutput}
  Maxima said:
  rat: replaced 1.498320841708742e-4 by 1329822/8875415485 = 1.498320841708742e-4
  
  rat: replaced 5.986466935998108e-4 by 398723/666040595 = 5.986466935998098e-4
  
  rat: replaced 0.001345398955533032 by 4525441/3363642421 = 0.001345398955533032
  
  rat: replaced 0.002389014203700413 by 1071627/448564516 = 0.00238901420370041
  
  rat: replaced 0.00372838808577948 by 661903/177530607 = 0.003728388085779485
  
  rat: replaced 0.005362386673832029 by 5230891/975478144 = 0.005362386673832028
  
  rat: replaced 0.007289846577694006 by 32241346/4422774287 = 0.007289846577694006
  
  rat: replaced 0.009509575061314376 by 2146493/225719129 = 0.009509575061314364
  
  rat: replaced 0.01202035016202822 by 1789188/148846579 = 0.01202035016202825
  
  rat: replaced 0.01482092081275069 by 2581665/174190594 = 0.01482092081275066
  
  rat: replaced 0.01791000696708436 by 5107285/285163764 = 0.01791000696708436
  
  rat: replaced 0.02128629972732062 by 3323295/156123659 = 0.02128629972732064
  
  rat: replaced 0.02494846147533059 by 4548287/182307314 = 0.02494846147533061
  
  rat: replaced 0.02889512600632479 by 3147802/108938857 = 0.02889512600632481
  
  rat: replaced 0.0331248986654725 by 5858625/176864692 = 0.03312489866547248
  
  rat: replaced 0.03763635648736519 by 10043830/266865099 = 0.03763635648736518
  
  rat: replaced 0.04242804833831373 by 4635713/109260576 = 0.04242804833831372
  
  rat: replaced 0.04749849506145984 by 5610259/118114458 = 0.04749849506145979
  
  rat: replaced 0.05284618962468965 by 4237503/80185592 = 0.05284618962468968
  
  rat: replaced 0.05846959727133633 by 3317197/56733707 = 0.05846959727133642
  
  rat: replaced 0.06436715567365475 by 13427433/208606903 = 0.06436715567365477
  
  rat: replaced 0.07053727508905278 by 8025659/113778977 = 0.07053727508905269
  
  rat: replaced 0.07697833851906408 by 6306881/81930594 = 0.07697833851906408
  
  rat: replaced 0.08368870187104593 by 4282086/51166835 = 0.08368870187104596
  
  rat: replaced 0.09066669412258874 by 2175091/23989967 = 0.09066669412258883
  
  rat: replaced 0.09791061748861546 by 8290049/84669561 = 0.09791061748861554
  
  rat: replaced 0.1054187475911595 by 8501563/80645646 = 0.1054187475911595
  
  rat: replaced 0.1131893336318011 by 6539019/57770629 = 0.113189333631801
  
  rat: replaced 0.121220598566744 by 5779101/47674249 = 0.1212205985667441
  
  rat: replaced 0.1295107392845216 by 7134865/55090914 = 0.1295107392845216
  
  rat: replaced 0.1380579267863034 by 6113057/44278928 = 0.1380579267863034
  
  rat: replaced 0.1468603063687953 by 6311140/42973763 = 0.1468603063687953
  
  rat: replaced 0.1559159978097077 by 4027079/25828517 = 0.1559159978097078
  
  rat: replaced 0.1652230955557758 by 10597125/64138279 = 0.1652230955557757
  
  rat: replaced 0.1747796689133147 by 9649007/55206690 = 0.1747796689133147
  
  rat: replaced 0.1845837622412855 by 6871913/37229239 = 0.1845837622412857
  
  rat: replaced 0.1946333951468589 by 39341769/202132676 = 0.1946333951468589
  
  rat: replaced 0.2049265626834523 by 10758647/52500012 = 0.2049265626834523
  
  rat: replaced 0.2154612355512225 by 33702610/156420759 = 0.2154612355512225
  
  rat: replaced 0.2262353602999955 by 2338161/10335082 = 0.2262353602999957
  
  rat: replaced 0.2372468595346078 by 7573078/31920667 = 0.2372468595346081
  
  rat: replaced 0.2484936321226457 by 3764353/15148690 = 0.2484936321226456
  
  rat: replaced 0.2599735534045555 by 26335713/101301508 = 0.2599735534045554
  
  rat: replaced 0.2716844754061095 by 29831699/109802737 = 0.2716844754061094
  
  rat: replaced 0.2836242270531998 by 15100773/53242183 = 0.2836242270531995
  
  rat: replaced 0.2957906143889442 by 2942977/9949528 = 0.2957906143889439
  
  rat: replaced 0.3081814207930817 by 12077608/39189929 = 0.3081814207930818
  
  rat: replaced 0.3207944072036307 by 9185023/28632117 = 0.3207944072036308
  
  rat: replaced 0.333627312340794 by 5228336/15671187 = 0.3336273123407946
  
  rat: replaced 0.346677852933085 by 15615111/45042136 = 0.3466778529330847
  
  rat: replaced 0.3599437239456539 by 7564465/21015688 = 0.3599437239456543
  
  rat: replaced 0.3734225988107874 by 7702871/20627758 = 0.3734225988107869
  
  rat: replaced 0.3871121296605642 by 97723109/252441351 = 0.3871121296605642
  
  rat: replaced 0.4010099475616409 by 3146543/7846546 = 0.4010099475616405
  
  rat: replaced 0.4151136627521425 by 6219049/14981557 = 0.4151136627521425
  
  rat: replaced 0.4294208648806354 by 26148647/60892819 = 0.4294208648806356
  
  rat: replaced 0.4439291232471635 by 19525684/43983787 = 0.4439291232471638
  
  rat: replaced 0.458635987046313 by 38604672/84172793 = 0.4586359870463132
  
  rat: replaced 0.4735389856122937 by 11146199/23538081 = 0.4735389856122935
  
  rat: replaced 0.488635628666001 by 13946471/28541658 = 0.4886356286660011
  
  rat: replaced 0.5039234065640431 by 5948069/11803518 = 0.503923406564043
  
  rat: replaced 0.5193997905497036 by 24027011/46259185 = 0.5193997905497038
  
  rat: replaced 0.5350622330058146 by 7363779/13762472 = 0.5350622330058147
  
  rat: replaced 0.5509081677095147 by 8130825/14758948 = 0.5509081677095142
  
  rat: replaced 0.5669350100888726 by 10250363/18080314 = 0.5669350100888735
  
  rat: replaced 0.5831401574813392 by 37655026/64572857 = 0.5831401574813393
  
  rat: replaced 0.5995209893940125 by 30778651/51338738 = 0.5995209893940128
  
  rat: replaced 0.6160748677656853 by 23698401/38466755 = 0.616074867765685
  
  rat: replaced 0.6327991372306488 by 5052598/7984521 = 0.6327991372306492
  
  rat: replaced 0.6496911253842265 by 60646047/93345968 = 0.6496911253842266
  
  rat: replaced 0.666748143050013 by 30125566/45182827 = 0.6667481430500132
  
  rat: replaced 0.6839674845487889 by 8953739/13090884 = 0.6839674845487899
  
  rat: replaced 0.7013464279690875 by 7888577/11247761 = 0.7013464279690864
  
  rat: replaced 0.7188822354393821 by 16662338/23178119 = 0.7188822354393815
  
  rat: replaced 0.7365721534018723 by 13899283/18870226 = 0.7365721534018723
  
  rat: replaced 0.7544134128878366 by 16270763/21567436 = 0.754413412887837
  
  rat: replaced 0.7724032297945274 by 8203205/10620366 = 0.7724032297945287
  
  rat: replaced 0.7905388051635788 by 10794522/13654639 = 0.7905388051635784
  
  rat: replaced 0.8088173254609005 by 16745047/20703126 = 0.808817325460899
  
  rat: replaced 0.8272359628580275 by 20291194/24528907 = 0.827235962858027
  
  rat: replaced 0.8457918755149025 by 10996366/13001267 = 0.8457918755149018
  
  rat: replaced 0.8644822078640563 by 9158500/10594203 = 0.8644822078640555
  
  rat: replaced 0.8833040908961625 by 13759446/15577247 = 0.8833040908961641
  
  rat: replaced 0.9022546424469358 by 19827819/21975857 = 0.9022546424469362
  
  rat: replaced 0.9213309674853474 by 60458149/65620446 = 0.9213309674853475
  
  rat: replaced 0.9405301584031224 by 11658841/12396031 = 0.9405301584031212
  
  rat: replaced 0.9598492953055026 by 26214088/27310629 = 0.9598492953055018
  
  rat: replaced 0.9792854463032298 by 35089005/35831233 = 0.9792854463032293
  
  rat: replaced 0.9988356678057343 by 15735752/15754095 = 0.9988356678057356
  
  rat: replaced 1.018497004815491 by 16202286/15908035 = 1.018497004815491
  
  rat: replaced 1.038266491223517 by 17763365/17108676 = 1.038266491223517
  
  rat: replaced 1.058141150105979 by 33730321/31876958 = 1.058141150105979
  
  rat: replaced 1.078117994021884 by 51996446/48228901 = 1.078117994021883
  
  rat: replaced 1.098194025311821 by 124719922/113568203 = 1.098194025311821
  
  rat: replaced 1.118366236397724 by 92837336/83011569 = 1.118366236397724
  
  rat: replaced 1.13863161008363 by 20601995/18093644 = 1.138631610083629
  
  rat: replaced 1.158987119857388 by 20626233/17796775 = 1.15898711985739
  
  rat: replaced 1.17942973019332 by 4098089/3474636 = 1.179429730193321
  
  rat: replaced 1.199956396855759 by 17442145/14535649 = 1.199956396855758
  
  rat: replaced 5.049958083474387e-5 by 102157/2022927682 = 5.049958083474385e-5
  
  rat: replaced 2.039932534230044e-4 by 284619/1395237319 = 2.039932534230043e-4
  
  rat: replaced 4.634656435254722e-4 by 573493/1237401322 = 4.634656435254721e-4
  
  rat: replaced 8.31890779119604e-4 by 332331/399488741 = 8.318907791196046e-4
  
  rat: replaced 0.001312231792998733 by 448125/341498356 = 0.001312231792998734
  
  rat: replaced 0.001907440626462018 by 276030/144712237 = 0.001907440626462018
  
  rat: replaced 0.002620457734122131 by 2586613/987084419 = 0.002620457734122131
  
  rat: replaced 0.00345421178986248 by 3402379/984994322 = 0.00345421178986248
  
  rat: replaced 0.004411619393972596 by 966955/219183686 = 0.004411619393972597
  
  rat: replaced 0.005495584781489732 by 2798484/509224061 = 0.005495584781489734
  
  rat: replaced 0.006708999531778753 by 6054060/902378957 = 0.006708999531778753
  
  rat: replaced 0.008054742279375651 by 806546/100133061 = 0.00805474227937564
  
  rat: replaced 0.009535678426127348 by 4115324/431571181 = 0.009535678426127346
  
  rat: replaced 0.01115465985465333 by 2266398/203179481 = 0.01115465985465334
  
  rat: replaced 0.01291452464316009 by 2106925/163143829 = 0.01291452464316012
  
  rat: replaced 0.01481809678163515 by 2779203/187554653 = 0.01481809678163516
  
  rat: replaced 0.01686818588945119 by 7427428/440321683 = 0.01686818588945119
  
  rat: replaced 0.01906758693440599 by 2278085/119474216 = 0.019067586934406
  
  rat: replaced 0.02141907995322798 by 2316386/108145915 = 0.02141907995322801
  
  rat: replaced 0.02392542977357476 by 1665518/69612877 = 0.02392542977357479
  
  rat: replaced 0.02658938573755304 by 3678645/138350131 = 0.02658938573755308
  
  rat: replaced 0.02941368142678652 by 4053557/137811957 = 0.0294136814267865
  
  rat: replaced 0.03240103438906003 by 2629160/81144323 = 0.03240103438906009
  
  rat: replaced 0.03555414586656669 by 1834427/51595305 = 0.03555414586656674
  
  rat: replaced 0.03887570052578646 by 1643964/42287701 = 0.03887570052578645
  
  rat: replaced 0.04236836618902146 by 3055464/72116635 = 0.04236836618902143
  
  rat: replaced 0.04603479356761608 by 3139251/68193007 = 0.0460347935676161
  
  rat: replaced 0.04987761599688789 by 5203437/104324092 = 0.04987761599688785
  
  rat: replaced 0.05389944917279615 by 4533622/84112585 = 0.0538994491727962
  
  rat: replaced 0.05810289089037535 by 11687290/201148167 = 0.05810289089037535
  
  rat: replaced 0.06249052078395612 by 3949243/63197473 = 0.06249052078395603
  
  rat: replaced 0.0670649000692059 by 3281728/48933615 = 0.067064900069206
  
  rat: replaced 0.07182857128700804 by 4146139/57722699 = 0.07182857128700791
  
  rat: replaced 0.07678405804921068 by 1198255/15605518 = 0.07678405804921054
  
  rat: replaced 0.08193386478626702 by 5956639/72700574 = 0.08193386478626702
  
  rat: replaced 0.08728047649679532 by 2799808/32078285 = 0.08728047649679527
  
  rat: replaced 0.09282635849907966 by 10292829/110882611 = 0.09282635849907972
  
  rat: replaced 0.09857395618454184 by 4198057/42587892 = 0.09857395618454184
  
  rat: replaced 0.1045256947732028 by 30563827/292404916 = 0.1045256947732028
  
  rat: replaced 0.1106839790711635 by 8949559/80856860 = 0.1106839790711635
  
  rat: replaced 0.1170511932301264 by 9911603/84677505 = 0.1170511932301265
  
  rat: replaced 0.1236297005089814 by 19561703/158228184 = 0.1236297005089814
  
  rat: replaced 0.1304218430374826 by 4975231/38147222 = 0.1304218430374825
  
  rat: replaced 0.137429941582038 by 3502939/25488907 = 0.137429941582038
  
  rat: replaced 0.1446562953136327 by 15521432/107298697 = 0.1446562953136327
  
  rat: replaced 0.1521031815779155 by 18080502/118869979 = 0.1521031815779155
  
  rat: replaced 0.1597728556674664 by 37419026/234201397 = 0.1597728556674664
  
  rat: replaced 0.1676675505962674 by 22585897/134706429 = 0.1676675505962674
  
  rat: replaced 0.1757894768764047 by 15940893/90681725 = 0.1757894768764048
  
  rat: replaced 0.1841408222970185 by 7944795/43145213 = 0.1841408222970182
  
  rat: replaced 0.1927237517055264 by 1392861/7227241 = 0.1927237517055264
  
  rat: replaced 0.2015404067911402 by 1735485/8611102 = 0.2015404067911401
  
  rat: replaced 0.2105929058706983 by 10627754/50465869 = 0.2105929058706985
  
  rat: replaced 0.2198833436768368 by 9372347/42624179 = 0.2198833436768366
  
  rat: replaced 0.2294137911485169 by 7405273/32279110 = 0.2294137911485168
  
  rat: replaced 0.239186295223934 by 27692337/115777273 = 0.239186295223934
  
  rat: replaced 0.2492028786358237 by 8925310/35815437 = 0.249202878635824
  
  rat: replaced 0.2594655397091927 by 11150701/42975653 = 0.259465539709193
  
  rat: replaced 0.2699762521614856 by 11249087/41666950 = 0.2699762521614853
  
  rat: replaced 0.280736964905216 by 12097010/43090193 = 0.2807369649052164
  
  rat: replaced 0.2917496018530771 by 14831788/50837389 = 0.2917496018530771
  
  rat: replaced 0.3030160617255513 by 18597622/61375037 = 0.3030160617255514
  
  rat: replaced 0.3145382178610399 by 11102944/35299189 = 0.3145382178610392
  
  rat: replaced 0.3263179180285316 by 13053510/40002431 = 0.3263179180285318
  
  rat: replaced 0.3383569842428258 by 13796661/40775458 = 0.3383569842428257
  
  rat: replaced 0.3506572125823338 by 33496033/95523582 = 0.3506572125823338
  
  rat: replaced 0.3632203730094723 by 23086207/63559780 = 0.3632203730094724
  
  rat: replaced 0.376048209193667 by 22674222/60296051 = 0.3760482091936668
  
  rat: replaced 0.3891424383369902 by 33246815/85436107 = 0.3891424383369902
  
  rat: replaced 0.402504751002439 by 7793813/19363282 = 0.4025047510024385
  
  rat: replaced 0.4161368109448825 by 10481453/25187517 = 0.4161368109448819
  
  rat: replaced 0.4300402549446862 by 19565443/45496771 = 0.4300402549446861
  
  rat: replaced 0.4442166926440365 by 16102633/36249500 = 0.4442166926440365
  
  rat: replaced 0.4586677063859775 by 19404529/42306290 = 0.4586677063859771
  
  rat: replaced 0.4733948510561774 by 10262860/21679281 = 0.4733948510561766
  
  rat: replaced 0.4883996539274416 by 4159841/8517289 = 0.488399653927441
  
  rat: replaced 0.5036836145069872 by 13202363/26211619 = 0.5036836145069864
  
  rat: replaced 0.5192482043864929 by 12221370/23536663 = 0.5192482043864927
  
  rat: replaced 0.5350948670949413 by 52965833/98984005 = 0.5350948670949413
  
  rat: replaced 0.551225017954267 by 14288533/25921416 = 0.551225017954266
  
  rat: replaced 0.5676400439378262 by 25565995/45039097 = 0.5676400439378259
  
  rat: replaced 0.5843413035316997 by 18888222/32323955 = 0.5843413035316997
  
  rat: replaced 0.6013301265988455 by 5789399/9627655 = 0.6013301265988447
  
  rat: replaced 0.6186078142461149 by 4803773/7765458 = 0.618607814246114
  
  rat: replaced 0.6361756386941407 by 13914515/21872128 = 0.6361756386941407
  
  rat: replaced 0.6540348431501183 by 48160581/73636109 = 0.6540348431501181
  
  rat: replaced 0.6721866416834846 by 6617334/9844489 = 0.6721866416834841
  
  rat: replaced 0.690632219104513 by 16840135/24383651 = 0.6906322191045139
  
  rat: replaced 0.7093727308458327 by 29189494/41148317 = 0.7093727308458326
  
  rat: replaced 0.7284093028468864 by 13153959/18058472 = 0.7284093028468854
  
  rat: replaced 0.7477430314413382 by 12236470/16364539 = 0.7477430314413379
  
  rat: replaced 0.7673749832474404 by 39576757/51574208 = 0.7673749832474402
  
  rat: replaced 0.7873061950613714 by 6818881/8661028 = 0.7873061950613714
  
  rat: replaced 0.8075376737535601 by 20498953/25384516 = 0.807537673753559
  
  rat: replaced 0.8280703961679966 by 6989671/8440914 = 0.8280703961679979
  
  rat: replaced 0.84890530902455 by 25231431/29722315 = 0.8489053090245494
  
  rat: replaced 0.8700433288242969 by 9721738/11173855 = 0.8700433288242957
  
  rat: replaced 0.8914853417578728 by 33469619/37543656 = 0.8914853417578725
  
  rat: replaced 0.9132322036168524 by 21961040/24047597 = 0.913232203616852
  part: invalid index of list or matrix.
  #0: lineIntersection(g=[-1,3,6],h=[3,1,6])
   -- an error. To debug this try: debugmode(true);
  
  Error in:
  ... ersection(c,h) // Q titik potong garis c=BC dan h ...
                                                       ^
\end{euleroutput}
\begin{eulerprompt}
>$projectToLine(A,lineThrough(B,C)) // proyeksi A pada BC
\end{eulerprompt}
\begin{euleroutput}
  Maxima said:
  rat: replaced 5.049958083474387e-5 by 102157/2022927682 = 5.049958083474385e-5
  
  rat: replaced 2.039932534230044e-4 by 284619/1395237319 = 2.039932534230043e-4
  
  rat: replaced 4.634656435254722e-4 by 573493/1237401322 = 4.634656435254721e-4
  
  rat: replaced 8.31890779119604e-4 by 332331/399488741 = 8.318907791196046e-4
  
  rat: replaced 0.001312231792998733 by 448125/341498356 = 0.001312231792998734
  
  rat: replaced 0.001907440626462018 by 276030/144712237 = 0.001907440626462018
  
  rat: replaced 0.002620457734122131 by 2586613/987084419 = 0.002620457734122131
  
  rat: replaced 0.00345421178986248 by 3402379/984994322 = 0.00345421178986248
  
  rat: replaced 0.004411619393972596 by 966955/219183686 = 0.004411619393972597
  
  rat: replaced 0.005495584781489732 by 2798484/509224061 = 0.005495584781489734
  
  rat: replaced 0.006708999531778753 by 6054060/902378957 = 0.006708999531778753
  
  rat: replaced 0.008054742279375651 by 806546/100133061 = 0.00805474227937564
  
  rat: replaced 0.009535678426127348 by 4115324/431571181 = 0.009535678426127346
  
  rat: replaced 0.01115465985465333 by 2266398/203179481 = 0.01115465985465334
  
  rat: replaced 0.01291452464316009 by 2106925/163143829 = 0.01291452464316012
  
  rat: replaced 0.01481809678163515 by 2779203/187554653 = 0.01481809678163516
  
  rat: replaced 0.01686818588945119 by 7427428/440321683 = 0.01686818588945119
  
  rat: replaced 0.01906758693440599 by 2278085/119474216 = 0.019067586934406
  
  rat: replaced 0.02141907995322798 by 2316386/108145915 = 0.02141907995322801
  
  rat: replaced 0.02392542977357476 by 1665518/69612877 = 0.02392542977357479
  
  rat: replaced 0.02658938573755304 by 3678645/138350131 = 0.02658938573755308
  
  rat: replaced 0.02941368142678652 by 4053557/137811957 = 0.0294136814267865
  
  rat: replaced 0.03240103438906003 by 2629160/81144323 = 0.03240103438906009
  
  rat: replaced 0.03555414586656669 by 1834427/51595305 = 0.03555414586656674
  
  rat: replaced 0.03887570052578646 by 1643964/42287701 = 0.03887570052578645
  
  rat: replaced 0.04236836618902146 by 3055464/72116635 = 0.04236836618902143
  
  rat: replaced 0.04603479356761608 by 3139251/68193007 = 0.0460347935676161
  
  rat: replaced 0.04987761599688789 by 5203437/104324092 = 0.04987761599688785
  
  rat: replaced 0.05389944917279615 by 4533622/84112585 = 0.0538994491727962
  
  rat: replaced 0.05810289089037535 by 11687290/201148167 = 0.05810289089037535
  
  rat: replaced 0.06249052078395612 by 3949243/63197473 = 0.06249052078395603
  
  rat: replaced 0.0670649000692059 by 3281728/48933615 = 0.067064900069206
  
  rat: replaced 0.07182857128700804 by 4146139/57722699 = 0.07182857128700791
  
  rat: replaced 0.07678405804921068 by 1198255/15605518 = 0.07678405804921054
  
  rat: replaced 0.08193386478626702 by 5956639/72700574 = 0.08193386478626702
  
  rat: replaced 0.08728047649679532 by 2799808/32078285 = 0.08728047649679527
  
  rat: replaced 0.09282635849907966 by 10292829/110882611 = 0.09282635849907972
  
  rat: replaced 0.09857395618454184 by 4198057/42587892 = 0.09857395618454184
  
  rat: replaced 0.1045256947732028 by 30563827/292404916 = 0.1045256947732028
  
  rat: replaced 0.1106839790711635 by 8949559/80856860 = 0.1106839790711635
  
  rat: replaced 0.1170511932301264 by 9911603/84677505 = 0.1170511932301265
  
  rat: replaced 0.1236297005089814 by 19561703/158228184 = 0.1236297005089814
  
  rat: replaced 0.1304218430374826 by 4975231/38147222 = 0.1304218430374825
  
  rat: replaced 0.137429941582038 by 3502939/25488907 = 0.137429941582038
  
  rat: replaced 0.1446562953136327 by 15521432/107298697 = 0.1446562953136327
  
  rat: replaced 0.1521031815779155 by 18080502/118869979 = 0.1521031815779155
  
  rat: replaced 0.1597728556674664 by 37419026/234201397 = 0.1597728556674664
  
  rat: replaced 0.1676675505962674 by 22585897/134706429 = 0.1676675505962674
  
  rat: replaced 0.1757894768764047 by 15940893/90681725 = 0.1757894768764048
  
  rat: replaced 0.1841408222970185 by 7944795/43145213 = 0.1841408222970182
  
  rat: replaced 0.1927237517055264 by 1392861/7227241 = 0.1927237517055264
  
  rat: replaced 0.2015404067911402 by 1735485/8611102 = 0.2015404067911401
  
  rat: replaced 0.2105929058706983 by 10627754/50465869 = 0.2105929058706985
  
  rat: replaced 0.2198833436768368 by 9372347/42624179 = 0.2198833436768366
  
  rat: replaced 0.2294137911485169 by 7405273/32279110 = 0.2294137911485168
  
  rat: replaced 0.239186295223934 by 27692337/115777273 = 0.239186295223934
  
  rat: replaced 0.2492028786358237 by 8925310/35815437 = 0.249202878635824
  
  rat: replaced 0.2594655397091927 by 11150701/42975653 = 0.259465539709193
  
  rat: replaced 0.2699762521614856 by 11249087/41666950 = 0.2699762521614853
  
  rat: replaced 0.280736964905216 by 12097010/43090193 = 0.2807369649052164
  
  rat: replaced 0.2917496018530771 by 14831788/50837389 = 0.2917496018530771
  
  rat: replaced 0.3030160617255513 by 18597622/61375037 = 0.3030160617255514
  
  rat: replaced 0.3145382178610399 by 11102944/35299189 = 0.3145382178610392
  
  rat: replaced 0.3263179180285316 by 13053510/40002431 = 0.3263179180285318
  
  rat: replaced 0.3383569842428258 by 13796661/40775458 = 0.3383569842428257
  
  rat: replaced 0.3506572125823338 by 33496033/95523582 = 0.3506572125823338
  
  rat: replaced 0.3632203730094723 by 23086207/63559780 = 0.3632203730094724
  
  rat: replaced 0.376048209193667 by 22674222/60296051 = 0.3760482091936668
  
  rat: replaced 0.3891424383369902 by 33246815/85436107 = 0.3891424383369902
  
  rat: replaced 0.402504751002439 by 7793813/19363282 = 0.4025047510024385
  
  rat: replaced 0.4161368109448825 by 10481453/25187517 = 0.4161368109448819
  
  rat: replaced 0.4300402549446862 by 19565443/45496771 = 0.4300402549446861
  
  rat: replaced 0.4442166926440365 by 16102633/36249500 = 0.4442166926440365
  
  rat: replaced 0.4586677063859775 by 19404529/42306290 = 0.4586677063859771
  
  rat: replaced 0.4733948510561774 by 10262860/21679281 = 0.4733948510561766
  
  rat: replaced 0.4883996539274416 by 4159841/8517289 = 0.488399653927441
  
  rat: replaced 0.5036836145069872 by 13202363/26211619 = 0.5036836145069864
  
  rat: replaced 0.5192482043864929 by 12221370/23536663 = 0.5192482043864927
  
  rat: replaced 0.5350948670949413 by 52965833/98984005 = 0.5350948670949413
  
  rat: replaced 0.551225017954267 by 14288533/25921416 = 0.551225017954266
  
  rat: replaced 0.5676400439378262 by 25565995/45039097 = 0.5676400439378259
  
  rat: replaced 0.5843413035316997 by 18888222/32323955 = 0.5843413035316997
  
  rat: replaced 0.6013301265988455 by 5789399/9627655 = 0.6013301265988447
  
  rat: replaced 0.6186078142461149 by 4803773/7765458 = 0.618607814246114
  
  rat: replaced 0.6361756386941407 by 13914515/21872128 = 0.6361756386941407
  
  rat: replaced 0.6540348431501183 by 48160581/73636109 = 0.6540348431501181
  
  rat: replaced 0.6721866416834846 by 6617334/9844489 = 0.6721866416834841
  
  rat: replaced 0.690632219104513 by 16840135/24383651 = 0.6906322191045139
  
  rat: replaced 0.7093727308458327 by 29189494/41148317 = 0.7093727308458326
  
  rat: replaced 0.7284093028468864 by 13153959/18058472 = 0.7284093028468854
  
  rat: replaced 0.7477430314413382 by 12236470/16364539 = 0.7477430314413379
  
  rat: replaced 0.7673749832474404 by 39576757/51574208 = 0.7673749832474402
  
  rat: replaced 0.7873061950613714 by 6818881/8661028 = 0.7873061950613714
  
  rat: replaced 0.8075376737535601 by 20498953/25384516 = 0.807537673753559
  
  rat: replaced 0.8280703961679966 by 6989671/8440914 = 0.8280703961679979
  
  rat: replaced 0.84890530902455 by 25231431/29722315 = 0.8489053090245494
  
  rat: replaced 0.8700433288242969 by 9721738/11173855 = 0.8700433288242957
  
  rat: replaced 0.8914853417578728 by 33469619/37543656 = 0.8914853417578725
  
  rat: replaced 0.9132322036168524 by 21961040/24047597 = 0.913232203616852
  
  rat: replaced 1.498320841708742e-4 by 1329822/8875415485 = 1.498320841708742e-4
  
  rat: replaced 5.986466935998108e-4 by 398723/666040595 = 5.986466935998098e-4
  
  rat: replaced 0.001345398955533032 by 4525441/3363642421 = 0.001345398955533032
  
  rat: replaced 0.002389014203700413 by 1071627/448564516 = 0.00238901420370041
  
  rat: replaced 0.00372838808577948 by 661903/177530607 = 0.003728388085779485
  
  rat: replaced 0.005362386673832029 by 5230891/975478144 = 0.005362386673832028
  
  rat: replaced 0.007289846577694006 by 32241346/4422774287 = 0.007289846577694006
  
  rat: replaced 0.009509575061314376 by 2146493/225719129 = 0.009509575061314364
  
  rat: replaced 0.01202035016202822 by 1789188/148846579 = 0.01202035016202825
  
  rat: replaced 0.01482092081275069 by 2581665/174190594 = 0.01482092081275066
  
  rat: replaced 0.01791000696708436 by 5107285/285163764 = 0.01791000696708436
  
  rat: replaced 0.02128629972732062 by 3323295/156123659 = 0.02128629972732064
  
  rat: replaced 0.02494846147533059 by 4548287/182307314 = 0.02494846147533061
  
  rat: replaced 0.02889512600632479 by 3147802/108938857 = 0.02889512600632481
  
  rat: replaced 0.0331248986654725 by 5858625/176864692 = 0.03312489866547248
  
  rat: replaced 0.03763635648736519 by 10043830/266865099 = 0.03763635648736518
  
  rat: replaced 0.04242804833831373 by 4635713/109260576 = 0.04242804833831372
  
  rat: replaced 0.04749849506145984 by 5610259/118114458 = 0.04749849506145979
  
  rat: replaced 0.05284618962468965 by 4237503/80185592 = 0.05284618962468968
  
  rat: replaced 0.05846959727133633 by 3317197/56733707 = 0.05846959727133642
  
  rat: replaced 0.06436715567365475 by 13427433/208606903 = 0.06436715567365477
  
  rat: replaced 0.07053727508905278 by 8025659/113778977 = 0.07053727508905269
  
  rat: replaced 0.07697833851906408 by 6306881/81930594 = 0.07697833851906408
  
  rat: replaced 0.08368870187104593 by 4282086/51166835 = 0.08368870187104596
  
  rat: replaced 0.09066669412258874 by 2175091/23989967 = 0.09066669412258883
  
  rat: replaced 0.09791061748861546 by 8290049/84669561 = 0.09791061748861554
  
  rat: replaced 0.1054187475911595 by 8501563/80645646 = 0.1054187475911595
  
  rat: replaced 0.1131893336318011 by 6539019/57770629 = 0.113189333631801
  
  rat: replaced 0.121220598566744 by 5779101/47674249 = 0.1212205985667441
  
  rat: replaced 0.1295107392845216 by 7134865/55090914 = 0.1295107392845216
  
  rat: replaced 0.1380579267863034 by 6113057/44278928 = 0.1380579267863034
  
  rat: replaced 0.1468603063687953 by 6311140/42973763 = 0.1468603063687953
  
  rat: replaced 0.1559159978097077 by 4027079/25828517 = 0.1559159978097078
  
  rat: replaced 0.1652230955557758 by 10597125/64138279 = 0.1652230955557757
  
  rat: replaced 0.1747796689133147 by 9649007/55206690 = 0.1747796689133147
  
  rat: replaced 0.1845837622412855 by 6871913/37229239 = 0.1845837622412857
  
  rat: replaced 0.1946333951468589 by 39341769/202132676 = 0.1946333951468589
  
  rat: replaced 0.2049265626834523 by 10758647/52500012 = 0.2049265626834523
  
  rat: replaced 0.2154612355512225 by 33702610/156420759 = 0.2154612355512225
  
  rat: replaced 0.2262353602999955 by 2338161/10335082 = 0.2262353602999957
  
  rat: replaced 0.2372468595346078 by 7573078/31920667 = 0.2372468595346081
  
  rat: replaced 0.2484936321226457 by 3764353/15148690 = 0.2484936321226456
  
  rat: replaced 0.2599735534045555 by 26335713/101301508 = 0.2599735534045554
  
  rat: replaced 0.2716844754061095 by 29831699/109802737 = 0.2716844754061094
  
  rat: replaced 0.2836242270531998 by 15100773/53242183 = 0.2836242270531995
  
  rat: replaced 0.2957906143889442 by 2942977/9949528 = 0.2957906143889439
  
  rat: replaced 0.3081814207930817 by 12077608/39189929 = 0.3081814207930818
  
  rat: replaced 0.3207944072036307 by 9185023/28632117 = 0.3207944072036308
  
  rat: replaced 0.333627312340794 by 5228336/15671187 = 0.3336273123407946
  
  rat: replaced 0.346677852933085 by 15615111/45042136 = 0.3466778529330847
  
  rat: replaced 0.3599437239456539 by 7564465/21015688 = 0.3599437239456543
  
  rat: replaced 0.3734225988107874 by 7702871/20627758 = 0.3734225988107869
  
  rat: replaced 0.3871121296605642 by 97723109/252441351 = 0.3871121296605642
  
  rat: replaced 0.4010099475616409 by 3146543/7846546 = 0.4010099475616405
  
  rat: replaced 0.4151136627521425 by 6219049/14981557 = 0.4151136627521425
  
  rat: replaced 0.4294208648806354 by 26148647/60892819 = 0.4294208648806356
  
  rat: replaced 0.4439291232471635 by 19525684/43983787 = 0.4439291232471638
  
  rat: replaced 0.458635987046313 by 38604672/84172793 = 0.4586359870463132
  
  rat: replaced 0.4735389856122937 by 11146199/23538081 = 0.4735389856122935
  
  rat: replaced 0.488635628666001 by 13946471/28541658 = 0.4886356286660011
  
  rat: replaced 0.5039234065640431 by 5948069/11803518 = 0.503923406564043
  
  rat: replaced 0.5193997905497036 by 24027011/46259185 = 0.5193997905497038
  
  rat: replaced 0.5350622330058146 by 7363779/13762472 = 0.5350622330058147
  
  rat: replaced 0.5509081677095147 by 8130825/14758948 = 0.5509081677095142
  
  rat: replaced 0.5669350100888726 by 10250363/18080314 = 0.5669350100888735
  
  rat: replaced 0.5831401574813392 by 37655026/64572857 = 0.5831401574813393
  
  rat: replaced 0.5995209893940125 by 30778651/51338738 = 0.5995209893940128
  
  rat: replaced 0.6160748677656853 by 23698401/38466755 = 0.616074867765685
  
  rat: replaced 0.6327991372306488 by 5052598/7984521 = 0.6327991372306492
  
  rat: replaced 0.6496911253842265 by 60646047/93345968 = 0.6496911253842266
  
  rat: replaced 0.666748143050013 by 30125566/45182827 = 0.6667481430500132
  
  rat: replaced 0.6839674845487889 by 8953739/13090884 = 0.6839674845487899
  
  rat: replaced 0.7013464279690875 by 7888577/11247761 = 0.7013464279690864
  
  rat: replaced 0.7188822354393821 by 16662338/23178119 = 0.7188822354393815
  
  rat: replaced 0.7365721534018723 by 13899283/18870226 = 0.7365721534018723
  
  rat: replaced 0.7544134128878366 by 16270763/21567436 = 0.754413412887837
  
  rat: replaced 0.7724032297945274 by 8203205/10620366 = 0.7724032297945287
  
  rat: replaced 0.7905388051635788 by 10794522/13654639 = 0.7905388051635784
  
  rat: replaced 0.8088173254609005 by 16745047/20703126 = 0.808817325460899
  
  rat: replaced 0.8272359628580275 by 20291194/24528907 = 0.827235962858027
  
  rat: replaced 0.8457918755149025 by 10996366/13001267 = 0.8457918755149018
  
  rat: replaced 0.8644822078640563 by 9158500/10594203 = 0.8644822078640555
  
  rat: replaced 0.8833040908961625 by 13759446/15577247 = 0.8833040908961641
  
  rat: replaced 0.9022546424469358 by 19827819/21975857 = 0.9022546424469362
  
  rat: replaced 0.9213309674853474 by 60458149/65620446 = 0.9213309674853475
  
  rat: replaced 0.9405301584031224 by 11658841/12396031 = 0.9405301584031212
  
  rat: replaced 0.9598492953055026 by 26214088/27310629 = 0.9598492953055018
  
  rat: replaced 0.9792854463032298 by 35089005/35831233 = 0.9792854463032293
  
  rat: replaced 0.9988356678057343 by 15735752/15754095 = 0.9988356678057356
  
  rat: replaced 1.018497004815491 by 16202286/15908035 = 1.018497004815491
  
  rat: replaced 1.038266491223517 by 17763365/17108676 = 1.038266491223517
  
  rat: replaced 1.058141150105979 by 33730321/31876958 = 1.058141150105979
  
  rat: replaced 1.078117994021884 by 51996446/48228901 = 1.078117994021883
  
  rat: replaced 1.098194025311821 by 124719922/113568203 = 1.098194025311821
  
  rat: replaced 1.118366236397724 by 92837336/83011569 = 1.118366236397724
  
  rat: replaced 1.13863161008363 by 20601995/18093644 = 1.138631610083629
  
  rat: replaced 1.158987119857388 by 20626233/17796775 = 1.15898711985739
  
  rat: replaced 1.17942973019332 by 4098089/3474636 = 1.179429730193321
  
  rat: replaced 1.199956396855759 by 17442145/14535649 = 1.199956396855758
  part: invalid index of list or matrix.
  #0: lineIntersection(g=[3,1,6],h=[-1,3,6])
  #1: projectToLine(a=[2,0],g=[-1,3,6])
   -- an error. To debug this try: debugmode(true);
  
  Error in:
   $projectToLine(A,lineThrough(B,C)) // proyeksi A pada BC ...
                                     ^
\end{euleroutput}
\begin{eulerprompt}
>$distance(A,Q) // jarak AQ
>cc &= circleThrough(A,B,C); $cc // (titik pusat dan jari-jari) lingkaran melalui A, B, C
\end{eulerprompt}
\begin{euleroutput}
  Maxima said:
  rat: replaced -9.96658350028018e-5 by -86001/862893488 = -9.966583500280164e-5
  
  rat: replaced -3.973200535106469e-4 by -1080775/2720162223 = -3.973200535106468e-4
  
  rat: replaced -8.90932907016237e-4 by -1194571/1340809157 = -8.90932907016237e-4
  
  rat: replaced -0.001578455051312488 by -2522953/1598368606 = -0.001578455051312488
  
  rat: replaced -0.002457817751424091 by -519814/211494119 = -0.002457817751424095
  
  rat: replaced -0.003526933088480816 by -11813191/3349423055 = -0.003526933088480816
  
  rat: replaced -0.004783694168506353 by -866601/181157275 = -0.004783694168506343
  
  rat: replaced -0.006225975333106509 by -4878061/783501498 = -0.00622597533310651
  
  rat: replaced -0.00785163237203354 by -1241039/158061272 = -0.007851632372033549
  
  rat: replaced -0.009658502737604657 by -6031380/624463249 = -0.009658502737604659
  
  rat: replaced -0.01164440576095599 by -2532373/217475503 = -0.01164440576095598
  
  rat: replaced -0.01380714287010623 by -1331361/96425525 = -0.01380714287010623
  
  rat: replaced -0.01614449780981353 by -7953293/492631799 = -0.01614449780981353
  
  rat: replaced -0.0186542368631985 by -865030/46371771 = -0.01865423686319852
  
  rat: replaced -0.02133410907511396 by -2814913/131944249 = -0.02133410907511399
  
  rat: replaced -0.02418184647723809 by -2509632/103781653 = -0.02418184647723813
  
  rat: replaced -0.02719516431487051 by -1827823/67211324 = -0.02719516431487051
  
  rat: replaced -0.03037176127540547 by -9190485/302599672 = -0.03037176127540548
  
  rat: replaced -0.03370931971846053 by -3905653/115862706 = -0.03370931971846057
  
  rat: replaced -0.03720550590763916 by -2032371/54625544 = -0.03720550590763911
  
  rat: replaced -0.04085797024390259 by -2827822/69211025 = -0.04085797024390261
  
  rat: replaced -0.04466434750052761 by -3719233/83270734 = -0.04466434750052762
  
  rat: replaced -0.04862225705962725 by -2754536/56651751 = -0.04862225705962733
  
  rat: replaced -0.05272930315021007 by -5066672/96088355 = -0.05272930315021003
  
  rat: replaced -0.05698307508775641 by -9699307/170213822 = -0.05698307508775639
  
  rat: replaced -0.06138114751528378 by -7938451/129330443 = -0.06138114751528378
  
  rat: replaced -0.0659210806458812 by -2936449/44544916 = -0.06592108064588112
  
  rat: replaced -0.07060042050668569 by -4716201/66801316 = -0.07060042050668583
  
  rat: replaced -0.07541669918427674 by -2448749/32469586 = -0.07541669918427664
  
  rat: replaced -0.08036743507146715 by -2461511/30628214 = -0.08036743507146711
  
  rat: replaced -0.08545013311546024 by -13954421/163304848 = -0.08545013311546024
  
  rat: replaced -0.09066228506735385 by -4103116/45257143 = -0.0906622850673539
  
  rat: replaced -0.09600136973296292 by -16995415/177033047 = -0.0960013697329629
  
  rat: replaced -0.1014648532249364 by -7634177/75239620 = -0.1014648532249365
  
  rat: replaced -0.107050189216145 by -3894269/36377974 = -0.1070501892161449
  
  rat: replaced -0.1127548191943103 by -9512927/84368252 = -0.1127548191943102
  
  rat: replaced -0.1185761727178553 by -6978418/58851773 = -0.1185761727178551
  
  rat: replaced -0.1245116676729451 by -3380435/27149544 = -0.1245116676729451
  
  rat: replaced -0.1305587105316969 by -7571267/57991282 = -0.1305587105316968
  
  rat: replaced -0.136714696611531 by -8109727/59318619 = -0.136714696611531
  
  rat: replaced -0.1429770103356357 by -5513427/38561633 = -0.142977010335636
  
  rat: replaced -0.1493430254945241 by -2259975/15132779 = -0.1493430254945242
  
  rat: replaced -0.1558101055086514 by -1315594/8443573 = -0.1558101055086514
  
  rat: replaced -0.1623756036920724 by -5159837/31777169 = -0.1623756036920721
  
  rat: replaced -0.1690368635171068 by -3076049/18197504 = -0.1690368635171065
  
  rat: replaced -0.1757912188799892 by -8356449/47536214 = -0.1757912188799891
  
  rat: replaced -0.1826359943674788 by -20067867/109879036 = -0.1826359943674788
  
  rat: replaced -0.1895685055243976 by -10432363/55032153 = -0.1895685055243977
  
  rat: replaced -0.1965860591220733 by -4406725/22416264 = -0.1965860591220732
  
  rat: replaced -0.2036859534276606 by -3912367/19207839 = -0.2036859534276604
  
  rat: replaced -0.2108654784743126 by -11495573/54516145 = -0.2108654784743125
  
  rat: replaced -0.2181219163321738 by -13126833/60181174 = -0.2181219163321739
  
  rat: replaced -0.225452541380172 by -5509494/24437489 = -0.2254525413801721
  
  rat: replaced -0.232854620578578 by -20847643/89530725 = -0.2328546205785779
  
  rat: replaced -0.2403254137423072 by -9494831/39508227 = -0.2403254137423074
  
  rat: replaced -0.2478621738149347 by -6380796/25743323 = -0.2478621738149345
  
  rat: replaced -0.2554621471434013 by -34172111/133765849 = -0.2554621471434013
  
  rat: replaced -0.2631225737533733 by -13929723/52940053 = -0.2631225737533734
  
  rat: replaced -0.2708406876252407 by -56284033/207812325 = -0.2708406876252407
  
  rat: replaced -0.2786137169707144 by -23181966/83204683 = -0.2786137169707142
  
  rat: replaced -0.2864388845100037 by -11672339/40749841 = -0.2864388845100034
  
  rat: replaced -0.2943134077495424 by -9731821/33066183 = -0.2943134077495428
  
  rat: replaced -0.3022344992602357 by -9353258/30947023 = -0.3022344992602358
  
  rat: replaced -0.3101993669561991 by -31708610/102220099 = -0.3101993669561991
  
  rat: replaced -0.3182052143739678 by -9026555/28367087 = -0.318205214373968
  
  rat: replaced -0.3262492409521378 by -20870146/63969945 = -0.3262492409521378
  
  rat: replaced -0.3343286423114211 by -163875765/490163702 = -0.3343286423114211
  
  rat: replaced -0.3424406105350815 by -14035276/40986015 = -0.3424406105350813
  
  rat: replaced -0.350582334449723 by -3184915/9084642 = -0.350582334449723
  
  rat: replaced -0.3587509999064056 by -5293418/14755131 = -0.3587509999064055
  
  rat: replaced -0.3669437900620574 by -16784286/45740755 = -0.3669437900620574
  
  rat: replaced -0.3751578856611566 by -29031339/77384323 = -0.3751578856611564
  
  rat: replaced -0.3833904653176554 by -32029406/83542521 = -0.3833904653176554
  
  rat: replaced -0.3916387057971147 by -42559854/108671215 = -0.3916387057971147
  
  rat: replaced -0.3998997822990269 by -5994245/14989368 = -0.3998997822990269
  
  rat: replaced -0.4081708687392926 by -35695538/87452439 = -0.4081708687392927
  
  rat: replaced -0.416449138032827 by -10407784/24991729 = -0.4164491380328268
  
  rat: replaced -0.4247317623762659 by -20393053/48013958 = -0.4247317623762656
  
  rat: replaced -0.4330159135307439 by -16978376/39209589 = -0.4330159135307437
  
  rat: replaced -0.4412987631047152 by -13590227/30795978 = -0.4412987631047145
  
  rat: replaced -0.4495774828367916 by -27127361/60339679 = -0.4495774828367914
  
  rat: replaced -0.4578492448785652 by -14370001/31385879 = -0.4578492448785647
  
  rat: replaced -0.4661112220773918 by -24206411/51932693 = -0.4661112220773916
  
  rat: replaced -0.4743605882591027 by -11052217/23299189 = -0.474360588259102
  
  rat: replaced -0.4825945185106215 by -34783885/72076834 = -0.4825945185106216
  
  rat: replaced -0.4908101894624506 by -9304730/18957899 = -0.4908101894624505
  
  rat: replaced -0.4990047795710082 by -12351268/24751803 = -0.4990047795710074
  
  rat: replaced -0.5071754694007786 by -11299519/22279309 = -0.507175469400779
  
  rat: replaced -0.5153194419062543 by -6871877/13335179 = -0.5153194419062541
  
  rat: replaced -0.5234338827136382 by -4491460/8580759 = -0.5234338827136388
  
  rat: replaced -0.5315159804022782 by -13987981/26317141 = -0.5315159804022785
  
  rat: replaced -0.5395629267858069 by -42101104/78028163 = -0.5395629267858069
  
  rat: replaced -0.5475719171929583 by -3020462/5516101 = -0.5475719171929593
  
  rat: replaced -0.5555401507480326 by -19638186/35349715 = -0.5555401507480329
  
  rat: replaced -0.5634648306509809 by -21674756/38466919 = -0.563464830650981
  
  rat: replaced -0.5713431644570837 by -35597565/62305051 = -0.5713431644570839
  
  rat: replaced -0.5791723643561919 by -1443012/2491507 = -0.5791723643561909
  
  rat: replaced -0.5869496474515068 by -23586220/40184401 = -0.5869496474515074
  
  rat: replaced -0.5946722360378665 by -17526553/29472627 = -0.5946722360378666
  
  rat: replaced -1.501654158375457e-4 by -374996/2497219469 = -1.501654158375457e-4
  
  rat: replaced -6.013133069336513e-4 by -664019/1104281233 = -6.013133069336514e-4
  
  rat: replaced -0.001354398550541709 by -654983/483596944 = -0.001354398550541709
  
  rat: replaced -0.002410345830432092 by -1208607/501424727 = -0.002410345830432092
  
  rat: replaced -0.003770049544422824 by -471953/125184827 = -0.003770049544422824
  
  rat: replaced -0.005434373714942833 by -1223803/225196695 = -0.005434373714942841
  
  rat: replaced -0.007404151902628484 by -1775171/239753455 = -0.007404151902628473
  
  rat: replaced -0.009680187122968989 by -1826977/188733645 = -0.009680187122968986
  
  rat: replaced -0.01226325176600614 by -549289/44791464 = -0.01226325176600613
  
  rat: replaced -0.01515408751909439 by -5645196/372519691 = -0.01515408751909439
  
  rat: replaced -0.01835340529273474 by -1469077/80043838 = -0.01835340529273471
  
  rat: replaced -0.02186188514948188 by -3538485/161856353 = -0.0218618851494819
  
  rat: replaced -0.02568017623594088 by -2586227/100709083 = -0.0256801762359409
  
  rat: replaced -0.02980889671785183 by -3946092/132379673 = -0.02980889671785184
  
  rat: replaced -0.03424863371827405 by -13149221/383934177 = -0.03424863371827406
  
  rat: replaced -0.03899994325887324 by -13884089/356002800 = -0.03899994325887324
  
  rat: replaced -0.0440633502043217 by -1745785/39619888 = -0.04406335020432163
  
  rat: replaced -0.04943934820981147 by -5036973/101881865 = -0.04943934820981143
  
  rat: replaced -0.0551283996716885 by -7433459/134839013 = -0.05512839967168849
  
  rat: replaced -0.06113093568121392 by -4757027/77817016 = -0.06113093568121399
  
  rat: replaced -0.06744735598145563 by -3884855/57598329 = -0.06744735598145564
  
  rat: replaced -0.07407802892731413 by -2885255/38948863 = -0.07407802892731426
  
  rat: replaced -0.08102329144868728 by -6021225/74314742 = -0.08102329144868727
  
  rat: replaced -0.08828344901677676 by -4377003/49578976 = -0.08828344901677679
  
  rat: replaced -0.09585877561354286 by -7052907/73576018 = -0.09585877561354299
  
  rat: replaced -0.1037495137043052 by -5876631/56642492 = -0.1037495137043052
  
  rat: replaced -0.1119558742134973 by -7474079/66759150 = -0.1119558742134973
  
  rat: replaced -0.1204780365035736 by -7358791/61079938 = -0.1204780365035734
  
  rat: replaced -0.1293161483570729 by -5061354/39139381 = -0.1293161483570729
  
  rat: replaced -0.1384703259618425 by -5586207/40342268 = -0.1384703259618423
  
  rat: replaced -0.1479406538994164 by -14042248/94918115 = -0.1479406538994164
  
  rat: replaced -0.1577271851365598 by -1401295/8884296 = -0.1577271851365601
  
  rat: replaced -0.167829941019971 by -12121567/72225295 = -0.1678299410199709
  
  rat: replaced -0.178248911274147 by -15269783/85665505 = -0.178248911274147
  
  rat: replaced -0.188984054002412 by -7617649/40308422 = -0.1889840540024117
  
  rat: replaced -0.2000352956911056 by -20506971/102516763 = -0.2000352956911056
  
  rat: replaced -0.2114025312169349 by -5553806/26271237 = -0.2114025312169351
  
  rat: replaced -0.2230856238574869 by -19624843/87970003 = -0.223085623857487
  
  rat: replaced -0.2350844053048997 by -15894843/67613345 = -0.2350844053048995
  
  rat: replaced -0.2473986756826945 by -10672172/43137547 = -0.2473986756826947
  
  rat: replaced -0.2600282035657621 by -13234131/50894983 = -0.2600282035657621
  
  rat: replaced -0.2729727260035054 by -10344911/37897233 = -0.2729727260035053
  
  rat: replaced -0.286231948546134 by -21050803/73544561 = -0.2862319485461338
  
  rat: replaced -0.2998055452741104 by -5811723/19384975 = -0.2998055452741105
  
  rat: replaced -0.3136931588307395 by -10410719/33187587 = -0.3136931588307399
  
  rat: replaced -0.3278944004579047 by -32013736/97634287 = -0.3278944004579047
  
  rat: replaced -0.3424088500349452 by -8881888/25939423 = -0.3424088500349449
  
  rat: replaced -0.357236056120665 by -14764623/41330159 = -0.3572360561206648
  
  rat: replaced -0.372375535998478 by -13197485/35441332 = -0.3723755359984777
  
  rat: replaced -0.3878267757246791 by -14427400/37200629 = -0.3878267757246793
  
  rat: replaced -0.403589230179839 by -12432000/30803597 = -0.4035892301798391
  
  rat: replaced -0.419662323123314 by -8483178/20214295 = -0.4196623231233145
  
  rat: replaced -0.4360454472508704 by -62882267/144210351 = -0.4360454472508704
  
  rat: replaced -0.4527379642554148 by -20707559/45738508 = -0.4527379642554147
  
  rat: replaced -0.4697392048908241 by -17485109/37223014 = -0.4697392048908237
  
  rat: replaced -0.4870484690388687 by -31502783/64681002 = -0.4870484690388686
  
  rat: replaced -0.504665025779225 by -2755088/5459241 = -0.5046650257792247
  
  rat: replaced -0.5225881134625661 by -18908824/36183035 = -0.5225881134625661
  
  rat: replaced -0.5408169397867263 by -7900905/14609204 = -0.5408169397867262
  
  rat: replaced -0.5593506818759304 by -9811459/17540801 = -0.5593506818759303
  
  rat: replaced -0.5781884863630807 by -55545197/96067629 = -0.5781884863630807
  
  rat: replaced -0.5973294694750937 by -15924011/26658673 = -0.5973294694750936
  
  rat: replaced -0.6167727171212756 by -14518184/23538953 = -0.6167727171212756
  
  rat: replaced -0.6365172849847307 by -12285310/19300827 = -0.6365172849847315
  
  rat: replaced -0.6565621986167935 by -11361539/17304589 = -0.6565621986167947
  
  rat: replaced -0.6769064535344717 by -8579417/12674450 = -0.6769064535344729
  
  rat: replaced -0.6975490153208934 by -31186954/44709337 = -0.6975490153208938
  
  rat: replaced -0.7184888197287485 by -37348784/51982415 = -0.7184888197287487
  
  rat: replaced -0.7397247727867132 by -12753239/17240519 = -0.7397247727867126
  
  rat: replaced -0.7612557509088446 by -14688094/19294559 = -0.7612557509088443
  
  rat: replaced -0.7830806010069399 by -11512064/14700995 = -0.7830806010069387
  
  rat: replaced -0.8051981406058428 by -15691583/19487853 = -0.805198140605843
  
  rat: replaced -0.8276071579616919 by -21531267/26016289 = -0.8276071579616908
  
  rat: replaced -0.8503064121830922 by -19412384/22829869 = -0.8503064121830922
  
  rat: replaced -0.8732946333552043 by -19396479/22210693 = -0.8732946333552042
  
  rat: replaced -0.8965705226667342 by -45044189/50240542 = -0.896570522666734
  
  rat: replaced -0.9201327525398142 by -6632592/7208299 = -0.9201327525398155
  
  rat: replaced -0.9439799667627587 by -19068047/20199631 = -0.9439799667627592
  
  rat: replaced -0.9681107806256851 by -17605779/18185707 = -0.9681107806256859
  
  rat: replaced -0.9925237810589822 by -24449409/24633575 = -0.9925237810589815
  
  rat: replaced -1.017217526774618 by -31992660/31451149 = -1.017217526774618
  
  rat: replaced -1.042190548410265 by -17752654/17033981 = -1.042190548410263
  
  rat: replaced -1.067441348676237 by -34132828/31976303 = -1.067441348676237
  
  rat: replaced -1.092968402505218 by -17060687/15609497 = -1.092968402505218
  
  rat: replaced -1.118770157204762 by -23160047/20701345 = -1.118770157204761
  
  rat: replaced -1.144845032612569 by -13646839/11920250 = -1.144845032612571
  
  rat: replaced -1.171191421254493 by -12158144/10381005 = -1.171191421254493
  
  rat: replaced -1.197807688505292 by -18610041/15536752 = -1.197807688505294
  
  rat: replaced -1.224692172752087 by -1607117/1312262 = -1.224692172752088
  
  rat: replaced -1.251843185560525 by -23622341/18870048 = -1.251843185560524
  
  rat: replaced -1.279259011843616 by -11838497/9254183 = -1.279259011843617
  
  rat: replaced -1.306937910033247 by -16769881/12831429 = -1.306937910033247
  
  rat: replaced -1.33487811225433 by -18675990/13990783 = -1.334878112254332
  
  rat: replaced -1.363077824501593 by -60815834/44616553 = -1.363077824501592
  
  rat: replaced -1.391535226818977 by -16386165/11775602 = -1.391535226818977
  
  rat: replaced -1.420248473481634 by -17316077/12192287 = -1.420248473481636
  
  rat: replaced -1.449215693180489 by -19585953/13514864 = -1.449215693180486
  
  rat: replaced -1.478434989209379 by -157494279/106527700 = -1.478434989209379
  
  rat: replaced -1.507904439654719 by -39585536/26252019 = -1.507904439654718
  part: invalid index of list or matrix.
  #0: lineIntersection(g=[2,-2,0],h=[-1,-3,-7])
  #1: circleThrough(a=[2,0],b=[0,2],c=[3,3])
   -- an error. To debug this try: debugmode(true);
  
  Error in:
  cc &= circleThrough(A,B,C); $cc // (titik pusat dan jari-jari) ...
                            ^
\end{euleroutput}
\begin{eulerprompt}
>r&=getCircleRadius(cc); $r , $float(r) // tampilkan nilai jari-jari
>$computeAngle(A,C,B) // nilai <ACB
>$solve(getLineEquation(angleBisector(A,C,B),x,y),y)[1] // persamaan garis bagi <ACB
\end{eulerprompt}
\begin{euleroutput}
  Maxima said:
  solve: all variables must not be numbers.
   -- an error. To debug this try: debugmode(true);
  
  Error in:
  ... (getLineEquation(angleBisector(A,C,B),x,y),y)[1] // persamaan  ...
                                                       ^
\end{euleroutput}
\begin{eulerprompt}
>P &= lineIntersection(angleBisector(A,C,B),angleBisector(C,B,A)); $P // titik potong 2
\end{eulerprompt}
\begin{euleroutput}
  Maxima said:
  rat: replaced -9.96658350028018e-5 by -86001/862893488 = -9.966583500280164e-5
  
  rat: replaced -3.973200535106469e-4 by -1080775/2720162223 = -3.973200535106468e-4
  
  rat: replaced -8.90932907016237e-4 by -1194571/1340809157 = -8.90932907016237e-4
  
  rat: replaced -0.001578455051312488 by -2522953/1598368606 = -0.001578455051312488
  
  rat: replaced -0.002457817751424091 by -519814/211494119 = -0.002457817751424095
  
  rat: replaced -0.003526933088480816 by -11813191/3349423055 = -0.003526933088480816
  
  rat: replaced -0.004783694168506353 by -866601/181157275 = -0.004783694168506343
  
  rat: replaced -0.006225975333106509 by -4878061/783501498 = -0.00622597533310651
  
  rat: replaced -0.00785163237203354 by -1241039/158061272 = -0.007851632372033549
  
  rat: replaced -0.009658502737604657 by -6031380/624463249 = -0.009658502737604659
  
  rat: replaced -0.01164440576095599 by -2532373/217475503 = -0.01164440576095598
  
  rat: replaced -0.01380714287010623 by -1331361/96425525 = -0.01380714287010623
  
  rat: replaced -0.01614449780981353 by -7953293/492631799 = -0.01614449780981353
  
  rat: replaced -0.0186542368631985 by -865030/46371771 = -0.01865423686319852
  
  rat: replaced -0.02133410907511396 by -2814913/131944249 = -0.02133410907511399
  
  rat: replaced -0.02418184647723809 by -2509632/103781653 = -0.02418184647723813
  
  rat: replaced -0.02719516431487051 by -1827823/67211324 = -0.02719516431487051
  
  rat: replaced -0.03037176127540547 by -9190485/302599672 = -0.03037176127540548
  
  rat: replaced -0.03370931971846053 by -3905653/115862706 = -0.03370931971846057
  
  rat: replaced -0.03720550590763916 by -2032371/54625544 = -0.03720550590763911
  
  rat: replaced -0.04085797024390259 by -2827822/69211025 = -0.04085797024390261
  
  rat: replaced -0.04466434750052761 by -3719233/83270734 = -0.04466434750052762
  
  rat: replaced -0.04862225705962725 by -2754536/56651751 = -0.04862225705962733
  
  rat: replaced -0.05272930315021007 by -5066672/96088355 = -0.05272930315021003
  
  rat: replaced -0.05698307508775641 by -9699307/170213822 = -0.05698307508775639
  
  rat: replaced -0.06138114751528378 by -7938451/129330443 = -0.06138114751528378
  
  rat: replaced -0.0659210806458812 by -2936449/44544916 = -0.06592108064588112
  
  rat: replaced -0.07060042050668569 by -4716201/66801316 = -0.07060042050668583
  
  rat: replaced -0.07541669918427674 by -2448749/32469586 = -0.07541669918427664
  
  rat: replaced -0.08036743507146715 by -2461511/30628214 = -0.08036743507146711
  
  rat: replaced -0.08545013311546024 by -13954421/163304848 = -0.08545013311546024
  
  rat: replaced -0.09066228506735385 by -4103116/45257143 = -0.0906622850673539
  
  rat: replaced -0.09600136973296292 by -16995415/177033047 = -0.0960013697329629
  
  rat: replaced -0.1014648532249364 by -7634177/75239620 = -0.1014648532249365
  
  rat: replaced -0.107050189216145 by -3894269/36377974 = -0.1070501892161449
  
  rat: replaced -0.1127548191943103 by -9512927/84368252 = -0.1127548191943102
  
  rat: replaced -0.1185761727178553 by -6978418/58851773 = -0.1185761727178551
  
  rat: replaced -0.1245116676729451 by -3380435/27149544 = -0.1245116676729451
  
  rat: replaced -0.1305587105316969 by -7571267/57991282 = -0.1305587105316968
  
  rat: replaced -0.136714696611531 by -8109727/59318619 = -0.136714696611531
  
  rat: replaced -0.1429770103356357 by -5513427/38561633 = -0.142977010335636
  
  rat: replaced -0.1493430254945241 by -2259975/15132779 = -0.1493430254945242
  
  rat: replaced -0.1558101055086514 by -1315594/8443573 = -0.1558101055086514
  
  rat: replaced -0.1623756036920724 by -5159837/31777169 = -0.1623756036920721
  
  rat: replaced -0.1690368635171068 by -3076049/18197504 = -0.1690368635171065
  
  rat: replaced -0.1757912188799892 by -8356449/47536214 = -0.1757912188799891
  
  rat: replaced -0.1826359943674788 by -20067867/109879036 = -0.1826359943674788
  
  rat: replaced -0.1895685055243976 by -10432363/55032153 = -0.1895685055243977
  
  rat: replaced -0.1965860591220733 by -4406725/22416264 = -0.1965860591220732
  
  rat: replaced -0.2036859534276606 by -3912367/19207839 = -0.2036859534276604
  
  rat: replaced -0.2108654784743126 by -11495573/54516145 = -0.2108654784743125
  
  rat: replaced -0.2181219163321738 by -13126833/60181174 = -0.2181219163321739
  
  rat: replaced -0.225452541380172 by -5509494/24437489 = -0.2254525413801721
  
  rat: replaced -0.232854620578578 by -20847643/89530725 = -0.2328546205785779
  
  rat: replaced -0.2403254137423072 by -9494831/39508227 = -0.2403254137423074
  
  rat: replaced -0.2478621738149347 by -6380796/25743323 = -0.2478621738149345
  
  rat: replaced -0.2554621471434013 by -34172111/133765849 = -0.2554621471434013
  
  rat: replaced -0.2631225737533733 by -13929723/52940053 = -0.2631225737533734
  
  rat: replaced -0.2708406876252407 by -56284033/207812325 = -0.2708406876252407
  
  rat: replaced -0.2786137169707144 by -23181966/83204683 = -0.2786137169707142
  
  rat: replaced -0.2864388845100037 by -11672339/40749841 = -0.2864388845100034
  
  rat: replaced -0.2943134077495424 by -9731821/33066183 = -0.2943134077495428
  
  rat: replaced -0.3022344992602357 by -9353258/30947023 = -0.3022344992602358
  
  rat: replaced -0.3101993669561991 by -31708610/102220099 = -0.3101993669561991
  
  rat: replaced -0.3182052143739678 by -9026555/28367087 = -0.318205214373968
  
  rat: replaced -0.3262492409521378 by -20870146/63969945 = -0.3262492409521378
  
  rat: replaced -0.3343286423114211 by -163875765/490163702 = -0.3343286423114211
  
  rat: replaced -0.3424406105350815 by -14035276/40986015 = -0.3424406105350813
  
  rat: replaced -0.350582334449723 by -3184915/9084642 = -0.350582334449723
  
  rat: replaced -0.3587509999064056 by -5293418/14755131 = -0.3587509999064055
  
  rat: replaced -0.3669437900620574 by -16784286/45740755 = -0.3669437900620574
  
  rat: replaced -0.3751578856611566 by -29031339/77384323 = -0.3751578856611564
  
  rat: replaced -0.3833904653176554 by -32029406/83542521 = -0.3833904653176554
  
  rat: replaced -0.3916387057971147 by -42559854/108671215 = -0.3916387057971147
  
  rat: replaced -0.3998997822990269 by -5994245/14989368 = -0.3998997822990269
  
  rat: replaced -0.4081708687392926 by -35695538/87452439 = -0.4081708687392927
  
  rat: replaced -0.416449138032827 by -10407784/24991729 = -0.4164491380328268
  
  rat: replaced -0.4247317623762659 by -20393053/48013958 = -0.4247317623762656
  
  rat: replaced -0.4330159135307439 by -16978376/39209589 = -0.4330159135307437
  
  rat: replaced -0.4412987631047152 by -13590227/30795978 = -0.4412987631047145
  
  rat: replaced -0.4495774828367916 by -27127361/60339679 = -0.4495774828367914
  
  rat: replaced -0.4578492448785652 by -14370001/31385879 = -0.4578492448785647
  
  rat: replaced -0.4661112220773918 by -24206411/51932693 = -0.4661112220773916
  
  rat: replaced -0.4743605882591027 by -11052217/23299189 = -0.474360588259102
  
  rat: replaced -0.4825945185106215 by -34783885/72076834 = -0.4825945185106216
  
  rat: replaced -0.4908101894624506 by -9304730/18957899 = -0.4908101894624505
  
  rat: replaced -0.4990047795710082 by -12351268/24751803 = -0.4990047795710074
  
  rat: replaced -0.5071754694007786 by -11299519/22279309 = -0.507175469400779
  
  rat: replaced -0.5153194419062543 by -6871877/13335179 = -0.5153194419062541
  
  rat: replaced -0.5234338827136382 by -4491460/8580759 = -0.5234338827136388
  
  rat: replaced -0.5315159804022782 by -13987981/26317141 = -0.5315159804022785
  
  rat: replaced -0.5395629267858069 by -42101104/78028163 = -0.5395629267858069
  
  rat: replaced -0.5475719171929583 by -3020462/5516101 = -0.5475719171929593
  
  rat: replaced -0.5555401507480326 by -19638186/35349715 = -0.5555401507480329
  
  rat: replaced -0.5634648306509809 by -21674756/38466919 = -0.563464830650981
  
  rat: replaced -0.5713431644570837 by -35597565/62305051 = -0.5713431644570839
  
  rat: replaced -0.5791723643561919 by -1443012/2491507 = -0.5791723643561909
  
  rat: replaced -0.5869496474515068 by -23586220/40184401 = -0.5869496474515074
  
  rat: replaced -0.5946722360378665 by -17526553/29472627 = -0.5946722360378666
  
  rat: replaced 1.66665833335744e-7 by 15819/94914474571 = 1.66665833335744e-7
  
  rat: replaced 4.999958333473664e-5 by 201389/4027813565 = 4.99995833347366e-5
  
  rat: replaced 1.33330666692022e-6 by 31771/23828726570 = 1.333306666920221e-6
  
  rat: replaced 1.999933334222437e-4 by 200030/1000183339 = 1.999933334222437e-4
  
  rat: replaced 4.499797504338432e-6 by 24036/5341573699 = 4.499797504338431e-6
  
  rat: replaced 4.499662510124569e-4 by 1162901/2584418270 = 4.499662510124571e-4
  
  rat: replaced 1.066581336583994e-5 by 58861/5518660226 = 1.066581336583993e-5
  
  rat: replaced 7.998933390220841e-4 by 1137431/1421978337 = 7.998933390220838e-4
  
  rat: replaced 2.083072932167196e-5 by 35635/1710693824 = 2.0830729321672e-5
  
  rat: replaced 0.001249739605033717 by 567943/454449069 = 0.001249739605033716
  
  rat: replaced 3.599352055540239e-5 by 98277/2730408098 = 3.599352055540234e-5
  
  rat: replaced 0.00179946006479581 by 479561/266502719 = 0.001799460064795812
  
  rat: replaced 5.71526624672386e-5 by 51154/895041417 = 5.715266246723866e-5
  
  rat: replaced 0.002448999746720415 by 1946227/794702818 = 0.002448999746720415
  
  rat: replaced 8.530603082730626e-5 by 121691/1426522824 = 8.530603082730627e-5
  
  rat: replaced 0.003198293697380561 by 2986741/933854512 = 0.003198293697380562
  
  rat: replaced 1.214508019889565e-4 by 158455/1304684674 = 1.214508019889563e-4
  
  rat: replaced 0.004047266988005727 by 2125334/525128193 = 0.004047266988005727
  
  rat: replaced 1.665833531718508e-4 by 142521/855553675 = 1.66583353171851e-4
  
  rat: replaced 0.004995834721974179 by 1957223/391770967 = 0.004995834721974179
  
  rat: replaced 2.216991628251896e-4 by 179571/809975995 = 2.216991628251896e-4
  
  rat: replaced 0.006043902043303184 by 1800665/297930871 = 0.006043902043303193
  
  rat: replaced 2.877927110806339e-4 by 1167733/4057548906 = 2.877927110806339e-4
  
  rat: replaced 0.00719136414613375 by 2476362/344352191 = 0.007191364146133747
  
  rat: replaced 3.658573803051457e-4 by 386279/1055818526 = 3.658573803051454e-4
  
  rat: replaced 0.00843810628521191 by 2079855/246483622 = 0.008438106285211924
  
  rat: replaced 4.5688535576352e-4 by 262978/575588595 = 4.568853557635206e-4
  
  rat: replaced 0.009784003787362772 by 1752551/179124113 = 0.009784003787362787
  
  rat: replaced 5.618675264007778e-4 by 150595/268025812 = 5.618675264007782e-4
  
  rat: replaced 0.01122892206395776 by 5450241/485375263 = 0.01122892206395776
  
  rat: replaced 6.817933857540259e-4 by 192316/282073725 = 6.817933857540258e-4
  
  rat: replaced 0.01277271662437307 by 3258991/255152533 = 0.01277271662437308
  
  rat: replaced 8.176509330039827e-4 by 105841/129445214 = 8.176509330039812e-4
  
  rat: replaced 0.01441523309043924 by 2330472/161667313 = 0.01441523309043925
  
  rat: replaced 9.704265741758145e-4 by 651321/671169790 = 9.704265741758132e-4
  
  rat: replaced 0.01615630721187855 by 19391318/1200232067 = 0.01615630721187855
  
  rat: replaced 0.001141105023499428 by 1259907/1104111343 = 0.001141105023499428
  
  rat: replaced 0.01799576488272969 by 4765614/264818641 = 0.01799576488272969
  
  rat: replaced 0.001330669204938795 by 1231154/925214167 = 0.001330669204938796
  
  rat: replaced 0.01993342215875837 by 2504519/125644206 = 0.01993342215875836
  
  rat: replaced 0.001540100153900437 by 276884/179783113 = 0.001540100153900439
  
  rat: replaced 0.02196908527585173 by 1298306/59096953 = 0.0219690852758517
  
  rat: replaced 0.001770376919130678 by 644389/363984072 = 0.001770376919130681
  
  rat: replaced 0.02410255066939448 by 2001286/83032125 = 0.02410255066939453
  
  rat: replaced 0.002022476464811601 by 1271955/628909667 = 0.002022476464811599
  
  rat: replaced 0.02633360499462523 by 2978115/113091808 = 0.02633360499462525
  
  rat: replaced 0.002297373572865413 by 1020913/444382669 = 0.002297373572865417
  
  rat: replaced 0.02866202514797045 by 1770713/61779061 = 0.02866202514797044
  
  rat: replaced 0.002596040745477063 by 1097643/422814242 = 0.002596040745477065
  
  rat: replaced 0.03108757828935527 by 5034207/161936287 = 0.03108757828935525
  
  rat: replaced 0.002919448107844891 by 906221/310408326 = 0.002919448107844891
  
  rat: replaced 0.03361002186548678 by 4553215/135471944 = 0.03361002186548678
  
  rat: replaced 0.003268563311168871 by 1379071/421919623 = 0.003268563311168867
  
  rat: replaced 0.03622910363410947 by 3082649/85087642 = 0.0362291036341094
  
  rat: replaced 0.003644351435886262 by 5966577/1637212301 = 0.003644351435886261
  
  rat: replaced 0.03894456168922911 by 4913415/126164342 = 0.03894456168922911
  
  rat: replaced 0.004047774895164447 by 572425/141417202 = 0.004047774895164451
  
  rat: replaced 0.04175612448730281 by 1734727/41544253 = 0.04175612448730273
  
  rat: replaced 0.004479793338660443 by 2952779/659132861 = 0.004479793338660444
  
  rat: replaced 0.04466351087439402 by 4691119/105032473 = 0.04466351087439405
  
  rat: replaced 0.0049413635565565 by 2524919/510976165 = 0.004941363556556498
  
  rat: replaced 0.04766643011428662 by 3536207/74186529 = 0.04766643011428665
  
  rat: replaced 0.005433439383882244 by 1361584/250593391 = 0.005433439383882235
  
  rat: replaced 0.05076458191755917 by 7710025/151878036 = 0.05076458191755916
  
  rat: replaced 0.005956971605131645 by 1447422/242979503 = 0.005956971605131648
  
  rat: replaced 0.0539576564716131 by 3377975/62604183 = 0.05395765647161309
  
  rat: replaced 0.006512907859185624 by 3695063/567344584 = 0.006512907859185626
  
  rat: replaced 0.05724533447165381 by 2560865/44734912 = 0.05724533447165382
  
  rat: replaced 0.007102192544548636 by 1363981/192050693 = 0.007102192544548642
  
  rat: replaced 0.06062728715262111 by 8274761/136485754 = 0.06062728715262107
  
  rat: replaced 0.007725766724910044 by 1464384/189545459 = 0.007725766724910038
  
  rat: replaced 0.06410317632206519 by 5287663/82486755 = 0.06410317632206528
  
  rat: replaced 0.00838456803503801 by 1113589/132814117 = 0.008384568035038023
  
  rat: replaced 0.06767265439396564 by 2921400/43169579 = 0.06767265439396572
  
  rat: replaced 0.009079530587017326 by 433906/47789475 = 0.00907953058701733
  
  rat: replaced 0.07133536442348987 by 7236103/101437808 = 0.07133536442348991
  
  rat: replaced 0.009811584876838586 by 1363090/138926587 = 0.009811584876838586
  
  rat: replaced 0.07509094014268702 by 9209133/122639735 = 0.07509094014268704
  
  rat: replaced 0.0105816576913495 by 1163729/109976058 = 0.01058165769134951
  
  rat: replaced 0.07893900599711501 by 5197067/65836489 = 0.07893900599711506
  
  rat: replaced 0.01139067201557714 by 13426050/1178688139 = 0.01139067201557714
  
  rat: replaced 0.08287917718339499 by 11217158/135343501 = 0.082879177183395
  
  rat: replaced 0.01223954694042984 by 2283101/186534764 = 0.01223954694042983
  
  rat: replaced 0.08691105968769186 by 5213115/59982182 = 0.08691105968769192
  
  rat: replaced 0.01312919757078923 by 3499615/266552086 = 0.01312919757078922
  
  rat: replaced 0.09103425032511492 by 5893225/64736349 = 0.09103425032511488
  
  rat: replaced 0.01406053493400045 by 2280713/162206702 = 0.01406053493400045
  
  rat: replaced 0.09524833678003664 by 9601787/100807923 = 0.09524833678003662
  
  rat: replaced 0.01503446588876983 by 200490/13335359 = 0.01503446588876985
  
  rat: replaced 0.09955289764732322 by 5687088/57126293 = 0.09955289764732328
  
  rat: replaced 0.01605189303448024 by 951971/59305840 = 0.01605189303448025
  
  rat: replaced 0.1039475024744748 by 10260011/98703776 = 0.1039475024744747
  
  rat: replaced 0.01711371462093175 by 9432386/551159477 = 0.01711371462093176
  
  rat: replaced 0.1084317118046711 by 14939691/137779721 = 0.1084317118046712
  
  rat: replaced 0.01822082445851714 by 2559788/140486947 = 0.01822082445851713
  
  rat: replaced 0.113005077220716 by 8478529/75027859 = 0.1130050772207161
  
  rat: replaced 0.01937411182884202 by 2983799/154009589 = 0.01937411182884203
  
  rat: replaced 0.1176671413898787 by 7123715/60541243 = 0.1176671413898786
  
  rat: replaced 0.02057446139579705 by 7167743/348380590 = 0.02057446139579705
  
  rat: replaced 0.1224174381096274 by 12172179/99431741 = 0.1224174381096274
  
  rat: replaced 0.02182275311709253 by 7415562/339808729 = 0.02182275311709253
  
  rat: replaced 0.1272554923542488 by 7277933/57191504 = 0.127255492354249
  
  rat: replaced 0.02311986215626333 by 2988661/129268115 = 0.02311986215626336
  
  rat: replaced 0.1321808203223502 by 3633064/27485561 = 0.1321808203223503
  
  rat: replaced 0.02446665879515308 by 1991976/81415939 = 0.02446665879515312
  
  rat: replaced 0.1371929294852391 by 56235017/409897341 = 0.1371929294852391
  
  rat: replaced 0.02586400834688696 by 5000736/193347293 = 0.02586400834688697
  
  rat: replaced 0.1422913186361759 by 9349741/65708443 = 0.1422913186361759
  
  rat: replaced 0.02731277106934082 by 858413/31428997 = 0.02731277106934084
  
  rat: replaced 0.1474754779404944 by 1549881/10509415 = 0.1474754779404943
  
  rat: replaced 0.02881380207911666 by 3754753/130310918 = 0.02881380207911666
  
  rat: replaced 0.152744888986584 by 5264425/34465474 = 0.1527448889865841
  
  rat: replaced 0.03036795126603076 by 4118329/135614318 = 0.03036795126603077
  
  rat: replaced 0.1580990248377314 by 5442776/34426373 = 0.1580990248377312
  
  rat: replaced 0.03197606320812652 by 3497683/109384416 = 0.03197606320812647
  
  rat: replaced 0.1635373500848132 by 12328488/75386375 = 0.1635373500848131
  
  rat: replaced 0.0336389770872163 by 3971799/118071337 = 0.03363897708721635
  
  rat: replaced 0.1690593208998367 by 20896917/123607009 = 0.1690593208998367
  
  rat: replaced 0.03535752660496472 by 1815732/51353479 = 0.03535752660496478
  
  rat: replaced 0.1746643850903219 by 2841592/16268869 = 0.1746643850903219
  
  rat: replaced 0.03713253989951881 by 3333721/89778965 = 0.03713253989951878
  
  rat: replaced 0.1803519821545206 by 4461007/24735004 = 0.1803519821545208
  
  rat: replaced 0.03896483946269502 by 8785771/225479461 = 0.03896483946269501
  
  rat: replaced 0.1861215433374662 by 4381209/23539505 = 0.1861215433374661
  
  rat: replaced 0.0408552420577305 by 3189084/78058135 = 0.04085524205773043
  
  rat: replaced 0.1919724916878484 by 72809759/379271834 = 0.1919724916878484
  
  rat: replaced 0.04280455863760801 by 7646593/178639688 = 0.04280455863760801
  
  rat: replaced 0.1979042421157076 by 26318167/132984350 = 0.1979042421157076
  
  rat: replaced 0.04481359426396048 by 20610430/459914683 = 0.04481359426396048
  
  rat: replaced 0.2039162014509444 by 8519416/41779005 = 0.2039162014509441
  
  rat: replaced 0.04688314802656623 by 3439140/73355569 = 0.04688314802656633
  
  rat: replaced 0.2100077685026351 by 50962787/242670961 = 0.2100077685026351
  
  rat: replaced 0.04901401296344043 by 4006732/81746663 = 0.04901401296344048
  
  rat: replaced 0.216178334119151 by 1347531/6233423 = 0.2161783341191509
  
  rat: replaced 0.05120697598153157 by 4148974/81023609 = 0.0512069759815315
  
  rat: replaced 0.2224272812490723 by 23234851/104460437 = 0.2224272812490723
  
  rat: replaced 0.05346281777803219 by 11998448/224426031 = 0.05346281777803218
  
  rat: replaced 0.2287539850028937 by 8185268/35781969 = 0.2287539850028935
  
  rat: replaced 0.05578231276230905 by 1398019/25062048 = 0.05578231276230897
  
  rat: replaced 0.2351578127155118 by 12642104/53760085 = 0.2351578127155119
  
  rat: replaced 0.05816622897846346 by 4451048/76522891 = 0.05816622897846345
  
  rat: replaced 0.2416381240094921 by 8002142/33116223 = 0.2416381240094923
  
  rat: replaced 0.06061532802852698 by 2146337/35409146 = 0.06061532802852686
  
  rat: replaced 0.2481942708591053 by 8882901/35790113 = 0.2481942708591057
  
  rat: replaced 0.0631303649963022 by 14651447/232082406 = 0.06313036499630222
  
  rat: replaced 0.2548255976551299 by 868346/3407609 = 0.25482559765513
  
  rat: replaced 0.06571208837185505 by 4240309/64528599 = 0.06571208837185509
  
  rat: replaced 0.2615314412704124 by 8212450/31401387 = 0.2615314412704127
  
  rat: replaced 0.06836123997666599 by 2716643/39739522 = 0.06836123997666604
  
  rat: replaced 0.2683111311261794 by 34459769/128432126 = 0.2683111311261794
  
  rat: replaced 0.07107855488944881 by 3146673/44270357 = 0.07107855488944893
  
  rat: replaced 0.2751639892590951 by 12552159/45617012 = 0.2751639892590949
  
  rat: replaced 0.07386476137264342 by 12898997/174629915 = 0.0738647613726434
  
  rat: replaced 0.2820893303890569 by 11134456/39471383 = 0.2820893303890568
  
  rat: replaced 0.07672058079958999 by 5073506/66129661 = 0.07672058079959007
  
  rat: replaced 0.2890864619877229 by 9583357/33150487 = 0.2890864619877228
  
  rat: replaced 0.07964672758239233 by 5672399/71219486 = 0.07964672758239227
  
  rat: replaced 0.2961546843477643 by 11052271/37319251 = 0.2961546843477647
  
  rat: replaced 0.08264390910047736 by 4686067/56701904 = 0.08264390910047748
  
  rat: replaced 0.3032932906528349 by 9918077/32701274 = 0.3032932906528351
  
  rat: replaced 0.0857128256298576 by 3585977/41837111 = 0.08571282562985766
  
  rat: replaced 0.3105015670482534 by 9320011/30015987 = 0.3105015670482533
  
  rat: replaced 0.08885417027310427 by 5751353/64728003 = 0.0888541702731042
  
  rat: replaced 0.3177787927123868 by 248395525/781661743 = 0.3177787927123868
  
  rat: replaced 0.09206862889003742 by 7305460/79347983 = 0.09206862889003745
  
  rat: replaced 0.3251242399287333 by 13842845/42577093 = 0.3251242399287335
  
  rat: replaced 0.09535688002914089 by 5971998/62627867 = 0.09535688002914103
  
  rat: replaced 0.3325371741586922 by 9318229/28021616 = 0.3325371741586923
  
  rat: replaced 0.0987195948597075 by 9821211/99485933 = 0.09871959485970745
  
  rat: replaced 0.3400168541150183 by 13391981/39386227 = 0.3400168541150184
  
  rat: replaced 0.1021574371047232 by 8336413/81603584 = 0.1021574371047232
  
  rat: replaced 0.3475625318359485 by 10097818/29053241 = 0.347562531835949
  
  rat: replaced 0.1056710629744951 by 5741011/54329074 = 0.105671062974495
  
  rat: replaced 0.3551734527599992 by 15867851/44676343 = 0.3551734527599987
  
  rat: replaced 0.1092611211010309 by 5551873/50812887 = 0.1092611211010309
  
  rat: replaced 0.3628488558014202 by 6897641/19009681 = 0.3628488558014203
  
  rat: replaced 0.1129282524731764 by 11548693/102265755 = 0.1129282524731764
  
  rat: replaced 0.3705879734263036 by 23358661/63031352 = 0.3705879734263038
  
  rat: replaced 0.1166730903725168 by 5656228/48479285 = 0.1166730903725168
  
  rat: replaced 0.3783900317293359 by 14241382/37636779 = 0.3783900317293358
  
  rat: replaced 0.1204962603100498 by 4057613/33674182 = 0.12049626031005
  
  rat: replaced 0.3862542505111889 by 3461217/8960981 = 0.3862542505111884
  
  rat: replaced 0.1243983799636342 by 7966447/64039797 = 0.1243983799636342
  
  rat: replaced 0.3941798433565377 by 5314214/13481699 = 0.3941798433565384
  
  rat: replaced 0.1283800591162231 by 796346/6203035 = 0.1283800591162229
  
  rat: replaced 0.4021660177127022 by 11567173/28762184 = 0.4021660177127022
  
  rat: replaced 0.1324418995948859 by 4716124/35609003 = 0.1324418995948862
  
  rat: replaced 0.4102119749689023 by 11320633/27597032 = 0.4102119749689024
  
  rat: replaced 0.1365844952106265 by 612971/4487852 = 0.1365844952106264
  
  rat: replaced 0.418316910536117 by 12225195/29224721 = 0.4183169105361177
  
  rat: replaced 0.140808431699002 by 10431632/74083859 = 0.1408084316990021
  
  rat: replaced 0.4264800139275439 by 7978696/18708253 = 0.4264800139275431
  
  rat: replaced 0.1451142866615502 by 3554077/24491572 = 0.1451142866615504
  
  rat: replaced 0.4347004688396462 by 20489554/47134879 = 0.4347004688396463
  
  rat: replaced 0.1495026295080298 by 26759297/178988805 = 0.1495026295080298
  
  rat: replaced 0.4429774532337832 by 23449796/52936771 = 0.4429774532337834
  
  rat: replaced 0.1539740213994798 by 16145763/104860306 = 0.1539740213994798
  
  rat: replaced 0.451310139418413 by 8841241/19590167 = 0.4513101394184133
  part: invalid index of list or matrix.
  #0: lineIntersection(g=[2,-2,0],h=[3-sqrt(10)/sqrt(2),1+sqrt(10)/sqrt(2),(5-sqrt(10)/sqrt(2))*(1+sqrt(10)/sqrt(2))/2+(3-sqrt(10)/sqrt(...)
   -- an error. To debug this try: debugmode(true);
  
  Error in:
  ... ection(angleBisector(A,C,B),angleBisector(C,B,A)); $P // titik ...
                                                       ^
\end{euleroutput}
\begin{eulerprompt}
>P() // 
\end{eulerprompt}
\begin{euleroutput}
  0
\end{euleroutput}
\eulersubheading{Garis dan Lingkaran yang berpotongan}
\begin{eulerprompt}
>A &:= [2,0]; c=circleWithCenter(A,4);
>B &:= [2,3]; C &:= [3,2]; l=lineThrough(B,C);
>setPlotRange(5); plotCircle(c); plotLine(l);
>\{P1,P2,f\}=lineCircleIntersections(l,c);
>P1, P2,
\end{eulerprompt}
\begin{euleroutput}
  [5.89792,  -0.897916]
  [1.10208,  3.89792]
\end{euleroutput}
\begin{eulerprompt}
>plotPoint(P1); plotPoint(P2):
\end{eulerprompt}
\begin{eulercomment}
\end{eulercomment}
\begin{eulerprompt}
>c &= circleWithCenter(A,4) // lingkaran dengan pusat A jari-jari 4
\end{eulerprompt}
\begin{euleroutput}
  
                                [2, 0, 4]
  
\end{euleroutput}
\begin{eulerprompt}
>l &= lineThrough(B,C) // garis l melalui B dan C
\end{eulerprompt}
\begin{euleroutput}
  
                                [1, 1, 5]
  
\end{euleroutput}
\begin{eulerprompt}
>$lineCircleIntersections(l,c) | radcan, // titik potong lingkaran c dan garis l
\end{eulerprompt}
\begin{euleroutput}
  Maxima said:
  rat: replaced -4.98329175014009e-5 by -86001/1725786976 = -4.983291750140082e-5
  
  rat: replaced -1.986600267553235e-4 by -1133306/5704751069 = -1.986600267553234e-4
  
  rat: replaced -4.454664535081185e-4 by -474290/1064704191 = -4.454664535081181e-4
  
  rat: replaced -7.892275256562442e-4 by -1190199/1508055613 = -7.892275256562439e-4
  
  rat: replaced -0.001228908875712045 by -259907/211494119 = -0.001228908875712047
  
  rat: replaced -0.001763466544240408 by -5854594/3319934829 = -0.001763466544240408
  
  rat: replaced -0.002391847084253176 by -866601/362314550 = -0.002391847084253172
  
  rat: replaced -0.003112987666553255 by -5049204/1621980085 = -0.003112987666553255
  
  rat: replaced -0.00392581618601677 by -1241039/316122544 = -0.003925816186016774
  
  rat: replaced -0.004829251368802329 by -3015690/624463249 = -0.00482925136880233
  
  rat: replaced -0.005822202880477995 by -2532373/434951006 = -0.005822202880477991
  
  rat: replaced -0.006903571435053116 by -1331361/192851050 = -0.006903571435053115
  
  rat: replaced -0.008072248904906765 by -7953293/985263598 = -0.008072248904906766
  
  rat: replaced -0.009327118431599252 by -432515/46371771 = -0.009327118431599259
  
  rat: replaced -0.01066705453755698 by -2950074/276559381 = -0.01066705453755698
  
  rat: replaced -0.01209092323861904 by -1254816/103781653 = -0.01209092323861907
  
  rat: replaced -0.01359758215743526 by -1827823/134422648 = -0.01359758215743526
  
  rat: replaced -0.01518588063770274 by -9199276/605778237 = -0.01518588063770274
  
  rat: replaced -0.01685465985923026 by -2516580/149310637 = -0.01685465985923026
  
  rat: replaced -0.01860275295381958 by -2032371/109251088 = -0.01860275295381955
  
  rat: replaced -0.02042898512195129 by -1413911/69211025 = -0.02042898512195131
  
  rat: replaced -0.02233217375026381 by -3647892/163346929 = -0.02233217375026377
  
  rat: replaced -0.02431112852981362 by -1377268/56651751 = -0.02431112852981367
  
  rat: replaced -0.02636465157510504 by -2533336/96088355 = -0.02636465157510502
  
  rat: replaced -0.0284915375438782 by -9699307/340427644 = -0.02849153754387819
  
  rat: replaced -0.03069057375764189 by -7938451/258660886 = -0.03069057375764189
  
  rat: replaced -0.0329605403229406 by -2936449/89089832 = -0.03296054032294056
  
  rat: replaced -0.03530021025334285 by -5224432/148000025 = -0.03530021025334287
  
  rat: replaced -0.03770834959213837 by -2448749/64939172 = -0.03770834959213832
  
  rat: replaced -0.04018371753573358 by -2461511/61256428 = -0.04018371753573356
  
  rat: replaced -0.04272506655773012 by -13954421/326609696 = -0.04272506655773012
  
  rat: replaced -0.04533114253367693 by -2051558/45257143 = -0.04533114253367695
  
  rat: replaced -0.04800068486648146 by -16995415/354066094 = -0.04800068486648145
  
  rat: replaced -0.05073242661246818 by -3970295/78259513 = -0.05073242661246818
  
  rat: replaced -0.05352509460807248 by -3894269/72755948 = -0.05352509460807246
  
  rat: replaced -0.05637740959715515 by -11093364/196769665 = -0.05637740959715513
  
  rat: replaced -0.05928808635892763 by -3489209/58851773 = -0.05928808635892754
  
  rat: replaced -0.06225583383647254 by -3380435/54299088 = -0.06225583383647254
  
  rat: replaced -0.06527935526584844 by -7571267/115982564 = -0.06527935526584841
  
  rat: replaced -0.06835734830576551 by -8050241/117767017 = -0.06835734830576544
  
  rat: replaced -0.07148850516781785 by -5513427/77123266 = -0.07148850516781798
  
  rat: replaced -0.07467151274726203 by -2259975/30265558 = -0.07467151274726208
  
  rat: replaced -0.07790505275432569 by -657797/8443573 = -0.07790505275432569
  
  rat: replaced -0.08118780184603619 by -4832180/59518547 = -0.08118780184603633
  
  rat: replaced -0.08451843175855339 by -3076049/36395008 = -0.08451843175855327
  
  rat: replaced -0.08789560943999458 by -7150621/81353563 = -0.08789560943999465
  
  rat: replaced -0.0913179971837394 by -20067867/219758072 = -0.0913179971837394
  
  rat: replaced -0.09478425276219882 by -5487749/57897265 = -0.09478425276219869
  
  rat: replaced -0.09829302956103664 by -4406725/44832528 = -0.09829302956103658
  
  rat: replaced -0.1018429767138303 by -3912367/38415678 = -0.1018429767138302
  
  rat: replaced -0.1054327392371563 by -8451941/80164293 = -0.1054327392371564
  
  rat: replaced -0.1090609581660869 by -13126833/120362348 = -0.1090609581660869
  
  rat: replaced -0.112726270690086 by -2754747/24437489 = -0.112726270690086
  
  rat: replaced -0.116427310289289 by -22239618/191017193 = -0.116427310289289
  
  rat: replaced -0.1201627068711536 by -9494831/79016454 = -0.1201627068711537
  
  rat: replaced -0.1239310869074673 by -3190398/25743323 = -0.1239310869074672
  
  rat: replaced -0.1277310735717007 by -15999330/125257931 = -0.1277310735717006
  
  rat: replaced -0.1315612868766867 by -13929723/105880106 = -0.1315612868766867
  
  rat: replaced -0.1354203438126204 by -28035370/207024803 = -0.1354203438126204
  
  rat: replaced -0.1393068584853572 by -11590983/83204683 = -0.1393068584853571
  
  rat: replaced -0.1432194422550018 by -12738764/88945773 = -0.1432194422550018
  
  rat: replaced -0.1471567038747712 by -5246589/35653075 = -0.147156703874771
  
  rat: replaced -0.1511172496301179 by -4676629/30947023 = -0.1511172496301179
  
  rat: replaced -0.1550996834780995 by -15854305/102220099 = -0.1550996834780995
  
  rat: replaced -0.1591026071869839 by -9026555/56734174 = -0.159102607186984
  
  rat: replaced -0.1631246204760689 by -10435073/63969945 = -0.1631246204760689
  
  rat: replaced -0.1671643211557106 by -164873401/986295400 = -0.1671643211557106
  
  rat: replaced -0.1712203052675407 by -7017638/40986015 = -0.1712203052675406
  
  rat: replaced -0.1752911672248615 by -3184915/18169284 = -0.1752911672248615
  
  rat: replaced -0.1793754999532028 by -2646709/14755131 = -0.1793754999532027
  
  rat: replaced -0.1834718950310287 by -8392143/45740755 = -0.1834718950310287
  
  rat: replaced -0.1875789428305783 by -12888313/68708741 = -0.1875789428305781
  
  rat: replaced -0.1916952326588277 by -16014703/83542521 = -0.1916952326588277
  
  rat: replaced -0.1958193528985573 by -21279927/108671215 = -0.1958193528985574
  
  rat: replaced -0.1999498911495134 by -5994245/29978736 = -0.1999498911495134
  
  rat: replaced -0.2040854343696463 by -17847769/87452439 = -0.2040854343696464
  
  rat: replaced -0.2082245690164135 by -5203892/24991729 = -0.2082245690164134
  
  rat: replaced -0.2123658811881329 by -20393053/96027916 = -0.2123658811881328
  
  rat: replaced -0.2165079567653719 by -8489188/39209589 = -0.2165079567653719
  
  rat: replaced -0.2206493815523576 by -14881929/67446049 = -0.2206493815523575
  
  rat: replaced -0.2247887414183958 by -11437558/50881365 = -0.2247887414183955
  
  rat: replaced -0.2289246224392826 by -17547464/76651711 = -0.2289246224392825
  
  rat: replaced -0.2330556110386959 by -11148764/47837355 = -0.2330556110386956
  
  rat: replaced -0.2371802941295513 by -11052217/46598378 = -0.237180294129551
  
  rat: replaced -0.2412972592553108 by -36037383/149348497 = -0.2412972592553108
  
  rat: replaced -0.2454050947312253 by -4652365/18957899 = -0.2454050947312252
  
  rat: replaced -0.2495023897855041 by -6175634/24751803 = -0.2495023897855037
  
  rat: replaced -0.2535877347003893 by -11299519/44558618 = -0.2535877347003895
  
  rat: replaced -0.2576597209531272 by -6871877/26670358 = -0.2576597209531271
  
  rat: replaced -0.2617169413568191 by -2245730/8580759 = -0.2617169413568194
  
  rat: replaced -0.2657579902011391 by -10500993/39513367 = -0.2657579902011388
  
  rat: replaced -0.2697814633929034 by -21050552/78028163 = -0.2697814633929034
  
  rat: replaced -0.2737859585964791 by -1510231/5516101 = -0.2737859585964796
  
  rat: replaced -0.2777700753740163 by -9819093/35349715 = -0.2777700753740164
  
  rat: replaced -0.2817324153254904 by -10837378/38466919 = -0.2817324153254905
  
  rat: replaced -0.2856715822285418 by -17041418/59653879 = -0.2856715822285421
  
  rat: replaced -0.289586182178096 by -721506/2491507 = -0.2895861821780955
  
  rat: replaced -0.2934748237257534 by -11793110/40184401 = -0.2934748237257537
  
  rat: replaced -0.2973361180189332 by -15390047/51759763 = -0.2973361180189329
  
  rat: replaced 5.016624916807239e-5 by 153117/3052191514 = 5.016624916807235e-5
  
  rat: replaced 2.013266400891639e-4 by 232411/1154397649 = 2.013266400891639e-4
  
  rat: replaced 4.544660485167953e-4 by 444871/978887205 = 4.544660485167952e-4
  
  rat: replaced 8.105591523879241e-4 by 1425236/1758336817 = 8.105591523879239e-4
  
  rat: replaced 0.001270570334355389 by 696221/547959433 = 0.00127057033435539
  
  rat: replaced 0.001835453585351213 by 1018402/554850315 = 0.001835453585351213
  
  rat: replaced 0.002506152409187654 by 484773/193433168 = 0.002506152409187653
  
  rat: replaced 0.003283599728207867 by 1007483/306822720 = 0.003283599728207872
  
  rat: replaced 0.004168717789994683 by 897113/215201183 = 0.004168717789994677
  
  rat: replaced 0.00516241807514603 by 757433/146720585 = 0.005162418075146034
  
  rat: replaced 0.006265601206128374 by 1194190/190594639 = 0.006265601206128363
  
  rat: replaced 0.007479156857214384 by 1971251/263565939 = 0.007479156857214391
  
  rat: replaced 0.008803963665517056 by 365844/41554465 = 0.008803963665517051
  
  rat: replaced 0.01024088914312629 by 1345773/131411734 = 0.01024088914312629
  
  rat: replaced 0.01179078959035854 by 1519715/128890011 = 0.01179078959035856
  
  rat: replaced 0.0134545100101271 by 2242921/166704027 = 0.01345451001012711
  
  rat: replaced 0.01523288402344322 by 1950407/128039247 = 0.01523288402344322
  
  rat: replaced 0.01712673378605437 by 1362867/79575418 = 0.01712673378605438
  
  rat: replaced 0.01913686990622912 by 1694449/88543686 = 0.01913686990622911
  
  rat: replaced 0.02126409136369717 by 9814128/461535263 = 0.02126409136369716
  
  rat: replaced 0.02350918542975217 by 2315819/98506986 = 0.02350918542975216
  
  rat: replaced 0.02587292758852516 by 3386321/130882792 = 0.02587292758852516
  
  rat: replaced 0.02835608145943683 by 10230271/360778728 = 0.02835608145943682
  
  rat: replaced 0.03095939872083586 by 14307719/462144602 = 0.03095939872083587
  
  rat: replaced 0.03368361903483233 by 4712088/139892569 = 0.03368361903483236
  
  rat: replaced 0.03652946997333167 by 4111522/112553563 = 0.03652946997333172
  
  rat: replaced 0.03949766694527834 by 8626745/218411508 = 0.03949766694527836
  
  rat: replaced 0.04258891312511537 by 3115258/73147159 = 0.04258891312511536
  
  rat: replaced 0.04580389938246726 by 2358579/51492974 = 0.04580389938246721
  
  rat: replaced 0.04914330421305446 by 2180747/44375262 = 0.04914330421305456
  
  rat: replaced 0.05260779367084312 by 4975224/94571995 = 0.05260779367084304
  
  rat: replaced 0.05619802130144141 by 1396735/24853811 = 0.05619802130144146
  
  rat: replaced 0.05991462807674475 by 6603037/110207427 = 0.05991462807674477
  
  rat: replaced 0.06375824233083943 by 6198842/97224167 = 0.0637582423308394
  
  rat: replaced 0.06772947969716975 by 4012504/59243095 = 0.06772947969716978
  
  rat: replaced 0.07182894304697524 by 5813372/80933559 = 0.07182894304697511
  
  rat: replaced 0.07605722242900365 by 14672328/192911699 = 0.07605722242900365
  
  rat: replaced 0.08041489501050719 by 3507279/43614793 = 0.0804148950105071
  
  rat: replaced 0.08490252501952561 by 2460362/28978667 = 0.08490252501952557
  
  rat: replaced 0.08952066368846451 by 4304415/48082921 = 0.08952066368846436
  
  rat: replaced 0.09426984919897213 by 3898288/41352437 = 0.09426984919897224
  
  rat: replaced 0.0991506066281217 by 11428253/115261554 = 0.09915060662812164
  
  rat: replaced 0.1041634478959041 by 7209817/69216382 = 0.1041634478959042
  
  rat: replaced 0.1093088717140371 by 3826731/35008421 = 0.109308871714037
  
  rat: replaced 0.1145873635360931 by 5173172/45146095 = 0.1145873635360932
  
  rat: replaced 0.1199993955089551 by 23218093/193485083 = 0.1199993955089551
  
  rat: replaced 0.1255454264256029 by 2445819/19481546 = 0.125545426425603
  
  rat: replaced 0.1312259016792331 by 9111136/69430927 = 0.131225901679233
  
  rat: replaced 0.1370412532187207 by 16597683/121114501 = 0.1370412532187207
  
  rat: replaced 0.1429918995054244 by 34253454/239548213 = 0.1429918995054244
  
  rat: replaced 0.1490782454713414 by 11997679/80479073 = 0.1490782454713414
  
  rat: replaced 0.1553006824786136 by 13065213/84128497 = 0.1553006824786136
  
  rat: replaced 0.1616595882803922 by 12686167/78474572 = 0.1616595882803923
  
  rat: replaced 0.1681553269830629 by 4527449/26924208 = 0.168155326983063
  
  rat: replaced 0.1747882490098353 by 23565700/134824281 = 0.1747882490098353
  
  rat: replaced 0.1815586910657007 by 4563713/25136296 = 0.1815586910657004
  
  rat: replaced 0.1884669761037622 by 8213146/43578701 = 0.1884669761037623
  
  rat: replaced 0.1955134132929397 by 7172626/36686107 = 0.1955134132929395
  
  rat: replaced 0.202698297987053 by 17668607/87167022 = 0.2026982979870529
  
  rat: replaced 0.2100219116952866 by 8269584/39374863 = 0.2100219116952864
  
  rat: replaced 0.2174845220540395 by 56596301/260231397 = 0.2174845220540395
  
  rat: replaced 0.2250863828001612 by 8187128/36373271 = 0.2250863828001611
  
  rat: replaced 0.2328277337455789 by 10320856/44328293 = 0.2328277337455787
  
  rat: replaced 0.2407088007533156 by 16964872/70478819 = 0.2407088007533157
  
  rat: replaced 0.2487297957149048 by 11063220/44478869 = 0.2487297957149045
  
  rat: replaced 0.2568909165292014 by 17200949/66958183 = 0.2568909165292015
  
  rat: replaced 0.2651923470825914 by 8866093/33432688 = 0.2651923470825918
  
  rat: replaced 0.2736342572306039 by 12664159/46281336 = 0.2736342572306037
  
  rat: replaced 0.2822168027809259 by 8116045/28758192 = 0.2822168027809259
  
  rat: replaced 0.2909401254778209 by 24764749/85119744 = 0.290940125477821
  
  rat: replaced 0.2998043529879556 by 28498628/95057419 = 0.2998043529879556
  
  rat: replaced 0.3088095988876323 by 13390352/43361191 = 0.308809598887632
  
  rat: replaced 0.3179559626514321 by 26241235/82531036 = 0.3179559626514321
  
  rat: replaced 0.3272435296422674 by 8247573/25203166 = 0.3272435296422679
  
  rat: replaced 0.3366723711028454 by 10805861/32096073 = 0.3366723711028449
  
  rat: replaced 0.3462425441485439 by 20967050/60555961 = 0.3462425441485438
  
  rat: replaced 0.3559540917617003 by 19053013/53526602 = 0.3559540917617001
  
  rat: replaced 0.3658070427873129 by 10401097/28433288 = 0.3658070427873132
  
  rat: replaced 0.3758014119301566 by 5923743/15762961 = 0.375801411930157
  
  rat: replaced 0.3859371997533123 by 2934328/7603123 = 0.3859371997533119
  
  rat: replaced 0.396214392678111 by 30414315/76762267 = 0.396214392678111
  
  rat: replaced 0.4066329629854911 by 13711485/33719561 = 0.4066329629854908
  
  rat: replaced 0.4171928688187707 by 20838614/49949593 = 0.4171928688187709
  
  rat: replaced 0.4278940541878331 by 16106690/37641771 = 0.427894054187833
  
  rat: replaced 0.4387364489747257 by 4869080/11097961 = 0.4387364489747261
  
  rat: replaced 0.4497199689406718 by 4550581/10118699 = 0.4497199689406711
  
  rat: replaced 0.4608445157344944 by 7970699/17295853 = 0.4608445157344943
  
  rat: replaced 0.4721099769024512 by 25424083/53852035 = 0.4721099769024513
  
  rat: replaced 0.48351622589948 by 17675673/36556525 = 0.4835162258994803
  
  rat: replaced 0.4950631221018528 by 7053395/14247466 = 0.495063122101853
  
  rat: replaced 0.5067505108212387 by 13754758/27143057 = 0.5067505108212388
  
  rat: replaced 0.5185782233201719 by 21662467/41772805 = 0.518578223320172
  
  rat: replaced 0.5305460768289253 by 10488897/19770002 = 0.530546076828925
  
  rat: replaced 0.5426538745637882 by 22388393/41257225 = 0.5426538745637886
  
  rat: replaced 0.5549014057467435 by 9960301/17949677 = 0.5549014057467441
  
  rat: replaced 0.5672884456265459 by 28078535/49496046 = 0.5672884456265456
  
  rat: replaced 0.5798147555011964 by 18086313/31193261 = 0.5798147555011962
  
  rat: replaced 0.5924800827418131 by 20592707/34756792 = 0.5924800827418134
  
  rat: replaced 0.6052841608178928 by 26813845/44299598 = 0.6052841608178927
  part: invalid index of list or matrix.
  #0: lineIntersection(g=[1,-1,2],h=[1,1,5])
  #1: projectToLine(a=[2,0],g=[1,1,5])
  #2: lineCircleIntersections(g=[1,1,5],c=[2,0,4])
   -- an error. To debug this try: debugmode(true);
  
  Error in:
   $lineCircleIntersections(l,c) | radcan, // titik potong lingka ...
                                        ^
\end{euleroutput}
\begin{eulercomment}
\end{eulercomment}
\begin{eulerprompt}
>C=A+normalize([-3,-4])*4; plotPoint(C); plotSegment(P1,C); plotSegment(P2,C);
>degprint(computeAngle(P1,C,P2))
\end{eulerprompt}
\begin{euleroutput}
  57°58'20.06''
\end{euleroutput}
\begin{eulerprompt}
>C=A+normalize([-4,-5])*4; plotPoint(C); plotSegment(P1,C); plotSegment(P2,C);
>degprint(computeAngle(P1,C,P2))
\end{eulerprompt}
\begin{euleroutput}
  57°58'20.06''
\end{euleroutput}
\begin{eulerprompt}
>insimg;
\end{eulerprompt}
\eulersubheading{ Garis Sumbu}
\begin{eulerprompt}
>A=[3,3]; B=[-2,-3];
>c1=circleWithCenter(A,distance(A,B));
>c2=circleWithCenter(B,distance(A,B));
>\{P1,P2,f\}=circleCircleIntersections(c1,c2);
>l=lineThrough(P1,P2);
>setPlotRange(5); plotCircle(c1); plotCircle(c2);
>plotPoint(A); plotPoint(B); plotSegment(A,B); plotLine(l):
>A &= [a1,a2]; B &= [b1,b2];
>c1 &= circleWithCenter(A,distance(A,B));
>c2 &= circleWithCenter(B,distance(A,B));
>P &= circleCircleIntersections(c1,c2); P1 &= P[1]; P2 &= P[2];
>g &= getLineEquation(lineThrough(P1,P2),x,y);
>$solve(g,y)
\end{eulerprompt}
\begin{euleroutput}
  Maxima said:
  solve: all variables must not be numbers.
   -- an error. To debug this try: debugmode(true);
  
  Error in:
   $solve(g,y) ...
             ^
\end{euleroutput}
\begin{eulerprompt}
>$solve(getLineEquation(middlePerpendicular(A,B),x,y),y)
\end{eulerprompt}
\begin{euleroutput}
  Maxima said:
  solve: all variables must not be numbers.
   -- an error. To debug this try: debugmode(true);
  
  Error in:
  ... (getLineEquation(middlePerpendicular(A,B),x,y),y) ...
                                                       ^
\end{euleroutput}
\begin{eulerprompt}
>h &=getLineEquation(lineThrough(A,B),x,y);
>$solve(h,y)
\end{eulerprompt}
\begin{euleroutput}
  Maxima said:
  solve: all variables must not be numbers.
   -- an error. To debug this try: debugmode(true);
  
  Error in:
   $solve(h,y) ...
             ^
\end{euleroutput}
\eulersubheading{Garis Euler dan Parabola}
\begin{eulerprompt}
>A::=[-1.5,-1.5]; B::=[3,0]; C::=[1.5,3];
>setPlotRange(3); plotPoint(A,"A"); plotPoint(B,"B"); plotPoint(C,"C");
\end{eulerprompt}
\begin{eulercomment}
\end{eulercomment}
\begin{eulerprompt}
>plotSegment(A,B,""); plotSegment(B,C,""); plotSegment(C,A,""):
>$areaTriangle(A,B,C)
\end{eulerprompt}
\begin{eulercomment}
\end{eulercomment}
\begin{eulerprompt}
>c &= lineThrough(A,B)
\end{eulerprompt}
\begin{euleroutput}
  
                                 3  9    9
                              [- -, -, - -]
                                 2  2    2
  
\end{euleroutput}
\begin{eulercomment}
\end{eulercomment}
\begin{eulerprompt}
>$getLineEquation(c,x,y)
>$getHesseForm(c,x,y,C), $at(%,[x=C[1],y=C[2]])
\end{eulerprompt}
\begin{eulercomment}
\end{eulercomment}
\begin{eulerprompt}
>LL &= circleThrough(A,B,C); $getCircleEquation(LL,x,y)
\end{eulerprompt}
\begin{euleroutput}
  Maxima said:
  rat: replaced -7.57493712521158e-5 by -291512/3848375177 = -7.574937125211583e-5
  
  rat: replaced -3.059898801345065e-4 by -367004/1199399143 = -3.059898801345067e-4
  
  rat: replaced -6.951984652882083e-4 by -649868/934794929 = -6.951984652882086e-4
  
  rat: replaced -0.001247836168679406 by -996993/798977482 = -0.001247836168679407
  
  rat: replaced -0.0019683476894981 by -1171852/595348071 = -0.001968347689498099
  
  rat: replaced -0.002861160939693026 by -414045/144712237 = -0.002861160939693027
  
  rat: replaced -0.003930686601183196 by -1414939/359972479 = -0.003930686601183198
  
  rat: replaced -0.00518131768479372 by -1585327/305969851 = -0.005181317684793722
  
  rat: replaced -0.006617429090958894 by -2655242/401249785 = -0.00661742909095889
  
  rat: replaced -0.008243377172234598 by -1494085/181246711 = -0.00824337717223459
  
  rat: replaced -0.01006349929766813 by -2785964/276838495 = -0.01006349929766812
  
  rat: replaced -0.01208211341906348 by -403273/33377687 = -0.01208211341906346
  
  rat: replaced -0.01430351763919102 by -2688199/187939713 = -0.01430351763919103
  
  rat: replaced -0.01673198978198 by -3399597/203179481 = -0.01673198978198
  
  rat: replaced -0.01937178696474014 by -4095384/211409717 = -0.01937178696474013
  
  rat: replaced -0.02222714517245272 by -1488848/66983321 = -0.02222714517245271
  
  rat: replaced -0.02530227883417678 by -11141142/440321683 = -0.02530227883417678
  
  rat: replaced -0.02860138040160899 by -5896067/206146239 = -0.02860138040160898
  
  rat: replaced -0.03212861992984196 by -3474579/108145915 = -0.03212861992984201
  
  rat: replaced -0.03588814466036214 by -2498277/69612877 = -0.03588814466036219
  
  rat: replaced -0.03988407860632956 by -5523906/138499025 = -0.03988407860632954
  
  rat: replaced -0.04412052214017978 by -4053557/91874638 = -0.04412052214017975
  
  rat: replaced -0.04860155158359004 by -3943740/81144323 = -0.04860155158359014
  
  rat: replaced -0.05333121879985003 by -1834427/34396870 = -0.0533312187998501
  
  rat: replaced -0.05831355078867968 by -2465946/42287701 = -0.05831355078867967
  
  rat: replaced -0.06355254928353218 by -4583196/72116635 = -0.06355254928353216
  
  rat: replaced -0.06905219035142413 by -9155887/132593723 = -0.0690521903514241
  
  rat: replaced -0.07481642399533184 by -2967077/39658097 = -0.0748164239953319
  
  rat: replaced -0.08084917375919423 by -6800433/84112585 = -0.0808491737591943
  
  rat: replaced -0.08715433633556302 by -5843645/67049389 = -0.08715433633556303
  
  rat: replaced -0.09373578117593417 by -7402616/78973215 = -0.09373578117593415
  
  rat: replaced -0.1005973501038089 by -1640864/16311205 = -0.100597350103809
  
  rat: replaced -0.1077428569305121 by -20150833/187027090 = -0.107742856930512
  
  rat: replaced -0.115176087073816 by -3594765/31211036 = -0.1151760870738158
  
  rat: replaced -0.1229007971794005 by -4862770/39566627 = -0.1229007971794007
  
  rat: replaced -0.130920714745193 by -4199712/32078285 = -0.1309207147451929
  
  rat: replaced -0.1392395377486195 by -36213847/260083074 = -0.1392395377486195
  
  rat: replaced -0.1478609342768128 by -4198057/28391928 = -0.1478609342768128
  
  rat: replaced -0.1567885421598042 by -14899832/95031383 = -0.1567885421598042
  
  rat: replaced -0.1660259686067453 by -12607897/75939307 = -0.1660259686067454
  
  rat: replaced -0.1755767898451896 by -9911603/56451670 = -0.1755767898451897
  
  rat: replaced -0.185444550763472 by -7194550/38796233 = -0.1854445507634723
  
  rat: replaced -0.1956327645562239 by -14925693/76294444 = -0.1956327645562238
  
  rat: replaced -0.206144912373057 by -10508817/50977814 = -0.206144912373057
  
  rat: replaced -0.2169844429704491 by -5288053/24370655 = -0.2169844429704495
  
  rat: replaced -0.2281547723668733 by -3759235/16476688 = -0.2281547723668737
  
  rat: replaced -0.2396592835011996 by -18130307/75650343 = -0.2396592835011997
  
  rat: replaced -0.2515013258944011 by -9665078/38429531 = -0.2515013258944014
  
  rat: replaced -0.2636842153146071 by -16839380/63861919 = -0.2636842153146071
  
  rat: replaced -0.2762112334455278 by -9903377/35854360 = -0.2762112334455279
  
  rat: replaced -0.2890856275582896 by -4178583/14454482 = -0.2890856275582895
  
  rat: replaced -0.3023106101867103 by -5206455/17222204 = -0.3023106101867101
  
  rat: replaced -0.3158893588060475 by -15779177/49951594 = -0.3158893588060473
  
  rat: replaced -0.3298250155152552 by -20176073/61172052 = -0.3298250155152552
  
  rat: replaced -0.3441206867227753 by -22215819/64558220 = -0.3441206867227752
  
  rat: replaced -0.358779442835901 by -40621537/113221473 = -0.3587794428359009
  
  rat: replaced -0.3738043179537355 by -4462655/11938479 = -0.373804317953736
  
  rat: replaced -0.3891983095637891 by -17279077/44396588 = -0.389198309563789
  
  rat: replaced -0.4049643782422284 by -15521239/38327418 = -0.4049643782422286
  
  rat: replaced -0.4211054473578241 by -18145515/43090193 = -0.4211054473578246
  
  rat: replaced -0.4376244027796156 by -12318025/28147482 = -0.4376244027796163
  
  rat: replaced -0.4545240925883269 by -18977389/41752218 = -0.4545240925883267
  
  rat: replaced -0.4718073267915598 by -11534269/24446990 = -0.47180732679156
  
  rat: replaced -0.4894768770427974 by -19580265/40002431 = -0.4894768770427977
  
  rat: replaced -0.5075354763642387 by -14211341/28000685 = -0.5075354763642389
  
  rat: replaced -0.5259858188735007 by -33496033/63682388 = -0.5259858188735008
  
  rat: replaced -0.5448305595142084 by -33841376/62113579 = -0.5448305595142087
  
  rat: replaced -0.5640723137905005 by -15307610/27137673 = -0.5640723137905007
  
  rat: replaced -0.5837136575054853 by -47878079/82023229 = -0.5837136575054854
  
  rat: replaced -0.6037571265036585 by -12602624/20873665 = -0.6037571265036591
  
  rat: replaced -0.6242052164173237 by -10481453/16791678 = -0.624205216417323
  
  rat: replaced -0.6450603824170293 by -7607359/11793251 = -0.6450603824170282
  
  rat: replaced -0.6663250389660548 by -21098582/31664099 = -0.6663250389660542
  
  rat: replaced -0.6880015595789664 by -31067245/45155777 = -0.6880015595789659
  
  rat: replaced -0.7100922765842661 by -5131430/7226427 = -0.710092276584265
  
  rat: replaced -0.7325994808911623 by -12479523/17034578 = -0.7325994808911614
  
  rat: replaced -0.7555254217604808 by -16394539/21699520 = -0.7555254217604813
  
  rat: replaced -0.7788723065797394 by -17129047/21992112 = -0.7788723065797409
  
  rat: replaced -0.8026423006424119 by -78532681/97842689 = -0.8026423006424118
  
  rat: replaced -0.8268375269314006 by -14288533/17280944 = -0.8268375269313991
  
  rat: replaced -0.8514600659067393 by -29344334/34463547 = -0.8514600659067391
  
  rat: replaced -0.8765119552975495 by -17808806/20317813 = -0.876511955297551
  
  rat: replaced -0.9019951898982683 by -17368197/19255310 = -0.901995189898267
  
  rat: replaced -0.9279117213691723 by -4803773/5176972 = -0.927911721369171
  
  rat: replaced -0.9542634580412112 by -20199596/21167735 = -0.9542634580412123
  
  rat: replaced -0.9810522647251774 by -22272134/22702291 = -0.9810522647251768
  
  rat: replaced -1.008279962525227 by -9926001/9844489 = -1.008279962525226
  
  rat: replaced -1.035948328656769 by -20447689/19738136 = -1.035948328656769
  
  rat: replaced -1.064059096268749 by -43784241/41148317 = -1.064059096268749
  
  rat: replaced -1.092613954270329 by -24228202/22174531 = -1.092613954270329
  
  rat: replaced -1.121614547162007 by -18354705/16364539 = -1.121614547162007
  
  rat: replaced -1.15106247487116 by -22035757/19143841 = -1.151062474871161
  
  rat: replaced -1.180959292592057 by -20456643/17322056 = -1.180959292592057
  
  rat: replaced -1.21130651063034 by -30900377/25509957 = -1.211306510630339
  
  rat: replaced -1.242105594251995 by -6989671/5627276 = -1.242105594251997
  
  rat: replaced -1.273357963536825 by -22090312/17348077 = -1.273357963536823
  
  rat: replaced -1.305064993236445 by -14582607/11173855 = -1.305064993236444
  
  rat: replaced -1.337228012636809 by -33469619/25029104 = -1.337228012636809
  
  rat: replaced -1.369848305425279 by -32941560/24047597 = -1.369848305425278
  
  rat: replaced -2.254981225063221e-4 by -476777/2114328025 = -2.254981225063221e-4
  
  rat: replaced -9.039699204008572e-4 by -554629/613548070 = -9.039699204008579e-4
  
  rat: replaced -0.002038347522069071 by -1271429/623754775 = -0.00203834752206907
  
  rat: replaced -0.003631517465696898 by -2066351/569004836 = -0.0036315174656969
  
  rat: replaced -0.005686320410616744 by -635713/111796901 = -0.005686320410616749
  
  rat: replaced -0.008205550853247354 by -2741742/334132595 = -0.008205550853247347
  
  rat: replaced -0.01119195684764358 by -1145556/102355291 = -0.01119195684764357
  
  rat: replaced -0.01464823973069444 by -3593060/245289541 = -0.01464823973069443
  
  rat: replaced -0.01857705385199264 by -2624072/141253399 = -0.01857705385199261
  
  rat: replaced -0.02298100630839936 by -2189611/95279161 = -0.02298100630839938
  
  rat: replaced -0.0278626566833399 by -3478181/124833071 = -0.02786265668333995
  
  rat: replaced -0.03322451679084377 by -2100144/63210671 = -0.03322451679084375
  
  rat: replaced -0.03906905042436903 by -3541941/90658487 = -0.03906905042436899
  
  rat: replaced -0.04539867311042303 by -2490333/54854753 = -0.04539867311042308
  
  rat: replaced -0.05221575186701224 by -4506215/86299916 = -0.05221575186701224
  
  rat: replaced -0.0595226049669409 by -10922963/183509492 = -0.0595226049669409
  
  rat: replaced -0.06732150170598852 by -4631344/68794425 = -0.06732150170598852
  
  rat: replaced -0.07561466217598092 by -14346317/189729301 = -0.07561466217598092
  
  rat: replaced -0.0844042570427819 by -3521587/41722860 = -0.08440425704278182
  
  rat: replaced -0.09369240732922907 by -5174175/55225126 = -0.0936924073292291
  
  rat: replaced -0.1034811842030341 by -2097183/20266322 = -0.103481184203034
  
  rat: replaced -0.1137726087696672 by -11392983/100138189 = -0.1137726087696673
  
  rat: replaced -0.1245686518702483 by -6834267/54863458 = -0.1245686518702485
  
  rat: replaced -0.1358712338844633 by -8117277/59742425 = -0.1358712338844632
  
  rat: replaced -0.1476822245385299 by -6303644/42683837 = -0.1476822245385297
  
  rat: replaced -0.1600034427182252 by -9148317/57175751 = -0.1600034427182251
  
  rat: replaced -0.1728366562869992 by -22187021/128369881 = -0.1728366562869993
  
  rat: replaced -0.1861835819091898 by -17269805/92756863 = -0.1861835819091898
  
  rat: replaced -0.200045884878356 by -7864089/39311426 = -0.2000458848783557
  
  rat: replaced -0.2144251789507544 by -7172489/33449845 = -0.2144251789507545
  
  rat: replaced -0.2293230261839593 by -10351388/45138895 = -0.2293230261839595
  
  rat: replaced -0.244740936780663 by -8538850/34889341 = -0.2447409367806632
  
  rat: replaced -0.2606803689376539 by -18287762/70153967 = -0.260680368937654
  
  rat: replaced -0.277142728699999 by -8575365/30942053 = -0.2771427286999993
  
  rat: replaced -0.2941293698204409 by -4485287/15249368 = -0.2941293698204411
  
  rat: replaced -0.3116415936240235 by -9259347/29711525 = -0.311641593624023
  
  rat: replaced -0.3296806488779594 by -11717987/35543448 = -0.3296806488779592
  
  rat: replaced -0.3482477316667564 by -36675861/105315434 = -0.3482477316667564
  
  rat: replaced -0.3673439852726074 by -9533778/25953271 = -0.3673439852726078
  
  rat: replaced -0.386970500061066 by -7376119/19061192 = -0.3869705000610665
  
  rat: replaced -0.4071283133720089 by -28281643/69466166 = -0.4071283133720091
  
  rat: replaced -0.4278184094159029 by -15961498/37309049 = -0.4278184094159034
  
  rat: replaced -0.4490417191753848 by -13063519/29091994 = -0.4490417191753855
  
  rat: replaced -0.4707991203121662 by -38805045/82423784 = -0.4707991203121662
  
  rat: replaced -0.493091437079264 by -14163447/28723774 = -0.493091437079264
  
  rat: replaced -0.5159194402385774 by -24505042/47497807 = -0.5159194402385777
  
  rat: replaced -0.5392838469838155 by -46178693/85629661 = -0.5392838469838156
  
  rat: replaced -0.5631853208687732 by -40347511/71641624 = -0.5631853208687732
  
  rat: replaced -0.58762447174098 by -29058549/49450883 = -0.5876244717409799
  
  rat: replaced -0.6126018556807142 by -18023233/29420794 = -0.6126018556807135
  
  rat: replaced -0.6381179749453973 by -17107844/26809845 = -0.6381179749453979
  
  rat: replaced -0.6641732779193661 by -21974679/33085762 = -0.6641732779193661
  
  rat: replaced -0.6907681590690352 by -10003471/14481662 = -0.690768159069035
  
  rat: replaced -0.7179029589034526 by -28073639/39105061 = -0.7179029589034525
  
  rat: replaced -0.7455779639402473 by -23235768/31164773 = -0.7455779639402476
  
  rat: replaced -0.773793406676978 by -8384139/10835113 = -0.7737934066769769
  
  rat: replaced -0.8025494655678836 by -7868837/9804800 = -0.8025494655678851
  
  rat: replaced -0.8318462650060389 by -30346636/36481063 = -0.8318462650060389
  
  rat: replaced -0.861683875310914 by -16277728/18890603 = -0.8616838753109152
  
  rat: replaced -0.8920623127213426 by -37841947/42420744 = -0.8920623127213422
  
  rat: replaced -0.9229815393938994 by -70218740/76078163 = -0.9229815393938994
  
  rat: replaced -0.954441463406683 by -24594815/25768804 = -0.9544414634066836
  
  rat: replaced -0.9864419387685092 by -15639887/15854848 = -0.9864419387685079
  
  rat: replaced -1.018982765433508 by -45306349/44462331 = -1.018982765433508
  
  rat: replaced -1.052063689321131 by -30825646/29300171 = -1.052063689321131
  
  rat: replaced -1.085684402341557 by -38378733/35349806 = -1.085684402341557
  
  rat: replaced -1.119844542426501 by -24894563/22230374 = -1.119844542426502
  
  rat: replaced -1.15454369356542 by -13440326/11641245 = -1.154543693565422
  
  rat: replaced -1.189781385847118 by -34027123/28599475 = -1.189781385847118
  
  rat: replaced -1.22555709550673 by -35019680/28574499 = -1.225557095506731
  
  rat: replaced -1.261870244978105 by -35214941/27906943 = -1.261870244978105
  
  rat: replaced -1.298720202951555 by -33824443/26044442 = -1.298720202951555
  
  rat: replaced -1.336106284436991 by -5396397/4038898 = -1.336106284436992
  
  rat: replaced -1.374027750832421 by -21232969/15453086 = -1.374027750832423
  
  rat: replaced -1.412483809997805 by -11919739/8438850 = -1.412483809997808
  
  rat: replaced -1.451473616334275 by -8780993/6049709 = -1.451473616334273
  
  rat: replaced -1.490996270868687 by -44673937/29962474 = -1.490996270868687
  
  rat: replaced -1.531050821343523 by -64281527/41985234 = -1.531050821343523
  
  rat: replaced -1.571636262312116 by -16616467/10572718 = -1.571636262312113
  
  rat: replaced -1.612751535239189 by -65392401/40547102 = -1.612751535239189
  
  rat: replaced -1.654395528606713 by -27646066/16710675 = -1.654395528606714
  
  rat: replaced -1.696567078025054 by -7906291/4660170 = -1.696567078025051
  
  rat: replaced -1.739264966349412 by -21773512/12518801 = -1.739264966349413
  
  rat: replaced -1.782487923801538 by -19245269/10796858 = -1.782487923801536
  
  rat: replaced -1.826234628096705 by -18221771/9977782 = -1.826234628096705
  
  rat: replaced -1.870503704575938 by -10770134/5757879 = -1.870503704575939
  
  rat: replaced -1.915293726343482 by -16780009/8761063 = -1.915293726343481
  
  rat: replaced -1.960603214409484 by -45722957/23320862 = -1.960603214409484
  
  rat: replaced -2.006430637837895 by -146296719/72913918 = -2.006430637837895
  
  rat: replaced -2.052774413899562 by -58550872/28522799 = -2.052774413899562
  
  rat: replaced -2.099632908230499 by -93949097/44745487 = -2.099632908230499
  
  rat: replaced -2.147004434995322 by -25274650/11772053 = -2.147004434995323
  
  rat: replaced -2.194887257055829 by -28867932/13152353 = -2.194887257055829
  
  rat: replaced -2.243279586144718 by -38403199/17119221 = -2.24327958614472
  
  rat: replaced -2.292179583044406 by -15612340/6811133 = -2.292179583044407
  
  rat: replaced -2.341585357770954 by -20809175/8886789 = -2.341585357770956
  
  rat: replaced -2.391494969763059 by -22142156/9258709 = -2.391494969763063
  
  rat: replaced -2.441906428076114 by -36070003/14771247 = -2.441906428076113
  
  rat: replaced -2.492817691581298 by -26575204/10660709 = -2.492817691581301
  part: invalid index of list or matrix.
  #0: lineIntersection(g=[-9/2,-3/2,-9/4],h=[-3,-9/2,-27/8])
  #1: circleThrough(a=[-3/2,-3/2],b=[3,0],c=[3/2,3])
   -- an error. To debug this try: debugmode(true);
  
  Error in:
  LL &= circleThrough(A,B,C); $getCircleEquation(LL,x,y) ...
                            ^
\end{euleroutput}
\begin{eulerprompt}
>O &= getCircleCenter(LL); $O
>plotCircle(LL()); plotPoint(O(),"O"):
\end{eulerprompt}
\begin{euleroutput}
  Function LL not found.
  Try list ... to find functions!
  Error in:
  plotCircle(LL()); plotPoint(O(),"O"): ...
                 ^
\end{euleroutput}
\begin{eulerprompt}
>H &= lineIntersection(perpendicular(A,lineThrough(C,B)),...
>  perpendicular(B,lineThrough(A,C))); $H
\end{eulerprompt}
\begin{euleroutput}
  Maxima said:
  rat: replaced -1.497487512542063e-4 by -299560/2000417349 = -1.497487512542064e-4
  
  rat: replaced -5.979800402663507e-4 by -330831/553247563 = -5.979800402663499e-4
  
  rat: replaced -0.001343149056780863 by -584699/435319518 = -0.001343149056780863
  
  rat: replaced -0.002383681297017493 by -756665/317435473 = -0.002383681297017489
  
  rat: replaced -0.003717972721118644 by -2149606/578166157 = -0.003717972721118646
  
  rat: replaced -0.005344389913554327 by -2779816/520137199 = -0.005344389913554327
  
  rat: replaced -0.007261270246460387 by -1541197/212248952 = -0.007261270246460392
  
  rat: replaced -0.009466922045900723 by -8174891/863521529 = -0.009466922045900722
  
  rat: replaced -0.01195962476103374 by -2414321/201872638 = -0.01195962476103374
  
  rat: replaced -0.01473762913616476 by -3432147/232883252 = -0.01473762913616476
  
  rat: replaced -0.01779915738567177 by -1659683/93245032 = -0.01779915738567176
  
  rat: replaced -0.0211424033717803 by -2960631/140032850 = -0.02114240337178026
  
  rat: replaced -0.02476553278517801 by -1738361/70192756 = -0.024765532785178
  
  rat: replaced -0.02866668332844304 by -1475047/51455098 = -0.02866668332844299
  
  rat: replaced -0.03284396490227211 by -4221724/128538805 = -0.03284396490227212
  
  rat: replaced -0.03729545979448817 by -6430372/172417019 = -0.03729545979448815
  
  rat: replaced -0.04201922287181174 by -2263313/53863752 = -0.04201922287181183
  
  rat: replaced -0.04701328177437193 by -1674543/35618509 = -0.04701328177437186
  
  rat: replaced -0.05227563711293993 by -3997444/76468585 = -0.05227563711293991
  
  rat: replaced -0.05780426266886693 by -3832681/66304470 = -0.05780426266886682
  
  rat: replaced -0.06359710559670453 by -5078877/79860191 = -0.06359710559670462
  
  rat: replaced -0.06965208662948744 by -10918553/156758448 = -0.06965208662948742
  
  rat: replaced -0.07596710028665828 by -5036501/66298450 = -0.07596710028665829
  
  rat: replaced -0.08254001508461323 by -6160264/74633667 = -0.08254001508461323
  
  rat: replaced -0.0893686737498502 by -3979484/44528847 = -0.08936867374985029
  
  rat: replaced -0.09645089343469301 by -4946630/51286513 = -0.09645089343469306
  
  rat: replaced -0.1037844659355751 by -7809283/75245201 = -0.1037844659355751
  
  rat: replaced -0.1113671579138579 by -6096479/54742162 = -0.1113671579138581
  
  rat: replaced -0.1191967111191618 by -7273952/61024771 = -0.1191967111191618
  
  rat: replaced -0.1272708426151914 by -9595393/75393490 = -0.1272708426151913
  
  rat: replaced -0.1355872450080251 by -5659716/41742245 = -0.1355872450080249
  
  rat: replaced -0.1441435866768541 by -2581028/17905951 = -0.144143586676854
  
  rat: replaced -0.1529375120071418 by -1082663/7079120 = -0.1529375120071421
  
  rat: replaced -0.161966641626183 by -5766177/35601016 = -0.1619666416261828
  
  rat: replaced -0.1712285726410404 by -5923297/34592924 = -0.1712285726410407
  
  rat: replaced -0.1807208788788305 by -55437725/306758828 = -0.1807208788788305
  
  rat: replaced -0.1904411111293399 by -5908417/31024903 = -0.1904411111293402
  
  rat: replaced -0.2003867973899436 by -6986853/34866833 = -0.2003867973899436
  
  rat: replaced -0.2105554431128032 by -4264228/20252281 = -0.2105554431128029
  
  rat: replaced -0.2209445314543208 by -9342805/42285749 = -0.2209445314543205
  
  rat: replaced -0.2315515235268193 by -4085380/17643503 = -0.2315515235268189
  
  rat: replaced -0.2423738586524308 by -187964237/775513655 = -0.2423738586524308
  
  rat: replaced -0.2534089546191609 by -7570461/29874481 = -0.2534089546191614
  
  rat: replaced -0.2646542079391092 by -6530305/24674858 = -0.2646542079391095
  
  rat: replaced -0.2761069941088149 by -10501531/38034281 = -0.2761069941088146
  
  rat: replaced -0.2877646678717041 by -78631265/273248504 = -0.2877646678717041
  
  rat: replaced -0.2996245634826158 by -7854364/26214019 = -0.2996245634826159
  
  rat: replaced -0.3116839949743722 by -12170593/39047860 = -0.3116839949743725
  
  rat: replaced -0.3239402564263729 by -11348921/35033994 = -0.3239402564263726
  
  rat: replaced -0.3363906222351865 by -33578595/99820247 = -0.3363906222351864
  
  rat: replaced -0.3490323473871076 by -11921804/34156731 = -0.349032347387108
  
  rat: replaced -0.3618626677326557 by -17595895/48625892 = -0.3618626677326557
  
  rat: replaced -0.3748788002629876 by -46177544/123179929 = -0.3748788002629876
  
  rat: replaced -0.3880779433881974 by -16190143/41718792 = -0.3880779433881978
  
  rat: replaced -0.401457277217472 by -60525431/150764314 = -0.401457277217472
  
  rat: replaced -0.415013963841077 by -12594557/30347309 = -0.4150139638410773
  
  rat: replaced -0.4287451476141481 by -27500639/64142158 = -0.4287451476141479
  
  rat: replaced -0.4426479554422498 by -14824369/33490201 = -0.4426479554422501
  
  rat: replaced -0.4567194970686855 by -21704313/47522195 = -0.4567194970686855
  
  rat: replaced -0.4709568653635186 by -16486730/35006879 = -0.470956865363519
  
  rat: replaced -0.4853571366142837 by -267196237/550514697 = -0.4853571366142837
  
  rat: replaced -0.4999173708183561 by -15013400/30031763 = -0.4999173708183566
  
  rat: replaced -0.5146346119769494 by -20942773/40694451 = -0.5146346119769499
  
  rat: replaced -0.5295058883907107 by -26908094/50817365 = -0.5295058883907106
  
  rat: replaced -0.5445282129568924 by -22151821/40680759 = -0.544528212956892
  
  rat: replaced -0.5596985834680561 by -41889600/74843141 = -0.5596985834680562
  
  rat: replaced -0.5750139829122923 by -11820697/20557234 = -0.5750139829122926
  
  rat: replaced -0.5904713797749195 by -13730652/23253713 = -0.5904713797749203
  
  rat: replaced -0.6060677283416327 by -16634707/27446944 = -0.6060677283416325
  
  rat: replaced -0.621799969003072 by -3133380/5039209 = -0.6217999690030717
  
  rat: replaced -0.6376650285607812 by -24667763/38684516 = -0.6376650285607812
  
  rat: replaced -0.6536598205345254 by -15672861/23977091 = -0.6536598205345261
  
  rat: replaced -0.6697812454709364 by -4173133/6230591 = -0.6697812454709353
  
  rat: replaced -0.6860261912534547 by -19587769/28552509 = -0.6860261912534552
  
  rat: replaced -0.7023915334135393 by -14227114/20255247 = -0.7023915334135397
  
  rat: replaced -0.7188741354431122 by -40957539/56974562 = -0.7188741354431123
  
  rat: replaced -0.7354708491082056 by -15902500/21622203 = -0.735470849108206
  
  rat: replaced -0.7521785147637838 by -19967209/26545838 = -0.7521785147637833
  
  rat: replaced -0.7689939616697044 by -16336853/21244449 = -0.7689939616697049
  
  rat: replaced -0.7859140083077888 by -21511393/27371179 = -0.7859140083077898
  
  rat: replaced -0.8029354626999737 by -9186705/11441399 = -0.8029354626999723
  
  rat: replaced -0.8200551227275041 by -39606167/48296957 = -0.8200551227275044
  
  rat: replaced -0.8372697764511438 by -3747190/4475487 = -0.8372697764511438
  
  rat: replaced -0.8545762024323653 by -43827549/51285712 = -0.8545762024323655
  
  rat: replaced -0.8719711700554936 by -21551370/24715691 = -0.8719711700554923
  
  rat: replaced -0.8894514398507607 by -14730441/16561265 = -0.8894514398507601
  
  rat: replaced -0.9070137638182549 by -23593213/26011968 = -0.9070137638182547
  
  rat: replaced -0.9246548857527144 by -17429936/18850207 = -0.9246548857527135
  
  rat: replaced -0.942371541569146 by -21072616/22361261 = -0.9423715415691449
  
  rat: replaced -0.9601604596292325 by -54513257/56775153 = -0.9601604596292326
  
  rat: replaced -0.978018361068492 by -77467650/79208789 = -0.978018361068492
  
  rat: replaced -0.9959419601241615 by -12215999/12265774 = -0.9959419601241634
  
  rat: replaced -1.013927964463772 by -21561239/21265060 = -1.013927964463773
  
  rat: replaced -1.031973075514378 by -37324525/36168119 = -1.031973075514378
  
  rat: replaced -1.050073988792411 by -29992669/28562434 = -1.050073988792412
  
  rat: replaced -1.068227394234129 by -23767018/22249025 = -1.068227394234129
  
  rat: replaced -1.086429976526613 by -11795807/10857402 = -1.086429976526613
  
  rat: replaced -1.104678415439305 by -16973206/15364839 = -1.104678415439303
  
  rat: replaced -1.122969386156019 by -33371626/29717307 = -1.12296938615602
  
  rat: replaced 2.254981225063221e-4 by 476777/2114328025 = 2.254981225063221e-4
  
  rat: replaced 9.039699204008572e-4 by 554629/613548070 = 9.039699204008579e-4
  
  rat: replaced 0.002038347522069071 by 1271429/623754775 = 0.00203834752206907
  
  rat: replaced 0.003631517465696898 by 2066351/569004836 = 0.0036315174656969
  
  rat: replaced 0.005686320410616744 by 635713/111796901 = 0.005686320410616749
  
  rat: replaced 0.008205550853247354 by 2741742/334132595 = 0.008205550853247347
  
  rat: replaced 0.01119195684764358 by 1145556/102355291 = 0.01119195684764357
  
  rat: replaced 0.01464823973069444 by 3593060/245289541 = 0.01464823973069443
  
  rat: replaced 0.01857705385199264 by 2624072/141253399 = 0.01857705385199261
  
  rat: replaced 0.02298100630839936 by 2189611/95279161 = 0.02298100630839938
  
  rat: replaced 0.0278626566833399 by 3478181/124833071 = 0.02786265668333995
  
  rat: replaced 0.03322451679084377 by 2100144/63210671 = 0.03322451679084375
  
  rat: replaced 0.03906905042436903 by 3541941/90658487 = 0.03906905042436899
  
  rat: replaced 0.04539867311042303 by 2490333/54854753 = 0.04539867311042308
  
  rat: replaced 0.05221575186701224 by 4506215/86299916 = 0.05221575186701224
  
  rat: replaced 0.0595226049669409 by 10922963/183509492 = 0.0595226049669409
  
  rat: replaced 0.06732150170598852 by 4631344/68794425 = 0.06732150170598852
  
  rat: replaced 0.07561466217598092 by 14346317/189729301 = 0.07561466217598092
  
  rat: replaced 0.0844042570427819 by 3521587/41722860 = 0.08440425704278182
  
  rat: replaced 0.09369240732922907 by 5174175/55225126 = 0.0936924073292291
  
  rat: replaced 0.1034811842030341 by 2097183/20266322 = 0.103481184203034
  
  rat: replaced 0.1137726087696672 by 11392983/100138189 = 0.1137726087696673
  
  rat: replaced 0.1245686518702483 by 6834267/54863458 = 0.1245686518702485
  
  rat: replaced 0.1358712338844633 by 8117277/59742425 = 0.1358712338844632
  
  rat: replaced 0.1476822245385299 by 6303644/42683837 = 0.1476822245385297
  
  rat: replaced 0.1600034427182252 by 9148317/57175751 = 0.1600034427182251
  
  rat: replaced 0.1728366562869992 by 22187021/128369881 = 0.1728366562869993
  
  rat: replaced 0.1861835819091898 by 17269805/92756863 = 0.1861835819091898
  
  rat: replaced 0.200045884878356 by 7864089/39311426 = 0.2000458848783557
  
  rat: replaced 0.2144251789507544 by 7172489/33449845 = 0.2144251789507545
  
  rat: replaced 0.2293230261839593 by 10351388/45138895 = 0.2293230261839595
  
  rat: replaced 0.244740936780663 by 8538850/34889341 = 0.2447409367806632
  
  rat: replaced 0.2606803689376539 by 18287762/70153967 = 0.260680368937654
  
  rat: replaced 0.277142728699999 by 8575365/30942053 = 0.2771427286999993
  
  rat: replaced 0.2941293698204409 by 4485287/15249368 = 0.2941293698204411
  
  rat: replaced 0.3116415936240235 by 9259347/29711525 = 0.311641593624023
  
  rat: replaced 0.3296806488779594 by 11717987/35543448 = 0.3296806488779592
  
  rat: replaced 0.3482477316667564 by 36675861/105315434 = 0.3482477316667564
  
  rat: replaced 0.3673439852726074 by 9533778/25953271 = 0.3673439852726078
  
  rat: replaced 0.386970500061066 by 7376119/19061192 = 0.3869705000610665
  
  rat: replaced 0.4071283133720089 by 28281643/69466166 = 0.4071283133720091
  
  rat: replaced 0.4278184094159029 by 15961498/37309049 = 0.4278184094159034
  
  rat: replaced 0.4490417191753848 by 13063519/29091994 = 0.4490417191753855
  
  rat: replaced 0.4707991203121662 by 38805045/82423784 = 0.4707991203121662
  
  rat: replaced 0.493091437079264 by 14163447/28723774 = 0.493091437079264
  
  rat: replaced 0.5159194402385774 by 24505042/47497807 = 0.5159194402385777
  
  rat: replaced 0.5392838469838155 by 46178693/85629661 = 0.5392838469838156
  
  rat: replaced 0.5631853208687732 by 40347511/71641624 = 0.5631853208687732
  
  rat: replaced 0.58762447174098 by 29058549/49450883 = 0.5876244717409799
  
  rat: replaced 0.6126018556807142 by 18023233/29420794 = 0.6126018556807135
  
  rat: replaced 0.6381179749453973 by 17107844/26809845 = 0.6381179749453979
  
  rat: replaced 0.6641732779193661 by 21974679/33085762 = 0.6641732779193661
  
  rat: replaced 0.6907681590690352 by 10003471/14481662 = 0.690768159069035
  
  rat: replaced 0.7179029589034526 by 28073639/39105061 = 0.7179029589034525
  
  rat: replaced 0.7455779639402473 by 23235768/31164773 = 0.7455779639402476
  
  rat: replaced 0.773793406676978 by 8384139/10835113 = 0.7737934066769769
  
  rat: replaced 0.8025494655678836 by 7868837/9804800 = 0.8025494655678851
  
  rat: replaced 0.8318462650060389 by 30346636/36481063 = 0.8318462650060389
  
  rat: replaced 0.861683875310914 by 16277728/18890603 = 0.8616838753109152
  
  rat: replaced 0.8920623127213426 by 37841947/42420744 = 0.8920623127213422
  
  rat: replaced 0.9229815393938994 by 70218740/76078163 = 0.9229815393938994
  
  rat: replaced 0.954441463406683 by 24594815/25768804 = 0.9544414634066836
  
  rat: replaced 0.9864419387685092 by 15639887/15854848 = 0.9864419387685079
  
  rat: replaced 1.018982765433508 by 45306349/44462331 = 1.018982765433508
  
  rat: replaced 1.052063689321131 by 30825646/29300171 = 1.052063689321131
  
  rat: replaced 1.085684402341557 by 38378733/35349806 = 1.085684402341557
  
  rat: replaced 1.119844542426501 by 24894563/22230374 = 1.119844542426502
  
  rat: replaced 1.15454369356542 by 13440326/11641245 = 1.154543693565422
  
  rat: replaced 1.189781385847118 by 34027123/28599475 = 1.189781385847118
  
  rat: replaced 1.22555709550673 by 35019680/28574499 = 1.225557095506731
  
  rat: replaced 1.261870244978105 by 35214941/27906943 = 1.261870244978105
  
  rat: replaced 1.298720202951555 by 33824443/26044442 = 1.298720202951555
  
  rat: replaced 1.336106284436991 by 5396397/4038898 = 1.336106284436992
  
  rat: replaced 1.374027750832421 by 21232969/15453086 = 1.374027750832423
  
  rat: replaced 1.412483809997805 by 11919739/8438850 = 1.412483809997808
  
  rat: replaced 1.451473616334275 by 8780993/6049709 = 1.451473616334273
  
  rat: replaced 1.490996270868687 by 44673937/29962474 = 1.490996270868687
  
  rat: replaced 1.531050821343523 by 64281527/41985234 = 1.531050821343523
  
  rat: replaced 1.571636262312116 by 16616467/10572718 = 1.571636262312113
  
  rat: replaced 1.612751535239189 by 65392401/40547102 = 1.612751535239189
  
  rat: replaced 1.654395528606713 by 27646066/16710675 = 1.654395528606714
  
  rat: replaced 1.696567078025054 by 7906291/4660170 = 1.696567078025051
  
  rat: replaced 1.739264966349412 by 21773512/12518801 = 1.739264966349413
  
  rat: replaced 1.782487923801538 by 19245269/10796858 = 1.782487923801536
  
  rat: replaced 1.826234628096705 by 18221771/9977782 = 1.826234628096705
  
  rat: replaced 1.870503704575938 by 10770134/5757879 = 1.870503704575939
  
  rat: replaced 1.915293726343482 by 16780009/8761063 = 1.915293726343481
  
  rat: replaced 1.960603214409484 by 45722957/23320862 = 1.960603214409484
  
  rat: replaced 2.006430637837895 by 146296719/72913918 = 2.006430637837895
  
  rat: replaced 2.052774413899562 by 58550872/28522799 = 2.052774413899562
  
  rat: replaced 2.099632908230499 by 93949097/44745487 = 2.099632908230499
  
  rat: replaced 2.147004434995322 by 25274650/11772053 = 2.147004434995323
  
  rat: replaced 2.194887257055829 by 28867932/13152353 = 2.194887257055829
  
  rat: replaced 2.243279586144718 by 38403199/17119221 = 2.24327958614472
  
  rat: replaced 2.292179583044406 by 15612340/6811133 = 2.292179583044407
  
  rat: replaced 2.341585357770954 by 20809175/8886789 = 2.341585357770956
  
  rat: replaced 2.391494969763059 by 22142156/9258709 = 2.391494969763063
  
  rat: replaced 2.441906428076114 by 36070003/14771247 = 2.441906428076113
  
  rat: replaced 2.492817691581298 by 26575204/10660709 = 2.492817691581301
  part: invalid index of list or matrix.
  #0: lineIntersection(g=[3/2,-3,9/4],h=[3,9/2,9])
   -- an error. To debug this try: debugmode(true);
  
  Error in:
    perpendicular(B,lineThrough(A,C))); $H ...
                                      ^
\end{euleroutput}
\begin{eulercomment}
\end{eulercomment}
\begin{eulerprompt}
>el &= lineThrough(H,O); $getLineEquation(el,x,y)
\end{eulerprompt}
\begin{eulercomment}
\end{eulercomment}
\begin{eulerprompt}
>plotPoint(H(),"H"); plotLine(el(),"Garis Euler"):
\end{eulerprompt}
\begin{euleroutput}
  Function H not found.
  Try list ... to find functions!
  Error in:
  plotPoint(H(),"H"); plotLine(el(),"Garis Euler"): ...
               ^
\end{euleroutput}
\begin{eulercomment}
\end{eulercomment}
\begin{eulerprompt}
>M &= (A+B+C)/3; $getLineEquation(el,x,y) with [x=M[1],y=M[2]]
>plotPoint(M(),"M"): // titik berat
\end{eulerprompt}
\begin{euleroutput}
  Variable C not found!
  Use global variables or parameters for string evaluation.
  Error in expression: (C+B+A)/3
  Error in:
  plotPoint(M(),"M"): // titik berat ...
               ^
\end{euleroutput}
\begin{eulerprompt}
>$distance(M,H)/distance(M,O)|radcan
\end{eulerprompt}
\begin{eulercomment}
\end{eulercomment}
\begin{eulerprompt}
>$computeAngle(A,C,B), degprint(%())
\end{eulerprompt}
\begin{euleroutput}
  Variable or function A not found.
  Error in expression: computeAngle(A,C,B)
  Error in:
   $computeAngle(A,C,B), degprint(%()) ...
                                    ^
\end{euleroutput}
\begin{eulerprompt}
>Q &= lineIntersection(angleBisector(A,C,B),angleBisector(C,B,A))|radcan; $Q
>r &= distance(Q,projectToLine(Q,lineThrough(A,B)))|ratsimp; $r
>LD &=  circleWithCenter(Q,r); // Lingkaran dalam
\end{eulerprompt}
\begin{eulercomment}
\end{eulercomment}
\begin{eulerprompt}
>color(5); plotCircle(LD()):
\end{eulerprompt}
\begin{euleroutput}
  Variable or function A not found.
  Error in expression: circleWithCenter(lineIntersection(angleBisector(A,C,B),angleBisector(C,B,A)),distance(lineIntersection(angleBisector(A,C,B),angleBisector(C,B,A)),projectToLine(lineIntersection(angleBisector(A,C,B),angleBisector(C,B,A)),lineThrough(A,B))))
  Error in:
  color(5); plotCircle(LD()): ...
                           ^
\end{euleroutput}
\eulersubheading{contoh lain dari materi trigonometri rasional}
\begin{eulerprompt}
>A&:=[2,3]; B&:=[5,4]; C&:=[0,5]; ...
>setPlotRange(-1,5,1,7); ...
>plotPoint(A,"A"); plotPoint(B,"B"); plotPoint(C,"C"); ...
>plotSegment(B,A,"c"); plotSegment(A,C,"b"); plotSegment(C,B,"a"); ...
>insimg;
\end{eulerprompt}
\begin{euleroutput}
  Function setPlotRange not found.
  Try list ... to find functions!
  Error in:
  ... ,3]; B&:=[5,4]; C&:=[0,5]; setPlotRange(-1,5,1,7); plotPoint(A ...
                                                       ^
\end{euleroutput}
\begin{eulerprompt}
>$distance(A,B)
>c &= quad(A,B); $c, b &= quad(A,C); $b, a &= quad(B,C); $a,
\end{eulerprompt}
\begin{eulercomment}
\end{eulercomment}
\begin{eulerprompt}
>wb &= computeAngle(A,B,C); $wb, $(wb/pi*180)()
\end{eulerprompt}
\begin{euleroutput}
  Function computeAngle not found.
  Try list ... to find functions!
  Error in expression: 180*computeAngle([2,3],[5,4],[0,5])/pi
  Error in:
  wb &= computeAngle(A,B,C); $wb, $(wb/pi*180)() ...
                                                ^
\end{euleroutput}
\begin{eulerprompt}
>$crosslaw(a,b,c,x), $solve(%,x), //(b+c-a)^=4b.c(1-x)
>sb &= spread(b,a,c); $sb
>$sin(computeAngle(A,B,C))^2
>ha &= c*sb; $ha
>$sqrt(ha)
>$sqrt(ha)*sqrt(a)/2
\end{eulerprompt}
\begin{eulercomment}
\end{eulercomment}
\begin{eulerprompt}
>$areaTriangle(B,A,C)
\end{eulerprompt}
\eulersubheading{Aturan penyebaran 3 kali lipat}
\begin{eulerprompt}
>setPlotRange(1); ...
>color(1); plotCircle(circleWithCenter([0,0],1)); ...
>A:=[cos(1),sin(1)]; B:=[cos(2),sin(2)]; C:=[cos(6),sin(6)]; ...
>plotPoint(A,"A"); plotPoint(B,"B"); plotPoint(C,"C"); ...
>color(3); plotSegment(A,B,"c"); plotSegment(A,C,"b"); plotSegment(C,B,"a"); ...
>color(1); O:=[0,0];  plotPoint(O,"0"); ...
>plotSegment(A,O); plotSegment(B,O); plotSegment(C,O,"r"); ...
>insimg;
\end{eulerprompt}
\begin{euleroutput}
  Function setPlotRange not found.
  Try list ... to find functions!
  Error in:
  setPlotRange(1); color(1); plotCircle(circleWithCenter([0,0],1 ...
                 ^
\end{euleroutput}
\begin{eulerprompt}
>&remvalue(a,b,c,r); // hapus nilai-nilai sebelumnya untuk perhitungan baru
>rabc &= rhs(solve(triplespread(spread(b,r,r),spread(a,r,r),spread(c,r,r)),r)[4]); $rabc
\end{eulerprompt}
\begin{euleroutput}
  Maxima said:
  part: invalid index of list or matrix.
   -- an error. To debug this try: debugmode(true);
  
  Error in:
  ... spread(b,r,r),spread(a,r,r),spread(c,r,r)),r)[4]); $rabc ...
                                                       ^
\end{euleroutput}
\begin{eulercomment}
\end{eulercomment}
\begin{eulerprompt}
>function periradius(a,b,c) &= rabc;
\end{eulerprompt}
\begin{eulercomment}
\end{eulercomment}
\begin{eulerprompt}
>a:=quadrance(B,C); b:=quadrance(A,C); c:=quadrance(A,B);
\end{eulerprompt}
\begin{euleroutput}
  Function quadrance not found.
  Try list ... to find functions!
  Error in:
  a:=quadrance(B,C); b:=quadrance(A,C); c:=quadrance(A,B); ...
                   ^
\end{euleroutput}
\begin{eulercomment}
\end{eulercomment}
\begin{eulerprompt}
>periradius(a,b,c)
\end{eulerprompt}
\begin{euleroutput}
  Variable rabc not found!
  Use global or local variables defined in function periradius.
  Try "trace errors" to inspect local variables after errors.
  periradius:
      useglobal; return rabc 
  Error in:
  periradius(a,b,c) ...
                   ^
\end{euleroutput}
\begin{eulerprompt}
>$spread(b,a,c)*rabc | ratsimp
>$doublespread(b/(4*r))-spread(b,r,r) | ratsimp
\end{eulerprompt}
\eulersubheading{Contoh 6: Jarak Minimal pada Bidang}
\begin{eulercomment}
\end{eulercomment}
\eulersubheading{Catatan awal}
\begin{eulercomment}
Fungsi yang, ke titik M di bidang, menetapkan jarak AM antara titik
tetap A dan M, memiliki garis level yang agak sederhana: lingkaran
berpusat di A.
\end{eulercomment}
\begin{eulerprompt}
>&remvalue();
>A=[-2,-2];
>function d1(x,y):=sqrt((x-A[1])^2+(y-A[2])^2)
>fcontour("d1",xmin=-2,xmax=0,ymin=-2,ymax=0,hue=1, ...
>title="If you see ellipses, please set your window square"):
\end{eulerprompt}
\begin{eulercomment}
dan grafiknya juga agak sederhana: bagian atas kerucut:
\end{eulercomment}
\begin{eulerprompt}
>plot3d("d1",xmin=-2,xmax=0,ymin=-2,ymax=0):
\end{eulerprompt}
\begin{eulercomment}
Ternyata setelah mencoba yang bisa hanya dengan memasukkan angka 1,
karena ketika memakai angka 2, plot tidak membentuk kerucut diatas.

\end{eulercomment}
\eulersubheading{Dua poin}
\begin{eulercomment}
\end{eulercomment}
\begin{eulerprompt}
>B=[2,-2];
>function d2(x,y):=d1(x,y)+sqrt((x-B[1])^2+(y-B[2])^2)
>fcontour("d2",xmin=-2,xmax=2,ymin=-3,ymax=1,hue=1):
\end{eulerprompt}
\begin{eulercomment}
Grafiknya lebih menarik:
\end{eulercomment}
\begin{eulerprompt}
>plot3d("d2",xmin=-2,xmax=2,ymin=-3,ymax=1):
\end{eulerprompt}
\begin{eulercomment}
Pembatasan garis (AB) lebih terkenal:
\end{eulercomment}
\begin{eulerprompt}
>plot2d("abs(x+1)+abs(x-1)",xmin=-3,xmax=3):
\end{eulerprompt}
\eulersubheading{Tiga poin}
\begin{eulercomment}
Contoh:
\end{eulercomment}
\begin{eulerprompt}
>C=[-3,2];
>function d3(x,y):=d2(x,y)+sqrt((x-C[1])^2+(y-C[2])^2)
>plot3d("d3",xmin=-5,xmax=3,ymin=-4,ymax=4);
>insimg;
>fcontour("d3",xmin=-4,xmax=1,ymin=-2,ymax=2,hue=1,title="The minimum is on A");
>P=(A_B_C_A)'; plot2d(P[1],P[2],add=1,color=12);
>insimg;
\end{eulerprompt}
\begin{eulercomment}
Tetapi jika semua sudut segitiga ABC kurang dari 120 °, minimumnya
adalah pada titik F di bagian dalam segitiga, yang merupakan
satu-satunya titik yang melihat sisi-sisi ABC dengan sudut yang sama
(maka masing-masing 120 ° ):
\end{eulercomment}
\begin{eulerprompt}
>C=[-1,2];
>plot3d("d3",xmin=-2,xmax=2,ymin=-2,ymax=2):
>fcontour("d3",xmin=-2,xmax=2,ymin=-2,ymax=2,hue=1,title="The Fermat point");
>P=(A_B_C_A)'; plot2d(P[1],P[2],add=1,color=12);
>insimg;
\end{eulerprompt}
\begin{eulercomment}
\end{eulercomment}
\eulersubheading{Empat poin}
\begin{eulercomment}
Langkah selanjutnya adalah menambahkan 4 titik D dan mencoba
meminimalkan MA+MB+MC+MD; katakan bahwa Anda adalah operator TV kabel
dan ingin mencari di bidang mana Anda harus meletakkan antena sehingga
Anda dapat memberi makan empat desa dan menggunakan panjang kabel
sesedikit mungkin!
\end{eulercomment}
\begin{eulerprompt}
>D=[2,21];
>function d4(x,y):=d3(x,y)+sqrt((x-D[1])^2+(y-D[2])^2)
>plot3d("d4",xmin=-1.5,xmax=1.5,ymin=-1.5,ymax=1.5):
>fcontour("d4",xmin=-1.5,xmax=1.5,ymin=-1.5,ymax=1.5,hue=1);
>P=(A_B_C_D)'; plot2d(P[1],P[2],points=1,add=1,color=12);
>insimg;
\end{eulerprompt}
\eulersubheading{Contoh 7: Bola Dandelin dengan Povray}
\begin{eulercomment}
\end{eulercomment}
\begin{eulerprompt}
>load geometry;
\end{eulerprompt}
\begin{eulercomment}
Pertama dua garis yang membentuk kerucut.
\end{eulercomment}
\begin{eulerprompt}
>g1 &= lineThrough([0,0],[2,a])
\end{eulerprompt}
\begin{euleroutput}
  
                               [- a, 2, 0]
  
\end{euleroutput}
\begin{eulerprompt}
>g2 &= lineThrough([0,0],[-2,a])
\end{eulerprompt}
\begin{euleroutput}
  
                              [- a, - 2, 0]
  
\end{euleroutput}
\begin{eulercomment}
\end{eulercomment}
\begin{eulerprompt}
>g &= lineThrough([-2,0],[2,2])
\end{eulerprompt}
\begin{euleroutput}
  
                               [- 2, 4, 4]
  
\end{euleroutput}
\begin{eulercomment}
\end{eulercomment}
\begin{eulerprompt}
>setPlotRange(-2,2,0,3);
>color(black); plotLine(g(),"")
>a:=2; color(blue); plotLine(g1(),""), plotLine(g2(),""):
\end{eulerprompt}
\begin{eulercomment}
Sekarang kita ambil titik umum pada sumbu y.
\end{eulercomment}
\begin{eulerprompt}
>P &= [0,u]
\end{eulerprompt}
\begin{euleroutput}
  
                                  [0, u]
  
\end{euleroutput}
\begin{eulercomment}
Hitung jarak ke g1.
\end{eulercomment}
\begin{eulerprompt}
>d1 &= distance(P,projectToLine(P,g1)); $d1
\end{eulerprompt}
\begin{eulercomment}
Hitung jarak ke g.
\end{eulercomment}
\begin{eulerprompt}
>d &= distance(P,projectToLine(P,g)); $d
\end{eulerprompt}
\begin{eulercomment}
Dan temukan pusat kedua lingkaran yang jaraknya sama.
\end{eulercomment}
\begin{eulerprompt}
>sol &= solve(d1^2=d^2,u); $sol
\end{eulerprompt}
\begin{eulercomment}
Ada dua solusi.

\end{eulercomment}
\begin{eulerprompt}
>u := sol()
\end{eulerprompt}
\begin{euleroutput}
  [0.558482,  4.77485]
\end{euleroutput}
\begin{eulerprompt}
>dd := d()
\end{eulerprompt}
\begin{euleroutput}
  [0.394906,  3.37633]
\end{euleroutput}
\begin{eulercomment}
Plot lingkaran ke dalam gambar.
\end{eulercomment}
\begin{eulerprompt}
>color(red);
>plotCircle(circleWithCenter([0,u[1]],dd[1]),"");
>plotCircle(circleWithCenter([0,u[2]],dd[2]),"");
>insimg;
\end{eulerprompt}
\eulersubheading{Latihan}
\begin{eulercomment}
1. Gambarlah segi-n beraturan jika diketahui titik pusat O, n, dan
jarak titik pusat ke titik-titik sudut segi-n tersebut (jari-jari
lingkaran luar segi-n), r.

Petunjuk:

- Besar sudut pusat yang menghadap masing-masing sisi segi-n adalah
(360/n).\\
- Titik-titik sudut segi-n merupakan perpotongan lingkaran luar segi-n
dan garis-garis yang melalui pusat dan saling membentuk sudut sebesar
kelipatan (360/n).\\
- Untuk n ganjil, pilih salah satu titik sudut adalah di atas.\\
- Untuk n genap, pilih 2 titik di kanan dan kiri lurus dengan titik
pusat.\\
- Anda dapat menggambar segi-3, 4, 5, 6, 7, dst beraturan.

Penyelesaian :
\end{eulercomment}
\begin{eulerprompt}
>load geometry
\end{eulerprompt}
\begin{euleroutput}
  Numerical and symbolic geometry.
\end{euleroutput}
\begin{eulerprompt}
>setPlotRange(-3.5,3.5,-3.5,3.5);
>A=[-2,-2]; plotPoint(A,"A");
>B=[2,-2]; plotPoint(B,"B");
>C=[0,3]; plotPoint(C,"C");
>plotSegment(A,B,"c");
>plotSegment(B,C,"a");
>plotSegment(A,C,"b");
>aspect(1):
>c=circleThrough(A,B,C);
>R=getCircleRadius(c);
>O=getCircleCenter(c);
>plotPoint(O,"O");
>l=angleBisector(A,C,B);
>color(2); plotLine(l); color(1);
>plotCircle(c,"Lingkaran luar segitiga ABC"):
\end{eulerprompt}
\begin{eulercomment}
2. Gambarlah suatu parabola yang melalui 3 titik yang diketahui.

Petunjuk:\\
- Misalkan persamaan parabolanya y= ax\textasciicircum{}2+bx+c.\\
- Substitusikan koordinat titik-titik yang diketahui ke persamaan
tersebut.\\
- Selesaikan SPL yang terbentuk untuk mendapatkan nilai-nilai a, b, c.

Penyelesaian :
\end{eulercomment}
\begin{eulerprompt}
>load geometry;
>setPlotRange(5); P=[2,0]; Q=[4,0]; R=[0,-4];
>plotPoint(P,"P"); plotPoint(Q,"Q"); plotPoint(R,"R"):
>sol &= solve([a+b=-c,16*a+4*b=-c,c=-4],[a,b,c])
\end{eulerprompt}
\begin{euleroutput}
  
                       [[a = - 1, b = 5, c = - 4]]
  
\end{euleroutput}
\begin{eulercomment}
Sehingga didapatkan nilai a = -1, b = 5 dan c = -4
\end{eulercomment}
\begin{eulerprompt}
>function y&=-x^2+5*x-4
\end{eulerprompt}
\begin{euleroutput}
  
                                 2
                              - x  + 5 x - 4
  
\end{euleroutput}
\begin{eulerprompt}
>plot2d("-x^2+5*x-4",-5,5,-5,5):
\end{eulerprompt}
\begin{eulercomment}
3. Gambarlah suatu segi-4 yang diketahui keempat titik sudutnya,
misalnya A, B, C, D.\\
\end{eulercomment}
\begin{eulerttcomment}
   - Tentukan apakah segi-4 tersebut merupakan segi-4 garis singgung
\end{eulerttcomment}
\begin{eulercomment}
(sisinya-sisintya merupakan garis singgung lingkaran yang sama yakni
lingkaran dalam segi-4 tersebut).\\
\end{eulercomment}
\begin{eulerttcomment}
   - Suatu segi-4 merupakan segi-4 garis singgung apabila keempat
\end{eulerttcomment}
\begin{eulercomment}
garis bagi sudutnya bertemu di satu titik.\\
\end{eulercomment}
\begin{eulerttcomment}
   - Jika segi-4 tersebut merupakan segi-4 garis singgung, gambar
\end{eulerttcomment}
\begin{eulercomment}
lingkaran dalamnya.\\
\end{eulercomment}
\begin{eulerttcomment}
   - Tunjukkan bahwa syarat suatu segi-4 merupakan segi-4 garis
\end{eulerttcomment}
\begin{eulercomment}
singgung apabila hasil kali panjang sisi-sisi yang berhadapan sama.

Penyelesaian :
\end{eulercomment}
\begin{eulerprompt}
>load geometry
\end{eulerprompt}
\begin{euleroutput}
  Numerical and symbolic geometry.
\end{euleroutput}
\begin{eulerprompt}
>setPlotRange(-4.5,4.5,-4.5,4.5);
>A=[-3,-3]; plotPoint(A,"A");
>B=[3,-3]; plotPoint(B,"B");
>C=[3,3]; plotPoint(C,"C");
>D=[-3,3]; plotPoint(D,"D");
>plotSegment(A,B,"");
>splotSegment(B,C,"");
\end{eulerprompt}
\begin{euleroutput}
  Function splotSegment not found.
  Try list ... to find functions!
  Error in:
  splotSegment(B,C,""); ...
                      ^
\end{euleroutput}
\begin{eulerprompt}
>plotSegment(C,D,"");
>plotSegment(A,D,"");
>aspect(1):
>l=angleBisector(A,B,C);
>m=angleBisector(B,C,D);
>P=lineIntersection(l,m);
>color(5); plotLine(l); plotLine(m); color(1);
>plotPoint(P,"P"):
\end{eulerprompt}
\begin{eulercomment}
Dari gambar diatas terlihat bahwa keempat garis bagi sudutnya bertemu
di satu titik yaitu titik P.
\end{eulercomment}
\begin{eulerprompt}
>r=norm(P-projectToLine(P,lineThrough(A,B)));
>plotCircle(circleWithCenter(P,r),"Lingkaran dalam segiempat ABCD"):
\end{eulerprompt}
\begin{eulercomment}
Dari gambar diatas, terlihat bahwa sisi-sisinya merupakan garis
singgung lingkaran yang sama yaitu lingkaran dalam segiempat.\\
Akan ditunjukkan bahwa hasil kali panjang sisi-sisi yang berhadapan
sama.
\end{eulercomment}
\begin{eulerprompt}
>AB=norm(A-B) //panjang sisi AB
\end{eulerprompt}
\begin{euleroutput}
  6
\end{euleroutput}
\begin{eulerprompt}
>CD=norm(C-D) //panjang sisi CD
\end{eulerprompt}
\begin{euleroutput}
  6
\end{euleroutput}
\begin{eulerprompt}
>AD=norm(A-D) //panjang sisi AD
\end{eulerprompt}
\begin{euleroutput}
  6
\end{euleroutput}
\begin{eulerprompt}
>BC=norm(B-C) //panjang sisi BC
\end{eulerprompt}
\begin{euleroutput}
  6
\end{euleroutput}
\begin{eulerprompt}
>AB.CD
\end{eulerprompt}
\begin{euleroutput}
  36
\end{euleroutput}
\begin{eulerprompt}
>AD.BC
\end{eulerprompt}
\begin{euleroutput}
  36
\end{euleroutput}
\begin{eulercomment}
Terbukti bahwa hasil kali panjang sisi-sisi yang berhadapan sama yaitu
36. Jadi dapat dipastikan bahwa segiempat tersebut merupakan segiempat
garis singgung.


4. Gambarlah suatu ellips jika diketahui kedua titik fokusnya,
misalnya P dan Q. Ingat ellips dengan fokus P dan Q adalah tempat
kedudukan titik-titik yang jumlah jarak ke P dan ke Q selalu sama
(konstan).

Penyelesaian :\\
Diketahui kedua titik fokus P = [-1,-1] dan Q = [1,-1]
\end{eulercomment}
\begin{eulerprompt}
>P=[-1,-1]; Q=[1,-1];
>function d1(x,y):=sqrt((x-P[1])^2+(y-P[2])^2)
>Q=[1,-1]; function d2(x,y):=sqrt((x-P[1])^2+(y-P[2])^2)+sqrt((x-Q[1])^2+(y-Q[2])^2)
>fcontour("d2",xmin=-2,xmax=2,ymin=-3,ymax=1,hue=1):
\end{eulerprompt}
\begin{eulercomment}
Grafik yang lebih menarik
\end{eulercomment}
\begin{eulerprompt}
>plot3d("d2",xmin=-2,xmax=2,ymin=-3,ymax=1):
\end{eulerprompt}
\begin{eulercomment}
Batasan ke garis PQ
\end{eulercomment}
\begin{eulerprompt}
>plot2d("abs(x+1)+abs(x-1)",xmin=-3,xmax=3):
\end{eulerprompt}
\begin{eulercomment}
5. Gambarlah suatu hiperbola jika diketahui kedua titik fokusnya,
misalnya P dan Q. Ingat ellips dengan fokus P dan Q adalah tempat
kedudukan titik-titik yang selisih jarak ke P dan ke Q selalu sama
(konstan).

Penyelesaian :
\end{eulercomment}
\begin{eulerprompt}
>P=[-1,-1]; Q=[1,-1];
>function d1(x,y):=sqrt((x-p[1])^2+(y-p[2])^2)
>Q=[1,-1]; function d2(x,y):=sqrt((x-P[1])^2+(y-P[2])^2)+sqrt((x+Q[1])^2+(y+Q[2])^2)
>fcontour("d2",xmin=-2,xmax=2,ymin=-3,ymax=1,hue=1):
\end{eulerprompt}
\begin{eulercomment}
Grafik yang lebih menarik
\end{eulercomment}
\begin{eulerprompt}
>plot3d("d2",xmin=-2,xmax=2,ymin=-3,ymax=1):
>plot2d("abs(x+1)+abs(x-1)",xmin=-3,xmax=3):
\end{eulerprompt}
\begin{eulercomment}
\begin{eulercomment}
\eulerheading{EMT untuk Statistika}
\begin{eulercomment}
Dalam buku catatan ini, kami mendemonstrasikan plot statistik utama,
pengujian, dan distribusi di Euler.

Mari kita mulai dengan beberapa statistik deskriptif. Ini bukan
pengantar statistik. Jadi, Anda mungkin memerlukan beberapa latar
belakang untuk memahami detailnya.

Asumsikan pengukuran berikut. Kami ingin menghitung nilai rata-rata
dan standar deviasi yang diukur.
\end{eulercomment}
\begin{eulerprompt}
>M=[1000,1004,998,997,1002,1001,998,1004,998,997]; ...
>mean(M), dev(M),
\end{eulerprompt}
\begin{euleroutput}
  999.9
  2.72641400622
\end{euleroutput}
\begin{eulercomment}
Kita dapat memplot plot kotak-dan-kumis untuk data. Dalam kasus kami
tidak ada outlier.
\end{eulercomment}
\begin{eulerprompt}
>aspect(1.75); boxplot(M):
\end{eulerprompt}
\eulerimg{15}{images/Davina Safa Felisa 1-6-669.png}
\begin{eulercomment}
Kami menghitung probabilitas bahwa suatu nilai lebih besar dari 1005,
dengan asumsi nilai terukur dan distribusi normal.

Semua fungsi untuk distribusi di Euler diakhiri dengan ...dis dan
menghitung distribusi probabilitas kumulatif (CPF).

\end{eulercomment}
\begin{eulerformula}
\[
\text{normaldis(x,m,d)}=\int_{-\infty}^x \frac{1}{d\sqrt{2\pi}}e^{-\frac{1}{2 }(\frac{t-m}{d})^2}\ dt.
\]
\end{eulerformula}
\begin{eulercomment}
Kami mencetak hasilnya dalam \% dengan akurasi 2 digit menggunakan
fungsi cetak.
\end{eulercomment}
\begin{eulerprompt}
>print((1-normaldis(1005,mean(M),dev(M)))*100,2,unit=" %")
\end{eulerprompt}
\begin{euleroutput}
        3.07 %
\end{euleroutput}
\begin{eulercomment}
Untuk contoh berikutnya, kami mengasumsikan jumlah pria berikut dalam
rentang ukuran yang diberikan.
\end{eulercomment}
\begin{eulerprompt}
>r=155.5:4:187.5; v=[22,71,136,169,139,71,32,8];
\end{eulerprompt}
\begin{eulercomment}
Berikut adalah plot distribusinya.
\end{eulercomment}
\begin{eulerprompt}
>plot2d(r,v,a=150,b=200,c=0,d=190,bar=1,style="\(\backslash\)/"):
\end{eulerprompt}
\eulerimg{15}{images/Davina Safa Felisa 1-6-671.png}
\begin{eulercomment}
Kita bisa memasukkan data mentah tersebut ke dalam sebuah tabel.

Tabel adalah metode untuk menyimpan data statistik. Tabel kita harus
berisi tiga kolom: Awal jangkauan, akhir jangkauan, jumlah orang dalam
jangkauan.

Tabel dapat dicetak dengan header. Kami menggunakan vektor string
untuk mengatur header.
\end{eulercomment}
\begin{eulerprompt}
>T:=r[1:8]' | r[2:9]' | v'; writetable(T,labc=["from","to","count"])
\end{eulerprompt}
\begin{euleroutput}
        from        to     count
       155.5     159.5        22
       159.5     163.5        71
       163.5     167.5       136
       167.5     171.5       169
       171.5     175.5       139
       175.5     179.5        71
       179.5     183.5        32
       183.5     187.5         8
\end{euleroutput}
\begin{eulercomment}
Jika kita membutuhkan nilai rata-rata dan statistik lain dari ukuran,
kita perlu menghitung titik tengah rentang. Kita dapat menggunakan dua
kolom pertama dari tabel kita untuk ini.

Sumbul "\textbar{}" digunakan untuk memisahkan kolom, fungsi "writetable"
digunakan untuk menulis tabel, dengan opsion "labc" adalah untuk
menentukan header kolom.
\end{eulercomment}
\begin{eulerprompt}
>(T[,1]+T[,2])/2 // the midpoint of each interval
\end{eulerprompt}
\begin{euleroutput}
          157.5 
          161.5 
          165.5 
          169.5 
          173.5 
          177.5 
          181.5 
          185.5 
\end{euleroutput}
\begin{eulercomment}
Tetapi lebih mudah, untuk melipat rentang dengan vektor [1/2.1/2].
\end{eulercomment}
\begin{eulerprompt}
>M=fold(r,[0.5,0.5])
\end{eulerprompt}
\begin{euleroutput}
  [157.5,  161.5,  165.5,  169.5,  173.5,  177.5,  181.5,  185.5]
\end{euleroutput}
\begin{eulercomment}
Sekarang kita dapat menghitung mean dan deviasi sampel dengan
frekuensi yang diberikan.
\end{eulercomment}
\begin{eulerprompt}
>\{m,d\}=meandev(M,v); m, d,
\end{eulerprompt}
\begin{euleroutput}
  169.901234568
  5.98912964449
\end{euleroutput}
\begin{eulercomment}
Mari kita tambahkan distribusi normal dari nilai-nilai ke plot batang
di atas. Rumus untuk distribusi normal dengan mean m dan standar
deviasi d adalah:

\end{eulercomment}
\begin{eulerformula}
\[
y=\frac{1}{d\sqrt{2\pi}}e^{\frac{-(x-m)^2}{2d^2}}.
\]
\end{eulerformula}
\begin{eulercomment}
Karena nilainya antara 0 dan 1, untuk memplotnya pada bar plot harus
dikalikan dengan 4 kali jumlah total data.
\end{eulercomment}
\begin{eulerprompt}
>plot2d("qnormal(x,m,d)*sum(v)*4", ...
>  xmin=min(r),xmax=max(r),thickness=3,add=1):
\end{eulerprompt}
\eulerimg{15}{images/Davina Safa Felisa 1-6-672.png}
\eulerheading{Meja}
\begin{eulercomment}
Di direktori notebook ini Anda menemukan file dengan tabel. Data
tersebut mewakili hasil survei. Berikut adalah empat baris pertama
dari file tersebut. Data berasal dari buku online Jerman "Einführung
in die Statistik mit R" oleh A. Handl.
\end{eulercomment}
\begin{eulerprompt}
>printfile("table.dat",4);
\end{eulerprompt}
\begin{euleroutput}
  Could not open the file
  table.dat
  for reading!
  Try "trace errors" to inspect local variables after errors.
  printfile:
      open(filename,"r");
\end{euleroutput}
\begin{eulercomment}
Tabel berisi 7 kolom angka atau token (string). Kami ingin membaca
tabel dari file. Pertama, kami menggunakan terjemahan kami sendiri
untuk token.

Untuk ini, kami mendefinisikan set token. Fungsi strtokens()
mendapatkan vektor string token dari string yang diberikan.
\end{eulercomment}
\begin{eulerprompt}
>mf:=["m","f"]; yn:=["y","n"]; ev:=strtokens("g vg m b vb");
\end{eulerprompt}
\begin{eulercomment}
Sekarang kita membaca tabel dengan terjemahan ini.

Argumen tok2, tok4 dll. adalah terjemahan dari kolom tabel. Argumen
ini tidak ada dalam daftar parameter readtable(), jadi Anda harus
menyediakannya dengan ":=".
\end{eulercomment}
\begin{eulerprompt}
>\{MT,hd\}=readtable("table.dat",tok2:=mf,tok4:=yn,tok5:=ev,tok7:=yn);
\end{eulerprompt}
\begin{euleroutput}
  Could not open the file
  table.dat
  for reading!
  Try "trace errors" to inspect local variables after errors.
  readtable:
      if filename!=none then open(filename,"r"); endif;
\end{euleroutput}
\begin{eulerprompt}
>load over statistics;
\end{eulerprompt}
\begin{eulercomment}
Untuk mencetak, kita perlu menentukan set token yang sama. Kami
mencetak empat baris pertama saja.
\end{eulercomment}
\begin{eulerprompt}
>writetable(MT[1:4],labc=hd,wc=5,tok2:=mf,tok4:=yn,tok5:=ev,tok7:=yn);
\end{eulerprompt}
\begin{euleroutput}
  MT is not a variable!
  Error in:
  writetable(MT[1:4],labc=hd,wc=5,tok2:=mf,tok4:=yn,tok5:=ev,tok ...
                    ^
\end{euleroutput}
\begin{eulercomment}
Titik "." mewakili nilai-nilai, yang tidak tersedia.

Jika kita tidak ingin menentukan token untuk terjemahan terlebih
dahulu, kita hanya perlu menentukan, kolom mana yang berisi token dan
bukan angka.
\end{eulercomment}
\begin{eulerprompt}
>ctok=[2,4,5,7]; \{MT,hd,tok\}=readtable("table.dat",ctok=ctok);
\end{eulerprompt}
\begin{euleroutput}
  Could not open the file
  table.dat
  for reading!
  Try "trace errors" to inspect local variables after errors.
  readtable:
      if filename!=none then open(filename,"r"); endif;
\end{euleroutput}
\begin{eulercomment}
Fungsi readtable() sekarang mengembalikan satu set token.
\end{eulercomment}
\begin{eulerprompt}
>tok
\end{eulerprompt}
\begin{euleroutput}
  Variable tok not found!
  Error in:
  tok ...
     ^
\end{euleroutput}
\begin{eulercomment}
Tabel berisi entri dari file dengan token yang diterjemahkan ke angka.

String khusus NA="." ditafsirkan sebagai "Tidak Tersedia", dan
mendapatkan NAN (bukan angka) dalam tabel. Terjemahan ini dapat diubah
dengan parameter NA, dan NAval.
\end{eulercomment}
\begin{eulerprompt}
>MT[1]
\end{eulerprompt}
\begin{euleroutput}
  MT is not a variable!
  Error in:
  MT[1] ...
       ^
\end{euleroutput}
\begin{eulercomment}
Berikut isi tabel dengan angka yang belum diterjemahkan.
\end{eulercomment}
\begin{eulerprompt}
>writetable(MT,wc=5)
\end{eulerprompt}
\begin{euleroutput}
  Variable or function MT not found.
  Error in:
  writetable(MT,wc=5) ...
               ^
\end{euleroutput}
\begin{eulercomment}
Untuk kenyamanan, Anda dapat memasukkan output readtable() ke dalam
daftar.
\end{eulercomment}
\begin{eulerprompt}
>Table=\{\{readtable("table.dat",ctok=ctok)\}\};
\end{eulerprompt}
\begin{euleroutput}
  Could not open the file
  table.dat
  for reading!
  Try "trace errors" to inspect local variables after errors.
  readtable:
      if filename!=none then open(filename,"r"); endif;
\end{euleroutput}
\begin{eulercomment}
Menggunakan kolom token yang sama dan token yang dibaca dari file,
kita dapat mencetak tabel. Kita dapat menentukan ctok, tok, dll. Atau
menggunakan daftar Tabel.
\end{eulercomment}
\begin{eulerprompt}
>writetable(Table,ctok=ctok,wc=5);
\end{eulerprompt}
\begin{euleroutput}
  Variable or function Table not found.
  Error in:
  writetable(Table,ctok=ctok,wc=5); ...
                  ^
\end{euleroutput}
\begin{eulercomment}
Fungsi tablecol() mengembalikan nilai kolom tabel, melewatkan setiap
baris dengan nilai NAN("." dalam file), dan indeks kolom, yang berisi
nilai-nilai ini.
\end{eulercomment}
\begin{eulerprompt}
>\{c,i\}=tablecol(MT,[5,6]);
\end{eulerprompt}
\begin{euleroutput}
  Variable or function MT not found.
  Error in:
  \{c,i\}=tablecol(MT,[5,6]); ...
                   ^
\end{euleroutput}
\begin{eulercomment}
Kita dapat menggunakan ini untuk mengekstrak kolom dari tabel untuk
tabel baru.
\end{eulercomment}
\begin{eulerprompt}
>j=[1,5,6]; writetable(MT[i,j],labc=hd[j],ctok=[2],tok=tok)
\end{eulerprompt}
\begin{euleroutput}
  Variable or function i not found.
  Error in:
  j=[1,5,6]; writetable(MT[i,j],labc=hd[j],ctok=[2],tok=tok) ...
                            ^
\end{euleroutput}
\begin{eulercomment}
Tentu saja, kita perlu mengekstrak tabel itu sendiri dari daftar Tabel
dalam kasus ini.
\end{eulercomment}
\begin{eulerprompt}
>MT=Table[1];
\end{eulerprompt}
\begin{euleroutput}
  Table is not a variable!
  Error in:
  MT=Table[1]; ...
             ^
\end{euleroutput}
\begin{eulercomment}
Tentu saja, kita juga dapat menggunakannya untuk menentukan nilai
rata-rata kolom atau nilai statistik lainnya.
\end{eulercomment}
\begin{eulerprompt}
>mean(tablecol(MT,6))
\end{eulerprompt}
\begin{euleroutput}
  Variable or function MT not found.
  Error in:
  mean(tablecol(MT,6)) ...
                  ^
\end{euleroutput}
\begin{eulercomment}
Fungsi getstatistics() mengembalikan elemen dalam vektor, dan
jumlahnya. Kami menerapkannya pada nilai "m" dan "f" di kolom kedua
tabel kami.
\end{eulercomment}
\begin{eulerprompt}
>\{xu,count\}=getstatistics(tablecol(MT,2)); xu, count,
\end{eulerprompt}
\begin{euleroutput}
  Variable or function MT not found.
  Error in:
  \{xu,count\}=getstatistics(tablecol(MT,2)); xu, count, ...
                                      ^
\end{euleroutput}
\begin{eulercomment}
Kami dapat mencetak hasilnya dalam tabel baru.
\end{eulercomment}
\begin{eulerprompt}
>writetable(count',labr=tok[xu])
\end{eulerprompt}
\begin{euleroutput}
  Variable count not found!
  Error in:
  writetable(count',labr=tok[xu]) ...
                   ^
\end{euleroutput}
\begin{eulercomment}
Fungsi selecttable() mengembalikan tabel baru dengan nilai dalam satu
kolom yang dipilih dari vektor indeks. Pertama kita mencari indeks
dari dua nilai kita di tabel token.
\end{eulercomment}
\begin{eulerprompt}
>v:=indexof(tok,["g","vg"])
\end{eulerprompt}
\begin{euleroutput}
  Variable or function tok not found.
  Error in:
  v:=indexof(tok,["g","vg"]) ...
                ^
\end{euleroutput}
\begin{eulercomment}
Sekarang kita dapat memilih baris tabel, yang memiliki salah satu
nilai dalam v di baris ke-5.
\end{eulercomment}
\begin{eulerprompt}
>MT1:=MT[selectrows(MT,5,v)]; i:=sortedrows(MT1,5);
\end{eulerprompt}
\begin{euleroutput}
  Variable or function MT not found.
  Error in:
  MT1:=MT[selectrows(MT,5,v)]; i:=sortedrows(MT1,5); ...
                       ^
\end{euleroutput}
\begin{eulercomment}
Sekarang kita dapat mencetak tabel, dengan nilai yang diekstrak dan
diurutkan di kolom ke-5.
\end{eulercomment}
\begin{eulerprompt}
>writetable(MT1[i],labc=hd,ctok=ctok,tok=tok,wc=7);
\end{eulerprompt}
\begin{euleroutput}
  Variable or function i not found.
  Error in:
  writetable(MT1[i],labc=hd,ctok=ctok,tok=tok,wc=7); ...
                  ^
\end{euleroutput}
\begin{eulercomment}
Untuk statistik berikutnya, kami ingin menghubungkan dua kolom tabel.
Jadi kami mengekstrak kolom 2 dan 4 dan mengurutkan tabel.
\end{eulercomment}
\begin{eulerprompt}
>i=sortedrows(MT,[2,4]);  ...
>  writetable(tablecol(MT[i],[2,4])',ctok=[1,2],tok=tok)
\end{eulerprompt}
\begin{euleroutput}
  Variable or function MT not found.
  Error in:
  i=sortedrows(MT,[2,4]);    writetable(tablecol(MT[i],[2,4])',c ...
                 ^
\end{euleroutput}
\begin{eulercomment}
Dengan getstatistics(), kita juga bisa menghubungkan hitungan dalam
dua kolom tabel satu sama lain.
\end{eulercomment}
\begin{eulerprompt}
>MT24=tablecol(MT,[2,4]); ...
>\{xu1,xu2,count\}=getstatistics(MT24[1],MT24[2]); ...
>writetable(count,labr=tok[xu1],labc=tok[xu2])
\end{eulerprompt}
\begin{euleroutput}
  Variable or function MT not found.
  Error in:
  MT24=tablecol(MT,[2,4]); \{xu1,xu2,count\}=getstatistics(MT24[1] ...
                  ^
\end{euleroutput}
\begin{eulercomment}
Tabel dapat ditulis ke file.
\end{eulercomment}
\begin{eulerprompt}
>filename="test.dat"; ...
>writetable(count,labr=tok[xu1],labc=tok[xu2],file=filename);
\end{eulerprompt}
\begin{euleroutput}
  Variable or function count not found.
  Error in:
  filename="test.dat"; writetable(count,labr=tok[xu1],labc=tok[x ...
                                       ^
\end{euleroutput}
\begin{eulercomment}
Kemudian kita bisa membaca tabel dari file tersebut.
\end{eulercomment}
\begin{eulerprompt}
>\{MT2,hd,tok2,hdr\}=readtable(filename,>clabs,>rlabs); ...
>writetable(MT2,labr=hdr,labc=hd)
\end{eulerprompt}
\begin{euleroutput}
  Could not open the file
  test.dat
  for reading!
  Try "trace errors" to inspect local variables after errors.
  readtable:
      if filename!=none then open(filename,"r"); endif;
\end{euleroutput}
\begin{eulercomment}
Dan hapus file tersebut.
\end{eulercomment}
\begin{eulerprompt}
>fileremove(filename);
\end{eulerprompt}
\eulerheading{Distribusi}
\begin{eulercomment}
Dengan plot2d, terdapat metode yang sangat mudah untuk memplot sebaran
data eksperimen.
\end{eulercomment}
\begin{eulerprompt}
>p=normal(1,1000); //1000 random normal-distributed sample p
>plot2d(p,distribution=20,style="\(\backslash\)/"); // plot the random sample p
>plot2d("qnormal(x,0,1)",add=1): // add the standard normal distribution plot
\end{eulerprompt}
\eulerimg{15}{images/Davina Safa Felisa 1-6-673.png}
\begin{eulercomment}
Harap perhatikan perbedaan antara plot batang (sampel) dan kurva
normal (distribusi nyata). Masukkan kembali tiga perintah untuk
melihat hasil pengambilan sampel lainnya.
\end{eulercomment}
\begin{eulercomment}
Berikut adalah perbandingan 10 simulasi dari 1000 nilai terdistribusi
normal menggunakan apa yang disebut plot kotak. Plot ini menunjukkan
median, kuartil 25\% dan 75\%, nilai minimal dan maksimal, dan outlier.
\end{eulercomment}
\begin{eulerprompt}
>p=normal(10,1000); boxplot(p):
\end{eulerprompt}
\eulerimg{15}{images/Davina Safa Felisa 1-6-674.png}
\begin{eulercomment}
Untuk membangkitkan bilangan bulat acak, Euler memiliki intrarandom.
Mari kita simulasikan lemparan dadu dan plot distribusinya.

Kita menggunakan fungsi getmultiplicities(v,x), yang menghitung
seberapa sering elemen v muncul di x. Kemudian kita memplot hasilnya
menggunakan columnplot().
\end{eulercomment}
\begin{eulerprompt}
>k=intrandom(1,6000,6);  ...
>columnsplot(getmultiplicities(1:6,k));  ...
>ygrid(1000,color=red):
\end{eulerprompt}
\eulerimg{15}{images/Davina Safa Felisa 1-6-675.png}
\begin{eulercomment}
Sementara intrandom(n,m,k) mengembalikan bilangan bulat yang
terdistribusi secara seragam dari 1 ke k, dimungkinkan untuk
menggunakan distribusi bilangan bulat lain yang diberikan dengan
randpint().

Dalam contoh berikut, probabilitas untuk 1,2,3 masing-masing adalah
0,4,0,1,0,5.
\end{eulercomment}
\begin{eulerprompt}
>randpint(1,1000,[0.4,0.1,0.5]); getmultiplicities(1:3,%)
\end{eulerprompt}
\begin{euleroutput}
  [378,  102,  520]
\end{euleroutput}
\begin{eulercomment}
Euler dapat menghasilkan nilai acak dari lebih banyak distribusi. Coba
lihat referensinya.

Misalnya, kami mencoba distribusi eksponensial. Variabel acak kontinu
X dikatakan memiliki distribusi eksponensial, jika PDF-nya diberikan
oleh\\
\end{eulercomment}
\begin{eulerformula}
\[
f_X(x)=\lambda e^{-\lambda x},\quad x>0,\quad \lambda>0,
\]
\end{eulerformula}
\begin{eulercomment}
dengan parameter\\
\end{eulercomment}
\begin{eulerformula}
\[
\lambda=\frac{1}{\mu},\quad \mu \text{ adalah rata-rata, dan dilambangkan dengan } X \sim \text{Eksponensial}(\lambda).
\]
\end{eulerformula}
\begin{eulerprompt}
>plot2d(randexponential(1,1000,2),>distribution):
\end{eulerprompt}
\eulerimg{15}{images/Davina Safa Felisa 1-6-678.png}
\begin{eulercomment}
Untuk banyak distribusi, Euler dapat menghitung fungsi distribusi dan
inversnya.
\end{eulercomment}
\begin{eulerprompt}
>plot2d("normaldis",-4,4): 
\end{eulerprompt}
\eulerimg{15}{images/Davina Safa Felisa 1-6-679.png}
\begin{eulercomment}
Berikut ini adalah salah satu cara untuk memplot kuantil.
\end{eulercomment}
\begin{eulerprompt}
>plot2d("qnormal(x,1,1.5)",-4,6);  ...
>plot2d("qnormal(x,1,1.5)",a=2,b=5,>add,>filled):
\end{eulerprompt}
\eulerimg{15}{images/Davina Safa Felisa 1-6-680.png}
\begin{eulerformula}
\[
\text{normaldis(x,m,d)}=\int_{-\infty}^x \frac{1}{d\sqrt{2\pi}}e^{-\frac{1}{2 }(\frac{t-m}{d})^2}\ dt.
\]
\end{eulerformula}
\begin{eulercomment}
Probabilitas untuk berada di area hijau adalah sebagai berikut.
\end{eulercomment}
\begin{eulerprompt}
>normaldis(5,1,1.5)-normaldis(2,1,1.5)
\end{eulerprompt}
\begin{euleroutput}
  0.248662156979
\end{euleroutput}
\begin{eulercomment}
Ini dapat dihitung secara numerik dengan integral berikut.\\
\end{eulercomment}
\begin{eulerformula}
\[
\int_2^5 \frac{1}{1.5\sqrt{2\pi}}e^{-\frac{1}{2}(\frac{x-1}{1.5})^2}\ dx .
\]
\end{eulerformula}
\begin{eulerprompt}
>gauss("qnormal(x,1,1.5)",2,5)
\end{eulerprompt}
\begin{euleroutput}
  0.248662156979
\end{euleroutput}
\begin{eulercomment}
Mari kita bandingkan distribusi binomial dengan distribusi normal mean
dan deviasi yang sama. Fungsi invbindis() memecahkan interpolasi
linier antara nilai bilangan bulat.
\end{eulercomment}
\begin{eulerprompt}
>invbindis(0.95,1000,0.5), invnormaldis(0.95,500,0.5*sqrt(1000))
\end{eulerprompt}
\begin{euleroutput}
  525.516721219
  526.007419394
\end{euleroutput}
\begin{eulercomment}
Fungsi qdis() adalah kepadatan distribusi chi-kuadrat. Seperti biasa,
Euler memetakan vektor ke fungsi ini. Jadi kita mendapatkan plot dari
semua distribusi chi-kuadrat dengan derajat 5 sampai 30 dengan mudah
dengan cara berikut.
\end{eulercomment}
\begin{eulerprompt}
>plot2d("qchidis(x,(5:5:50)')",0,50):
\end{eulerprompt}
\eulerimg{15}{images/Davina Safa Felisa 1-6-683.png}
\begin{eulercomment}
Euler memiliki fungsi yang akurat untuk mengevaluasi distribusi. Mari
kita periksa chidis() dengan integral.

Penamaan mencoba untuk konsisten. Misalnya.,

- distribusi chi-kuadrat adalah chidis(),\\
- fungsi kebalikannya adalah invchidis(),\\
- densitasnya adalah qchidis().

Pelengkap distribusi (ekor atas) adalah chicdis().
\end{eulercomment}
\begin{eulerprompt}
>chidis(1.5,2), integrate("qchidis(x,2)",0,1.5)
\end{eulerprompt}
\begin{euleroutput}
  0.527633447259
  0.527633447259
\end{euleroutput}
\eulerheading{Distribusi Diskrit}
\begin{eulercomment}
Untuk menentukan distribusi diskrit Anda sendiri, Anda dapat
menggunakan metode berikut.

Pertama kita mengatur fungsi distribusi.
\end{eulercomment}
\begin{eulerprompt}
>wd = 0|((1:6)+[-0.01,0.01,0,0,0,0])/6
\end{eulerprompt}
\begin{euleroutput}
  [0,  0.165,  0.335,  0.5,  0.666667,  0.833333,  1]
\end{euleroutput}
\begin{eulercomment}
Artinya dengan probabilitas wd[i+1]-wd[i] kita menghasilkan nilai acak
i.

Ini hampir merupakan distribusi yang seragam. Mari kita tentukan
generator angka acak untuk ini. Fungsi find(v,x) menemukan nilai x
dalam vektor v. Fungsi ini juga berlaku untuk vektor x.
\end{eulercomment}
\begin{eulerprompt}
>function wrongdice (n,m) := find(wd,random(n,m))
\end{eulerprompt}
\begin{eulercomment}
Kesalahannya sangat halus sehingga kami melihatnya hanya dengan
iterasi yang sangat banyak.
\end{eulercomment}
\begin{eulerprompt}
>columnsplot(getmultiplicities(1:6,wrongdice(1,1000000))):
\end{eulerprompt}
\eulerimg{15}{images/Davina Safa Felisa 1-6-684.png}
\begin{eulercomment}
Berikut adalah fungsi sederhana untuk memeriksa distribusi seragam
dari nilai 1...K dalam v. Kami menerima hasilnya, jika untuk semua
frekuensi

\end{eulercomment}
\begin{eulerformula}
\[
\left|f_i-\frac{1}{K}\right| < \frac{\delta}{\sqrt{n}}.
\]
\end{eulerformula}
\begin{eulerprompt}
>function checkrandom (v, delta=1) ...
\end{eulerprompt}
\begin{eulerudf}
    K=max(v); n=cols(v);
    fr=getfrequencies(v,1:K);
    return max(fr/n-1/K)<delta/sqrt(n);
    endfunction
\end{eulerudf}
\begin{eulercomment}
Memang fungsi menolak distribusi seragam.
\end{eulercomment}
\begin{eulerprompt}
>checkrandom(wrongdice(1,1000000))
\end{eulerprompt}
\begin{euleroutput}
  0
\end{euleroutput}
\begin{eulercomment}
Dan itu menerima generator acak bawaan.
\end{eulercomment}
\begin{eulerprompt}
>checkrandom(intrandom(1,1000000,6))
\end{eulerprompt}
\begin{euleroutput}
  1
\end{euleroutput}
\begin{eulercomment}
Kita dapat menghitung distribusi binomial. Pertama ada binomialsum(),
yang mengembalikan probabilitas i atau kurang hit dari n percobaan.
\end{eulercomment}
\begin{eulerprompt}
>bindis(410,1000,0.4)
\end{eulerprompt}
\begin{euleroutput}
  0.751401349654
\end{euleroutput}
\begin{eulercomment}
Fungsi Beta terbalik digunakan untuk menghitung interval kepercayaan
Clopper-Pearson untuk parameter p. Level default adalah alfa.

Arti interval ini adalah jika p berada di luar interval, hasil
pengamatan 410 dalam 1000 jarang terjadi.
\end{eulercomment}
\begin{eulerprompt}
>clopperpearson(410,1000)
\end{eulerprompt}
\begin{euleroutput}
  [0.37932,  0.441212]
\end{euleroutput}
\begin{eulercomment}
Perintah berikut adalah cara langsung untuk mendapatkan hasil di atas.
Tapi untuk n besar, penjumlahan langsungnya tidak akurat dan lambat.
\end{eulercomment}
\begin{eulerprompt}
>p=0.4; i=0:410; n=1000; sum(bin(n,i)*p^i*(1-p)^(n-i))
\end{eulerprompt}
\begin{euleroutput}
  0.751401349655
\end{euleroutput}
\begin{eulercomment}
Omong-omong, invbinsum() menghitung kebalikan dari binomialsum().
\end{eulercomment}
\begin{eulerprompt}
>invbindis(0.75,1000,0.4)
\end{eulerprompt}
\begin{euleroutput}
  409.932733047
\end{euleroutput}
\begin{eulercomment}
Di Bridge, kami mengasumsikan 5 kartu beredar (dari 52) dengan dua
tangan (26 kartu). Mari kita hitung probabilitas distribusi yang lebih
buruk dari 3:2 (mis. 0:5, 1:4, 4:1, atau 5:0).
\end{eulercomment}
\begin{eulerprompt}
>2*hypergeomsum(1,5,13,26)
\end{eulerprompt}
\begin{euleroutput}
  0.321739130435
\end{euleroutput}
\begin{eulercomment}
Ada juga simulasi distribusi multinomial.
\end{eulercomment}
\begin{eulerprompt}
>randmultinomial(10,1000,[0.4,0.1,0.5])
\end{eulerprompt}
\begin{euleroutput}
            381           100           519 
            376            91           533 
            417            80           503 
            440            94           466 
            406           112           482 
            408            94           498 
            395           107           498 
            399            96           505 
            428            87           485 
            400            99           501 
\end{euleroutput}
\eulerheading{Merencanakan Data}
\begin{eulercomment}
Untuk memplot data, kami mencoba hasil pemilu Jerman sejak 1990, yang
diukur dalam jumlah kursi.
\end{eulercomment}
\begin{eulerprompt}
>BW := [ ...
>1990,662,319,239,79,8,17; ...
>1994,672,294,252,47,49,30; ...
>1998,669,245,298,43,47,36; ...
>2002,603,248,251,47,55,2; ...
>2005,614,226,222,61,51,54; ...
>2009,622,239,146,93,68,76; ...
>2013,631,311,193,0,63,64];
\end{eulerprompt}
\begin{eulercomment}
Untuk partai, kami menggunakan rangkaian nama.
\end{eulercomment}
\begin{eulerprompt}
>P:=["CDU/CSU","SPD","FDP","Gr","Li"];
\end{eulerprompt}
\begin{eulercomment}
Mari kita cetak persentasenya dengan baik.

Pertama, kami mengekstrak kolom yang diperlukan. Kolom 3 sampai 7
adalah kursi masing-masing partai, dan kolom 2 adalah jumlah kursi.
kolom adalah tahun pemilihan.
\end{eulercomment}
\begin{eulerprompt}
>BT:=BW[,3:7]; BT:=BT/sum(BT); YT:=BW[,1]';
\end{eulerprompt}
\begin{eulercomment}
Kemudian kami mencetak statistik dalam bentuk tabel. Kami menggunakan
nama sebagai tajuk kolom, dan tahun sebagai tajuk untuk baris. Lebar
default untuk kolom adalah wc=10, tetapi kami lebih memilih hasil yang
lebih padat. Kolom akan diperluas untuk label kolom, jika perlu.
\end{eulercomment}
\begin{eulerprompt}
>writetable(BT*100,wc=6,dc=0,>fixed,labc=P,labr=YT)
\end{eulerprompt}
\begin{euleroutput}
         CDU/CSU   SPD   FDP    Gr    Li
    1990      48    36    12     1     3
    1994      44    38     7     7     4
    1998      37    45     6     7     5
    2002      41    42     8     9     0
    2005      37    36    10     8     9
    2009      38    23    15    11    12
    2013      49    31     0    10    10
\end{euleroutput}
\begin{eulercomment}
Perkalian matriks berikut mengekstrak jumlah persentase dari dua
partai besar yang menunjukkan bahwa partai-partai kecil telah
mendapatkan rekaman di parlemen hingga tahun 2009.
\end{eulercomment}
\begin{eulerprompt}
>BT1:=(BT.[1;1;0;0;0])'*100
\end{eulerprompt}
\begin{euleroutput}
  [84.29,  81.25,  81.1659,  82.7529,  72.9642,  61.8971,  79.8732]
\end{euleroutput}
\begin{eulercomment}
Ada juga plot statistik sederhana. Kami menggunakannya untuk
menampilkan garis dan titik secara bersamaan. Alternatifnya adalah
memanggil plot2d dua kali dengan \textgreater{}add.
\end{eulercomment}
\begin{eulerprompt}
>statplot(YT,BT1,"b"):
\end{eulerprompt}
\eulerimg{15}{images/Davina Safa Felisa 1-6-686.png}
\begin{eulercomment}
Tentukan beberapa warna untuk masing-masing pihak.
\end{eulercomment}
\begin{eulerprompt}
>CP:=[rgb(0.5,0.5,0.5),red,yellow,green,rgb(0.8,0,0)];
\end{eulerprompt}
\begin{eulercomment}
Sekarang kita bisa memplot hasil pemilu 2009 dan mengubahnya menjadi
satu plot menggunakan gambar. Kita dapat menambahkan vektor kolom ke
setiap plot.
\end{eulercomment}
\begin{eulerprompt}
>figure(2,1);  ...
>figure(1); columnsplot(BW[6,3:7],P,color=CP); ...
>figure(2); columnsplot(BW[6,3:7]-BW[5,3:7],P,color=CP);  ...
>figure(0):
\end{eulerprompt}
\eulerimg{15}{images/Davina Safa Felisa 1-6-687.png}
\begin{eulercomment}
Plot data menggabungkan deretan data statistik dalam satu plot.
\end{eulercomment}
\begin{eulerprompt}
>J:=BW[,1]'; DP:=BW[,3:7]'; ...
>dataplot(YT,BT',color=CP);  ...
>labelbox(P,colors=CP,styles="[]",>points,w=0.2,x=0.3,y=0.4):
\end{eulerprompt}
\eulerimg{15}{images/Davina Safa Felisa 1-6-688.png}
\begin{eulercomment}
Plot kolom 3D menunjukkan deretan data statistik dalam bentuk kolom.
Kami menyediakan label untuk baris dan kolom. sudut adalah sudut
pandang.
\end{eulercomment}
\begin{eulerprompt}
>columnsplot3d(BT,scols=P,srows=YT, ...
>  angle=30°,ccols=CP):
\end{eulerprompt}
\eulerimg{15}{images/Davina Safa Felisa 1-6-689.png}
\begin{eulercomment}
Representasi lainnya adalah plot mozaik. Perhatikan bahwa kolom plot
mewakili kolom matriks di sini. Karena panjang label CDU/CSU, kami
mengambil jendela yang lebih kecil dari biasanya.
\end{eulercomment}
\begin{eulerprompt}
>shrinkwindow(>smaller);  ...
>mosaicplot(BT',srows=YT,scols=P,color=CP,style="#"); ...
>shrinkwindow():
\end{eulerprompt}
\eulerimg{15}{images/Davina Safa Felisa 1-6-690.png}
\begin{eulercomment}
Kita juga bisa membuat diagram lingkaran. Karena hitam dan kuning
membentuk koalisi, kami menyusun ulang elemennya.
\end{eulercomment}
\begin{eulerprompt}
>i=[1,3,5,4,2]; piechart(BW[6,3:7][i],color=CP[i],lab=P[i]):
\end{eulerprompt}
\eulerimg{15}{images/Davina Safa Felisa 1-6-691.png}
\begin{eulercomment}
Ini jenis plot lainnya.
\end{eulercomment}
\begin{eulerprompt}
>starplot(normal(1,10)+4,lab=1:10,>rays):
\end{eulerprompt}
\eulerimg{15}{images/Davina Safa Felisa 1-6-692.png}
\begin{eulercomment}
Beberapa plot di plot2d bagus untuk statika. Berikut adalah plot
impuls dari data acak, terdistribusi secara seragam di [0,1].
\end{eulercomment}
\begin{eulerprompt}
>plot2d(makeimpulse(1:10,random(1,10)),>bar):
\end{eulerprompt}
\eulerimg{15}{images/Davina Safa Felisa 1-6-693.png}
\begin{eulercomment}
Tetapi untuk data yang terdistribusi secara eksponensial, kita mungkin
memerlukan plot logaritmik.
\end{eulercomment}
\begin{eulerprompt}
>logimpulseplot(1:10,-log(random(1,10))*10):
\end{eulerprompt}
\eulerimg{15}{images/Davina Safa Felisa 1-6-694.png}
\begin{eulercomment}
Fungsi columnplot() lebih mudah digunakan, karena hanya membutuhkan
vektor nilai. Selain itu, ia dapat mengatur labelnya ke apa pun yang
kita inginkan, kita telah mendemonstrasikannya di tutorial ini.

Ini adalah aplikasi lain, di mana kami menghitung karakter dalam
sebuah kalimat dan memplot statistik.
\end{eulercomment}
\begin{eulerprompt}
>v=strtochar("the quick brown fox jumps over the lazy dog"); ...
>w=ascii("a"):ascii("z"); x=getmultiplicities(w,v); ...
>cw=[]; for k=w; cw=cw|char(k); end; ...
>columnsplot(x,lab=cw,width=0.05):
\end{eulerprompt}
\eulerimg{15}{images/Davina Safa Felisa 1-6-695.png}
\begin{eulercomment}
Dimungkinkan juga untuk mengatur sumbu secara manual.
\end{eulercomment}
\begin{eulerprompt}
>n=10; p=0.4; i=0:n; x=bin(n,i)*p^i*(1-p)^(n-i); ...
>columnsplot(x,lab=i,width=0.05,<frame,<grid); ...
>yaxis(0,0:0.1:1,style="->",>left); xaxis(0,style="."); ...
>label("p",0,0.25), label("i",11,0); ...
>textbox(["Binomial distribution","with p=0.4"]):
\end{eulerprompt}
\eulerimg{15}{images/Davina Safa Felisa 1-6-696.png}
\begin{eulercomment}
Berikut ini adalah cara memplot frekuensi bilangan dalam vektor.

Kami membuat vektor bilangan acak bilangan bulat 1 hingga 6.
\end{eulercomment}
\begin{eulerprompt}
>v:=intrandom(1,10,10)
\end{eulerprompt}
\begin{euleroutput}
  [8,  5,  8,  8,  6,  8,  8,  3,  5,  5]
\end{euleroutput}
\begin{eulercomment}
Kemudian ekstrak angka unik di v.
\end{eulercomment}
\begin{eulerprompt}
>vu:=unique(v)
\end{eulerprompt}
\begin{euleroutput}
  [3,  5,  6,  8]
\end{euleroutput}
\begin{eulercomment}
Dan plot frekuensi dalam plot kolom.
\end{eulercomment}
\begin{eulerprompt}
>columnsplot(getmultiplicities(vu,v),lab=vu,style="/"):
\end{eulerprompt}
\eulerimg{15}{images/Davina Safa Felisa 1-6-697.png}
\begin{eulercomment}
Kami ingin menunjukkan fungsi untuk distribusi nilai empiris.
\end{eulercomment}
\begin{eulerprompt}
>x=normal(1,20);
\end{eulerprompt}
\begin{eulercomment}
Fungsi empdist(x,vs) membutuhkan array nilai yang diurutkan. Jadi kita
harus mengurutkan x sebelum kita dapat menggunakannya.
\end{eulercomment}
\begin{eulerprompt}
>xs=sort(x);
\end{eulerprompt}
\begin{eulercomment}
Kemudian kami memplot distribusi empiris dan beberapa batang kepadatan
menjadi satu plot. Alih-alih plot batang untuk distribusi, kali ini
kami menggunakan plot gigi gergaji.
\end{eulercomment}
\begin{eulerprompt}
>figure(2,1); ...
>figure(1); plot2d("empdist",-4,4;xs); ...
>figure(2); plot2d(histo(x,v=-4:0.2:4,<bar));  ...
>figure(0):
\end{eulerprompt}
\eulerimg{15}{images/Davina Safa Felisa 1-6-698.png}
\begin{eulercomment}
Plot pencar mudah dilakukan di Euler dengan plot titik biasa. Grafik
berikut menunjukkan bahwa X dan X+Y jelas berkorelasi positif.
\end{eulercomment}
\begin{eulerprompt}
>x=normal(1,100); plot2d(x,x+rotright(x),>points,style=".."):
\end{eulerprompt}
\eulerimg{15}{images/Davina Safa Felisa 1-6-699.png}
\begin{eulercomment}
Seringkali, kami ingin membandingkan dua sampel dari distribusi yang
berbeda. Ini dapat dilakukan dengan plot kuantil-kuantil.

Untuk pengujian, kami mencoba distribusi student-t dan distribusi
eksponensial.
\end{eulercomment}
\begin{eulerprompt}
>x=randt(1,1000,5); y=randnormal(1,1000,mean(x),dev(x)); ...
>plot2d("x",r=6,style="--",yl="normal",xl="student-t",>vertical); ...
>plot2d(sort(x),sort(y),>points,color=red,style="x",>add):
\end{eulerprompt}
\eulerimg{15}{images/Davina Safa Felisa 1-6-700.png}
\begin{eulercomment}
Plot jelas menunjukkan bahwa nilai terdistribusi normal cenderung
lebih kecil di ujung ekstrim.

Jika kita memiliki dua distribusi dengan ukuran berbeda, kita dapat
memperluas yang lebih kecil atau mengecilkan yang lebih besar. Fungsi
berikut ini baik untuk keduanya. Dibutuhkan nilai median dengan
persentase antara 0 dan 1.
\end{eulercomment}
\begin{eulerprompt}
>function medianexpand (x,n) := median(x,p=linspace(0,1,n-1));
\end{eulerprompt}
\begin{eulercomment}
Mari kita bandingkan dua distribusi yang sama.
\end{eulercomment}
\begin{eulerprompt}
>x=random(1000); y=random(400); ...
>plot2d("x",0,1,style="--"); ...
>plot2d(sort(medianexpand(x,400)),sort(y),>points,color=red,style="x",>add):
\end{eulerprompt}
\eulerimg{15}{images/Davina Safa Felisa 1-6-701.png}
\eulerheading{Regresi dan Korelasi}
\begin{eulercomment}
Regresi linier dapat dilakukan dengan fungsi polyfit() atau berbagai
fungsi fit.

Sebagai permulaan, kami menemukan garis regresi untuk data univariat
dengan polyfit(x,y,1).
\end{eulercomment}
\begin{eulerprompt}
>x=1:10; y=[2,3,1,5,6,3,7,8,9,8]; writetable(x'|y',labc=["x","y"])
\end{eulerprompt}
\begin{euleroutput}
           x         y
           1         2
           2         3
           3         1
           4         5
           5         6
           6         3
           7         7
           8         8
           9         9
          10         8
\end{euleroutput}
\begin{eulercomment}
Kami ingin membandingkan kecocokan yang tidak berbobot dan berbobot.
Pertama koefisien fit linier.
\end{eulercomment}
\begin{eulerprompt}
>p=polyfit(x,y,1)
\end{eulerprompt}
\begin{euleroutput}
  [0.733333,  0.812121]
\end{euleroutput}
\begin{eulercomment}
Sekarang koefisien dengan bobot yang menekankan nilai terakhir.
\end{eulercomment}
\begin{eulerprompt}
>w &= "exp(-(x-10)^2/10)"; pw=polyfit(x,y,1,w=w(x))
\end{eulerprompt}
\begin{euleroutput}
  [4.71566,  0.38319]
\end{euleroutput}
\begin{eulercomment}
Kami memasukkan semuanya ke dalam satu plot untuk poin dan garis
regresi, dan untuk bobot yang digunakan.
\end{eulercomment}
\begin{eulerprompt}
>figure(2,1);  ...
>figure(1); statplot(x,y,"b",xl="Regression"); ...
>  plot2d("evalpoly(x,p)",>add,color=blue,style="--"); ...
>  plot2d("evalpoly(x,pw)",5,10,>add,color=red,style="--"); ...
>figure(2); plot2d(w,1,10,>filled,style="/",fillcolor=red,xl=w); ...
>figure(0):
\end{eulerprompt}
\eulerimg{15}{images/Davina Safa Felisa 1-6-702.png}
\begin{eulercomment}
Sebagai contoh lain kita membaca survei siswa, umur mereka, umur orang
tua mereka dan jumlah saudara dari sebuah file.

Tabel ini berisi "m" dan "f" di kolom kedua. Kami menggunakan variabel
tok2 untuk menyetel terjemahan yang tepat alih-alih membiarkan
readtable() mengumpulkan terjemahan.
\end{eulercomment}
\begin{eulerprompt}
>\{MS,hd\}:=readtable("table1.dat",tok2:=["m","f"]);  ...
>writetable(MS,labc=hd,tok2:=["m","f"]);
\end{eulerprompt}
\begin{euleroutput}
  Could not open the file
  table1.dat
  for reading!
  Try "trace errors" to inspect local variables after errors.
  readtable:
      if filename!=none then open(filename,"r"); endif;
\end{euleroutput}
\begin{eulercomment}
Bagaimana usia bergantung satu sama lain? Kesan pertama berasal dari
sebar berpasangan.
\end{eulercomment}
\begin{eulerprompt}
>scatterplots(tablecol(MS,3:5),hd[3:5]):
\end{eulerprompt}
\begin{euleroutput}
  Variable or function MS not found.
  Error in:
  scatterplots(tablecol(MS,3:5),hd[3:5]): ...
                          ^
\end{euleroutput}
\begin{eulercomment}
Jelas terlihat bahwa usia ayah dan ibu saling bergantung. Mari kita
tentukan dan plot garis regresi.
\end{eulercomment}
\begin{eulerprompt}
>cs:=MS[,4:5]'; ps:=polyfit(cs[1],cs[2],1)
\end{eulerprompt}
\begin{euleroutput}
  MS is not a variable!
  Error in:
  cs:=MS[,4:5]'; ps:=polyfit(cs[1],cs[2],1) ...
              ^
\end{euleroutput}
\begin{eulercomment}
Ini jelas model yang salah. Garis regresi adalah s=17+0,74t, di mana t
adalah umur ibu dan s adalah umur ayah. Perbedaan usia mungkin sedikit
bergantung pada usia, tetapi tidak terlalu banyak.

Sebaliknya, kami mencurigai fungsi seperti s=a+t. Maka a adalah
rata-rata dari s-t. Ini adalah perbedaan usia rata-rata antara ayah
dan ibu.
\end{eulercomment}
\begin{eulerprompt}
>da:=mean(cs[2]-cs[1])
\end{eulerprompt}
\begin{euleroutput}
  cs is not a variable!
  Error in:
  da:=mean(cs[2]-cs[1]) ...
                ^
\end{euleroutput}
\begin{eulercomment}
Mari kita plot ini menjadi satu plot pencar.
\end{eulercomment}
\begin{eulerprompt}
>plot2d(cs[1],cs[2],>points);  ...
>plot2d("evalpoly(x,ps)",color=red,style=".",>add);  ...
>plot2d("x+da",color=blue,>add):
\end{eulerprompt}
\begin{euleroutput}
  cs is not a variable!
  Error in:
  plot2d(cs[1],cs[2],>points);  plot2d("evalpoly(x,ps)",color=re ...
              ^
\end{euleroutput}
\begin{eulercomment}
Berikut adalah plot kotak dari dua zaman. Ini hanya menunjukkan, bahwa
usianya berbeda.
\end{eulercomment}
\begin{eulerprompt}
>boxplot(cs,["mothers","fathers"]):
\end{eulerprompt}
\begin{euleroutput}
  Variable or function cs not found.
  Error in:
  boxplot(cs,["mothers","fathers"]): ...
            ^
\end{euleroutput}
\begin{eulercomment}
Menariknya, perbedaan median tidak sebesar perbedaan rata-rata.
\end{eulercomment}
\begin{eulerprompt}
>median(cs[2])-median(cs[1])
\end{eulerprompt}
\begin{euleroutput}
  cs is not a variable!
  Error in:
  median(cs[2])-median(cs[1]) ...
              ^
\end{euleroutput}
\begin{eulercomment}
Koefisien korelasi menunjukkan korelasi positif.
\end{eulercomment}
\begin{eulerprompt}
>correl(cs[1],cs[2])
\end{eulerprompt}
\begin{euleroutput}
  cs is not a variable!
  Error in:
  correl(cs[1],cs[2]) ...
              ^
\end{euleroutput}
\begin{eulercomment}
Korelasi peringkat adalah ukuran untuk urutan yang sama di kedua
vektor. Ini juga cukup positif.
\end{eulercomment}
\begin{eulerprompt}
>rankcorrel(cs[1],cs[2])
\end{eulerprompt}
\begin{euleroutput}
  cs is not a variable!
  Error in:
  rankcorrel(cs[1],cs[2]) ...
                  ^
\end{euleroutput}
\eulerheading{Membuat Fungsi baru}
\begin{eulercomment}
Tentu saja, bahasa EMT dapat digunakan untuk memprogram fungsi-fungsi
baru. Misalnya, kita mendefinisikan fungsi skewness.

\end{eulercomment}
\begin{eulerformula}
\[
\text{sk}(x) = \dfrac{\sqrt{n} \sum_i (x_i-m)^3}{\left(\sum_i (x_i-m)^2\right)^{3/2 }}
\]
\end{eulerformula}
\begin{eulercomment}
di mana m adalah rata-rata dari x.
\end{eulercomment}
\begin{eulerprompt}
>function skew (x:vector) ...
\end{eulerprompt}
\begin{eulerudf}
  m=mean(x);
  return sqrt(cols(x))*sum((x-m)^3)/(sum((x-m)^2))^(3/2);
  endfunction
\end{eulerudf}
\begin{eulercomment}
Seperti yang Anda lihat, kita dapat dengan mudah menggunakan bahasa
matriks untuk mendapatkan implementasi yang sangat singkat dan
efisien. Mari kita coba fungsi ini.
\end{eulercomment}
\begin{eulerprompt}
>data=normal(20); skew(normal(10))
\end{eulerprompt}
\begin{euleroutput}
  -0.198710316203
\end{euleroutput}
\begin{eulercomment}
Ini adalah fungsi lain, yang disebut koefisien kemiringan Pearson.
\end{eulercomment}
\begin{eulerprompt}
>function skew1 (x) := 3*(mean(x)-median(x))/dev(x)
>skew1(data)
\end{eulerprompt}
\begin{euleroutput}
  -0.0801873249135
\end{euleroutput}
\eulerheading{Simulasi Monte Carlo}
\begin{eulercomment}
Euler dapat digunakan untuk mensimulasikan kejadian acak. Kita telah
melihat contoh sederhana di atas. Ini satu lagi, yang mensimulasikan
1000 kali 3 lemparan dadu, dan meminta distribusi jumlahnya.
\end{eulercomment}
\begin{eulerprompt}
>ds:=sum(intrandom(1000,3,6))';  fs=getmultiplicities(3:18,ds)
\end{eulerprompt}
\begin{euleroutput}
  [5,  17,  35,  44,  75,  97,  114,  116,  143,  116,  104,  53,  40,
  22,  13,  6]
\end{euleroutput}
\begin{eulercomment}
Kita bisa merencanakan ini sekarang.
\end{eulercomment}
\begin{eulerprompt}
>columnsplot(fs,lab=3:18):
\end{eulerprompt}
\eulerimg{15}{images/Davina Safa Felisa 1-6-704.png}
\begin{eulercomment}
Untuk menentukan distribusi yang diharapkan tidak begitu mudah. Kami
menggunakan rekursi lanjutan untuk ini.

Fungsi berikut menghitung banyaknya cara bilangan k dapat dinyatakan
sebagai jumlah dari n bilangan dalam rentang 1 sampai m. Ini bekerja
secara rekursif dengan cara yang jelas.
\end{eulercomment}
\begin{eulerprompt}
>function map countways (k; n, m) ...
\end{eulerprompt}
\begin{eulerudf}
    if n==1 then return k>=1 && k<=m
    else
      sum=0; 
      loop 1 to m; sum=sum+countways(k-#,n-1,m); end;
      return sum;
    end;
  endfunction
\end{eulerudf}
\begin{eulercomment}
Inilah hasil lemparan dadu sebanyak tiga kali.
\end{eulercomment}
\begin{eulerprompt}
>cw=countways(3:18,3,6)
\end{eulerprompt}
\begin{euleroutput}
  [1,  3,  6,  10,  15,  21,  25,  27,  27,  25,  21,  15,  10,  6,  3,
  1]
\end{euleroutput}
\begin{eulercomment}
Kami menambahkan nilai yang diharapkan ke plot.
\end{eulercomment}
\begin{eulerprompt}
>plot2d(cw/6^3*1000,>add); plot2d(cw/6^3*1000,>points,>add):
\end{eulerprompt}
\eulerimg{15}{images/Davina Safa Felisa 1-6-705.png}
\begin{eulercomment}
Untuk simulasi lain, penyimpangan nilai rata-rata n 0-1-variabel acak
terdistribusi normal adalah 1/sqrt(n).
\end{eulercomment}
\begin{eulerprompt}
>longformat; 1/sqrt(10)
\end{eulerprompt}
\begin{euleroutput}
  0.316227766017
\end{euleroutput}
\begin{eulercomment}
Mari kita periksa ini dengan simulasi. Kami menghasilkan 10.000 kali
10 vektor acak.
\end{eulercomment}
\begin{eulerprompt}
>M=normal(10000,10); dev(mean(M)')
\end{eulerprompt}
\begin{euleroutput}
  0.319493614817
\end{euleroutput}
\begin{eulerprompt}
>plot2d(mean(M)',>distribution):
\end{eulerprompt}
\eulerimg{15}{images/Davina Safa Felisa 1-6-706.png}
\begin{eulercomment}
Median dari 10 bilangan acak terdistribusi 0-1-normal memiliki deviasi
yang lebih besar.
\end{eulercomment}
\begin{eulerprompt}
>dev(median(M)')
\end{eulerprompt}
\begin{euleroutput}
  0.374460271535
\end{euleroutput}
\begin{eulercomment}
Karena kita dapat dengan mudah membuat jalan acak, kita dapat
mensimulasikan proses Wiener. Kami mengambil 1000 langkah dari 1000
proses. Kami kemudian memplot standar deviasi dan rata-rata langkah
ke-n dari proses ini bersama dengan nilai yang diharapkan dalam warna
merah.
\end{eulercomment}
\begin{eulerprompt}
>n=1000; m=1000; M=cumsum(normal(n,m)/sqrt(m)); ...
>t=(1:n)/n; figure(2,1); ...
>figure(1); plot2d(t,mean(M')'); plot2d(t,0,color=red,>add); ...
>figure(2); plot2d(t,dev(M')'); plot2d(t,sqrt(t),color=red,>add); ...
>figure(0):
\end{eulerprompt}
\eulerimg{15}{images/Davina Safa Felisa 1-6-707.png}
\eulerheading{Tes}
\begin{eulercomment}
Tes adalah alat penting dalam statistik. Di Euler, banyak tes yang
diterapkan. Semua tes ini mengembalikan kesalahan yang kami terima
jika kami menolak hipotesis nol.

Sebagai contoh, kami menguji lemparan dadu untuk distribusi seragam.
Pada 600 lemparan, kami mendapat nilai berikut, yang kami masukkan ke
uji chi-square.
\end{eulercomment}
\begin{eulerprompt}
>chitest([90,103,114,101,103,89],dup(100,6)')
\end{eulerprompt}
\begin{euleroutput}
  0.498830517952
\end{euleroutput}
\begin{eulercomment}
Tes chi-kuadrat juga memiliki mode yang menggunakan simulasi Monte
Carlo untuk menguji statistik. Hasilnya harus hampir sama. Parameter
\textgreater{}p menginterpretasikan vektor-y sebagai vektor probabilitas.
\end{eulercomment}
\begin{eulerprompt}
>chitest([90,103,114,101,103,89],dup(1/6,6)',>p,>montecarlo)
\end{eulerprompt}
\begin{euleroutput}
  0.526
\end{euleroutput}
\begin{eulercomment}
Kesalahan ini terlalu besar. Jadi kita tidak bisa menolak pemerataan
distribusi. Ini tidak membuktikan bahwa dadu kami adil. Tapi kita
tidak bisa menolak hipotesis kita.

Selanjutnya kami menghasilkan 1000 lemparan dadu menggunakan generator
angka acak, dan melakukan pengujian yang sama.
\end{eulercomment}
\begin{eulerprompt}
>n=1000; t=random([1,n*6]); chitest(count(t*6,6),dup(n,6)')
\end{eulerprompt}
\begin{euleroutput}
  0.528028118442
\end{euleroutput}
\begin{eulercomment}
Mari kita uji nilai rata-rata 100 dengan uji-t.
\end{eulercomment}
\begin{eulerprompt}
>s=200+normal([1,100])*10; ...
>ttest(mean(s),dev(s),100,200)
\end{eulerprompt}
\begin{euleroutput}
  0.0218365848476
\end{euleroutput}
\begin{eulercomment}
Fungsi ttest() membutuhkan nilai rata-rata, simpangan, jumlah data,
dan nilai rata-rata untuk diuji.

Sekarang mari kita periksa dua pengukuran untuk rata-rata yang sama.
Kami menolak hipotesis bahwa mereka memiliki rata-rata yang sama, jika
hasilnya \textless{}0,05.
\end{eulercomment}
\begin{eulerprompt}
>tcomparedata(normal(1,10),normal(1,10))
\end{eulerprompt}
\begin{euleroutput}
  0.38722000942
\end{euleroutput}
\begin{eulercomment}
Jika kami menambahkan bias ke satu distribusi, kami mendapat lebih
banyak penolakan. Ulangi simulasi ini beberapa kali untuk melihat
efeknya.
\end{eulercomment}
\begin{eulerprompt}
>tcomparedata(normal(1,10),normal(1,10)+2)
\end{eulerprompt}
\begin{euleroutput}
  5.60009101758e-07
\end{euleroutput}
\begin{eulercomment}
Dalam contoh berikutnya, kami menghasilkan 20 lemparan dadu acak 100
kali dan menghitungnya. Harus ada rata-rata 20/6=3,3.
\end{eulercomment}
\begin{eulerprompt}
>R=random(100,20); R=sum(R*6<=1)'; mean(R)
\end{eulerprompt}
\begin{euleroutput}
  3.28
\end{euleroutput}
\begin{eulercomment}
Kami sekarang membandingkan jumlah satu dengan distribusi binomial.
Pertama kita memplot distribusi satuan.
\end{eulercomment}
\begin{eulerprompt}
>plot2d(R,distribution=max(R)+1,even=1,style="\(\backslash\)/"):
\end{eulerprompt}
\eulerimg{15}{images/Davina Safa Felisa 1-6-708.png}
\begin{eulerprompt}
>t=count(R,21);
\end{eulerprompt}
\begin{eulercomment}
Kemudian kami menghitung nilai yang diharapkan.
\end{eulercomment}
\begin{eulerprompt}
>n=0:20; b=bin(20,n)*(1/6)^n*(5/6)^(20-n)*100;
\end{eulerprompt}
\begin{eulercomment}
Kita harus mengumpulkan beberapa angka untuk mendapatkan kategori yang
cukup besar.
\end{eulercomment}
\begin{eulerprompt}
>t1=sum(t[1:2])|t[3:7]|sum(t[8:21]); ...
>b1=sum(b[1:2])|b[3:7]|sum(b[8:21]);
\end{eulerprompt}
\begin{eulercomment}
Uji chi-square menolak hipotesis bahwa distribusi kita adalah
distribusi binomial, jika hasilnya \textless{}0,05.
\end{eulercomment}
\begin{eulerprompt}
>chitest(t1,b1)
\end{eulerprompt}
\begin{euleroutput}
  0.53921579764
\end{euleroutput}
\begin{eulercomment}
Contoh berikut berisi hasil dari dua kelompok orang (pria dan wanita,
katakanlah) memilih satu dari enam partai.
\end{eulercomment}
\begin{eulerprompt}
>A=[23,37,43,52,64,74;27,39,41,49,63,76];  ...
>  writetable(A,wc=6,labr=["m","f"],labc=1:6)
\end{eulerprompt}
\begin{euleroutput}
             1     2     3     4     5     6
       m    23    37    43    52    64    74
       f    27    39    41    49    63    76
\end{euleroutput}
\begin{eulercomment}
Kami ingin menguji independensi suara dari jenis kelamin. Tes tabel
chi\textasciicircum{}2 melakukan ini. Hasilnya terlalu besar untuk menolak kemerdekaan.
Jadi kami tidak bisa mengatakan, jika pemungutan suara tergantung pada
jenis kelamin dari data tersebut.
\end{eulercomment}
\begin{eulerprompt}
>tabletest(A)
\end{eulerprompt}
\begin{euleroutput}
  0.990701632326
\end{euleroutput}
\begin{eulercomment}
Berikut adalah tabel yang diharapkan, jika kita mengasumsikan
frekuensi pemungutan suara yang diamati.
\end{eulercomment}
\begin{eulerprompt}
>writetable(expectedtable(A),wc=6,dc=1,labr=["m","f"],labc=1:6)
\end{eulerprompt}
\begin{euleroutput}
             1     2     3     4     5     6
       m  24.9  37.9  41.9  50.3  63.3  74.7
       f  25.1  38.1  42.1  50.7  63.7  75.3
\end{euleroutput}
\begin{eulercomment}
Kita dapat menghitung koefisien kontingensi yang dikoreksi. Karena
sangat mendekati 0, kami menyimpulkan bahwa pemungutan suara tidak
bergantung pada jenis kelamin.
\end{eulercomment}
\begin{eulerprompt}
>contingency(A)
\end{eulerprompt}
\begin{euleroutput}
  0.0427225484717
\end{euleroutput}
\eulerheading{Beberapa Tes Lagi}
\begin{eulercomment}
Selanjutnya kami menggunakan analisis varians (F-test) untuk menguji
tiga sampel data yang terdistribusi normal untuk nilai rata-rata yang
sama. Metode tersebut dinamakan ANOVA (analysis of variance). Di
Euler, fungsi varanalysis() digunakan.
\end{eulercomment}
\begin{eulerprompt}
>x1=[109,111,98,119,91,118,109,99,115,109,94]; mean(x1),
\end{eulerprompt}
\begin{euleroutput}
  106.545454545
\end{euleroutput}
\begin{eulerprompt}
>x2=[120,124,115,139,114,110,113,120,117]; mean(x2),
\end{eulerprompt}
\begin{euleroutput}
  119.111111111
\end{euleroutput}
\begin{eulerprompt}
>x3=[120,112,115,110,105,134,105,130,121,111]; mean(x3)
\end{eulerprompt}
\begin{euleroutput}
  116.3
\end{euleroutput}
\begin{eulerprompt}
>varanalysis(x1,x2,x3)
\end{eulerprompt}
\begin{euleroutput}
  0.0138048221371
\end{euleroutput}
\begin{eulercomment}
Ini berarti, kami menolak hipotesis dengan nilai rata-rata yang sama.
Kami melakukan ini dengan probabilitas kesalahan 1,3\%.

Ada juga uji median yang menolak sampel data dengan distribusi
rata-rata yang berbeda menguji median sampel bersatu.
\end{eulercomment}
\begin{eulerprompt}
>a=[56,66,68,49,61,53,45,58,54];
>b=[72,81,51,73,69,78,59,67,65,71,68,71];
>mediantest(a,b)
\end{eulerprompt}
\begin{euleroutput}
  0.0241724220052
\end{euleroutput}
\begin{eulercomment}
Tes lain tentang kesetaraan adalah tes peringkat. Ini jauh lebih tajam
daripada tes median.
\end{eulercomment}
\begin{eulerprompt}
>ranktest(a,b)
\end{eulerprompt}
\begin{euleroutput}
  0.00199969612469
\end{euleroutput}
\begin{eulercomment}
Dalam contoh berikut, kedua distribusi memiliki rata-rata yang sama.
\end{eulercomment}
\begin{eulerprompt}
>ranktest(random(1,100),random(1,50)*3-1)
\end{eulerprompt}
\begin{euleroutput}
  0.129608141484
\end{euleroutput}
\begin{eulercomment}
Mari kita coba mensimulasikan dua perlakuan a dan b yang diterapkan
pada orang yang berbeda.
\end{eulercomment}
\begin{eulerprompt}
>a=[8.0,7.4,5.9,9.4,8.6,8.2,7.6,8.1,6.2,8.9];
>b=[6.8,7.1,6.8,8.3,7.9,7.2,7.4,6.8,6.8,8.1];
\end{eulerprompt}
\begin{eulercomment}
Tes signum memutuskan, jika a lebih baik dari b.
\end{eulercomment}
\begin{eulerprompt}
>signtest(a,b)
\end{eulerprompt}
\begin{euleroutput}
  0.0546875
\end{euleroutput}
\begin{eulercomment}
Ini terlalu banyak kesalahan. Kita tidak dapat menolak bahwa a sama
baiknya dengan b.

Tes Wilcoxon lebih tajam dari tes ini, tetapi bergantung pada nilai
kuantitatif perbedaannya.
\end{eulercomment}
\begin{eulerprompt}
>wilcoxon(a,b)
\end{eulerprompt}
\begin{euleroutput}
  0.0296680599405
\end{euleroutput}
\begin{eulercomment}
Mari kita coba dua tes lagi menggunakan rangkaian yang dihasilkan.
\end{eulercomment}
\begin{eulerprompt}
>wilcoxon(normal(1,20),normal(1,20)-1)
\end{eulerprompt}
\begin{euleroutput}
  0.0068706451766
\end{euleroutput}
\begin{eulerprompt}
>wilcoxon(normal(1,20),normal(1,20))
\end{eulerprompt}
\begin{euleroutput}
  0.275145971064
\end{euleroutput}
\eulerheading{Angka Acak}
\begin{eulercomment}
Berikut ini adalah tes untuk generator angka acak. Euler menggunakan
generator yang sangat bagus, jadi kita tidak perlu berharap ada
masalah.

Pertama kami menghasilkan sepuluh juta angka acak di [0,1].
\end{eulercomment}
\begin{eulerprompt}
>n:=10000000; r:=random(1,n);
\end{eulerprompt}
\begin{eulercomment}
Selanjutnya kita menghitung jarak antara dua angka kurang dari 0,05.
\end{eulercomment}
\begin{eulerprompt}
>a:=0.05; d:=differences(nonzeros(r<a));
\end{eulerprompt}
\begin{eulercomment}
Akhirnya, kami memplot berapa kali, setiap jarak terjadi, dan
membandingkannya dengan nilai yang diharapkan.
\end{eulercomment}
\begin{eulerprompt}
>m=getmultiplicities(1:100,d); plot2d(m); ...
>  plot2d("n*(1-a)^(x-1)*a^2",color=red,>add):
\end{eulerprompt}
\eulerimg{15}{images/Davina Safa Felisa 1-6-709.png}
\begin{eulercomment}
Hapus datanya.
\end{eulercomment}
\begin{eulerprompt}
>remvalue n;
\end{eulerprompt}
\eulerheading{Pengantar untuk Pengguna Proyek R}
\begin{eulercomment}
Jelas, EMT tidak bersaing dengan R sebagai paket statistik. Namun, ada
banyak prosedur dan fungsi statistik yang tersedia di EMT juga. Jadi
EMT dapat memenuhi kebutuhan dasar. Lagi pula, EMT hadir dengan paket
numerik dan sistem aljabar komputer.

Notebook ini cocok untuk Anda jika sudah familiar dengan R, namun
perlu mengetahui perbedaan sintaks EMT dan R. Kami mencoba memberikan
gambaran umum tentang hal-hal yang jelas dan kurang jelas yang perlu
Anda ketahui.

Selain itu, kami mencari cara untuk bertukar data antara kedua sistem.
\end{eulercomment}
\begin{eulercomment}
Perhatikan bahwa ini adalah pekerjaan yang sedang berjalan.
\end{eulercomment}
\eulerheading{Sintaks Dasar}
\begin{eulercomment}
Hal pertama yang Anda pelajari di R adalah membuat vektor. Di EMT,
perbedaan utamanya adalah operator : dapat mengambil ukuran langkah.
Apalagi daya ikatnya rendah.
\end{eulercomment}
\begin{eulerprompt}
>n=10; 0:n/20:n-1
\end{eulerprompt}
\begin{euleroutput}
  [0,  0.5,  1,  1.5,  2,  2.5,  3,  3.5,  4,  4.5,  5,  5.5,  6,  6.5,
  7,  7.5,  8,  8.5,  9]
\end{euleroutput}
\begin{eulercomment}
Fungsi c() tidak ada. Dimungkinkan untuk menggunakan vektor untuk
menggabungkan berbagai hal.

Contoh berikut, seperti banyak lainnya, dari "Introduction to R" yang
disertakan dengan proyek R. Jika Anda membaca PDF ini, Anda akan
menemukan bahwa saya mengikuti jalannya dalam tutorial ini.
\end{eulercomment}
\begin{eulerprompt}
>x=[10.4, 5.6, 3.1, 6.4, 21.7]; [x,0,x]
\end{eulerprompt}
\begin{euleroutput}
  [10.4,  5.6,  3.1,  6.4,  21.7,  0,  10.4,  5.6,  3.1,  6.4,  21.7]
\end{euleroutput}
\begin{eulercomment}
Operator titik dua dengan ukuran langkah EMT diganti dengan fungsi
seq() di R. Kita bisa menulis fungsi ini di EMT.
\end{eulercomment}
\begin{eulerprompt}
>function seq(a,b,c) := a:b:c; ...
>seq(0,-0.1,-1)
\end{eulerprompt}
\begin{euleroutput}
  [0,  -0.1,  -0.2,  -0.3,  -0.4,  -0.5,  -0.6,  -0.7,  -0.8,  -0.9,  -1]
\end{euleroutput}
\begin{eulercomment}
Fungsi rep() dari R tidak ada di EMT. Untuk input vektor, dapat
ditulis sebagai berikut.
\end{eulercomment}
\begin{eulerprompt}
>function rep(x:vector,n:index) := flatten(dup(x,n)); ...
>rep(x,2)
\end{eulerprompt}
\begin{euleroutput}
  [10.4,  5.6,  3.1,  6.4,  21.7,  10.4,  5.6,  3.1,  6.4,  21.7]
\end{euleroutput}
\begin{eulercomment}
Perhatikan bahwa "=" atau ":=" digunakan untuk tugas. Operator "-\textgreater{}"
digunakan untuk unit di EMT.
\end{eulercomment}
\begin{eulerprompt}
>125km -> " miles"
\end{eulerprompt}
\begin{euleroutput}
  77.6713990297 miles
\end{euleroutput}
\begin{eulercomment}
The "\textless{}-" operator for assignment is misleading anyway, and not a good
idea of R. The following will compare a and -4 in EMT.
\end{eulercomment}
\begin{eulerprompt}
>a=2; a<-4
\end{eulerprompt}
\begin{euleroutput}
  0
\end{euleroutput}
\begin{eulercomment}
Di R, "a\textless{}-4\textless{}3" berfungsi, tetapi "a\textless{}-4\textless{}-3" tidak. Saya juga memiliki
ambiguitas serupa di EMT, tetapi mencoba menghilangkannya sedikit demi
sedikit.

EMT dan R memiliki vektor tipe boolean. Namun dalam EMT, angka 0 dan 1
digunakan untuk mewakili salah dan benar. Di R, nilai benar dan salah
tetap bisa digunakan dalam aritmatika biasa seperti di EMT.
\end{eulercomment}
\begin{eulerprompt}
>x<5, %*x
\end{eulerprompt}
\begin{euleroutput}
  [0,  0,  1,  0,  0]
  [0,  0,  3.1,  0,  0]
\end{euleroutput}
\begin{eulercomment}
EMT melempar kesalahan atau menghasilkan NAN tergantung pada bendera
"kesalahan".
\end{eulercomment}
\begin{eulerprompt}
>errors off; 0/0, isNAN(sqrt(-1)), errors on;
\end{eulerprompt}
\begin{euleroutput}
  NAN
  1
\end{euleroutput}
\begin{eulercomment}
String sama di R dan EMT. Keduanya berada di lokal saat ini, bukan di
Unicode.

Di R ada paket untuk Unicode. Di EMT, sebuah string dapat berupa
string Unicode. String unicode dapat diterjemahkan ke pengkodean lokal
dan sebaliknya. Selain itu, u"..." dapat berisi entitas HTML.
\end{eulercomment}
\begin{eulerprompt}
>u"&#169; Ren&eacut; Grothmann"
\end{eulerprompt}
\begin{euleroutput}
  © René Grothmann
\end{euleroutput}
\begin{eulercomment}
Berikut ini mungkin atau mungkin tidak ditampilkan dengan benar di
sistem Anda sebagai A dengan titik dan garis di atasnya. Itu
tergantung pada font yang Anda gunakan.
\end{eulercomment}
\begin{eulerprompt}
>chartoutf([480])
\end{eulerprompt}
\begin{euleroutput}
  Ǡ
\end{euleroutput}
\begin{eulercomment}
Penggabungan string dilakukan dengan "+" atau "\textbar{}". Itu bisa termasuk
angka, yang akan dicetak dalam format saat ini.
\end{eulercomment}
\begin{eulerprompt}
>"pi = "+pi
\end{eulerprompt}
\begin{euleroutput}
  pi = 3.14159265359
\end{euleroutput}
\eulerheading{Pengindeksan}
\begin{eulercomment}
Sebagian besar waktu, ini akan berfungsi seperti di R.

Tetapi EMT akan menginterpretasikan indeks negatif dari belakang
vektor, sedangkan R menginterpretasikan x[n] sebagai x tanpa elemen
ke-n.
\end{eulercomment}
\begin{eulerprompt}
>x, x[1:3], x[-2]
\end{eulerprompt}
\begin{euleroutput}
  [10.4,  5.6,  3.1,  6.4,  21.7]
  [10.4,  5.6,  3.1]
  6.4
\end{euleroutput}
\begin{eulercomment}
Perilaku R dapat dicapai dalam EMT dengan drop().
\end{eulercomment}
\begin{eulerprompt}
>drop(x,2)
\end{eulerprompt}
\begin{euleroutput}
  [10.4,  3.1,  6.4,  21.7]
\end{euleroutput}
\begin{eulercomment}
Vektor logis tidak diperlakukan berbeda sebagai indeks di EMT, berbeda
dengan R. Anda perlu mengekstraksi elemen bukan nol terlebih dahulu di
EMT.
\end{eulercomment}
\begin{eulerprompt}
>x, x>5, x[nonzeros(x>5)]
\end{eulerprompt}
\begin{euleroutput}
  [10.4,  5.6,  3.1,  6.4,  21.7]
  [1,  1,  0,  1,  1]
  [10.4,  5.6,  6.4,  21.7]
\end{euleroutput}
\begin{eulercomment}
Sama seperti di R, vektor indeks dapat berisi pengulangan.
\end{eulercomment}
\begin{eulerprompt}
>x[[1,2,2,1]]
\end{eulerprompt}
\begin{euleroutput}
  [10.4,  5.6,  5.6,  10.4]
\end{euleroutput}
\begin{eulercomment}
Tetapi nama untuk indeks tidak dimungkinkan di EMT. Untuk paket
statistik, hal ini sering diperlukan untuk memudahkan akses ke elemen
vektor.

Untuk meniru perilaku ini, kita dapat mendefinisikan fungsi sebagai
berikut.
\end{eulercomment}
\begin{eulerprompt}
>function sel (v,i,s) := v[indexof(s,i)]; ...
>s=["first","second","third","fourth"]; sel(x,["first","third"],s)
\end{eulerprompt}
\begin{euleroutput}
  
  Trying to overwrite protected function sel!
  Error in:
  function sel (v,i,s) := v[indexof(s,i)]; ... ...
               ^
  
  Trying to overwrite protected function sel!
  Error in:
  function sel (v,i,s) := v[indexof(s,i)]; ... ...
               ^
  
  Trying to overwrite protected function sel!
  Error in:
  function sel (v,i,s) := v[indexof(s,i)]; ... ...
               ^
  [10.4,  3.1]
\end{euleroutput}
\eulerheading{Tipe Data}
\begin{eulercomment}
EMT memiliki lebih banyak tipe data tetap daripada R. Jelas, di R
terdapat vektor yang tumbuh. Anda dapat menyetel vektor numerik kosong
v dan menetapkan nilai ke elemen v[17]. Ini tidak mungkin di EMT.

Berikut ini agak tidak efisien.
\end{eulercomment}
\begin{eulerprompt}
>v=[]; for i=1 to 10000; v=v|i; end;
\end{eulerprompt}
\begin{eulercomment}
EMT sekarang akan membuat vektor dengan v dan i ditambahkan pada
tumpukan dan menyalin vektor itu kembali ke variabel global v.

Semakin efisien pra-mendefinisikan vektor.
\end{eulercomment}
\begin{eulerprompt}
>v=zeros(10000); for i=1 to 10000; v[i]=i; end;
\end{eulerprompt}
\begin{eulercomment}
Untuk mengubah jenis tanggal di EMT, Anda dapat menggunakan fungsi
seperti complex().
\end{eulercomment}
\begin{eulerprompt}
>complex(1:4)
\end{eulerprompt}
\begin{euleroutput}
  [ 1+0i ,  2+0i ,  3+0i ,  4+0i  ]
\end{euleroutput}
\begin{eulercomment}
Konversi ke string hanya dimungkinkan untuk tipe data dasar. Format
saat ini digunakan untuk penggabungan string sederhana. Tapi ada
fungsi seperti print() atau frac().

Untuk vektor, Anda dapat dengan mudah menulis fungsi Anda sendiri.
\end{eulercomment}
\begin{eulerprompt}
>function tostr (v) ...
\end{eulerprompt}
\begin{eulerudf}
  s="[";
  loop 1 to length(v);
     s=s+print(v[#],2,0);
     if #<length(v) then s=s+","; endif;
  end;
  return s+"]";
  endfunction
\end{eulerudf}
\begin{eulerprompt}
>tostr(linspace(0,1,10))
\end{eulerprompt}
\begin{euleroutput}
  [0.00,0.10,0.20,0.30,0.40,0.50,0.60,0.70,0.80,0.90,1.00]
\end{euleroutput}
\begin{eulercomment}
Untuk komunikasi dengan Maxima, terdapat fungsi convertmxm(), yang
juga dapat digunakan untuk memformat vektor untuk output.
\end{eulercomment}
\begin{eulerprompt}
>convertmxm(1:10)
\end{eulerprompt}
\begin{euleroutput}
  [1,2,3,4,5,6,7,8,9,10]
\end{euleroutput}
\begin{eulercomment}
Untuk Lateks, perintah tex dapat digunakan untuk mendapatkan perintah
Lateks.
\end{eulercomment}
\begin{eulerprompt}
>tex(&[1,2,3])
\end{eulerprompt}
\begin{euleroutput}
  \(\backslash\)left[ 1 , 2 , 3 \(\backslash\)right] 
\end{euleroutput}
\eulerheading{Faktor dan Tabel}
\begin{eulercomment}
Dalam pengantar R ada contoh dengan apa yang disebut faktor.

Berikut ini adalah daftar wilayah dari 30 negara bagian.
\end{eulercomment}
\begin{eulerprompt}
>austates = ["tas", "sa", "qld", "nsw", "nsw", "nt", "wa", "wa", ...
>"qld", "vic", "nsw", "vic", "qld", "qld", "sa", "tas", ...
>"sa", "nt", "wa", "vic", "qld", "nsw", "nsw", "wa", ...
>"sa", "act", "nsw", "vic", "vic", "act"];
\end{eulerprompt}
\begin{eulercomment}
Asumsikan, kita memiliki pendapatan yang sesuai di setiap negara
bagian.
\end{eulercomment}
\begin{eulerprompt}
>incomes = [60, 49, 40, 61, 64, 60, 59, 54, 62, 69, 70, 42, 56, ...
>61, 61, 61, 58, 51, 48, 65, 49, 49, 41, 48, 52, 46, ...
>59, 46, 58, 43];
\end{eulerprompt}
\begin{eulercomment}
Sekarang, kami ingin menghitung rata-rata pendapatan di wilayah
tersebut. Menjadi program statistik, R memiliki factor() dan tappy()
untuk ini.

EMT dapat melakukannya dengan menemukan indeks wilayah di daftar unik
wilayah.
\end{eulercomment}
\begin{eulerprompt}
>auterr=sort(unique(austates)); f=indexofsorted(auterr,austates)
\end{eulerprompt}
\begin{euleroutput}
  [6,  5,  4,  2,  2,  3,  8,  8,  4,  7,  2,  7,  4,  4,  5,  6,  5,  3,
  8,  7,  4,  2,  2,  8,  5,  1,  2,  7,  7,  1]
\end{euleroutput}
\begin{eulercomment}
Pada saat itu, kita dapat menulis fungsi loop kita sendiri untuk
melakukan sesuatu hanya untuk satu faktor.

Atau kita bisa meniru fungsi tapply() dengan cara berikut.
\end{eulercomment}
\begin{eulerprompt}
>function map tappl (i; f$:call, cat, x) ...
\end{eulerprompt}
\begin{eulerudf}
  u=sort(unique(cat));
  f=indexof(u,cat);
  return f$(x[nonzeros(f==indexof(u,i))]);
  endfunction
\end{eulerudf}
\begin{eulercomment}
Ini sedikit tidak efisien, karena menghitung wilayah unik untuk setiap
i, tetapi berhasil.
\end{eulercomment}
\begin{eulerprompt}
>tappl(auterr,"mean",austates,incomes)
\end{eulerprompt}
\begin{euleroutput}
  [44.5,  57.3333333333,  55.5,  53.6,  55,  60.5,  56,  52.25]
\end{euleroutput}
\begin{eulercomment}
Perhatikan bahwa ini berfungsi untuk setiap vektor wilayah.
\end{eulercomment}
\begin{eulerprompt}
>tappl(["act","nsw"],"mean",austates,incomes)
\end{eulerprompt}
\begin{euleroutput}
  [44.5,  57.3333333333]
\end{euleroutput}
\begin{eulercomment}
Sekarang, paket statistik EMT mendefinisikan tabel seperti pada R.
Fungsi readtable() dan writetable() dapat digunakan untuk input dan
output.

Sehingga kita bisa mencetak rata-rata pendapatan negara di daerah
dengan cara yang bersahabat.
\end{eulercomment}
\begin{eulerprompt}
>writetable(tappl(auterr,"mean",austates,incomes),labc=auterr,wc=7)
\end{eulerprompt}
\begin{euleroutput}
      act    nsw     nt    qld     sa    tas    vic     wa
     44.5  57.33   55.5   53.6     55   60.5     56  52.25
\end{euleroutput}
\begin{eulercomment}
Kami juga dapat mencoba meniru perilaku R sepenuhnya.

Faktor jelas harus disimpan dalam kumpulan dengan jenis dan kategori
(negara bagian dan teritori dalam contoh kita). Untuk EMT, kami
menambahkan indeks yang telah dihitung sebelumnya.
\end{eulercomment}
\begin{eulerprompt}
>function makef (t) ...
\end{eulerprompt}
\begin{eulerudf}
  ## Factor data
  ## Returns a collection with data t, unique data, indices.
  ## See: tapply
  u=sort(unique(t));
  return \{\{t,u,indexofsorted(u,t)\}\};
  endfunction
\end{eulerudf}
\begin{eulerprompt}
>statef=makef(austates);
\end{eulerprompt}
\begin{eulercomment}
Sekarang elemen ketiga dari koleksi akan berisi indeks.
\end{eulercomment}
\begin{eulerprompt}
>statef[3]
\end{eulerprompt}
\begin{euleroutput}
  [6,  5,  4,  2,  2,  3,  8,  8,  4,  7,  2,  7,  4,  4,  5,  6,  5,  3,
  8,  7,  4,  2,  2,  8,  5,  1,  2,  7,  7,  1]
\end{euleroutput}
\begin{eulercomment}
Sekarang kita bisa meniru tapply() dengan cara berikut. Ini akan
mengembalikan tabel sebagai kumpulan data tabel dan judul kolom.
\end{eulercomment}
\begin{eulerprompt}
>function tapply (t:vector,tf,f$:call) ...
\end{eulerprompt}
\begin{eulerudf}
  ## Makes a table of data and factors
  ## tf : output of makef()
  ## See: makef
  uf=tf[2]; f=tf[3]; x=zeros(length(uf));
  for i=1 to length(uf);
     ind=nonzeros(f==i);
     if length(ind)==0 then x[i]=NAN;
     else x[i]=f$(t[ind]);
     endif;
  end;
  return \{\{x,uf\}\};
  endfunction
\end{eulerudf}
\begin{eulercomment}
Kami tidak menambahkan banyak pengecekan tipe di sini. Satu-satunya
tindakan pencegahan menyangkut kategori (faktor) tanpa data. Tetapi
orang harus memeriksa panjang t yang benar dan kebenaran koleksi tf.

Tabel ini dapat dicetak sebagai tabel dengan writetable().
\end{eulercomment}
\begin{eulerprompt}
>writetable(tapply(incomes,statef,"mean"),wc=7)
\end{eulerprompt}
\begin{euleroutput}
      act    nsw     nt    qld     sa    tas    vic     wa
     44.5  57.33   55.5   53.6     55   60.5     56  52.25
\end{euleroutput}
\eulerheading{Array}
\begin{eulercomment}
EMT hanya memiliki dua dimensi untuk array. Tipe datanya disebut
matriks. Namun, akan mudah untuk menulis fungsi untuk dimensi yang
lebih tinggi atau pustaka C untuk ini.

R memiliki lebih dari dua dimensi. Di R array adalah vektor dengan
bidang dimensi.

Dalam EMT, vektor adalah matriks dengan satu baris. Itu dapat dibuat
menjadi matriks dengan redim().
\end{eulercomment}
\begin{eulerprompt}
>shortformat; X=redim(1:20,4,5)
\end{eulerprompt}
\begin{euleroutput}
          1         2         3         4         5 
          6         7         8         9        10 
         11        12        13        14        15 
         16        17        18        19        20 
\end{euleroutput}
\begin{eulercomment}
Ekstraksi baris dan kolom, atau sub-matriks, sangat mirip dengan R.
\end{eulercomment}
\begin{eulerprompt}
>X[,2:3]
\end{eulerprompt}
\begin{euleroutput}
          2         3 
          7         8 
         12        13 
         17        18 
\end{euleroutput}
\begin{eulercomment}
Namun, dalam R dimungkinkan untuk menetapkan daftar indeks spesifik
vektor ke suatu nilai. Hal yang sama dimungkinkan di EMT hanya dengan
satu putaran.
\end{eulercomment}
\begin{eulerprompt}
>function setmatrixvalue (M, i, j, v) ...
\end{eulerprompt}
\begin{eulerudf}
  loop 1 to max(length(i),length(j),length(v))
     M[i\{#\},j\{#\}] = v\{#\};
  end;
  endfunction
\end{eulerudf}
\begin{eulercomment}
Kami mendemonstrasikan ini untuk menunjukkan bahwa matriks dilewatkan
dengan referensi di EMT. Jika Anda tidak ingin mengubah matriks asli
M, Anda perlu menyalinnya ke dalam fungsi.
\end{eulercomment}
\begin{eulerprompt}
>setmatrixvalue(X,1:3,3:-1:1,0); X,
\end{eulerprompt}
\begin{euleroutput}
          1         2         0         4         5 
          6         0         8         9        10 
          0        12        13        14        15 
         16        17        18        19        20 
\end{euleroutput}
\begin{eulercomment}
Produk luar di EMT hanya dapat dilakukan di antara vektor. Ini
otomatis karena bahasa matriks. Satu vektor harus berupa vektor kolom
dan yang lainnya vektor baris.
\end{eulercomment}
\begin{eulerprompt}
>(1:5)*(1:5)'
\end{eulerprompt}
\begin{euleroutput}
          1         2         3         4         5 
          2         4         6         8        10 
          3         6         9        12        15 
          4         8        12        16        20 
          5        10        15        20        25 
\end{euleroutput}
\begin{eulercomment}
Dalam pengantar PDF untuk R ada contoh, yang menghitung distribusi
ab-cd untuk a,b,c,d dipilih dari 0 sampai n secara acak. Solusi dalam
R adalah membentuk matriks 4 dimensi dan menjalankan table() di
atasnya.

Tentu saja, ini bisa dicapai dengan satu putaran. Tapi loop tidak
efektif di EMT atau R. Di EMT, kita bisa menulis loop di C dan itu
akan menjadi solusi tercepat.

Tapi kami ingin meniru perilaku R. Untuk ini, kami perlu meratakan
perkalian ab dan membuat matriks ab-cd.
\end{eulercomment}
\begin{eulerprompt}
>a=0:6; b=a'; p=flatten(a*b); q=flatten(p-p'); ...
>u=sort(unique(q)); f=getmultiplicities(u,q); ...
>statplot(u,f,"h"):
\end{eulerprompt}
\eulerimg{15}{images/Davina Safa Felisa 1-6-710.png}
\begin{eulercomment}
Selain perkalian yang tepat, EMT dapat menghitung frekuensi dalam
vektor.
\end{eulercomment}
\begin{eulerprompt}
>getfrequencies(q,-50:10:50)
\end{eulerprompt}
\begin{euleroutput}
  [0,  23,  132,  316,  602,  801,  333,  141,  53,  0]
\end{euleroutput}
\begin{eulercomment}
Cara paling mudah untuk memplot ini sebagai distribusi adalah sebagai
berikut.
\end{eulercomment}
\begin{eulerprompt}
>plot2d(q,distribution=11):
\end{eulerprompt}
\eulerimg{15}{images/Davina Safa Felisa 1-6-711.png}
\begin{eulercomment}
Tetapi juga memungkinkan untuk melakukan pra-perhitungan hitungan
dalam interval yang dipilih sebelumnya. Tentu saja, berikut ini
menggunakan getfrequencies() secara internal.

Karena fungsi histo() mengembalikan frekuensi, kita perlu
menskalakannya sehingga integral di bawah grafik batang adalah 1.
\end{eulercomment}
\begin{eulerprompt}
>\{x,y\}=histo(q,v=-55:10:55); y=y/sum(y)/differences(x); ...
>plot2d(x,y,>bar,style="/"):
\end{eulerprompt}
\eulerimg{15}{images/Davina Safa Felisa 1-6-712.png}
\eulerheading{Daftar}
\begin{eulercomment}
EMT memiliki dua jenis daftar. Salah satunya adalah daftar global yang
bisa berubah, dan yang lainnya adalah tipe daftar yang tidak bisa
diubah. Kami tidak peduli dengan daftar global di sini.

Jenis daftar yang tidak dapat diubah disebut koleksi di EMT. Ini
berperilaku seperti struktur di C, tetapi elemennya hanya diberi nomor
dan tidak diberi nama.
\end{eulercomment}
\begin{eulerprompt}
>L=\{\{"Fred","Flintstone",40,[1990,1992]\}\}
\end{eulerprompt}
\begin{euleroutput}
  Fred
  Flintstone
  40
  [1990,  1992]
\end{euleroutput}
\begin{eulercomment}
Saat ini elemen tidak memiliki nama, meskipun nama dapat diatur untuk
tujuan khusus. Mereka diakses oleh nomor.
\end{eulercomment}
\begin{eulerprompt}
>(L[4])[2]
\end{eulerprompt}
\begin{euleroutput}
  1992
\end{euleroutput}
\eulerheading{File Input dan Output (Membaca dan Menulis Data)}
\begin{eulercomment}
Anda sering ingin mengimpor matriks data dari sumber lain ke EMT.
Tutorial ini memberitahu Anda tentang banyak cara untuk mencapai hal
ini. Fungsi sederhana adalah writematrix() dan readmatrix().

Mari kita tunjukkan cara membaca dan menulis vektor real ke file.
\end{eulercomment}
\begin{eulerprompt}
>a=random(1,100); mean(a), dev(a),
\end{eulerprompt}
\begin{euleroutput}
  0.49815
  0.28037
\end{euleroutput}
\begin{eulercomment}
Untuk menulis data ke file, kami menggunakan fungsi writematrix().

Karena pengantar ini kemungkinan besar ada di direktori, di mana
pengguna tidak memiliki akses tulis, kami menulis data ke direktori
home pengguna. Untuk buku catatan sendiri, hal ini tidak diperlukan,
karena file data akan ditulis ke dalam direktori yang sama.
\end{eulercomment}
\begin{eulerprompt}
>filename="test.dat";
\end{eulerprompt}
\begin{eulercomment}
Sekarang kita menulis vektor kolom a' ke file. Ini menghasilkan satu
nomor di setiap baris file.
\end{eulercomment}
\begin{eulerprompt}
>writematrix(a',filename);
\end{eulerprompt}
\begin{eulercomment}
Untuk membaca data, kami menggunakan readmatrix().
\end{eulercomment}
\begin{eulerprompt}
>a=readmatrix(filename)';
\end{eulerprompt}
\begin{eulercomment}
Dan hapus file tersebut.
\end{eulercomment}
\begin{eulerprompt}
>fileremove(filename);
>mean(a), dev(a),
\end{eulerprompt}
\begin{euleroutput}
  0.49815
  0.28037
\end{euleroutput}
\begin{eulercomment}
Fungsi writematrix() atau writetable() dapat dikonfigurasi untuk
bahasa lain.

Misalnya, jika Anda memiliki sistem bahasa Indonesia (titik desimal
dengan koma), Excel Anda memerlukan nilai dengan koma desimal yang
dipisahkan oleh titik koma dalam file csv (defaultnya adalah nilai
yang dipisahkan koma). File berikut "test.csv" akan muncul di folder
cuurent Anda.
\end{eulercomment}
\begin{eulerprompt}
>filename="test.csv"; ...
>writematrix(random(5,3),file=filename,separator=",");
\end{eulerprompt}
\begin{eulercomment}
Anda sekarang dapat membuka file ini dengan Excel bahasa Indonesia
secara langsung.
\end{eulercomment}
\begin{eulerprompt}
>fileremove(filename);
\end{eulerprompt}
\begin{eulercomment}
Terkadang kami memiliki string dengan token seperti berikut ini.
\end{eulercomment}
\begin{eulerprompt}
>s1:="f m m f m m m f f f m m f";  ...
>s2:="f f f m m f f";
\end{eulerprompt}
\begin{eulercomment}
Untuk menandai ini, kami mendefinisikan vektor token.
\end{eulercomment}
\begin{eulerprompt}
>tok:=["f","m"]
\end{eulerprompt}
\begin{euleroutput}
  f
  m
\end{euleroutput}
\begin{eulercomment}
Kemudian kita dapat menghitung berapa kali setiap token muncul dalam
string, dan memasukkan hasilnya ke dalam tabel.
\end{eulercomment}
\begin{eulerprompt}
>M:=getmultiplicities(tok,strtokens(s1))_ ...
>  getmultiplicities(tok,strtokens(s2));
\end{eulerprompt}
\begin{eulercomment}
Tulis tabel dengan header token.
\end{eulercomment}
\begin{eulerprompt}
>writetable(M,labc=tok,labr=1:2,wc=8)
\end{eulerprompt}
\begin{euleroutput}
                 f       m
         1       6       7
         2       5       2
\end{euleroutput}
\begin{eulercomment}
Untuk statika, EMT dapat membaca dan menulis tabel.
\end{eulercomment}
\begin{eulerprompt}
>file="test.dat"; open(file,"w"); ...
>writeln("A,B,C"); writematrix(random(3,3)); ...
>close();
\end{eulerprompt}
\begin{eulercomment}
The file looks like this.
\end{eulercomment}
\begin{eulerprompt}
>printfile(file)
\end{eulerprompt}
\begin{euleroutput}
  A,B,C
  0.7003664386138074,0.1875530821001213,0.3262339279660414
  0.5926249243193858,0.1522927283984059,0.368140583062521
  0.8065535209872989,0.7265910840408142,0.7332619844597152
  
\end{euleroutput}
\begin{eulercomment}
Fungsi readtable() dalam bentuknya yang paling sederhana dapat membaca
ini dan mengembalikan kumpulan nilai dan baris heading.
\end{eulercomment}
\begin{eulerprompt}
>L=readtable(file,>list);
\end{eulerprompt}
\begin{eulercomment}
Koleksi ini dapat dicetak dengan writetable() ke notebook, atau ke
file.
\end{eulercomment}
\begin{eulerprompt}
>writetable(L,wc=10,dc=5)
\end{eulerprompt}
\begin{euleroutput}
           A         B         C
     0.70037   0.18755   0.32623
     0.59262   0.15229   0.36814
     0.80655   0.72659   0.73326
\end{euleroutput}
\begin{eulercomment}
Matriks nilai adalah elemen pertama dari L. Perhatikan bahwa mean()
dalam EMT menghitung nilai rata-rata dari baris matriks.
\end{eulercomment}
\begin{eulerprompt}
>mean(L[1])
\end{eulerprompt}
\begin{euleroutput}
    0.40472 
    0.37102 
    0.75547 
\end{euleroutput}
\eulerheading{File CSV}
\begin{eulercomment}
Pertama, mari kita menulis matriks ke dalam file. Untuk hasilnya, kami
membuat file di direktori kerja saat ini.
\end{eulercomment}
\begin{eulerprompt}
>file="test.csv";  ...
>M=random(3,3); writematrix(M,file);
\end{eulerprompt}
\begin{eulercomment}
Berikut adalah isi dari file ini.
\end{eulercomment}
\begin{eulerprompt}
>printfile(file)
\end{eulerprompt}
\begin{euleroutput}
  0.8221197733097619,0.821531098722547,0.7771240608094004
  0.8482947121863489,0.3237767724883862,0.6501422353377985
  0.1482301827518109,0.3297459716109594,0.6261901074210923
  
\end{euleroutput}
\begin{eulercomment}
CVS ini dapat dibuka pada sistem bahasa Inggris ke dalam Excel dengan
klik dua kali. Jika Anda mendapatkan file seperti itu di sistem
Jerman, Anda perlu mengimpor data ke Excel dengan memperhatikan titik
desimal.

Tetapi titik desimal juga merupakan format default untuk EMT. Anda
dapat membaca matriks dari file dengan readmatrix().
\end{eulercomment}
\begin{eulerprompt}
>readmatrix(file)
\end{eulerprompt}
\begin{euleroutput}
    0.82212   0.82153   0.77712 
    0.84829   0.32378   0.65014 
    0.14823   0.32975   0.62619 
\end{euleroutput}
\begin{eulercomment}
Dimungkinkan untuk menulis beberapa matriks ke satu file. Perintah
open() dapat membuka file untuk ditulis dengan parameter "w".
Standarnya adalah "r" untuk membaca.
\end{eulercomment}
\begin{eulerprompt}
>open(file,"w"); writematrix(M); writematrix(M'); close();
\end{eulerprompt}
\begin{eulercomment}
Matriks dipisahkan oleh garis kosong. Untuk membaca matriks, buka file
dan panggil readmatrix() beberapa kali.
\end{eulercomment}
\begin{eulerprompt}
>open(file); A=readmatrix(); B=readmatrix(); A==B, close();
\end{eulerprompt}
\begin{euleroutput}
          1         0         0 
          0         1         0 
          0         0         1 
\end{euleroutput}
\begin{eulercomment}
Di Excel atau spreadsheet serupa, Anda dapat mengekspor matriks
sebagai CSV (nilai yang dipisahkan koma). Di Excel 2007, gunakan
"simpan sebagai" dan "format lain", lalu pilih "CSV". Pastikan, tabel
saat ini hanya berisi data yang ingin Anda ekspor.

Ini sebuah contoh.
\end{eulercomment}
\begin{eulerprompt}
>printfile("excel-data.csv")
\end{eulerprompt}
\begin{euleroutput}
  Could not open the file
  excel-data.csv
  for reading!
  Try "trace errors" to inspect local variables after errors.
  printfile:
      open(filename,"r");
\end{euleroutput}
\begin{eulercomment}
Seperti yang Anda lihat, sistem Jerman saya menggunakan titik koma
sebagai pemisah dan koma desimal. Anda dapat mengubahnya di pengaturan
sistem atau di Excel, tetapi tidak perlu membaca matriks ke dalam EMT.

Cara termudah untuk membaca ini ke Euler adalah readmatrix(). Semua
koma diganti dengan titik dengan parameter \textgreater{}koma. Untuk CSV bahasa
Inggris, hilangkan saja parameter ini.
\end{eulercomment}
\begin{eulerprompt}
>M=readmatrix("excel-data.csv",>comma)
\end{eulerprompt}
\begin{euleroutput}
  Could not open the file
  excel-data.csv
  for reading!
  Try "trace errors" to inspect local variables after errors.
  readmatrix:
      if filename<>"" then open(filename,"r"); endif;
\end{euleroutput}
\begin{eulercomment}
Let us plot this.
\end{eulercomment}
\begin{eulerprompt}
>plot2d(M'[1],M'[2:3],>points,color=[red,green]'):
\end{eulerprompt}
\eulerimg{15}{images/Davina Safa Felisa 1-6-713.png}
\begin{eulercomment}
Ada cara yang lebih mendasar untuk membaca data dari file. Anda dapat
membuka file dan membaca angka baris demi baris. Fungsi
getvectorline() akan membaca angka dari baris data. Secara default,
ini mengharapkan titik desimal. Tapi itu juga bisa menggunakan koma
desimal, jika Anda memanggil setdecimaldot(",") sebelum Anda
menggunakan fungsi ini.

Fungsi berikut adalah contoh untuk ini. Itu akan berhenti di akhir
file atau baris kosong.
\end{eulercomment}
\begin{eulerprompt}
>function myload (file) ...
\end{eulerprompt}
\begin{eulerudf}
  open(file);
  M=[];
  repeat
     until eof();
     v=getvectorline(3);
     if length(v)>0 then M=M_v; else break; endif;
  end;
  return M;
  close(file);
  endfunction
\end{eulerudf}
\begin{eulerprompt}
>myload(file)
\end{eulerprompt}
\begin{euleroutput}
    0.82212         0   0.82153         0   0.77712 
    0.84829         0   0.32378         0   0.65014 
    0.14823         0   0.32975         0   0.62619 
\end{euleroutput}
\begin{eulercomment}
Dimungkinkan juga untuk membaca semua angka dalam file itu dengan
getvector().
\end{eulercomment}
\begin{eulerprompt}
>open(file); v=getvector(10000); close(); redim(v[1:9],3,3)
\end{eulerprompt}
\begin{euleroutput}
    0.82212         0   0.82153 
          0   0.77712   0.84829 
          0   0.32378         0 
\end{euleroutput}
\begin{eulercomment}
Thus it is very easy to save a vector of values, one value in each
line and read back this vector.
\end{eulercomment}
\begin{eulerprompt}
>v=random(1000); mean(v)
\end{eulerprompt}
\begin{euleroutput}
  0.50303
\end{euleroutput}
\begin{eulerprompt}
>writematrix(v',file); mean(readmatrix(file)')
\end{eulerprompt}
\begin{euleroutput}
  0.50303
\end{euleroutput}
\eulerheading{Menggunakan Tabel}
\begin{eulercomment}
Tabel dapat digunakan untuk membaca atau menulis data numerik. Sebagai
contoh, kami menulis tabel dengan tajuk baris dan kolom ke file.
\end{eulercomment}
\begin{eulerprompt}
>file="test.tab"; M=random(3,3);  ...
>open(file,"w");  ...
>writetable(M,separator=",",labc=["one","two","three"]);  ...
>close(); ...
>printfile(file)
\end{eulerprompt}
\begin{euleroutput}
  one,two,three
        0.09,      0.39,      0.86
        0.39,      0.86,      0.71
         0.2,      0.02,      0.83
\end{euleroutput}
\begin{eulercomment}
Ini dapat diimpor ke Excel.

Untuk membaca file di EMT, kami menggunakan readtable().
\end{eulercomment}
\begin{eulerprompt}
>\{M,headings\}=readtable(file,>clabs); ...
>writetable(M,labc=headings)
\end{eulerprompt}
\begin{euleroutput}
         one       two     three
        0.09      0.39      0.86
        0.39      0.86      0.71
         0.2      0.02      0.83
\end{euleroutput}
\eulerheading{Menganalisis Garis}
\begin{eulercomment}
Anda bahkan dapat mengevaluasi setiap baris dengan tangan. Misalkan,
kita memiliki garis dengan format berikut.
\end{eulercomment}
\begin{eulerprompt}
>line="2020-11-03,Tue,1'114.05"
\end{eulerprompt}
\begin{euleroutput}
  2020-11-03,Tue,1'114.05
\end{euleroutput}
\begin{eulercomment}
Pertama kita dapat menandai garis.
\end{eulercomment}
\begin{eulerprompt}
>vt=strtokens(line)
\end{eulerprompt}
\begin{euleroutput}
  2020-11-03
  Tue
  1'114.05
\end{euleroutput}
\begin{eulercomment}
Kemudian kita dapat mengevaluasi setiap elemen garis menggunakan
evaluasi yang sesuai.
\end{eulercomment}
\begin{eulerprompt}
>day(vt[1]),  ...
>indexof(["mon","tue","wed","thu","fri","sat","sun"],tolower(vt[2])),  ...
>strrepl(vt[3],"'","")()
\end{eulerprompt}
\begin{euleroutput}
  7.3816e+05
  2
  1114
\end{euleroutput}
\begin{eulercomment}
Menggunakan ekspresi reguler, dimungkinkan untuk mengekstraksi hampir
semua informasi dari sebaris data.

Asumsikan kita memiliki baris berikut sebuah dokumen HTML.
\end{eulercomment}
\begin{eulerprompt}
>line="<tr><td>1145.45</td><td>5.6</td><td>-4.5</td><tr>"
\end{eulerprompt}
\begin{euleroutput}
  <tr><td>1145.45</td><td>5.6</td><td>-4.5</td><tr>
\end{euleroutput}
\begin{eulercomment}
Untuk mengekstrak ini, kami menggunakan ekspresi reguler, yang mencari

\end{eulercomment}
\begin{eulerttcomment}
 - tanda kurung tutup >,
 - string apa pun yang tidak mengandung tanda kurung dengan
\end{eulerttcomment}
\begin{eulercomment}
sub-pertandingan "(...)",\\
\end{eulercomment}
\begin{eulerttcomment}
 - braket pembuka dan penutup menggunakan solusi terpendek,
 - sekali lagi string apa pun yang tidak mengandung tanda kurung,
 - dan tanda kurung buka <.
\end{eulerttcomment}
\begin{eulercomment}

Ekspresi reguler agak sulit dipelajari tetapi sangat kuat.
\end{eulercomment}
\begin{eulerprompt}
>\{pos,s,vt\}=strxfind(line,">([^<>]+)<.+?>([^<>]+)<");
\end{eulerprompt}
\begin{eulercomment}
Hasilnya adalah posisi kecocokan, string yang cocok, dan vektor string
untuk sub-kecocokan.
\end{eulercomment}
\begin{eulerprompt}
>for k=1:length(vt); vt[k](), end;
\end{eulerprompt}
\begin{euleroutput}
  1145.5
  5.6
\end{euleroutput}
\begin{eulercomment}
Ini adalah fungsi yang membaca semua item numerik antara \textless{}td\textgreater{} dan
\textless{}/td\textgreater{}.
\end{eulercomment}
\begin{eulerprompt}
>function readtd (line) ...
\end{eulerprompt}
\begin{eulerudf}
  v=[]; cp=0;
  repeat
     \{pos,s,vt\}=strxfind(line,"<td.*?>(.+?)</td>",cp);
     until pos==0;
     if length(vt)>0 then v=v|vt[1]; endif;
     cp=pos+strlen(s);
  end;
  return v;
  endfunction
\end{eulerudf}
\begin{eulerprompt}
>readtd(line+"<td>non-numerical</td>")
\end{eulerprompt}
\begin{euleroutput}
  1145.45
  5.6
  -4.5
  non-numerical
\end{euleroutput}
\eulerheading{Membaca dari Web}
\begin{eulercomment}
Situs web atau file dengan URL dapat dibuka di EMT dan dapat dibaca
baris demi baris.

Dalam contoh, kami membaca versi terkini dari situs EMT. Kami
menggunakan ekspresi reguler untuk memindai "Versi ..." dalam judul.
\end{eulercomment}
\begin{eulerprompt}
>function readversion () ...
\end{eulerprompt}
\begin{eulerudf}
  urlopen("http://www.euler-math-toolbox.de/Programs/Changes.html");
  repeat
    until urleof();
    s=urlgetline();
    k=strfind(s,"Version ",1);
    if k>0 then substring(s,k,strfind(s,"<",k)-1), break; endif;
  end;
  urlclose();
  endfunction
\end{eulerudf}
\begin{eulerprompt}
>readversion
\end{eulerprompt}
\begin{euleroutput}
  Version 2024-01-12
\end{euleroutput}
\eulerheading{Input dan Output Variabel}
\begin{eulercomment}
Anda dapat menulis variabel dalam bentuk definisi Euler ke file atau
ke baris perintah.
\end{eulercomment}
\begin{eulerprompt}
>writevar(pi,"mypi");
\end{eulerprompt}
\begin{euleroutput}
  mypi = 3.141592653589793;
\end{euleroutput}
\begin{eulercomment}
Untuk pengujian, kami membuat file Euler di direktori kerja EMT.
\end{eulercomment}
\begin{eulerprompt}
>file="test.e"; ...
>writevar(random(2,2),"M",file); ...
>printfile(file,3)
\end{eulerprompt}
\begin{euleroutput}
  M = [ ..
  0.5991820585590205, 0.7960280262224293;
  0.5167243983231363, 0.2996684599070898];
\end{euleroutput}
\begin{eulercomment}
We can now load the file. It will define the matrix M.
\end{eulercomment}
\begin{eulerprompt}
>load(file); show M,
\end{eulerprompt}
\begin{euleroutput}
  M = 
    0.59918   0.79603 
    0.51672   0.29967 
\end{euleroutput}
\begin{eulercomment}
By the way, jika writevar() digunakan pada variabel, itu akan mencetak
definisi variabel dengan nama variabel ini.
\end{eulercomment}
\begin{eulerprompt}
>writevar(M); writevar(inch$)
\end{eulerprompt}
\begin{euleroutput}
  M = [ ..
  0.5991820585590205, 0.7960280262224293;
  0.5167243983231363, 0.2996684599070898];
  inch$ = 0.0254;
\end{euleroutput}
\begin{eulercomment}
Kami juga dapat membuka file baru atau menambahkan file yang sudah
ada. Dalam contoh kami menambahkan file yang dihasilkan sebelumnya.
\end{eulercomment}
\begin{eulerprompt}
>open(file,"a"); ...
>writevar(random(2,2),"M1"); ...
>writevar(random(3,1),"M2"); ...
>close();
>load(file); show M1; show M2;
\end{eulerprompt}
\begin{euleroutput}
  M1 = 
    0.30287   0.15372 
     0.7504   0.75401 
  M2 = 
    0.27213 
   0.053211 
    0.70249 
\end{euleroutput}
\begin{eulercomment}
Untuk menghapus file apa pun gunakan fileremove().
\end{eulercomment}
\begin{eulerprompt}
>fileremove(file);
\end{eulerprompt}
\begin{eulercomment}
Vektor baris dalam file tidak memerlukan koma, jika setiap angka
berada di baris baru. Mari kita buat file seperti itu, menulis setiap
baris satu per satu dengan writeln().
\end{eulercomment}
\begin{eulerprompt}
>open(file,"w"); writeln("M = ["); ...
>for i=1 to 5; writeln(""+random()); end; ...
>writeln("];"); close(); ...
>printfile(file)
\end{eulerprompt}
\begin{euleroutput}
  M = [
  0.344851384551
  0.0807510017715
  0.876519562911
  0.754157709472
  0.688392638934
  ];
\end{euleroutput}
\begin{eulerprompt}
>load(file); M
\end{eulerprompt}
\begin{euleroutput}
  [0.34485,  0.080751,  0.87652,  0.75416,  0.68839]
\end{euleroutput}
\end{eulernotebook}
\end{document}
